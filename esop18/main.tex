\documentclass{llncs}
%

%External Packages
\usepackage{makeidx}  % allows for indexgeneration
\usepackage[table]{xcolor}
\usepackage{amsmath}
\usepackage{listings}
\usepackage{hyperref}
\usepackage{graphicx}
\usepackage{tabularx}
\usepackage{xspace}
\usepackage{textcomp}
\usepackage{amssymb,amsfonts,latexsym,wasysym,mathrsfs,textcomp,stmaryrd}
\usepackage{mathpartir}
\usepackage{url}

\usepackage{upgreek}


%JavaScript 
\definecolor{SkyBlue}{rgb}{0.20,0.39,0.64}
\definecolor{Plum}{rgb}{0.46,0.31,0.48}
\definecolor{Chocolate}{rgb}{0.75,0.49,0.07}
\definecolor{Aluminium5}{rgb}{0.33,0.34,0.32}
\definecolor{DarkGreen}{rgb}{0.2,0.5,0.2}
\definecolor{ltblue}{rgb}{0,0.4,0.4}
\definecolor{dkblue}{rgb}{0,0.2,0.7}
\definecolor{dkgreen}{rgb}{0,0.4,0}
\definecolor{dkviolet}{rgb}{0.3,0,0.5}
\definecolor{dkred}{rgb}{0.6,0,0}
\definecolor{talkred}{rgb}{0.69,.20,0.22}
\definecolor{talkblue}{rgb}{0.04,0.40,0.80}
\definecolor{talkgreen}{rgb}{0.34,.81,0.10}
\definecolor{oldtalkblue}{rgb}{0.22,.20,0.69}
\definecolor{greenish}{rgb}{.0,.65,.0}

\lstdefinelanguage{JavaScript}{
  morekeywords=[1]{typeof, new, true, false, catch,
    function, return, null, catch, switch, var,
    if, in, while, do, else, case, break, continue},
  morekeywords=[2]{class, export, boolean, throw, implements, import, this},
  numbers=left,
  numbersep=4pt,
  numberstyle=\tiny\color{dkblue},
  columns=fullflexible,
  sensitive=false,
  comment=[l]{//},
  captionpos=b,   
  morecomment=[s]{/*}{*/},
  morestring=[b]',
  morestring=[b]",
  basicstyle=\scriptsize\texttt,
  identifierstyle=\ttfamily\color{Aluminium5},
  keywordstyle=[1]\ttfamily\color{Plum},
  keywordstyle=[2]\ttfamily\color{SkyBlue},
  stringstyle=\ttfamily\color{DarkGreen},
  commentstyle=\ttfamily
}[keywords,comments,strings]

\lstdefinelanguage{Scheme}{
  morekeywords=[1]{define, define-syntax, define-macro, lambda, define-stream, stream-lambda},
  morekeywords=[2]{begin, call-with-current-continuation, call/cc,
    call-with-input-file, call-with-output-file, case, cond,
    do, else, for-each, if,
    let*, let, let-syntax, letrec, letrec-syntax,
    let-values, let*-values,
    and, or, not, delay, force,
    quasiquote, quote, unquote, unquote-splicing,
    map, fold, syntax, syntax-rules, eval, environment, query },
  morekeywords=[3]{import, export},
  alsodigit=!\$\%&*+-./:<=>?@^_~,
  sensitive=true,
  morecomment=[l]{;},
  morecomment=[s]{\#|}{|\#},
  morestring=[b]",
  basicstyle=\scriptsize\ttfamily,
  keywordstyle=\bf\ttfamily\color[rgb]{0,.3,.7},
  commentstyle=\color[rgb]{0.133,0.545,0.133},
  stringstyle={\color[rgb]{0.75,0.49,0.07}},
  upquote=true,
  breaklines=true,
  breakatwhitespace=true,
  literate=*{`}{{`}}{1}
}

\lstnewenvironment{lstjs}{\lstset{language=JavaScript,basicstyle=\fontsize{8}{8}\ttfamily,escapeinside={~}{~}}}{}
\def\jsinline{\lstinline[language=JavaScript, basicstyle=\small]}


% The Acronyms of the project and some other stuff
\newcommand{\jsil}{JSIL\xspace}
\newcommand{\jsverify}{JSVerify\xspace}
\newcommand{\JSComp}{JS-2-JSIL\xspace}
\newcommand{\jsilverify}{JSILVerify\xspace}


% Tikz 
\usepackage{tikz}
\usetikzlibrary{calc,positioning,arrows,shapes,decorations.pathmorphing}
\usetikzlibrary{arrows,positioning} 
\tikzset{
    %Define standard arrow tip
    >=stealth',
    % Define arrow style
    pil/.style={
           ->,
           shorten <=2pt,
           shorten >=2pt,}
}

\newcommand{\runpic}{\includegraphics[width=0.06\picwidth]{running.pdf}}
\newcommand{\tickpic}{\resizebox{0.06\picwidth}{!}{\(\color{greenish} \checkmark \)}}
\tikzset{
  box/.style = {rectangle, draw=black,align=center,font=\scriptsize},
  sbox/.style = {rectangle,draw=black,align=left,font=\scriptsize,text width=1.7cm},
  p/.style = {-latex},
  dp/.style = {latex-latex},
  sz/.style n args={2}{minimum width=#2, minimum height=#1},
  m/.style = {midway,inner sep=0pt,fill=white},
  ll/.style = {font=\scriptsize,anchor=south west}
}



% Polishing...
\newcommand{\polish}[1]{{\color{red}#1}}

% macros_js as for Jose Santos
\usepackage{macros_js}
\usepackage{gdshojs}

\newcommand{\jilette}{Jilette\xspace}

\newcommand{\myparagraph}[1]{\smallskip\noindent {\bf #1.}\hspace{1pt}}



%
\title{\jilette: Symbolic Execution for JavaScript}

\author{}

\institute{Imperial College London}

\begin{document}
%

\maketitle 

\begin{abstract}

\end{abstract}


\section{Introduction}

JavaScript is the most widespread dynamic language: it is the de facto language for client-side Web
applications (used by 94.8\% of websites\footnote{w3techs.com/technologies/details/cp-javascript/all/all.});
it is used for server-side scripting via Node.js; and it is even run on small embedded devices with limited 
memory. It is the most active language in both GitHub and StackOverflow\footnote{JavaScript  has most active
 repos on GitHub, http://githut.info; JavaScript has most questions on 
 StackOverflow, \\https://exploratory.io/viz/Hidetaka-Ko/94368d12800a?cb=1469037012628.}.
The dynamic nature of JavaScript and its complex semantics make it a difficult target for
symbolic analysis and logic-based verification. 
This paper presents \jilette, a symbolic execution tool for JavaScript that improves existing solutions
in two ways: \dtag{1} it is grounded on the real semantics of the language and 
\dtag{2} it supports reasoning about symbolic strings and regular expressions. 
%
Furthermore, we present two possible uses for \jilette. It can both be used as \dtag{i} a tool for
 running symbolic tests for JavaScript programs and \dtag{ii} a debugging mechanism for separation 
 logic specifications of JavaScript programs. 

\myparagraph{Symbolic Testing} The most commonly used 
approach to obtaining trust in JavaScript code is running it against 
adhoc test batteries---given inputs, the code produces the expected
output. The main drawback of this approach is that tests, in general,
cannot guarantee exhaustiveness. We show how to use \jilette
for the symbolic testing of JavaScript programs: instead of using concrete 
inputs in tests, the developer uses symbolic inputs and states the 
constraints that the output needs to satisfy as simple, intuitive 
first-order assertions over these inputs. 
%This approach guarantees exhaustiveness.  Moreover, due to the
Furthermore, if a test fails, Jilette provides the concrete inputs that cause it 
to fail, immediately revealing bugs in the tested code.


\myparagraph{Debugging Separation Logic Specifications}
We show how to use \jilette as an auxiliary mechanism for debugging 
separation logic specifications of JavaScript programs. Verification tools for  
JavaScript code are still very rare and often require heavily annotating the 
original code for helping the solver prove entailments involving user-defined 
predicates, as well as coping with loops and recursion. Furthermore, when these 
tools cannot prove that a given function does not satisfy its specification, the 
user needs to go through a  proof trace and understand it in detail. 
Here, we demonstrate how to use \jilette to find errors in specifications by generating 
symbolic tests from these specifications and running the tests using \jilette. 
Then, if a symbolic test generated from a given specification fails, we can 
be sure that the code to be verified does not satisfy its specification. 
More importantly, in case of failure, \jilette generates a concrete counter-model that 
invalidates the specification. This information greatly simplifies the debugging of 
both specifications and code. 

\myparagraph{Architecture}
\jilette leverages on the \JSComp compiler and Rosette \cite{Rosette1}. 
\JSComp is a thoroughly tested compiler from JavaScript to \jsil, a simple 
intermediate language well suited for verification. Rosette is a 
solver-aided programming language with support for symbolic values 
and first-order assertions. The core of Jilette comprises a JSIL interpreter
written in Rosette. By piggybacking on Rosette and JS-2-JSIL, we 
are able to execute JavaScript programs on symbolic values, effectively 
obtaining a symbolic execution tool for JavaScript. 
This combination is non-trivial because \polish{many many things...} 


%We show how  to use Jilette for writing symbolic tests for client side 
%JavaScript code calling Web APIs. In particular, we demonstrate how to 
%checking the conformance of Web API requests with their specified signatures. 
%The existing solutions for this problem are still imprecise due to the 
%dynamicity of JavaScript combined with the difficulty of reasoning about
%operations on symbolic strings \cite{Idontknow}. Jilette is an excellent fit for
%this task as it leverages on Rosette's back-end
%constraint solver, Z3, which supports reasoning on symbolic strings
%and regular expressions, whereas JS-2-JSIL successfully
%contains the complexity of JavaScript itself.

\section{\jilette: Design and Implementation}
%!TEX root = ../main.tex

\begin{itemize}
  \item 2.0  - intro - explain the basic ideas of Jilette (use a diagram) - 
 
  \item 2.1 - describe the jsil language. give its formal syntax (extended with assert and solve). 
                   formally define symbolic execution for JSIL commands. soundness lemma. 
 
  \item 2.2 Implementation:  
               - encoding \jsil heaps in Rosette 
               - explain the \jsil interpreter implemented in Rosette and its connection to the \jsil semantics (as defined in appendix) 
               - give snippets of the interpreter 
               - discuss soundness, trust, and other issues
 
  \item 2.3 Symbolic execution for JavaScript 
              - explain that we have to extend the syntax of JavaScript with asserts  as well as constructs for creating symbolic values
              - give the example 
              - discuss challenges: abstraction level of the generated code needs to match the abstraction level of Rosette 
\end{itemize}

\subsection{Symbolic Execution for  \jsil}

\myparagraph{\jsil: Formal Semantics}
The basic memory model of \jsil is straightforward. 
\jsil values contain: numbers, $\jnumber$; booleans, $\jbool$; strings, $\jstring$;  the special values \jsinline|undefined| and \jsinline|null|; and object locations,  $\loc \in \locs$.
A \jsil heap, $\heap \in \heaps$, is a partial function mapping pairs of  object locations, and strings to heap values. 
 Given a heap $\heap$, we denote a heap cell by $\hcell{\loc}{\jstring}{\val}$ when $h(\loc,\jstring) = \val$, the union of two disjoint heaps by $\oheap_1 \dunion \oheap_2$,  a heap lookup by $\hread{\oheap}{\loc}{\jstring}$, and the empty heap by $\hemp$.
 Finally, a \jsil variable store, $\store \in \stores$, is a mapping from JSIL program variables $\jvar \in \jvars$ to JSIL values.

We introduce the \jsil semantic judgement for program behaviour; %needed to state our soundness result. 
the full \jsil semantics is given in the Appendix. 
The semantics of \jsil is defined in a small-step style. 
Transitions have the form:  $\semtrans{\heap, \store, \ctx[i]}{\heap', \store', \ctx'[j]}$, 
meaning that the evaluation of the $i$-th command of the current procedure in 
the heap $\heap$ and store $\store$ generates the heap $\heap'$ and store $\store'$ 
and $j$ is the next command to be evaluated. 
\polish{I have to talk about contexts here...} 


\myparagraph{\jsil: Symbolic Evaluation}
In order to symbolically execute \jsil programs, we extend the syntax of \jsil expressions with 
symbolic strings $\sstring \in \sstrings$ and symbolic numbers $\snumber \in \snumbers$. 
For convenience, we use $\svars$ to denote the union of $\sstrings$ and $\snumbers$ 
and $\svar$ to range over $\svars$. 
A symbolic expression $\sexpr$ is a \jsil expression with no program variables. 
\begin{equation}
\jsilexpr \triangleq \lit \mid \jvar \mid \sstring \mid \snumber \mid \unoper\ \jsilexpr \mid \jsilexpr \binoper \jsilexpr
\qquad 
\sexpr \triangleq \lit \mid \sstring \mid \snumber \mid \unoper\ \sexpr \mid \sexpr \binoper \sexpr
\end{equation}

We extend heaps, stores, and contexts with symbolic values, obtaining symbolic 
heaps, stores, and contexts, respectively ranged by $\sheap$, $\sstore$, and $\sctx$. 
A symbolic heap, $\sheap \in \sheaps$, is a partial function mapping pairs of  
object locations, and symbolic expressions to symbolic expressions. 
A symbolic store, $\sstore \in \sstores$, is a mapping from program variables 
$\jvar \in \jvars$ to symbolic expressions.
%
A \emph{symbolic state} $\sstate = (\sheap, \sstore, \pc)$ is a triple consisting of a 
symbolic heap $\sheap$, a symbolic store $\sstore$, and a path condition $\pc$. 
The path condition is a first order quantifier free formula over symbolic strings and 
numbers, which accumulates constraints on the given symbolic inputs that trigger 
the execution to follow the path that led to the current symbolic state. 

Figure~\ref{fig:symbexe:bcmds} presents the symbolic execution rules for \jsil basic commands. 
Rules have the form $\symbtrans{\sheap, \sstore, \pc, \bcmd}{\sheap', \sstore', \pc'}$, 
where: \dtag{1} $\sheap$ and $\sstore$ are the symbolic heap and store on which to evaluate $\bcmd$, 
\dtag{2} $\pc$ the current \emph{path condition}, and \dtag{3} $\sheap'$, $\sstore'$, and $\pc'$
the resulting symbolic heap, store, and path condition. 
Figure~\ref{fig:symbexe:cmds} presents the symbolic execution rules for \jsil commands. 
Rules have the form $\symbtrans{\sheap, \sstore, \pc, \sctx[i]}{\sheap, \sstore, \pc, \sctx'[j]}$; 
they differ from the rules for basic commands in that: \dtag{i} instead of the basic command to 
be executed on the left, they have the index of the command to be executed paired up 
with its associated execution context, and, \dtag{ii} on the right, they additionally have the 
index of the next command to be executed paired with the resulting execution context. 

%\begin{display}{}
\begin{figure}[ht!]
{\scriptsize
\begin{mathpar} 
%
\inferrule[\textsc{Skip}]{}
	{ \symbtrans{\sheap, \sstore, \pc, \jsilskip}{\sheap, \sstore, \pc}} 
 \and
 %
\inferrule[\textsc{Assignment}]
  {
      \symbeval{\jsilexpr}{\sstore} =  \sexpr
      \quad
      \sstore' = \sstore[\jvar \mapsto \sexpr]
  }{\symbtrans{\sheap, \sstore, \pc, \jvar := \jsilexpr}{\sheap, \sstore', \pc}} 
%
\and 
%
\inferrule[\textsc{Object Creation}]
  { 
    \sheap' = \sheap \dunion \hcell{\loc}{\protop}{\jsnull}
    \and (\loc,-) \notin \domain (\sheap)
  }{\symbtrans{\sheap, \sstore, \pc, \jvar := \jsilnew()}{\sheap', \sstore[\jvar \mapsto \loc], \pc}}
\\
%
\inferrule[\textsc{Property Access}]
  { 
 	\symbeval{\jsilexpr_1}{\sstore} =  \loc
  	\quad 
        \symbeval{\jsilexpr_2}{\sstore} =  \sexpr_p
        \quad
        \sheap = \sheap' \, \uplus \, \big((l, \sexprp_i) \mapsto \sexprv_i\big)\mid_{i = 0}^n   
        \quad
        (l, -) \not\in \domain(\sheap')
        \quad 
        0 \leq k \leq n
        \\\\
        \pc' = \pc \ \wedge \, \big( (\sexprp_k = \sexpr_p) \ \wedge \bigwedge_{i = 0, i \neq k}^n (\sexprp_i \neq \sexpr_p) \big)
  }{ \symbtrans{\sheap, \sstore, \pc, \jvar := [\jsilexpr_1, \jsilexpr_2]}{\sheap,  \sstore[\jvar \mapsto \sexprv_k], \pc'}}
 %
\\
%
\inferrule[\textsc{Property Assignment - Found}]
  {     \symbeval{\jsilexpr_1}{\sstore} =  \loc
  	\quad 
        \symbeval{\jsilexpr_2}{\sstore} =  \sexpr_p
        \quad
        \symbeval{\jsilexpr_3}{\sstore} =  \sexpr_v
       \quad 
        \sheap = \sheap' \, \uplus \, \big((l, \sexprp_i) \mapsto \sexprv_i\big)\mid_{i = 0}^n   
        \quad
        (l, -) \not\in \domain(\sheap')
        \quad 
        0 \leq k \leq n
        \\
          \pc' = \pc \ \wedge \, \big( (\sexprp_k = \sexpr_p) \ \wedge \bigwedge_{i = 0, i \neq k}^n (\sexprp_i \neq \sexpr_p) \big)
         \quad
         \sheap'' = \sheap' \, \uplus \,  \big((l, \sexprp_i) \mapsto \sexprv_i\big)\mid_{i = 0, i \neq k}^n \, \uplus \,  (l, \sexpr_p) \mapsto \sexpr_v
  }{\symbtrans{\sheap, \sstore, \pc,  [\jsilexpr_1, \jsilexpr_2] := \jsilexpr_3}{\sheap'', \sstore, \pc'}} 
\\
%
\inferrule[\textsc{Property Assignment - Not Found}]
  {     \symbeval{\jsilexpr_1}{\sstore} =  \loc
  	\quad 
        \symbeval{\jsilexpr_2}{\sstore} =  \sexpr_p
        \quad
        \symbeval{\jsilexpr_3}{\sstore} =  \sexpr_v
       \quad 
        \sheap = \sheap' \, \uplus \, \big((l, \sexprp_i) \mapsto \sexprv_i\big)\mid_{i = 0}^n   
        \quad
        (l, -) \not\in \domain(\sheap')
        \quad 
        0 \leq k \leq n
        \\
          \pc' = \pc \ \wedge \, \bigwedge_{i = 0}^n (\sexprp_i \neq \sexpr_p)
         \quad
         \sheap' = \sheap \, \uplus \,  (l, \sexpr_p) \mapsto \sexpr_v
  }{\symbtrans{\sheap, \sstore, \pc,  [\jsilexpr_1, \jsilexpr_2] := \jsilexpr_3}{\sheap', \sstore, \pc'}}   
%
\\
%
\inferrule[\textsc{Property Deletion}]
  { 
        \symbeval{\jsilexpr_1}{\sstore} =  \loc
  	\quad 
        \symbeval{\jsilexpr_2}{\sstore} =  \sexpr_p
       \quad 
        \sheap = \sheap' \, \uplus \, \big((l, \sexprp_i) \mapsto -\big)\mid_{i = 0}^n   
        \quad
        (l, -) \not\in \domain(\sheap')
        \quad 
        0 \leq k \leq n
     \\ 
      \pc' = \pc \ \wedge \, \big( (\sexprp_k = \sexpr_p) \ \wedge \bigwedge_{i = 0, i \neq k}^n (\sexprp_i \neq \sexpr_p) \big)
     \quad 
      \sheap'' = \sheap' \, \uplus \,  \big((l, \sexprp_i) \mapsto \sexprv_i\big)\mid_{i = 0, i \neq k}^n
   }{\symbtrans{\sheap, \sstore, \pc, \jsildelete(\jsilexpr_1, \jsilexpr_2)}{\sheap'', \sstore, \pc'}}
 \\
 %
\inferrule[\textsc{Member Check - True}]
  { 
      \symbeval{\jsilexpr_1}{\sstore} =  \loc
  	\quad 
        \symbeval{\jsilexpr_2}{\sstore} =  \sexpr_p
       \quad 
        \sheap = \sheap' \, \uplus \, \big((l, \sexprp_i) \mapsto -\big)\mid_{i = 0}^n   
        \quad
        (l, -) \not\in \domain(\sheap')
        \quad 
        0 \leq k \leq n
     \\ 
     \pc' = \pc \ \wedge \, \big( (\sexprp_k = \sexpr_p) \ \wedge \bigwedge_{i = 0, i \neq k}^n (\sexprp_i \neq \sexpr_p) \big)
  }{\symbtrans{\sheap, \sstore, \pc, \jvar := \hasfield(\jsilexpr_1, \jsilexpr_2)}{\sheap, \sstore[\jvar \mapsto \jtrue], \pc'}}
%
\\
%
\inferrule[\textsc{Member Check - False}]
  { 
      \symbeval{\jsilexpr_1}{\sstore} =  \loc
  	\quad 
        \symbeval{\jsilexpr_2}{\sstore} =  \sexpr_p
       \quad 
        \sheap = \sheap' \, \uplus \, \big((l, \sexprp_i) \mapsto -\big)\mid_{i = 0}^n   
        \quad
        (l, -) \not\in \domain(\sheap')
        \quad 
        0 \leq k \leq n
     \\ 
     \pc' = \pc \ \wedge \,  \bigwedge_{i = 0}^n (\sexprp_i \neq \sexpr_p) \big)
  }{\symbtrans{\sheap, \sstore, \pc, \jvar := \hasfield(\jsilexpr_1, \jsilexpr_2)}{\sheap, \sstore[\jvar \mapsto \jfalse], \pc'}}
\\
%
\inferrule[\textsc{Assert - True}]
  { 
      \symbeval{\jsilexpr}{\sstore} =  \sexpr
     \quad 
     \pc \vdash \sexpr 
  }{\symbtrans{\sheap, \sstore, \pc, \assert(\jsilexpr)}{\sheap, \sstore, \pc}} 
\quad
\inferrule[\textsc{Assert - False}]
  { 
      \symbeval{\jsilexpr}{\sstore} =  \sexpr
     \quad 
     \pc \not\vdash \sexpr 
  }{\symbtranserr{\sheap, \sstore, \pc, \assert(\jsilexpr)}} 
\end{mathpar}}
\caption{Symbolic Execution for Basic Commands: {\scriptsize$\symbtrans{\sheap, \sstore, \pc, \bcmd}{\sheap', \sstore', \pc'}$}\label{fig:symbexe:bcmds}}
\end{figure}
%\end{display}  


\begin{figure}[ht!]
{\scriptsize
\begin{mathpar} 
\inferrule[\textsc{Basic Command}]
   { 
     \prog_{\pid}(i) = \bcmd 
     \quad
     \symbtrans{\sheap, \sstore, \pc, \bcmd}{\sheap', \sstore', \pc'} 
   }{\symbtrans{\sheap, \sstore, \pc, \sctx[i]}{\sheap', \sstore', \pc', \sctx[i+1]}}
%
   \qquad
  %
  \inferrule[\textsc{Basic Command - Fail}]
   { 
     \prog_{\pid}(i) = \bcmd 
     \quad
     \symbtranserr{\sheap, \sstore, \pc, \bcmd} 
   }{\symbtranserr{\sheap, \sstore, \pc, \sctx[i]}}
 %
   \qquad
  %
  \inferrule[\textsc{Goto}]
   { \prog_{\pid}(i) = \goto \, j \quad}
   {\symbtrans{\sheap, \sstore, \pc, \sctx[i]}{\sheap, \sstore, \pc, \sctx[j]}}
  \\ 
  \inferrule[\textsc{Cond. Goto - True}]
   { \prog_{\pid}(i) =  \ifgoto{\jsilexpr}{j}{k} \quad
     \symbeval{\jsilexpr}{\sstore} =  \sexpr
   }
   {\symbtrans{\sheap, \sstore, \pc, \sctx[i]}{\sheap, \sstore, \pc \, \wedge \, \sexpr, \sctx[j]}}
  \and 
    \inferrule[\textsc{Cond. Goto - False}]
   { \prog_{\pid}(i) =  \ifgoto{\jsilexpr}{j}{k} \quad
     \symbeval{\jsilexpr}{\sstore} =  \sexpr
   }
   {\symbtrans{\sheap, \sstore, \pc, \sctx[i]}{\sheap, \sstore, \pc \, \wedge \, \neg\sexpr, \sctx[k]}}
   \\
    \inferrule[\textsc{Procedure Call}]
   { 
    \prog_{\pid}(i) =   \jsilcall{\jvar}{\jsilexpr}{\jsilexpr_i \mid_{i = 0}^{n}}{j}
     \quad
    \symbeval{\jsilexpr}{\sstore} =  \pid' 
    \quad
      \symbeval{\jsilexpr_i}{\sstore} =  \sexpr_i \mid_{i = 0}^{n} 
     \quad
     \args(\pid') = \jsillist{\jvar_1, ..., \jvar_{m}} 
     \quad 
      \sexpr_i = \jsundefined \mid_{i = n+1}^{m}  
   }
   {\symbtrans{\sheap, \sstore, \pc, \sctx[i]}{\sheap, [ \jvar_i \mapsto \sexpr_i \mid_{i = 0}^{m}] , \pc, ((\pid', \sstore, \jvar, i+1, j)::\sctx)[0]}}
    \\ 
  \inferrule[\textsc{Normal Return}]
   {
       \sctx = (-, \sstore', \jvar, i, -) :: \sctx' 
       \quad 
       \sstore(\procretvar) = \sexpr
   }  
   {\symbtrans{\sheap, \sstore, \pc, \sctx[\procretlab]}{\sheap, \sstore'[\jvar \mapsto \sexpr], \pc, \sctx'[i]}}
   \and 
     \inferrule[\textsc{Error Return}]
   {
       \sctx = (-, \sstore', \jvar, -, j) :: \sctx' 
       \quad 
       \sstore(\procerrvar) = \sexpr
   }  
   {\symbtrans{\sheap, \sstore, \pc, \sctx[\procerrlab]}{\sheap, \sstore'[\jvar \mapsto \sexpr], \pc, \sctx'[j]}}
 \end{mathpar}}
\caption{Symbolic Execution for Control Flow Commands: {\scriptsize$\symbtrans{\sheap, \sstore, \pc, \sctx[i]}{\sheap, \sstore, \pc, \sctx'[j]}$}\label{fig:symbexe:cmds}}
\end{figure}


\myparagraph{Soundness} To establish the soundness of symbolic execution we need to relate 
symbolic states to concrete states. To this end, we make use of \emph{symbolic environments} 
$\senv : \svars \rightharpoonup \lits$ mapping symbolic values to \jsil literals. 
A symbolic environment is said to be \emph{consistent} if it maps symbolic 
values to concrete values of the appropriate type. In the following, we will always 
assume consistent symbolic environments. 
%
Given a symbolic environment $\senv$, we define the interpretation of a symbolic 
expression $\sexpr$ under $\senv$ as follows: 
\begin{equation}
\semexpr{\lit}{\senv} \semeq \lit
\quad 
\semexpr{\svar}{\senv} \semeq \senv(\svar)
\quad 
\semexpr{\unoper\ \sexpr}{\senv} \semeq \unoper (\semexpr{\sexpr}{\senv})
\quad 
\semexpr{\sexpr_1 \binoper \sexpr_2}{\senv} \semeq \binoper(\semexpr{\sexpr_1}{\senv}, \semexpr{\sexpr_2}{\senv}) 
\end{equation}
We extend the interpretation function to symbolic states as defined below. 

\begin{display}{Interpretation of Symbolic States}
{
\begin{tabular}{l}
$\quad${\bf Symbolic Heaps:}  \\
$
\quad
 \semexpr{\hemp}{\senv} \semeq \hemp
\quad
\semexpr{\hcell{\loc}{\sexpr_p}{\sexpr_v}}{\senv} \semeq  \hcell{\loc}{\semexpr{\sexpr_p}{\senv}}{\semexpr{\sexpr_v}{\senv}}
\quad
\semexpr{\sheap_1 \dunion \sheap_2}{\senv} \semeq  \semexpr{\sheap_1}{\senv} \dunion \semexpr{\sheap_2}{\senv}
$%
%%
%%
\\[3pt]
$\quad${\bf Symbolic Stores:}  \\
$
\quad
 \semexpr{\storeemp}{\senv} \semeq \storeemp
\quad 
 \semexpr{(\jvar: \sexpr) \dunion \sstore}{\senv} \semeq (\jvar: \semexpr{\sheap_1}{\senv}) \dunion \semexpr{\sstore}{\senv}
$%
\\[3pt]
$\quad$ {\bf Symbolic States:}  $\semexpr{(\sheap, \sstore, \sctx)}{\senv} \semeq (\semexpr{\sheap}{\senv}, \semexpr{\sstore}{\senv}, \semexpr{\sctx}{\senv})$
\end{tabular}
}
\end{display} 

In the following, we write $\senv \vdash \pc$  if and only if $\semexpr{\pc}{\senv} \Leftrightarrow \ltrue$. 
For convenience, we define: 
\begin{equation}
\smodels{\sheap, \sstore, \sctx}{\pc} = \left\{ (\heap, \store, \ctx) \mid \exists \senv \, . \,  \semexpr{(\sheap, \sstore, \sctx)}{\senv} = (\heap, \store, \ctx) \, \wedge \,  \senv \vdash \pc  \right\} 
\end{equation}

\begin{theorem}[Soundess]
$$
\begin{array}{l}
\symbtrans{\sheap, \sstore, \pc, \sctx[i]}{\sheap', \sstore', \pc', \sctx'[j]} 
   \ \wedge \ 
      (\heap, \store, \ctx) \in \smodels{\sheap, \sstore, \sctx}{\pc'} \\ \quad \quad
      	 \ \Rightarrow \ \exists (\heap', \store', \ctx') \, . \, 
	 	 \semtrans{\heap, \store, \ctx[i]}{\heap', \store', \ctx'[j]}
		\, \wedge \, 
		(\heap', \store', \ctx') \in \smodels{\sheap', \sstore', \sctx'}{\pc'}  
\end{array}
$$
\end{theorem}



\subsection{Implementation}

The point here is to explain how writing a correct concrete \jsil interpreter in Rosette
yields the symbolic environments presented in the previous subsection. 




\subsection{Symbolic Execution for JavaScript}


 \begin{figure}[th!]
 \begin{lstjs}[firstnumber=1]
function Map () { this._contents = {} }

Map.prototype.get = function (k) {
    if (this._contents.hasOwnProperty(k)) {  return this._contents[k] } 
    	else { return null }  
}

Map.prototype.put = function (k, v) {
   var contents = this._contents;
   if (this.validKey(k)) {  contents[k] = v   } 
   	else { throw new Error("Invalid Key") } 
} 

Map.prototype.validKey = function (k) { ... }
\end{lstjs}
\caption{Map Implementation in JavaScript}
\end{figure}

 \begin{figure}[th!]
 \begin{lstjs}[firstnumber=1]
var m = new Map();  m.put (__s1, __n1); var r = m.get(__s1);  
assert(__n1 = r)
\end{lstjs}
\caption{Symbolic Test for the Map Library}
\end{figure}



\section{Debugging Separation Logic Specifications}
%!TEX root = ../main.tex

We show how to use \cosette for debugging \jsil code annotated with 
separation logic (SL) specifications. Tools that allow for SL-reasoning about
functional correctness properties in general, and those targeting 
JavaScript in particular, require the user to have substantial expertise 
and to go through a long and complex proof trace, whenever verification
is not successful. \cosette substantially simplifies this process by providing
concrete counter-models that invalidate the input specifications.

In \S\ref{subsec:sep:assertions}, we extend the 
 \jsil symbolic interpreter with a mechanism for asserting
SL-assertions. 
%
In \S\ref{subsec:fip}, we show how to implement this mechanism by giving 
a sound decision procedure for solving the frame inference problem (FIP)~\cite{}
in the context of symbolic execution.
%
Unlike verification tools, our emphasis is in the generation of counter-models 
for failing cases. 
%
Finally, in \S\ref{specs:to:symbolic:tests}, we present an algorithm  
for generating symbolic tests from SL-specifications, which guarantees 
that whenever a symbolic test fails, \cosette produces a concrete 
counter-model that invalidates the corresponding specification.

\vspace{-5pt}
\subsection{Symbolic Execution with SL-Assertions}\label{subsec:sep:assertions}

\jsil Logic assertions~\cite{javert}
provide a compositional way of describing \emph{partial} symbolic states. 
\jsil assertions include: boolean operations; the separating conjunction; 
and assertions for describing heaps. The $\lemp$ assertion describes 
an empty heap. The cell assertion, $(\lexpr_1,\lexpr_2) \pointsto \lexpr_3$,  describes an object 
at the location denoted by $\lexpr_1$ with a property denoted by $\lexpr_2$ that has the value 
denoted by $\lexpr_3$. The object domain assertion $\emptyfields{\lexpr_1}{\lexpr_2}$ states that the object at 
the location denoted by $\lexpr_1$ has no properties other than possibly those included in the
set denoted by $\lexpr_2$. The syntax of assertions is given below. 
We refer to assertions different from $- \sep -$ and $\lemp$ as \emph{simple assertions}
and use $\spass$ and $\sqass$ to range over them.

\vspace{2pt}
\begin{display}{\jsil Logic Assertions}
%
{\small
\begin{tabular}{l}
  %%%%
  $\lexpr \quad \ \triangleq \val \mid \jvar \mid \svar \mid \unoper\ \lexpr \mid \lexpr \binoper \lexpr \quad \quad \quad \quad \quad \quad \quad \ \ $   \text{ Logical Expressions} \\
  $\rass, \sass \ \triangleq \jtrue \mid \jfalse \mid  \neg \rass \mid \rass \land \sass \mid \rass \lor \sass  \mid \lexpr = \lexpr \mid \lexpr \leq \lexpr$  \quad \text{\hfill{Pure Asrts.}} \\
  $\pass, \qass \triangleq \sass \mid \lemp \mid (\lexpr, \lexpr)\pointsto \lexpr \mid \pass \sep \qass  \mid \emptyfields{\lexpr}{\lexpr} $ \quad \quad \quad \quad \ \  \text{\hfill Asrts.} \\
\end{tabular}}
\end{display}

\noindent Without loss of generality, we implicitly assume that different symbolic locations 
denote different concrete locations.\footnote{In order to express aliasing, the user has to write multiple assertions.}
 Furthermore, given a cell assertion $(\lexpr_1,\lexpr_2) \pointsto \lexpr_3$, we always assume 
 $\lexpr_1$ to be either a concrete location $\loc$ or a symbolic location $\sloc$. 
%
Unsurprisingly, every symbolic state can be thought of as an assertion. In particular, 
the symbolic state $\sstate = (\sheap, \sdom, \sstore, \pc)$ corresponds to the assertion: 
\begin{equation*}
{\small \begin{array}{l}
\big(\varoast_{(\sloc, \sexprp) \in \domain(\sheap)} (\sloc, \sexprp) \mapsto \sheap(\sloc, \sexprp)\big) 
  \sep \big(\varoast_{\sloc \in \domain(\sdom)} \, \emptyfields{\sloc}{\idom(\sloc)}\big)  \\
 %
 \qquad \sep \big(\bigwedge_{\jvar \in \domain(\sstore)} \, \jvar = \sstore(\jvar)\big) \sep \pc
\end{array}}
\end{equation*}

\noindent where $\varoast$ denotes the iterated separating conjunction~\cite{}. 
Analogously, every assertion can be trivially re-organised into a symbolic state: 
\dtag{1} cell assertions form the heap part of the state, 
\dtag{2} object domain assertions form the domain part,
\dtag{3} equalities involving program variables form the store, and 
\dtag{4} pure assertions form the path condition. 
Finally,  all the occurrences of program variables in the heap, domain, and path condition 
are replaced with their corresponding symbolic expressions given by the store. 
We use $\normaliser(\pass)$ to refer to the symbolic state corresponding 
to $\pass$. 
%
In the following, we use $\interpret{}{}(\sstate)$ for denoting the set 
$\{ (\jstate, \heap_f) \mid \exists \senv \, . \, (\jstate, \heap_f) \in \interpret{\symbconc}{\senv}(\sstate) \}$ 
and $\interpret{}{}(\pass)$ for denoting $\interpret{}{}(\normaliser(\pass))$. 

\myparagraph{Inductive Predicates}
\cosette does not support symbolic execution over inductive predicates, which are commonplace 
in SL-style specifications~\cite{smallf, berdine:aplas:2005}. 
As in~\cite{korat}, we deal with user-defined inductive predicates by \emph{unfolding} 
those predicates up to a fixed bound, given by the user. This unfolding mechanism 
is routine and is, therefore, omitted from the paper. 

\myparagraph{Asserting SL-Assertions}
We extend \jsil with a special construct, $\sepassert(P)$, for stating that 
the assertion $P$ must hold whenever that command is evaluated. 
The symbolic semantics is given below. 
\begin{mathpar}
\inferrule[\textsc{SL-Assert - True}]
  { 
     \ccmd{i}  = \sepassert(\pass)  \\\\
     %
     \unificationfun(\sstate, \pass) = \success{\subst, \sstate_f}
  }{
    \abssemrule{\sstate, \scs, i}{\sstate, \scs, i{+}1}{\top}{\top}{\symbolic}
}
\qquad
\inferrule[\textsc{SL-Assert - False}]
  { 
     \ccmd{i}  = \sepassert(\pass)  \\\\
     %
    \unificationfun(\sstate, \pass) = \fail{\pc_f}  \\\\ 
    %
     (\sstate.\pcsel \wedge \pc_f) \text{ SAT}
  }{
    \abssemrule{\sstate, \scs, i}{\sstate, \scs, i}{\top}{\bot}{\symbolic}
}
\end{mathpar}

\vspace{-3pt}
\noindent The rules make use of a partial decision procedure $\unificationfun(\sstate, \pass)$ for 
determining whether or not a given symbolic state $\sstate$ satisfies an assertion $P$, 
which is in general undecidable \cite{citemeplease}. More concretely, 
the decision procedure outputs: 
\dtag{1} $\success{\subst, \sstate_f}$ when it finds a substitution $\subst$ and 
a symbolic state frame $\sstate_f$ for which it holds that $\interpret{}{}(\sstate) \subseteq \interpret{}{}(\sstate_f \statecompose \normaliser(\subst(\pass)))$, 
where we use $\statecompose$ for the composition of two symbolic states, 
and \dtag{2} $\fail{\pc_f}$ when it finds a first order formula $\pc_f$ such that 
$\interpret{}{}(\sstate \, \wedge \, \pc_f) \cap \interpret{}{}(\pass) = \emptyset$. 
Note that every concrete state and heap frame in $\interpret{}{}(\sstate \, \wedge \, \pc_f)$ are counter models 
for $P$. By requiring that $(\sstate.\pcsel \, \wedge \, \pc_f)$ be satisfiable, 
the semantics only triggers an assertion failure when it finds a concrete witness for the failure ---
any instantiation of $(\sstate \, \wedge \, \pc_f)$. 



\subsection{Frame Inference Problem}\label{subsec:fip}

We describe a partial decision procedure, which we  
implement as part of the \jsil symbolic interpreter, for proving entailments 
between symbolic states \underline{and} finding counter 
models in case of failure.  
As it is customary~\cite{javert,jacobs2011verifast,sepwithsmt}, the decision procedure works by first using \emph{pattern-matching} 
on the spatial part of the symbolic state, and then discharging the pure part of the 
entailment to an external constraint solver (in our case, \rosette). 

\begin{table} 
{\small \begin{tabular}{@{}c@{}ccc@{}c@{}}\toprule
\emph{Argument} & & \textbf{IN}  & & \textbf{OUT}  \\
\cmidrule{1-1} \cmidrule{3-3} \cmidrule{5-5}

$\svar$                                                       & & $\{ \svar \}$                                                          & & $\{ \svar \}$    \\
$\jsillist{\lexpr_1, ..., \lexpr_n}$                     & & $\upin(\lexpr_1) \cup ... \cup \upin(\lexpr_n)$      & & $\upout(\lexpr_1) \cup ... \cup \upout(\lexpr_n)$ \\
$\lexpr_1 + \lexpr_2$                                    & & $\upin(\lexpr_1) \cup \upin(\lexpr_2)$                  & & $\emptyset$ \\
$\cdots$ & & $\cdots$ & & $\cdots$ \\[1pt]
%%
$(\lexpr_1, \lexpr_2)\pointsto \lexpr_3$   & & $\upin(\lexpr_1) \cup \upin(\lexpr_2)$ & & $\upout(\lexpr_3)$  \\
%
$\emptyfields{\lexpr_1}{\lexpr_2}$           &  & $\upin(\lexpr_1) \cup \upin(\lexpr_2)$ & & $\emptyset$ \\
%
$\lexpr_1 = \lexpr_2$                               & & $\upin(\lexpr_1)$/$\upin(\lexpr_2)$      & & $\upout(\lexpr_2)$/$\upout(\lexpr_1)$ \\
%
$\sass$                                                    & & $\fv(\sass)$                                           & & $\emptyset$ \\
\bottomrule
\end{tabular}}
\caption{\emph{in} and \emph{out} sets for assertions and logical expressions}
\vspace{-25pt}
\end{table}

\myparagraph{Unification Plan}
When solving $\unificationfun(\sstate, \pass)$, the symbolic variables of $\pass$ that are not 
in $\sstate$ are assumed to be existentially quantified. 
In order to find the appropriate bindings for these variables, we 
introduce the notion of \emph{unification plan} (Definition~\ref{def:up}).
Informally, a unification plan is an ordering of the simple assertions in $\pass$ that 
guarantees that the unification algorithm does not have to backtrack at unification~time. 

We define for each assertion an \emph{in} set and an \emph{out} set (these 
sets are reminiscent of the predicate parameter modes in~\cite{nguyen:vmcai:2008}).
Intuitively, the variables in the \emph{out} set of an assertion are those that can be computed 
using the variables in the \emph{in} set, together with the current state. 
For instance, given the assertion $(\svar_1, \svar_2) \pointsto \svar_3$, if we know the bindings
for $\svar_1$ and $\svar_2$, we can compute the bindings for $\svar_3$, given 
the current symbolic state. 
%
Analogously, we define \emph{in} and \emph{out} sets for logical expressions. 
The variables in the \emph{out} set of a logical expression are those that can be computed
given the value of the whole expression, whereas the variables in the \emph{in} set 
are those that need to be known for us to compute the value of the whole expression.  
Finally, the definition of unification plan is given below. 

\begin{definition}[Unification Plan]\label{def:up}
A \emph{unification plan} $\up$ is a sequence of simple assertions $\sqass_i \mid_{i=0}^n$ such that: 
$$
 \forall 0 \leq i \leq n \, . \, \upin(\sqass_i) \subseteq \big(\cup_{j = 0}^{i-1} \upout(\sqass_j)\big) \cup \upin_0
$$
where $\upin_0$ is the set of \underline{non}-existentially quantified logical variables. 
\end{definition}

\vspace{-3pt}
\noindent It is not always possible to generate a unification plan for an SL-assertion. We only 
consider assertions that admit a unification plan. 


\myparagraph{Unification Algorithm}
Given a symbolic state $\sstate$ and an assertion $\pass$, $\unificationfun(\sstate, \pass)$: 
\dtag{1} replaces all the occurrences of program variables in $\pass$ with their bindings 
given by $\sstate.\stosel$;
\dtag{2} computes an initial substitution $\subst_0$ mapping all the non-existentially quantified symbolic 
variables in $\pass$ to themselves (put formally: $\subst_0 = \identity_{\fv(\sstate) \cap \fv(\pass)}$); 
\dtag{3} creates a unification plan $\up$ for $\pass$; and, 
\dtag{4} calls Algorithm~\ref{fip:symb:states}. 
%
Algorithm~\ref{fip:symb:states} makes use of the following auxiliary functions: 

\begin{description}
\setlength{\itemsep}{0.2em}
  \item[FIP GetCell.] In case of success, $\GetCellV{\sstate, \sloc, \sexprp}$ returns 
          a pair consisting of the symbolic expression $\sexprv$ associated with 
          $(\sloc, \sexprp)$ in the heap component of $\sstate$ \underline{and} 
          the state obtained by removing that cell from $\sstate.\hpsel$.
  
  \item[FIP GetDomain.] In case of success, $\GetDomainV{\sstate, \sloc}$ returns 
          a pair consisting of the symbolic expression $\sexprv_d$ denoting the domain 
          of the object at location $\sloc$ in $\sstate$ \underline{and} 
          the state obtained by removing all the negative resource associated with 
          $\sloc$ from $\sstate$.  
  
  \item[Expression Unification.]  In case of success, $\unifylexpr(\sexprv, \sexprv', \subst, \pc)$ 
          returns a substitution $\subst'$ that extends $\subst$ such that $\pc \vdash \sexprv = \subst'(\sexprv')$. 
\end{description}
In case of failure, all auxiliary functions return a constraint $\pc_f$ under which 
the unification is guaranteed not to be possible. 
Below, we give selected rules for $\GetCellV{\sstate, \sloc, \sexprp}$ and $\GetDomainV{\sstate, \sloc}$.  

{\small \begin{algorithm}[t!]
\algblock[Name]{match}{end}
\caption{Frame Inference for Symbolic States}\label{fip:symb:states}
\begin{algorithmic}[1]
\Function{Unification}{$\sstate$, $\up$, $\subst$}
    \State $\textbf{match}$ $\up$ $\textbf{with}$
    \State $\mid~\lstemp:$ \Return $\success{\subst, \sstate}$
   % Cell ASS
    \State $\mid~(\sloc, \sexprp) \pointsto \sexprv \lstcons \up' :$ 
    \State $\qquad \textbf{match} \ \GetCellV{\sstate, \subst(\sloc), \subst(\sexprp)}$ $\textbf{with}$
    \State $\qquad \mid~\success{\sexprv', \sstate'}:$
    \State $\qquad \qquad \textbf{match} \ \unifylexpr(\sexprv, \sexprv', \subst, \sstate.\pcsel)$ $\textbf{with}$
     \State $\qquad \qquad \mid~ \success{\subst'}:$ \Return \Call{Unification}{$\sstate'$, $\up'$, $\subst'$}
      \State $\qquad \qquad \mid~ \fail{\pc_f} \ \, :$ \Return $\fail{\pc_f}$
      \State $\qquad \mid~\fail{\pc_f}:$ \Return $\fail{\pc_f}$
      % EF ASS
     \State $\mid~\emptyfields{\sloc}{\sexprv} \lstcons \up' :$  
       \State $\qquad \textbf{match} \ \GetDomainV{\sstate, \subst(\sloc)}$ $\textbf{with}$
       \State $\qquad \mid~\success{\sexprv', \sstate'}:$
         \State $\qquad \qquad \textbf{match} \ \unifylexpr(\sexprv, \sexprv', \subst, \sstate.\pcsel)$ $\textbf{with}$
       \State $\qquad \qquad \mid~ \success{\subst'}:$ \Return \Call{Unification}{$\sstate'$, $\up'$, $\subst'$}
       \State $\qquad \qquad \mid~ \fail{\pc_f} \ \, :$ \Return $\fail{\pc_f}$
       \State $\qquad \mid~\fail{\pc_f}:$ \Return $\fail{\pc_f}$
     % Logical Equality  
     \State $\mid~(\sexprv = \sexprv') \lstcons \up' :$  
        \State $\qquad \textbf{match} \ \unifylexpr(\sexprv, \sexprv', \subst, \sstate.\pcsel)$ $\textbf{with}$
         \State $\qquad \qquad \mid~ \success{\subst'}:$ \Return \Call{Unification}{$\sstate$, $\up'$, $\subst'$}
       \State $\qquad \qquad \mid~ \fail{\pc_f} \ \, :$ \Return $\fail{\pc_f}$
     % OTHER PURE ASS
     \State $\mid~\sass \lstcons \up' :$   
      \State $\qquad \textbf{if} \,(\sstate.\pcsel \vdash \subst(\sass))$
       \State $\qquad \qquad \textbf{then}$ \Return  \Call{Unification}{$\sstate$, $\up'$, $\subst$}
      \State $\qquad \qquad \textbf{else}$  \Return $\fail{\subst(\sass)}$
\EndFunction
\end{algorithmic}
\end{algorithm}}


\vspace{2pt}
\begin{display}{Selected FIP Rules}
\text{
{\scriptsize
\begin{mathpar} 
  \inferrule[\textsc{GetDomain}]
   { 
       \sheap = \sheap' \, \uplus \, \big((\sloc, \sexprp_i) \mapsto \sexprv_i \big)\mid_{i = 0}^m  
         \quad 
          (\sloc,-) \notin \domain (\sheap')  
         \quad
        \forall_{0 \leq i \leq n} \,  \sexprv_i \neq \none 
         \quad
           \forall_{n < i \leq m} \, \sexprv_i = \none
           \\\\ 
           \sexprv =  \{ \sexprp_i \mid_{i = n{+}1}^m\}
           \and
           \sheap'' = (\sloc, \sexprp_i) \mapsto \sexprv_i  \mid_{i=0}^n
           \and
           \sdom = \sdom' \dunion (\sloc \mapsto \sexprv')
   }{  \GetDomainV{(\sheap, \sdom, \sstore, \pc), \sloc} \semeq \success{(\sheap' \, \uplus \,  \sheap'', \sdom', \sstore, \pc), \sexprv' \backslash \sexprv}}
  \\
      \inferrule[\textsc{GetCell - Found}]
   { 
      (\sheap, \sdom, \sstore, \pc) = \sstate    
       \quad
       \sheap = \sheap' \, \uplus \, (\sloc, \sexprp') \mapsto \sexprv 
       \\\\
       {\color{blue} \pc \vdash (\sexprp = \sexprp')}
       \quad
       \sstate' = (\sheap', \sdom, \sstore, \pc)
   }{  \GetCellV{\sstate, \sloc, \sexprp} \semeq \success{\sstate', \sexprv}}
\quad
     \inferrule[\textsc{GetCell - Not Found}]
   { 
     (\sheap, \sdom, \sstore, \pc) = \sstate  
     \quad 
        { \color{blue} \pc \vdash \sexprp \not\in \sdom(\sloc)}
        \\\\
        \sstate' = (\sheap, \sdom[\sloc \mapsto \sdom(\sloc) \cup \jsilset{\sexprp}], \sstore,  \pc)
   }{  \GetCellV{\sstate, \sloc, \sexprp} \semeq \success{\sstate', \none}}
\\
     \inferrule[\textsc{GetCell - Fail with Domain Info}]
   { 
       \sheap = \sheap'' \, \uplus \, \big((\sloc, \sexprp_i) \mapsto \sexprv_i \big)\mid_{i = 0}^m
      \qquad
       (\sloc,-) \notin \domain (\sheap'')
       \qquad
        { \color{blue} \pc \not\vdash \sexprp \not\in \sdom(\sloc)}
        \qquad
          { \color{blue} \pc \not\vdash \sexprp = \sexprp_i \mid_{i=0}^m}
   }{  \GetCellV{(\sheap, \sdom, \sstore, \pc), \sloc, \sexprp} \semeq \fail{{\color{red} (\sexprp \in \sdom(\sloc)) \, \wedge \, \wedge_{i=0}^m (\sexprp_i \neq \sexprp)}}}
 \\
 \\
     \inferrule[\textsc{GetCell - Fail without Domain Info}]
   { 
       \sheap = \sheap'' \, \uplus \, \big((\sloc, \sexprp_i) \mapsto \sexprv_i \big)\mid_{i = 0}^m
      \qquad
       (\sloc,-) \notin \domain (\sheap'')
       \qquad
       \sloc \not\in \domain(\sdom)
        \qquad
          { \color{blue} \pc \not\vdash \sexprp = \sexprp_i \mid_{i=0}^m}
   }{  \GetCellV{(\sheap, \sdom, \sstore, \pc), \sloc, \sexprp} \semeq \fail{{\color{red} (\wedge_{i=0}^m (\sexprp_i \neq \sexprp)}}}
 \end{mathpar}}}
 \end{display}

The rules are analogous to the rules in \S\ref{subsec:symb:semantics} except that they return a new symbolic state 
from which the matched resource is removed and their corresponding constraints are lifted to the premise. 

\myparagraph{Formal Guarantees}
Theorem~\ref{teo:fip:soundness} states that the unification algorithm is sound: 
given an SL-assertion $\pass$ and a symbolic state $\sstate$, if
$\unificationfun(\sstate, \pass) = \success{\subst, \sstate_f}$, then there is a symbolic state
$\sstate'$ such that $\sstate = \sstate' \statecompose \sstate_f$ and $\sstate'$ satisfies $\pass$. 
The bug-finding theorem, Theorem~\ref{teo:fip:bugfinding}, is more subtle. It states that, 
in case of failure, to find a counter-model for $\pass$, one has to pick a concretisation of the 
symbolic state that is consistent with the failing constraint generated by the unification algorithm.


\begin{theorem}[Soundness of FIP]\label{teo:fip:soundness}
$$
\begin{array}{l}
	\unificationfun(\sstate, \pass) = \success{\subst, \sstate_f}
        \implies 
        \interpret{}{}(\sstate) \subseteq \interpret{}{}(\sstate_f \statecompose \normaliser(\subst(\pass)))
\end{array}
$$ 
\end{theorem}

\begin{theorem}[Bug-finding for SL]\label{teo:fip:bugfinding}
$$
\begin{array}{l}
\unificationfun(\sstate, \pass) = \fail{\pc_f} 
   \implies
   \interpret{}{}(\sstate \, \wedge \, \pc_f) \cap \interpret{}{}(\pass) = \emptyset
\end{array}
$$ 
\end{theorem}



\subsection{From Specifications to Symbolic Tests}\label{specs:to:symbolic:tests}

\jsil Logic specifications have the form $\specsig{\pass}{\pid(\jvec{x})}{\qass}{\flag}$, where $\pass$ and $\qass$ are the 
pre- and postconditions of the procedure with identifier $\pid$ and formal parameters $\jvec{x}$. 
Each specification is associated with a return mode $\flag \in \{ \fnormal, \ferror \}$, indicating if the function
 returns normally or with an error. 
 %If it returns normally, then its return value can be accessed  via a dedicated variable 
% $\retvar$, and $\errvar$ otherwise. 
 Intuitively, a specification $\specsig{\pass}{\pid(\jvec{x})}{\qass}{\flag}$ is 
valid for a given \jsil program $\prog$, if $\prog$ contains a procedure with identifier 
$\pid$ and ``whenever $\pid$ is executed in a state satisfying $P$, then, 
if it terminates, it does so in a state satisfying $Q$, with return mode $\flag$''.
The formal definition is given below. 


\begin{definition}[Validity of \jsil Logic Specifications]
A \jsil logic specification $\specsig{\pass}{\pid(\jvec{x})}{\qass}{\flag}$ is valid with respect to a program 
$\prog$, written $\prog \satisfies \specsig{\pass}{\pid(\jvec{x})}{\qass}{\flag}$, if and only if, for all logical 
contexts $(\iheap, \store, \senv)$, heaps $\heap_f$, stores $\store_f$, and flags $\flag'$, it holds that: 
$$
\begin{array}{l}
   (\jstate, \heap_f, \cs) \in \interpret{}{}(P) 
   \ \wedge \ 
    \abssemrule{\jstate \dunion \heap_f, \cs, 0}{\jstate', \cs', i_{\flag'}}{\top}{\top}{\concrete} \\ \quad \
   \implies
      \flag' = \flag \ \wedge \ \exists \jstate'' \, . \, \jstate' = \jstate'' \dunion \heap_f
          \ \wedge \   (\jstate'', \heap_f, \cs') \in \interpret{}{}(Q) 
\end{array}
$$
\end{definition}

\begin{figure}
{\small
$$
\begin{array}{lll}
\testify{}(\specsig{P}{\pid(\jvar_1, ..., \jvar_n)}{Q}{\flag}) \ \semeq                           &  \testify{\fnormal}(\pid, \svar_i\mid_{i=0}^n, Q) \ \semeq \\
%
\ \  \mathbf{let} \ \sstore =  [ \jvar_i \mapsto \svar_i \mid_{i=0}^n] \ \mathbf{in}        &  \ \  \darkmath{\sf proc} \jsilmain () \{    \\
%
 \ \  \mathbf{let} \ \sstate = \normaliser(\sstore(P)) \ \mathbf{in}                               &   \ \ \ \ 0_{\phantom{\sf nm}}: \jsilcall{\jvar}{\pid}{\svar_0, ..., \svar_n}{\errlab} \\
 %
\ \  \mathbf{let} \ Q' = \sstore(Q) \ \mathbf{in}                                                           &    \ \ \ \ \retlab \, : \sepassert(Q[\jvar/\retvar])  \\
 %
 \ \  \mathbf{let} \ \proc = \testify{\flag}(\pid, \svar_i\mid_{i=0}^n, Q')  \ \mathbf{in}  &    \ \ \ \ \errlab \, \, \, : \jassert(\jfalse)   \\
 %
 \ \ \ \ (\proc, \sstate)                                                                                                 &    \ \ \}  
\end{array}
$$}
\vspace*{-0.2cm}
\caption{Symbolic Test Generation Algorithm~\label{fig:test:generation}}
\vspace*{-0.2cm}
\end{figure}

Given a \jsil program $\prog$ containing a procedure $\pid$ with spec {\small $\specsig{\pass}{\pid(\jvec{x})}{\qass}{\flag}$}, 
our goal is to construct a symbolic test for checking whether or not $\pid$ behaves as its specification mandates.
A symbolic test is a pair $(\proc, \sheap)$ consisting of a \jsil procedure with the code of the test and the initial 
symbolic heap on which to execute the test. 
%
Figure~\ref{fig:test:generation} presents the test generation procedure. Intuitively, $\testify{}(\specsig{\pass}{\pid(\jvec{x})}{\qass}{\flag})$ 
returns the symbolic test for $\specsig{\pass}{\pid(\jvec{x})}{\qass}{\flag}$. The test generation function $\testifyfun{} \ $ is defined in terms 
of two auxiliary functions, $\testifyfun{\fnormal}$ and $\testifyfun{\ferror}$, for generating tests for $\fnormal$-mode and 
$\ferror$-mode specifications, respectively. 
For space reasons, we only present $\testifyfun{\fnormal}$, $\testifyfun{\ferror}$ is equivalent. 
The test program $\prog'$, denoted by $\prog[\jsilmain \mapsto \proc]$, is obtained from the original program $\prog$ and the test procedure $\proc$ by replacing the 
$\jsilmain$ of $\prog$ with the new test procedure, $\proc$. 

Finally, Theorem~\ref{teo:bug:finding:sl} states that if the symbolic execution of the 
test generated for $\specsig{\pass}{\pid(\jvec{x})}{\qass}{\flag}$ finds a bug, then the specification 
is not~valid.

\begin{theorem}[Bug-finding for SL Specifications]\label{teo:bug:finding:sl}
$$
\begin{array}{l}
\testify{}(\specsig{\pass}{\pid(\jvec{x})}{\qass}{\flag})  = (\proc, \sstate) \, \wedge \, 
  \prog' :  \transabssemrule{\sstate, \csmain, 0}{-}{\top}{\bot}{\symbolic} \\ \quad \quad 
    \implies  
         \prog \not\satisfies \specsig{\pass}{\pid(\jvec{x})}{\qass}{\flag}
\end{array}
$$
\noindent where:  $\csmain = [ (\jsilmain, -, -, -) ]$ 
and $\prog' = \prog[\jsilmain \mapsto \proc]$.
\end{theorem}





\section{Evaluation}
%!TEX root = ../main.tex

\myparagraph{Methodology}
Paragraph about the test262 tests

With \cosette, we can write symbolic tests, in which some of the concrete values of the program are replaced with symbolic values.
Symbolic tests improve on concrete tests for two main reasons.
First, they are by construction more comprehensive than concrete tests, because symbolic tests can account for the whole range of values that a variable can take, instead of focusing on a few specific examples.
Second, when \cosette finds a failing assertion inside a symbolic test, it can concretize the symbolic values into a counter-model that the developer can actually run in node, making debugging much easier compared to (the other things that we mention before).

Paragraph about spec-driven debugging

\myparagraph{Symbolic testing}
We used \cosette to analyse the code of two JavaScript data structure libraries: Buckets.js~\cite{buckets}, and queue-pri~\cite{priq}.
We chose these libraries because reasoning about data structure code requires a precise description of the control flow features of JavaScript, because they come equipped with unit test suites, and because they do not have external dependencies (\cosette is a whole-program analysiso); Buckets.js has over 65k downloads on npm.
For these two libraries, we wrote symbolic tests with the aim of obtaining 100\% line coverage, in order to compare them with the concrete unit tests that ship with the libraries.
In both cases, we were able to reduce the length of the tests by up to an average factor of 3, while retaining full coverage; we also discovered one bug in the Buckets.js library, and one in the queue-pri library.


\myparagraph{Results}
The results are presented in table~\ref{cosette:res}.
For each file in the library, we report the number of JS executable lines in the code itself / including dependencies, the corresponding numbers of JSIL lines, the number of symbolic and concrete test cases, the number of JS lines in the symbolic and concrete tests, the coverage measured as percentage of lines and the average \cosette run time for the symbolic tests.
Most of the execution time is spent in \cosette itself, with negligible time spent in solver calls (except for a limited number of cases).


\begin{table}[h]
{
\small
%\begin{center}
\setlength\tabcolsep{4pt}
\begin{tabular*}{\linewidth}{lrrrrrr}
\toprule
% Name || JS Loc/loc* || JSIL Loc/loc* || #tests || symb/conc loc || symb/conc cov || time
Name & \makecell{JS loc} & \makecell{JSIL loc} & \# Tests & \makecell{Test loc} & \makecell{Line\\cov. (\%)} & \makecell{Avg.\\time} \\
\midrule
\texttt{arrays} & 44/71 & 1251/1942 & 9/24 & 166/329 & 100/\_ & 20.2s \\
\texttt{bag} & 69/237 & 2041/7194 & 7/18 & 78/265 & 100/\_ & 1m14s \\
\texttt{bstree} & 143/326 & 3819/8052 & 11/31 & 216/759 & 100/\_ & 5m27s \\
\texttt{dict} & 57/84 & 1683/2374 & 7/14 & 116/170 & 100/\_ & 15.6s \\
\texttt{heap} & 57/128 & 2059/4001 & 4/15 & 92/626 & 100/\_ & 5m29s \\
\texttt{llist} & 126/153 & 2447/3138 & 9/21 & 149/370 & 100/\_ & 24.0s \\
\texttt{multidict} & 56/184 & 1871/5496 & 6/16 & 118/189 & 100/\_ & 1m15s \\
\texttt{pqueue} & 26/154 & 1066/5067 & 5/12 & 70/283 & 100/\_ & 5m49s \\
\texttt{queue} & 30/183 & 1095/4233 & 6/9 & 111/146 & 100/\_ & 20.7s \\
\texttt{set} & 40/124 & 1528/3902 & 6/12 & 86/271 & 100/\_ & 1m02s \\
\texttt{stack} & 23/176 & 941/4079 & 4/7 & 91/104 & 100/\_ & 26.4s \\
\midrule 
\texttt{queue-pri} & 0/0 & 0/0 & 0/0 & 0/0 & 100/0 & 42.0 \\
\bottomrule
%\end{center}
\end{tabular*}
}
\caption{Tests for the Buckets.js and queue-pri libraries}
\label{cosette:res}
\end{table}


\myparagraph{Bug-finding}
With \cosette, we found a bug in the implementation of the Buckets.js \jsinline{multidictionary} (a key-value map in which a single key might hold several values), as well as a bug in \jsinline{queue-pri}, a JS priority queue implementation~(\cite{priq}). 

%\myparagraph{\jsinline{multidictionary}}
%The implementation of the Buckets.js \jsinline{multidictionary} essentially comprises the \jsinline{get(key)} (which returns an array of values), \jsinline{set(key, value)}, and \jsinline{remove(key, value)} methods.
The implementation of the Buckets.js \jsinline{multidictionary} essentially comprises the \jsinline{get(key)}, \jsinline{set(key, value)}, and \jsinline{remove(key, value)} methods.
In the \jsinline{remove} function, the \jsinline{value} argument can either be \jsinline{undefined}, in which case the key is completely removed from the dictionary, or it can be an actual value.
In that case, the array associated with the key \jsinline{s} is retrieved, and, if the value is present in the array, it is removed.
However, the library does not consider the case where this array is not found; in that case, the library tries to remove an element from \jsinline{undefined}, which raises an error.
\begin{lstjs}
multiDict.remove = function (key, value) {
    if (value === undefined) { ... }
    var array = parent.get(key);
    if (buckets.arrays.remove(array, value, equalsF)) { ... }
    return false;
}
\end{lstjs}
We were able to expose this behavior with the following symbolic test:

\begin{lstjs}
var dict = new buckets.MultiDictionary()
var s = symb_string(s);
var x1 = symb_number(x1), x2 = symb_number(x2);
dict.set(s, x1); dict.set(s, x2);
dict.remove(s, x1);
var res = dict.remove(s, x2);
assert((x1 != x2 && res) || (x1 == x2 && !res));
\end{lstjs}

In this test, we create a new dictionary \jsinline{dict}, and insert two symbolic values \jsinline{x1} and \jsinline{x2} at key \jsinline{s}, then remove them in order.
However, if \jsinline{x1} is actually equal to \jsinline{x2}, the implementation does not store it twice.
This means that the key in the second call to \jsinline{remove} on line 6 is not in the dictionary any more, which triggers the bug.

When running \cosette on this test, we obtain the counter-model \jsinline{x1 = x2 = 0}, and running the instantiation of the test in Node triggers an error.
Moreover, we were able to easily fix the bug by adding a check for \jsinline{undefined} after line 3 in the code of \jsinline{remove}, after which \cosette was successfully able to discharge the assertion.

%\myparagraph{\jsinline{queue-pri}}
The other bug that we found is in the \jsinline{queue-pri} library, which implements a priority queue that stores data with an optional priority value.
The priority value is either an integer (the lower the value, the higher the priority) or the default \jsinline{null} value if no priority is provided, which puts the associated element at the end of the queue.
With symbolic tests of the \jsinline{enqueue} and \jsinline{dequeue} methods of the library, we found that elements enqueued with a priority value of \jsinline{0} were actually being enqueued at the end of the queue, when they should have had the highest priority instead.

We identified the bug to come from the following part of the \jsinline{enqueue} function: \jsinline!var payload = { data: data, priority: pri || null }!
Indeed, in JavaScript, the expression \jsinline{0 || null} actually evaluates to \jsinline{null}, whereas the developer believed that it reduced to 0.
Even though the developer wrote tests, they never used a priority value of 0 and therefore never triggered the bug.
This shows that \cosette is a useful tool for debugging JavaScript code, because it analyses all possible code paths, and does not make any assumptions about the subtle semantics of the language (type coercions in this case).






%We demonstrate the practical validity of \cosette as a tool for bug-finding in JavaScript.
%We wrote symbolic tests fully covering the code of real-world Node.js libraries, and found several  implementation bugs.
%Moreover, we measured evaluation times, including solver times, to assess the performance of \cosette.
%
%Throughout this section, we will be using the Buckets.js~\cite{buckets} (which has around 65k downloads on npm) data structure library as a running example.
%It implements most common data structures, comes with unit tests, and has no external dependencies, which is necessary for the whole-program analysis of \cosette.
%
%\subsection{Making Concrete Tests Symbolic}
%
%When writing libraries, developers more and more often include extensive test suites, called unit tests, along with actual code (JavaScript has several well-known unit test libraries, such as Jasmine~\cite{jasmine}, Ava~\cite{ava}, or Mocha~\cite{mocha}).
%These test suites allow the developer to have reasonable confidence in the well-behavedness of their code, up to the level of detail of the tests.
%However, being concrete by nature, these tests might miss corner cases or code paths that the developer hasn't reasoned about.
%
%The example code on the right shows the expressiveness of symbolic testing.
%We adapted this code from one of the tests bundled with the array library in Buckets.js.
%The snippet tests the \jsinline{swap} function, which swaps two elements of a JavaScript array, and returns \jsinline{true} if the swap was successful (the two index arguments were within the bounds of the array), or \jsinline{false} if the indices were out of bounds.
%In the original test code, the developer had to write several examples, each reflecting one of these particular cases.
%With the help of symbolic values, we are able to reduce this to a single example which encompasses all possible cases.
%We take \jsinline{ar} to be a JS array with some values, and \jsinline{i} is a symbolic number that we use as an index (for simplicity, we only make one of the two indices symbolic here).
%The assertion on line 7 accurately describes all the possible cases that can happen, \jsinline{i} being either inside the bounds or outside.
%Running \cosette on this code tells us that the assertion always holds, which means that \jsinline{swap} behaves as expected.
%
%Writing symbolic tests with \cosette therefore allows the developer to reduce the burden of test writing and maintenance, by increasing the expressive power of tests without losing their simplicity.
%
%\begin{lstjs}
%var n1 = symb_number(n1); % 0
%var n2 = symb_number(n2);
%var i = symb_number(i);
%var ar = [n1, n1, n2, n2];
%var res = buckets.arrays.swap(ar, 0, i);
%var len = ar.length;
%assert((i >= 0 && i < len && res) || ((i < 0 || i >= len) && !res));
%\end{lstjs}


\section{Related Work} 

The existing literature covers a wide range of analysis techniques for JavaScript programs, including: 
type systems~\cite{thiemann:esop:2005,anderson:ecoop:2005,jensen:sas:2009,typescript:toot:2014,feldthaus:oopsla:2014,bierman:ecoop:2014,rastogi:popl:2015},
control flow analysis~\cite{feldthaus2013efficient}, pointer analysis~\cite{jang2009points,sridharan:ecoop:12} and abstract
interpretation~\cite{kashyap:fse:14,jensen:sas:2009,andreasen:oopsla:2014,park:ecoop:15}, among others. 
Here, we focused on the existing work on logic-based analysis and symbolic execution for JavaScript. 

\myparagraph{Symbolic Execution} 




\myparagraph{Logic-based Analysis} 
%
\cite{gardner:popl:2012} have developed a separation logic for a small fragment of ECMAScript 3, to reason about the variable store emulated in the JavaScript heap.
%
\cite{rosu-serbanuta-2010-jlap} have developed $\mathbb{K}$, a term-rewriting framework  for  formalising the operational
semantics of programming languages.
 In particular, they have developed KJS~\cite{Park:2015} which provides a $\mathbb{K}$-interpretation of the core language and part of the built-in libraries of the ES5 standard. KJS has been tested against the official ECMAScript Test262 test suite and passed all 2782 tests for the core language; the testing results for the built-in libraries are not reported. 
\cite{stefanescu-park-yuwen-li-rosu-2016-oopsla} introduce a language-independent verification infrastructure 
that can be instantiated with a $\mathbb{K}$-interpretation of a  language to automatically generate a symbolic verification tool for that language based on the $\mathbb{K}$ reachability logic. They apply this infrastructure to KJS to generate a verification tool for JavaScript, which they use to verify functional correctness properties of operations for manipulating data structures such as binary search trees, AVL trees, and lists.


\section{Conclusions}\label{conclusions}

\newpage
\bibliographystyle{abbrv}
\bibliography{esop18}

\newpage
\appendix

%!TEX root = ../main.tex

\newtheorem{lemmax}{}
\newtheorem{temax}{}

\section{\jsil Syntax and Semantics}


\begin{figure}[ht!]
{\scriptsize
\begin{mathpar} 
%
\inferrule[\textsc{Skip}]{}
	{ \semtrans{\heap, \store, \jsilskip}{\heap, \store}} 
 \qquad
 %
\inferrule[\textsc{Assignment}]
  {
      \symbeval{\jsilexpr}{\store} =  \val
      \quad
      \store' = \store[\jvar \mapsto \val]
  }{\semtrans{\heap, \store, \jvar := \jsilexpr}{\heap, \store'}} 
%
\qquad 
%
\inferrule[\textsc{Object Creation}]
  { 
    \heap = \heap \dunion \hcell{\loc}{\protop}{\jsnull}
    \quad (\loc,-) \notin \domain (\heap)
  }{\semtrans{\heap, \store, \jvar := \jsilnew()}{\heap, \store[\jvar \mapsto \loc]}}
\\
%
\inferrule[\textsc{Property Access}]
  { 
 	\symbeval{\jsilexpr_1}{\store} =  \loc
  	\quad 
        \symbeval{\jsilexpr_2}{\store} =  \jstring
        \quad
        \heap = - \dunion \hcell{\loc}{\jstring}{\val}
  }{ \semtrans{\heap, \store, \jvar := [\jsilexpr_1, \jsilexpr_2]}{\heap,  \store[\jvar \mapsto \val]}}
 \and 
 \inferrule[\textsc{Property Deletion}]
  { 
        \symbeval{\jsilexpr_1}{\store} =  \loc
  	\quad 
        \symbeval{\jsilexpr_2}{\store} =  \jstring
        \quad
        \heap = \heap' \dunion \hcell{\loc}{\jstring}{-}
  }{\semtrans{\heap, \store, \jsildelete(\jsilexpr_1, \jsilexpr_2)}{\heap', \store}}
 %
\\
%
\inferrule[\textsc{Property Assignment - Found}]
  {     \symbeval{\jsilexpr_1}{\store} =  \loc
  	\quad 
        \symbeval{\jsilexpr_2}{\store} =  \jstring
        \quad
        \symbeval{\jsilexpr_3}{\store} =  \val
       \\\\
        \heap = \heap' \dunion  \hcell{\loc}{\jstring}{-}
  }{\semtrans{\heap, \store, [\jsilexpr_1, \jsilexpr_2] := \jsilexpr_3}{\heap' \dunion  \hcell{\loc}{\jstring}{\val}, \store}} 
 \and 
 \inferrule[\textsc{Property Assignment - Not Found}]
  {     \symbeval{\jsilexpr_1}{\store} =  \loc
  	\quad 
        \symbeval{\jsilexpr_2}{\store} =  \jstring
        \quad
        \symbeval{\jsilexpr_3}{\store} =  \val
       \\\\
        \heap = \heap' 
        \quad 
        (\loc, \jstring) \not\in \domain(\heap)
  }{\semtrans{\heap, \store, [\jsilexpr_1, \jsilexpr_2] := \jsilexpr_3}{\heap \dunion  \hcell{\loc}{\jstring}{\val}, \store}} 
\\
%
\inferrule[\textsc{Member Check - True}]
  { 
      \symbeval{\jsilexpr_1}{\store} =  \loc
  	\quad 
        \symbeval{\jsilexpr_2}{\store} =  \jstring
       \quad 
   	(\loc, \jstring) \in \domain(\heap) 
  }{\semtrans{\heap, \store,\jvar := \hasfield(\jsilexpr_1, \jsilexpr_2)}{\heap, \store[\jvar \mapsto \jtrue]}}
  \and 
 \inferrule[\textsc{Member Check - False}]
  { 
      \symbeval{\jsilexpr_1}{\store} =  \loc
  	\quad 
        \symbeval{\jsilexpr_2}{\store} =  \jstring
       \quad 
   	(\loc, \jstring) \not\in \domain(\heap) 
  }{\semtrans{\heap, \store,\jvar := \hasfield(\jsilexpr_1, \jsilexpr_2)}{\heap, \store[\jvar \mapsto \jfalse]}}
%
\\
%
\inferrule[\textsc{Assert - True}]
  { 
      \symbeval{\jsilexpr}{\store} =  \jtrue
  }{\semtrans{\heap, \store, \assert(\jsilexpr)}{\heap, \store}} 
\and
\inferrule[\textsc{Assert - False}]
  { 
      \symbeval{\jsilexpr}{\store} \neq \jtrue
  }{\semtranserr{\heap, \store, \assert(\jsilexpr)}} 
\end{mathpar}}
\caption{Symbolic Execution for Basic Commands: {\scriptsize$\semtrans{\heap, \store, \bcmd}{\heap', \store'}$}\label{fig:sem:basic:commands}}
\end{figure}




\begin{figure}[ht!]
{\scriptsize
\begin{mathpar} 
\inferrule[\textsc{Basic Command}]
   { 
     \prog_{\pid}(i) = \bcmd 
     \quad
     \semtrans{\heap, \store, \bcmd}{\heap', \store'} 
   }{\semtrans{\heap, \store, \ctx[i]}{\heap', \store', \ctx[i+1]}}
%
   \qquad
  %
  \inferrule[\textsc{Basic Command - Fail}]
   { 
     \prog_{\pid}(i) = \bcmd 
     \quad
     \semtranserr{\heap, \store, \bcmd} 
   }{\semtranserr{\heap, \store, \ctx[i]}}
 %
   \qquad
  %
  \inferrule[\textsc{Goto}]
   { \prog_{\pid}(i) = \goto \, j \quad}
   {\semtrans{\heap, \store, \ctx[i]}{\heap, \store, \ctx[j]}}
  \\ 
  \inferrule[\textsc{Cond. Goto - True}]
   { \prog_{\pid}(i) =  \ifgoto{\jsilexpr}{j}{k} \quad
     \symbeval{\jsilexpr}{\store} =  \jtrue
   }
   {\semtrans{\heap, \store, \ctx[i]}{\heap, \store, \ctx[j]}}
  \and 
    \inferrule[\textsc{Cond. Goto - False}]
   { \prog_{\pid}(i) =  \ifgoto{\jsilexpr}{j}{k} \quad
     \symbeval{\jsilexpr}{\store} =  \jfalse
   }
   {\semtrans{\heap, \store, \ctx[i]}{\heap, \store, \ctx[k]}}
   \\
    \inferrule[\textsc{Procedure Call}]
   { 
    \prog_{\pid}(i) =   \jsilcall{\jvar}{\jsilexpr}{\jsilexpr_i \mid_{i = 0}^{n}}{j}
     \quad
    \symbeval{\jsilexpr}{\sstore} =  \pid' 
        \quad
     \args(\pid') = \jsillist{\jvar_1, ..., \jvar_{m}} 
      \quad
      \val_i = \symbeval{\jsilexpr_i}{\sstore} \mid_{i = 0}^{n} 
     \ 
      \val_i = \jsundefined \mid_{i = n+1}^{m}  
   }
   {\semtrans{\heap, \store, \ctx[i]}{\heap, [ \jvar_i \mapsto \val_i \mid_{i = 0}^{m}] , ((\pid', \store, \jvar, i+1, j)::\ctx)[0]}}
    \\ 
  \inferrule[\textsc{Normal Return}]
   {
       \ctx = (-, \store', \jvar, i, -) :: \ctx' 
       \quad 
       \store(\procretvar) = \val
   }  
   {\semtrans{\heap, \store, \ctx[\procretlab]}{\heap, \store'[\jvar \mapsto \val], \ctx'[i]}}
   \and 
     \inferrule[\textsc{Error Return}]
   {
       \ctx = (-, \store', \jvar, -, j) :: \sctx' 
       \quad 
       \sstore(\procerrvar) = \val
   }
   {\semtrans{\heap, \store, \ctx[\procerrlab]}{\sheap, \store'[\jvar \mapsto \val], \ctx'[j]}}
 \end{mathpar}}
\caption{Symbolic Execution for Control Flow Commands: {\scriptsize$\semtrans{\heap, \store, \ctx[i]}{\heap', \store', \ctx'[j]}$}} 
\end{figure}


\begin{figure}[ht!]
{\scriptsize
\begin{mathpar} 
\inferrule[\textsc{Basic Command}]
   { 
     \ccmd[\prog][\ctx]{i} = \bcmd 
     \quad
     \semtrans{\heap, \store, \bcmd}{\heap', \store'} 
   }{\semtrans[\prog]{\heap, \store, i}{\heap', \store', i+1}[C]}
%
   \qquad
  %
  \inferrule[\textsc{Basic Command - Fail}]
   { 
     \ccmd[\prog][\ctx]{i} = \bcmd 
     \quad
     \semtranserr{\heap, \store, \bcmd} 
   }{\semtranserr[\prog]{\heap, \store, i}[C]}
 %
   \qquad
  %
  \inferrule[\textsc{Goto}]
   { \ccmd[\prog][\ctx]{i} = \goto \, j \quad}
   {\semtrans[\prog]{\heap, \store, i}{\heap, \store, j}[C]}
  \\ 
  \inferrule[\textsc{Cond. Goto - True}]
   { \ccmd[\prog][\ctx]{i} =  \ifgoto{\jsilexpr}{j}{k} \quad
     \symbeval{\jsilexpr}{\store} =  \jtrue
   }
   {\semtrans[\prog]{\heap, \store, i}{\heap, \store, j}[C]}
  \and 
    \inferrule[\textsc{Cond. Goto - False}]
   { \ccmd[\prog][\ctx]{i} =  \ifgoto{\jsilexpr}{j}{k} \quad
     \symbeval{\jsilexpr}{\store} =  \jfalse
   }
   {\semtrans[\prog]{\heap, \store, i}{\heap, \store, k}[C]}
   \\
    \inferrule[\textsc{Procedure Call}]
   { 
    \ccmd[\prog][\ctx]{i} =   \jsilcall{\jvar}{\jsilexpr}{\jsilexpr_i \mid_{i = 0}^{n}}{j}
     \quad
    \symbeval{\jsilexpr}{\sstore} =  \pid' 
        \quad
     \args(\pid') = \jsillist{\jvar_1, ..., \jvar_{m}} 
      \quad
      \val_i = \symbeval{\jsilexpr_i}{\sstore} \mid_{i = 0}^{n} 
     \ 
      \val_i = \jsundefined \mid_{i = n+1}^{m}  
   }
   {\semtrans[\prog]{\heap, \store, i}{\heap, [ \jvar_i \mapsto \val_i \mid_{i = 0}^{m}] , 0}[C][(\pid', \store, \jvar, i+1, j) :: \ctx]}
    \\ 
  \inferrule[\textsc{Normal Return}]
   {
       \ctx = (-, \store', \jvar, i, -) :: \ctx' 
       \quad 
       \store(\procretvar) = \val
   }  
   {\semtrans[\prog]{\heap, \store, \procretlab}{\heap, \store'[\jvar \mapsto \val], i}[C][C']}
   \and 
     \inferrule[\textsc{Error Return}]
   {
       \ctx = (-, \store', \jvar, -, j) :: \ctx' 
       \quad 
       \store(\procerrvar) = \val
   }
   {\semtrans[\prog]{\heap, \store, \procerrlab}{\heap, \store'[\jvar \mapsto \val], j}[C][C']}
 \end{mathpar}}
 \pmax{What happens when we exit from main, how do we stop? Basically, cmd, returns nothing and we can't reduce?}
 \vspace*{-0.4cm}
\caption{Symbolic Execution for Control Flow Commands: $\semtrans[\prog]{\heap, \store, i}{\heap', \store', j}[C][C']$}
\end{figure}

\begin{figure}[ht!]
{\scriptsize
\begin{mathpar} 
\inferrule[\textsc{Basic Command}]
   { 
     \ccmd{i} = \bcmd 
     \quad
     \semtrans{\heap, \store, \bcmd}{\heap', \store'} 
   }{\semtrans{\heap, \store, i}{\heap', \store', i+1}}
%
   \qquad
  %
  \inferrule[\textsc{Basic Command - Fail}]
   { 
     \ccmd{i} = \bcmd 
     \quad
     \semtranserr{\heap, \store, \bcmd} 
   }{\semtranserr{\heap, \store, i}}
 %
   \qquad
  %
  \inferrule[\textsc{Goto}]
   { \ccmd{i} = \goto \, j \quad}
   {\semtrans{\heap, \store, i}{\heap, \store, j}}
  \\ 
  \inferrule[\textsc{Cond. Goto - True}]
   { \ccmd{i} =  \ifgoto{\jsilexpr}{j}{k} \quad
     \symbeval{\jsilexpr}{\store} =  \jtrue
   }
   {\semtrans{\heap, \store, i}{\heap, \store, j}}
  \and 
    \inferrule[\textsc{Cond. Goto - False}]
   { \ccmd{i} =  \ifgoto{\jsilexpr}{j}{k} \quad
     \symbeval{\jsilexpr}{\store} =  \jfalse
   }
   {\semtrans{\heap, \store, i}{\heap, \store, k}}
   \\
    \inferrule[\textsc{Procedure Call}]
   { 
    \ccmd{i} = \jsilcall{\jvar}{\jsilexpr}{\jsilexpr_i \mid_{i = 0}^{n}}{j}
     \quad
    \symbeval{\jsilexpr}{\sstore} =  \pid' 
        \quad
     \args(\pid') = \jsillist{\jvar_1, ..., \jvar_{m}} 
      \quad
      \val_i = \symbeval{\jsilexpr_i}{\sstore} \mid_{i = 0}^{n} 
     \ 
      \val_i = \jsundefined \mid_{i = n+1}^{m}  
   }
   {\semtrans{\heap, \store, i}{\heap, [ \jvar_i \mapsto \val_i \mid_{i = 0}^{m}] , 0}[C][(\pid', \store, \jvar, i+1, j) :: \ctx]}
    \\ 
  \inferrule[\textsc{Normal Return}]
   {
       \ctx = (-, \store', \jvar, i, -) :: \ctx' 
       \quad 
       \store(\procretvar) = \val
   }  
   {\semtrans{\heap, \store, \procretlab}{\heap, \store'[\jvar \mapsto \val], i}[C][C']}
   \and 
     \inferrule[\textsc{Error Return}]
   {
       \ctx = (-, \store', \jvar, -, j) :: \ctx' 
       \quad 
       \store(\procerrvar) = \val
   }
   {\semtrans{\heap, \store, \procerrlab}{\heap, \store'[\jvar \mapsto \val], j}[C][C']}
 \end{mathpar}}
 \vspace*{-0.4cm}
\caption{Symbolic Execution for Control Flow Commands: $\semtrans[\prog]{\heap, \store, i}{\heap', \store', j}[C][C']$}
\end{figure}

\section{Proofs - Section~\ref{sec:jsil:symb:exec}}

\begin{lemma}[Soundess of symbolic execution for \jsil basic commands]
$$
\begin{array}{l}
\symbtrans{\sheap, \sstore, \bcmd, \pc}{\sheap', \sstore', \pc'}
   \ \wedge \ 
      (\heap, \store) \in \smodels{\sheap, \sstore}{\pc'} \\ \quad \quad
      	 \ \implies \ \exists (\heap', \store') \, . \, 
	 	 \semtrans{\heap, \store, \bcmd}{\heap', \store'}
		\, \wedge \, 
		(\heap', \store') \in \smodels{\sheap', \sstore'}{\pc'}  
\end{array}
$$
\end{lemma}
\begin{proof}
We proceed by case analysis on $\symbtrans{\sheap, \sstore, \bcmd, \pc}{\sheap', \sstore', \pc'}$. 
\vspace{5pt}

\noindent\prooflab{Skip} 
We conclude that $\bcmd = \jsilskip$, and 
that $\sheap' = \sheap$, $\sstore' = \sstore$, and $\pc' = \pc$. 
By picking $\heap' = \heap$, $\store' = \store$, the result follows. 
\vspace{6pt}

\noindent\prooflab{Assignment} 
We conclude that $\bcmd = \jvar := \jsilexpr$, for some variable $\jvar$ and expression $\jsilexpr$, 
and that $\sheap' = \sheap$, $\sstore' = \sstore[\jvar \mapsto \symbeval{\jsilexpr}{\store}]$, and $\pc' = \pc$. 
From $(\heap, \store) \in \smodels{\sheap, \sstore}{\pc'}$, we conclude that there is a symbolic environment 
$\senv$ such that $\heap = \semexpr{\sheap}{\senv}$ and $\store = \semexpr{\sstore}{\senv}$. 
Noting that: 
$$
 \semtrans{\heap, \store, \jvar := \jsilexpr}{\heap, \store[\jvar \mapsto \symbeval{\jsilexpr}{\store}]}
% \qquad 
 %\semexpr{\sstore[\jvar \mapsto \symbeval{\jsilexpr}{\store}]}{\senv} = \semexpr{\sstore}{\senv}[\jvar \mapsto \symbeval{\jsilexpr}{\store, \senv}]
$$
we pick $\heap' = \heap$ and $\store' =  \store[\jvar \mapsto \symbeval{\jsilexpr}{\store}]$. We 
now have to prove that $(\heap', \store') \in \smodels{\sheap', \sstore'}{\pc}$.
Observing that: 
$$
\heap' =  \semexpr{\sheap}{\senv} = \semexpr{\sheap'}{\senv} 
\quad 
\store' = \semexpr{\sstore}{\senv}[\jvar \mapsto \symbeval{\jsilexpr}{\semexpr{\sstore}{\senv}}]
   = \semexpr{\sstore[\jvar \mapsto \symbeval{\jsilexpr}{\store}]}{\senv} 
   = \semexpr{\sstore'}{\senv}
$$
%
the result follows. 
\vspace{6pt}

\noindent\prooflab{Object Creation}
We conclude that $\bcmd = \jvar := \jsilnew()$, for some variable $\jvar$, and that
$\sheap' = \sheap \dunion \hcell{\loc}{\protop}{\jsnull}$, $\sstore' = \sstore[\jvar \mapsto \loc]$, and $\pc' = \pc$, 
 where  $(\loc,-) \notin \domain (\sheap)$. 
 From $(\heap, \store) \in \smodels{\sheap, \sstore}{\pc'}$, we conclude that there is a symbolic environment
$\senv$ such that $\heap = \semexpr{\sheap}{\senv}$ and $\store = \semexpr{\sstore}{\senv}$. 
Noting that: 
$$
\semtrans{\heap, \store, \jvar := \jsilnew()}{\heap \dunion \hcell{\loc}{\protop}{\jsnull}, \store[\jvar \mapsto \loc]}
$$
where: $(\loc,-) \notin \domain (\heap)$, we pick $\heap' = \semexpr{\sheap}{\senv} \dunion \hcell{\loc}{\protop}{\jsnull}$ 
and $\store' = \semexpr{\sstore}{\senv}[\jvar \mapsto \loc]$. 
We now have to prove that $(\heap', \store') \in \smodels{\sheap', \sstore'}{\pc}$.
Noting that: 
$$
\begin{array}{l}
\heap' = \semexpr{\sheap}{\senv} \dunion \hcell{\loc}{\protop}{\jsnull} = \semexpr{\sheap \dunion \hcell{\loc}{\protop}{\jsnull}}{\senv}   
     = \semexpr{\sheap'}{\senv}  \\
%
\store' = \semexpr{\sstore}{\senv}[\jvar \mapsto \loc] = \semexpr{\sstore}{\senv}[\jvar \mapsto \symbeval{\loc}{\senv}] = 
      \semexpr{\sstore[\jvar \mapsto \loc]}{\senv} = \semexpr{\sstore'}{\senv} 
\end{array}
$$
the result follows. 
\vspace{6pt}

\noindent\prooflab{Property Access}
We conclude that $\bcmd = \jvar := [\jsilexpr_1, \jsilexpr_2]$, for some variable $\jvar$, and expressions $\jsilexpr_1$ and $\jsilexpr_2$, 
and that $\sheap' = \sheap$, $\sstore' = \sstore[\jvar \mapsto \sexprv_k]$, and: 
 $$\pc' =  \pc \ \wedge \, \big( (\sexprp_k = \sexpr_p) \ \wedge \bigwedge_{i = 0, i \neq k}^n (\sexprp_i \neq \sexpr_p) \big)$$
 where 
 $\symbeval{\jsilexpr_1}{\sstore} =  \loc$, $\symbeval{\jsilexpr_2}{\sstore} =  \sexpr_p$, 
 $\sheap = \sheap'' \, \uplus \, \big((l, \sexprp_i) \mapsto \sexprv_i\big)\mid_{i = 0}^n$, 
 $(l, -) \not\in \domain(\sheap')$, and $0 \leq k \leq n$. 
%
From $(\heap, \store) \in \smodels{\sheap, \sstore}{\pc'}$, we conclude that there is a symbolic environment
$\senv$ such that $\heap = \semexpr{\sheap}{\senv}$, $\store = \semexpr{\sstore}{\senv}$, and 
$\senv \vdash \pc'$. 
We now have to prove that we can apply the \prooflab{Property Access} rule in the concrete state.
To this end, we have to show that there is a concrete heap $\heap''$ such that:
$\heap = \heap'' \dunion \hcell{\symbeval{\jsilexpr_1}{\store}}{\symbeval{\jsilexpr_2}{\store}}{\symbeval{\jsilexpr_3}{\store}}$. 
Note that: 
$$
\begin{array}{l}
%
 \symbeval{\jsilexpr_1}{\store} = \symbeval{\jsilexpr_1}{\symbeval{\sstore}{\senv}} = \symbeval{\symbeval{\jsilexpr_1}{\sstore}}{\senv} 
    = \symbeval{\loc}{\senv} = \loc \\ 
 %
  \symbeval{\jsilexpr_2}{\store}  = \symbeval{\jsilexpr_2}{\semexpr{\sstore}{\senv}} =  \symbeval{\symbeval{\jsilexpr_2}{\sstore}}{\senv}
   =  \symbeval{\sexpr_p}{\senv} = \symbeval{\sexprp_k}{\senv}  \text{ (because $\senv \vdash \pc'$ and $\pc' \vdash \sexprp_k = \sexpr_p$)} \\
 %
 \heap = \semexpr{\sheap'' \, \uplus \, \big((l, \sexprp_i) \mapsto \sexprv_i\big)\mid_{i = 0}^n}{\senv} 
       =  \semexpr{\sheap'' \, \uplus \, \big((l, \sexprp_i) \mapsto \sexprv_i\big)\mid_{i = 0, i \neq k}^n}{\senv} \dunion \semexpr{(l, \sexprp_k) \mapsto \sexprv_k}{\senv} \\
         \qquad = \semexpr{\sheap'' \, \uplus \, \big((l, \sexprp_i) \mapsto \sexprv_i\big)\mid_{i = 0, i \neq k}^n}{\senv} \dunion (l, \semexpr{\sexprp_k}{\senv}) \mapsto \semexpr{\sexprv_k}{\senv}  \\ 
         \qquad =  \semexpr{\sheap'' \, \uplus \, \big((l, \sexprp_i) \mapsto \sexprv_i\big)\mid_{i = 0, i \neq k}^n}{\senv} \dunion (\symbeval{\jsilexpr_1}{\store}, \symbeval{\jsilexpr_2}{\store}) \mapsto \semexpr{\sexprv_k}{\senv}
\end{array}
$$
We can now apply the \prooflab{Property Access} rule of \jsil semantics, concluding: 
$$
   \semtrans{\heap, \store, \jvar := [\jsilexpr_1, \jsilexpr_2]}{\heap,  \store[\jvar \mapsto \semexpr{\sexprv_k}{\senv}]}
$$
meaning that: $\heap' = \heap$ and $\store' = \store[\jvar \mapsto \semexpr{\sexprv_k}{\senv}]$.
We have now to prove that $(\heap', \store') \in \smodels{\sheap', \sstore'}{\pc'}$.
Observe that: 
$$
\begin{array}{l}
\heap' = \heap = \semexpr{\sheap}{\senv}   = \semexpr{\sheap'}{\senv}  \text{ (because $\heap' = \heap$ and $ \sheap = \sheap'$)}
\\
 \store' =  \semexpr{\sstore}{\senv}[\jvar \mapsto \symbeval{\sexprv_k}{\senv}] 
    =  \semexpr{\sstore[\jvar \mapsto \sexprv_k]}{\senv} 
    =  \semexpr{\sstore'}{\senv}
\end{array}
$$
 which concludes the proof. 
\vspace{6pt}

\noindent\prooflab{Property Deletion}
We conclude that $\bcmd = \jsildelete(\jsilexpr_1, \jsilexpr_2)$, for some expressions $\jsilexpr_1$ and $\jsilexpr_2$
and that: 
$$
\begin{array}{l}
\sheap' = \sheap'' \, \uplus \,  \big((\loc, \sexprp_i) \mapsto \sexprv_i\big)\mid_{i = 0, i \neq k}^n
\quad 
\sstore' = \sstore
\\ 
 \pc' = \pc \ \wedge \, \big( (\sexprp_k = \sexpr_p) \ \wedge \bigwedge_{i = 0, i \neq k}^n (\sexprp_i \neq \sexpr_p) \big)
\end{array}
$$
where $\loc = \symbeval{\jsilexpr_1}{\sstore}$ and $\sexpr_p = \symbeval{\jsilexpr_2}{\sstore}$
From $(\heap, \store) \in \smodels{\sheap, \sstore}{\pc'}$, we conclude that there is a symbolic environment
$\senv$ such that $\heap = \semexpr{\sheap}{\senv}$, $\store = \semexpr{\sstore}{\senv}$, and 
$\senv \vdash \pc'$. 
We now have to prove that we can apply the \prooflab{Property Deletion} rule in the concrete state.
To this end, we have to show that:
$\heap = \heap' \dunion \hcell{\symbeval{\jsilexpr_1}{\store}}{\symbeval{\jsilexpr_2}{\store}}{-}$. 
Note that: 
$$
\begin{array}{l}
%
 \symbeval{\jsilexpr_1}{\store} = \symbeval{\jsilexpr_1}{\symbeval{\sstore}{\senv}} = \symbeval{\symbeval{\jsilexpr_1}{\sstore}}{\senv} 
    = \symbeval{\loc}{\senv} = \loc \\ 
 %
  \symbeval{\jsilexpr_2}{\store}  = \symbeval{\jsilexpr_2}{\semexpr{\sstore}{\senv}} =  \symbeval{\symbeval{\jsilexpr_2}{\sstore}}{\senv}
   =  \symbeval{\sexpr_p}{\senv} = \symbeval{\sexprp_k}{\senv}  \text{ (because $\senv \vdash \pc'$ and $\pc' \vdash \sexprp_k = \sexpr_p$)} \\
 %
 \heap = \semexpr{\sheap'' \, \uplus \, \big((l, \sexprp_i) \mapsto \sexprv_i\big)\mid_{i = 0}^n}{\senv} 
       =  \semexpr{\sheap'' \, \uplus \, \big((l, \sexprp_i) \mapsto \sexprv_i\big)\mid_{i = 0, i \neq k}^n}{\senv} \dunion \semexpr{(l, \sexprp_k) \mapsto \sexprv_k}{\senv} \\
         \qquad = \semexpr{\sheap'' \, \uplus \, \big((l, \sexprp_i) \mapsto \sexprv_i\big)\mid_{i = 0, i \neq k}^n}{\senv} \dunion (l, \semexpr{\sexprp_k}{\senv}) \mapsto \semexpr{\sexprv_k}{\senv}  \\ 
         \qquad =  \semexpr{\sheap'' \, \uplus \, \big((l, \sexprp_i) \mapsto \sexprv_i\big)\mid_{i = 0, i \neq k}^n}{\senv} \dunion (\symbeval{\jsilexpr_1}{\store}, \symbeval{\jsilexpr_2}{\store}) \mapsto \semexpr{\sexprv_k}{\senv} \\ 
         \qquad = \semexpr{\sheap'}{\senv} \dunion (\symbeval{\jsilexpr_1}{\store}, \symbeval{\jsilexpr_2}{\store}) \mapsto -
\end{array}
$$
We can now apply the \prooflab{Property Deletion} rule of \jsil semantics, concluding: 
$$
   \semtrans{\heap, \store, \jsildelete(\jsilexpr_1, \jsilexpr_2)}{\semexpr{\sheap'}{\senv},  \store}
$$
meaning that: $\heap' = \semexpr{\sheap'}{\senv}$ and $\store' = \store$.
We have now to prove that $(\heap', \store') \in \smodels{\sheap', \sstore'}{\pc'}$.
Noting that $\heap' = \semexpr{\sheap'}{\senv}$ and $\store' = \store = \semexpr{\sstore}{\senv} = \semexpr{\sstore'}{\senv}$, 
the result follows. 
\vspace{6pt}

\noindent\prooflab{Property Assignment - Found}
We conclude that  $\bcmd = [\jsilexpr_1, \jsilexpr_2] := \jsilexpr_3$ for some expressions $\jsilexpr_1$, $\jsilexpr_2$, 
and $\jsilexpr_3$, and that: 
$$
\begin{array}{l}
  \sheap =  \sheap'' \, \uplus \, \big((l, \sexprp_i) \mapsto \sexprv_i\big)\mid_{i = 0}^n    \\
  %
  \sheap' = \sheap'' \, \uplus \,  \big((l, \sexprp_i) \mapsto \sexprv_i\big)\mid_{i = 0, i \neq k}^n \, \uplus \,  (l, \sexpr_p) \mapsto \sexpr_v  \\
  %
  \sstore' = \sstore \\ 
  %
  \pc' = \pc \ \wedge \, \big( (\sexprp_k = \sexpr_p) \ \wedge \bigwedge_{i = 0, i \neq k}^n (\sexprp_i \neq \sexpr_p)
\end{array}
$$ 
where $\symbeval{\jsilexpr_1}{\sstore} =  \loc$, $\symbeval{\jsilexpr_2}{\sstore} =  \sexpr_p$, 
$\symbeval{\jsilexpr_3}{\sstore} =  \sexpr_v$.
From $(\heap, \store) \in \smodels{\sheap, \sstore}{\pc'}$, we conclude that there is a symbolic environment
$\senv$ such that $\heap = \semexpr{\sheap}{\senv}$, $\store = \semexpr{\sstore}{\senv}$, and 
$\senv \vdash \pc'$. 
We now have to prove that we can apply the \prooflab{Property Assignment - Found} rule in the concrete state.
To this end, we have to show that there is a concrete heap $\heap''$ such that:
$\heap = \heap'' \dunion \hcell{\symbeval{\jsilexpr_1}{\store}}{\symbeval{\jsilexpr_2}{\store}}{-}$. 
Note that: 
$$
\begin{array}{l}
%
 \symbeval{\jsilexpr_1}{\store} = \symbeval{\jsilexpr_1}{\symbeval{\sstore}{\senv}} = \symbeval{\symbeval{\jsilexpr_1}{\sstore}}{\senv} 
    = \symbeval{\loc}{\senv} = \loc \\ 
 %
  \symbeval{\jsilexpr_2}{\store}  = \symbeval{\jsilexpr_2}{\semexpr{\sstore}{\senv}} =  \symbeval{\symbeval{\jsilexpr_2}{\sstore}}{\senv}
   =  \symbeval{\sexpr_p}{\senv} = \symbeval{\sexprp_k}{\senv}  \text{ (because $\senv \vdash \pc'$ and $\pc' \vdash \sexprp_k = \sexpr_p$)} \\
 %
  \symbeval{\jsilexpr_3}{\store}  = \symbeval{\jsilexpr_3}{\semexpr{\sstore}{\senv}} =  \symbeval{\symbeval{\jsilexpr_3}{\sstore}}{\senv}
   =  \symbeval{\sexpr_v}{\senv} \\
 %
 \heap = \semexpr{\sheap'' \, \uplus \, \big((l, \sexprp_i) \mapsto \sexprv_i\big)\mid_{i = 0}^n}{\senv} 
       =  \semexpr{\sheap'' \, \uplus \, \big((l, \sexprp_i) \mapsto \sexprv_i\big)\mid_{i = 0, i \neq k}^n}{\senv} \dunion \semexpr{(l, \sexprp_k) \mapsto \sexprv_k}{\senv} \\
         \qquad = \semexpr{\sheap'' \, \uplus \, \big((l, \sexprp_i) \mapsto \sexprv_i\big)\mid_{i = 0, i \neq k}^n}{\senv} \dunion (l, \semexpr{\sexprp_k}{\senv}) \mapsto \semexpr{\sexprv_k}{\senv}  \\ 
         \qquad =  \semexpr{\sheap'' \, \uplus \, \big((l, \sexprp_i) \mapsto \sexprv_i\big)\mid_{i = 0, i \neq k}^n}{\senv} \dunion (\symbeval{\jsilexpr_1}{\store}, \symbeval{\jsilexpr_2}{\store}) \mapsto \semexpr{\sexprv_k}{\senv} \\ 
\end{array}
$$
We can now apply the \prooflab{Property Assignment - Found} rule of \jsil semantics, concluding: 
$$
   \semtrans{\heap, \store, [\jsilexpr_1, \jsilexpr_2] := \jsilexpr_3}
     {\semexpr{\sheap'' \, \uplus \, \big((l, \sexprp_i) \mapsto \sexprv_i\big)\mid_{i = 0, i \neq k}^n}{\senv} \dunion (\symbeval{\jsilexpr_1}{\store}, \symbeval{\jsilexpr_2}{\store}) \mapsto \symbeval{\jsilexpr_3}{\store},  \store}
$$
meaning that: 
$\heap' = \symbeval{\sheap'' \, \uplus \, \big((l, \sexprp_i) \mapsto \sexprv_i\big)\mid_{i = 0, i \neq k}^n}{\senv} \dunion (\symbeval{\jsilexpr_1}{\store}, \symbeval{\jsilexpr_2}{\store}) \mapsto \symbeval{\jsilexpr_3}{\store}$ and 
$\store' = \store$.
We have now to prove that $(\heap', \store') \in \smodels{\sheap', \sstore'}{\pc'}$.
Noting that:
$$
\begin{array}{l}
\heap' = \symbeval{\sheap'' \, \uplus \, \big((l, \sexprp_i) \mapsto \sexprv_i\big)\mid_{i = 0, i \neq k}^n}{\senv} \dunion (\symbeval{\jsilexpr_1}{\store}, \symbeval{\jsilexpr_2}{\store}) \mapsto \symbeval{\jsilexpr_3}{\store} \\ 
  \qquad = \symbeval{\sheap'' \, \uplus \, \big((l, \sexprp_i) \mapsto \sexprv_i\big)\mid_{i = 0, i \neq k}^n}{\senv} \dunion (\loc, \symbeval{\sexpr_p}{\senv}) \mapsto \symbeval{\sexpr_v}{\senv}  \\
    \qquad = \symbeval{\sheap'' \, \uplus \, \big((l, \sexprp_i) \mapsto \sexprv_i\big)\mid_{i = 0, i \neq k}^n \dunion (\loc, \sexpr_p) \mapsto \sexpr_v}{\senv}  \\
    \qquad = \symbeval{\sheap'}{\senv} \\[2pt]
 %
 \store' = \store = \symbeval{\sstore}{\senv} = \symbeval{\sstore'}{\senv} 
\end{array}
$$
the result follows. 
\vspace{6pt}

\noindent\prooflab{Property Assignment - Not Found}
We conclude that  $\bcmd = [\jsilexpr_1, \jsilexpr_2] := \jsilexpr_3$ for some expressions $\jsilexpr_1$, $\jsilexpr_2$, 
and $\jsilexpr_3$, and that: 
$$
\begin{array}{l}
  \sheap =   \sheap'' \, \uplus \, \big((l, \sexprp_i) \mapsto \sexprv_i\big)\mid_{i = 0}^n     \\
  %
  \sheap' =  \sheap \, \uplus \,  (l, \sexpr_p) \mapsto \sexpr_v  \\
  %
  \sstore' = \sstore \\ 
  %
    \pc' = \pc \ \wedge \, \bigwedge_{i = 0}^n (\sexprp_i \neq \sexpr_p)
\end{array}
$$ 
where $\symbeval{\jsilexpr_1}{\sstore} =  \loc$, $\symbeval{\jsilexpr_2}{\sstore} =  \sexpr_p$, 
$\symbeval{\jsilexpr_3}{\sstore} =  \sexpr_v$,  $(\loc, -) \not\in \domain(\sheap'')$, 
and $0 \leq k \leq n$. 
From $(\heap, \store) \in \smodels{\sheap, \sstore}{\pc'}$, we conclude that there is a symbolic environment
$\senv$ such that $\heap = \semexpr{\sheap}{\senv}$, $\store = \semexpr{\sstore}{\senv}$, and 
$\senv \vdash \pc'$. 
We have now to prove that we can apply the \prooflab{Property Assignment - Found} rule in the concrete state.
To this end, we have to show that:
$(\symbeval{\jsilexpr_1}{\store}, \symbeval{\jsilexpr_2}{\store}) \not\in \domain(\heap)$. 
Note that: 
$$
\begin{array}{l}
%
 \symbeval{\jsilexpr_1}{\store} = \symbeval{\jsilexpr_1}{\symbeval{\sstore}{\senv}} = \symbeval{\symbeval{\jsilexpr_1}{\sstore}}{\senv} 
    = \symbeval{\loc}{\senv} = \loc \\ 
 %
  \symbeval{\jsilexpr_2}{\store}  = \symbeval{\jsilexpr_2}{\semexpr{\sstore}{\senv}} =  \symbeval{\symbeval{\jsilexpr_2}{\sstore}}{\senv}
   =  \symbeval{\sexpr_p}{\senv} \\
 %
  \symbeval{\jsilexpr_3}{\store}  = \symbeval{\jsilexpr_3}{\semexpr{\sstore}{\senv}} =  \symbeval{\symbeval{\jsilexpr_3}{\sstore}}{\senv}
   =  \symbeval{\sexpr_v}{\senv} \\
 %
 \heap = \semexpr{\sheap'' \, \uplus \, \big((l, \sexprp_i) \mapsto \sexprv_i\big)\mid_{i = 0}^n}{\senv} \\
    \qquad = \semexpr{\sheap''}{\senv} \dunion \biguplus_{0 \leq i \leq n} ((l, \symbeval{\sexprp_i}{\senv}) \mapsto \symbeval{\sexprv_i}{\senv})
\end{array}
$$
From  $(\loc, -) \not\in \domain(\sheap'')$, we conclude that $(\loc, -) \not\in \domain(\semexpr{\sheap''}{\senv})$. 
Since $\senv \vdash \pc'$, we additionally conclude that: 
$
  \forall_{0 \leq i \leq n}  \, \symbeval{\sexprp_i}{\senv} \neq \symbeval{\sexpr_p}{\senv} 
$
Recalling that $\symbeval{\jsilexpr_2}{\store} = \symbeval{\sexpr_p}{\senv}$, we conclude that  
$
  \forall_{0 \leq i \leq n}  \, \symbeval{\sexprp_i}{\senv} \neq \symbeval{\jsilexpr_2}{\store}
$, from which it follows (together with $(\loc, -) \not\in \domain(\semexpr{\sheap''}{\senv})$) that 
$(\symbeval{\jsilexpr_1}{\store}, \symbeval{\jsilexpr_2}{\store}) \not\in \domain(\heap)$.
We can now apply the \prooflab{Property Assignment - Not Found} rule of \jsil semantics, concluding: 
$$
   \semtrans{\heap, \store, [\jsilexpr_1, \jsilexpr_2] := \jsilexpr_3}
     {\heap \dunion (\symbeval{\jsilexpr_1}{\store}, \symbeval{\jsilexpr_2}{\store}) \mapsto \symbeval{\jsilexpr_3}{\store},  \store}
$$
meaning that $\heap' = \heap \dunion (\symbeval{\jsilexpr_1}{\store}, \symbeval{\jsilexpr_2}{\store}) \mapsto \symbeval{\jsilexpr_3}{\store}$ 
and $\store' = \store$. 
%
We have now to prove that $(\heap', \store') \in \smodels{\sheap', \sstore'}{\pc'}$.
Noting that:
$$
\begin{array}{l}
\heap' = \symbeval{\sheap}{\senv} \dunion (\symbeval{\jsilexpr_1}{\store}, \symbeval{\jsilexpr_2}{\store}) \mapsto \symbeval{\jsilexpr_3}{\store} \\ 
  \qquad = \symbeval{\sheap}{\senv} \dunion (\loc, \symbeval{\sexpr_p}{\senv}) \mapsto \symbeval{\sexpr_v}{\senv}  \\
    \qquad = \symbeval{\sheap \dunion (\loc, \sexpr_p) \mapsto \sexpr_v}{\senv}  \\
    \qquad = \symbeval{\sheap'}{\senv} \\[2pt]
 %
 \store' = \store = \symbeval{\sstore}{\senv} = \symbeval{\sstore'}{\senv} 
\end{array}
$$
the result follows. 
\vspace{6pt}



\noindent\prooflab{Member Check - True}
We conclude that  $\bcmd = \jvar := \hasfield(\jsilexpr_1, \jsilexpr_2)$ for some variable $\jvar$ and expressions $\jsilexpr_1$ and $\jsilexpr_2$, and that: 
$$
\begin{array}{l}
  \sheap =   \sheap'' \, \uplus \, \big((l, \sexprp_i) \mapsto -\big)\mid_{i = 0}^n     \\
  %
  \sheap' =  \sheap \\
  %
  \sstore' = \sstore[\jvar \mapsto \jtrue] \\ 
  %
    \pc' = \pc \ \wedge \, \big( (\sexprp_k = \sexpr_p) \ \wedge \bigwedge_{i = 0, i \neq k}^n (\sexprp_i \neq \sexpr_p) \big)
\end{array}
$$ 
where $\symbeval{\jsilexpr_1}{\sstore} =  \loc$, $\symbeval{\jsilexpr_2}{\sstore} =  \sexpr_p$, 
$(\loc, -) \not\in \domain(\sheap'')$, and $0 \leq k \leq n$. 
%
From $(\heap, \store) \in \smodels{\sheap, \sstore}{\pc'}$, we conclude that there is a symbolic environment
$\senv$ such that $\heap = \semexpr{\sheap}{\senv}$, $\store = \semexpr{\sstore}{\senv}$, and 
$\senv \vdash \pc'$. 
We have now to prove that we can apply the \prooflab{Member Check - True} rule in the concrete state.
To this end, we have to show that:
$\heap = \heap'' \dunion (\symbeval{\jsilexpr_1}{\store}, \symbeval{\jsilexpr_2}{\store}) \mapsto -$, for 
some concrete heap $\heap''$. 
Note that: 
$$
\begin{array}{l}
%
 \symbeval{\jsilexpr_1}{\store} = \symbeval{\jsilexpr_1}{\symbeval{\sstore}{\senv}} = \symbeval{\symbeval{\jsilexpr_1}{\sstore}}{\senv} 
    = \symbeval{\loc}{\senv} = \loc \\ 
 %
  \symbeval{\jsilexpr_2}{\store}  = \symbeval{\jsilexpr_2}{\semexpr{\sstore}{\senv}} =  \symbeval{\symbeval{\jsilexpr_2}{\sstore}}{\senv}
   =  \symbeval{\sexpr_p}{\senv} \\
 %
 \heap = \semexpr{\sheap'' \, \uplus \, \big((l, \sexprp_i) \mapsto \sexprv_i\big)\mid_{i = 0}^n}{\senv} \\
    \qquad = \semexpr{\sheap'' \, \uplus \, \big((l, \sexprp_i) \mapsto \sexprv_i\big)\mid_{i = 0, i\neq k}^n \dunion (l, \sexprp_k) \mapsto \sexprv_k}{\senv} \\
    \qquad = \semexpr{\sheap'' \, \uplus \, \big((l, \sexprp_i) \mapsto \sexprv_i\big)\mid_{i = 0, i\neq k}^n}{\senv} \dunion \semexpr{(l, \sexprp_k) \mapsto \sexprv_k}{\senv} \\
    \qquad = \semexpr{\sheap'' \, \uplus \, \big((l, \sexprp_i) \mapsto \sexprv_i\big)\mid_{i = 0, i\neq k}^n}{\senv} \dunion (l, \semexpr{\sexprp_k}{\senv}) \mapsto \semexpr{\sexprv_k}{\senv} \\ 
     \qquad = \semexpr{\sheap'' \, \uplus \, \big((l, \sexprp_i) \mapsto \sexprv_i\big)\mid_{i = 0, i\neq k}^n}{\senv} \dunion (l, \semexpr{\sexpr_p}{\senv}) \mapsto \semexpr{\sexprv_k}{\senv}
      			\text{ (using $\senv \vdash \pc$)} \\ 
     \qquad = \semexpr{\sheap'' \, \uplus \, \big((l, \sexprp_i) \mapsto \sexprv_i\big)\mid_{i = 0, i\neq k}^n}{\senv} \dunion (\symbeval{\jsilexpr_1}{\store}, \symbeval{\jsilexpr_2}{\store}) \mapsto - \\
\end{array}
$$
We can now apply the \prooflab{Member Check - True} rule of \jsil semantics, concluding: 
$$
   \semtrans{\heap, \store, \jvar := \hasfield(\jsilexpr_1, \jsilexpr_2)}{\heap,  \store[\jvar \mapsto \jtrue]}
$$
meaning that $\heap' = \heap$ and $\store' = \store[\jvar \mapsto \jtrue]$. 
%
We have now to prove that $(\heap', \store') \in \smodels{\sheap', \sstore'}{\pc'}$.
Noting that:
$$
\begin{array}{l}
\heap' = \heap = \semexpr{\sheap}{\senv} = \semexpr{\sheap'}{\senv} \\
 %
 \store' = \store[\jvar \mapsto \jtrue] = \symbeval{\sstore}{\senv}[\jvar \mapsto \jtrue]  = \symbeval{\sstore[\jvar \mapsto \jtrue]}{\senv} = \symbeval{\sstore'}{\senv} 
\end{array}
$$
the result follows. 
\vspace{6pt}


\noindent\prooflab{Member Check - False}
We conclude that  $\bcmd = \jvar := \hasfield(\jsilexpr_1, \jsilexpr_2)$ for some variable $\jvar$ and expressions $\jsilexpr_1$ and $\jsilexpr_2$, and that: 
$$
\begin{array}{l}
  \sheap =  \sheap'' \, \uplus \, \big((l, \sexprp_i) \mapsto -\big)\mid_{i = 0}^n      \\
  %
  \sheap' =  \sheap \\
  %
  \sstore' = \sstore[\jvar \mapsto \jfalse] \\ 
  %
     \pc' = \pc \ \wedge \,  \bigwedge_{i = 0}^n (\sexprp_i \neq \sexpr_p) 
\end{array}
$$ 
where $\symbeval{\jsilexpr_1}{\sstore} =  \loc$, $\symbeval{\jsilexpr_2}{\sstore} =  \sexpr_p$, 
$(\loc, -) \not\in \domain(\sheap'')$, and $0 \leq k \leq n$. 
%
From $(\heap, \store) \in \smodels{\sheap, \sstore}{\pc'}$, we conclude that there is a symbolic environment
$\senv$ such that $\heap = \semexpr{\sheap}{\senv}$, $\store = \semexpr{\sstore}{\senv}$, and 
$\senv \vdash \pc'$. 
We have now to prove that we can apply the \prooflab{Member Check - False} rule in the concrete state.
To this end, we have to show that: $(\symbeval{\jsilexpr_1}{\store}, \symbeval{\jsilexpr_2}{\store}) \not\in \domain(\heap)$. 
Note that: 
$$
\begin{array}{l}
%
 \symbeval{\jsilexpr_1}{\store} = \symbeval{\jsilexpr_1}{\symbeval{\sstore}{\senv}} = \symbeval{\symbeval{\jsilexpr_1}{\sstore}}{\senv} 
    = \symbeval{\loc}{\senv} = \loc \\ 
 %
  \symbeval{\jsilexpr_2}{\store}  = \symbeval{\jsilexpr_2}{\semexpr{\sstore}{\senv}} =  \symbeval{\symbeval{\jsilexpr_2}{\sstore}}{\senv}
   =  \symbeval{\sexpr_p}{\senv} \\
 %
 \heap = \semexpr{\sheap'' \, \uplus \, \big((l, \sexprp_i) \mapsto \sexprv_i\big)\mid_{i = 0}^n}{\senv} \\
    \qquad = \semexpr{\sheap''}{\senv} \dunion \biguplus_{0 \leq i \leq n} (l, \semexpr{\sexprp_i}{\senv}) \mapsto \semexpr{\sexprv_i}{\senv}
\end{array}
$$
Since $(\loc, -) \not\in \domain(\sheap'')$, we conclude that $(\loc, -) \not\in \semexpr{\sheap''}{\senv}$. 
%
Since $\senv \vdash \pc'$, we additionally conclude that: 
$
  \forall_{0 \leq i \leq n}  \, \symbeval{\sexprp_i}{\senv} \neq \symbeval{\sexpr_p}{\senv} 
$
Recalling that $\symbeval{\jsilexpr_2}{\store} = \symbeval{\sexpr_p}{\senv}$, we conclude that  
$
  \forall_{0 \leq i \leq n}  \, \symbeval{\sexprp_i}{\senv} \neq \symbeval{\jsilexpr_2}{\store}
$, from which it follows (together with $(\loc, -) \not\in \domain(\semexpr{\sheap''}{\senv})$) that 
$(\symbeval{\jsilexpr_1}{\store}, \symbeval{\jsilexpr_2}{\store}) \not\in \domain(\heap)$.
%
We can now apply the \prooflab{Member Check - False} rule of \jsil semantics, concluding: 
$$
   \semtrans{\heap, \store, \jvar := \hasfield(\jsilexpr_1, \jsilexpr_2)}{\heap,  \store[\jvar \mapsto \jfalse]}
$$
meaning that $\heap' = \heap$ and $\store' = \store[\jvar \mapsto \jfalse]$. 
%
Now we have to prove that $(\heap', \store') \in \smodels{\sheap', \sstore'}{\pc'}$.
Noting that:
$$
\begin{array}{l}
\heap' = \heap = \semexpr{\sheap}{\senv} = \semexpr{\sheap'}{\senv} \\
 %
 \store' = \store[\jvar \mapsto \jfalse] = \symbeval{\sstore}{\senv}[\jvar \mapsto \jfalse]] = \symbeval{\sstore[\jvar \mapsto \jfalse]}{\senv} = \symbeval{\sstore'}{\senv} 
\end{array}
$$
the result follows. 
\vspace{6pt}


\noindent\prooflab{Assert - True}
We conclude that  $\bcmd = \assert(\jsilexpr)$ for some expression $\jsilexpr$, and that: 
$$
  \sheap' = \sheap 
  \quad
  \sstore' =  \sstore 
  \quad
  \pc' = \pc
  \quad
  \pc \vdash  \symbeval{\jsilexpr}{\sstore}
$$ 
From $(\heap, \store) \in \smodels{\sheap, \sstore}{\pc'}$, we conclude that there is a symbolic environment
$\senv$ such that $\heap = \semexpr{\sheap}{\senv}$, $\store = \semexpr{\sstore}{\senv}$, and 
$\senv \vdash \pc$. 
We have now to prove that we can apply the \prooflab{Assert - True} rule in the concrete state.
To this end, we have to show that: $\symbeval{\jsilexpr}{\store} = \jtrue$. 
Noting that:
$
  \symbeval{\jsilexpr}{\store} = \symbeval{\jsilexpr}{\symbeval{\sstore}{\senv}} 
         = \symbeval{\symbeval{\jsilexpr}{\sstore}}{\senv} 
$, we conclude (using $\senv \vdash \pc$ and $\pc \vdash  \symbeval{\jsilexpr}{\sstore}$) that 
$\symbeval{\jsilexpr}{\store} = \jtrue$. 
We can now apply the \prooflab{Assert - True} rule of \jsil semantics, concluding: 
$$
   \semtrans{\heap, \store, \assert(\jsilexpr)}{\heap,  \store}
$$
meaning that $\heap' = \heap$ and $\store' = \store$. 
%
Now we have to prove that $(\heap', \store') \in \smodels{\sheap', \sstore'}{\pc'}$.
Noting that:
$$
\begin{array}{l}
\heap' = \heap = \semexpr{\sheap}{\senv} = \semexpr{\sheap'}{\senv}
 %
 \quad 
 %
 \store' = \store = \symbeval{\sstore}{\senv} = \symbeval{\sstore}{\senv} = \symbeval{\sstore'}{\senv} 
\end{array}
$$
the result follows. 
\vspace{6pt}

\noindent\prooflab{Assert - False}
We conclude that  $\bcmd = \assert(\jsilexpr)$ for some expression $\jsilexpr$, and that: 
$\pc \not\vdash  \symbeval{\jsilexpr}{\sstore}$. 
From $(\heap, \store) \in \smodels{\sheap, \sstore}{\pc'}$, we conclude that there is a symbolic environment
$\senv$ such that $\heap = \semexpr{\sheap}{\senv}$, $\store = \semexpr{\sstore}{\senv}$, and 
$\senv \vdash \pc$. 
We have now to prove that we can apply the \prooflab{Assert - False} rule in the concrete state.
To this end, we have to show that: $\symbeval{\jsilexpr}{\store} = \jfalse$. 
Noting that:
$
  \symbeval{\jsilexpr}{\store} = \symbeval{\jsilexpr}{\symbeval{\sstore}{\senv}} 
         = \symbeval{\symbeval{\jsilexpr}{\sstore}}{\senv} 
$, we conclude (using $\senv \vdash \pc$ and $\pc \not\vdash  \symbeval{\jsilexpr}{\sstore}$) that 
$\symbeval{\jsilexpr}{\store} = \jfalse$, from which the result follows. 
\end{proof}


\begin{temax}[Theorem~\ref{teo:soundness:jsil:symb:exe} - Soundess of \jsil symbolic execution]
$$
\begin{array}{l}
\symbtrans{\sheap, \sstore, \sctx[i], \pc}{\sheap', \sstore', \sctx'[j], \pc'} 
   \ \wedge \ 
      (\heap, \store, \ctx) \in \smodels{\sheap, \sstore, \sctx}{\pc'} \\ \quad \quad
      	 \ \implies \ \exists (\heap', \store', \ctx') \, . \, 
	 	 \semtrans{\heap, \store, \ctx[i]}{\heap', \store', \ctx'[j]}
		\, \wedge \, 
		(\heap', \store', \ctx') \in \smodels{\sheap', \sstore', \sctx'}{\pc'}  
\end{array}
$$
\end{temax}
%
\begin{proof}
We proceed by case analysis on $\symbtrans{\sheap, \sstore, \sctx[i], \pc}{\sheap', \sstore', \sctx'[j], \pc'}$. 

\end{proof}



 

\end{document}