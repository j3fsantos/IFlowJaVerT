%!TEX root = ../main.tex

\begin{itemize}
  \item 2.0  - intro - explain the basic ideas of Jilette (use a diagram) - 
 
  \item 2.1 - describe the jsil language. give its formal syntax (extended with assert and solve). 
                   formally define symbolic execution for JSIL commands. soundness lemma. 
 
  \item 2.2 Implementation:  
               - encoding \jsil heaps in Rosette 
               - explain the \jsil interpreter implemented in Rosette and its connection to the \jsil semantics (as defined in appendix) 
               - give snippets of the interpreter 
               - discuss soundness, trust, and other issues
 
  \item 2.3 Symbolic execution for JavaScript 
              - explain that we have to extend the syntax of JavaScript with asserts  as well as constructs for creating symbolic values
              - give the example 
              - discuss challenges: abstraction level of the generated code needs to match the abstraction level of Rosette 
\end{itemize}

\subsection{Formalisation}

\myparagraph{\jsil: Formal Semantics}
The basic memory model of \jsil is straightforward. 
\jsil values contain: numbers, $\jnumber$; booleans, $\jbool$; strings, $\jstring$;  the special values \jsinline|undefined| and \jsinline|null|; and object locations,  $\loc \in \locs$.
A \jsil heap, $\heap \in \heaps$, is a partial function mapping pairs of  object locations, and strings to heap values. 
 Given a heap $\heap$, we denote a heap cell by $\hcell{\loc}{\jstring}{\val}$ when $h(\loc,\jstring) = \val$, the union of two disjoint heaps by $\oheap_1 \dunion \oheap_2$,  a heap lookup by $\hread{\oheap}{\loc}{\jstring}$, and the empty heap by $\hemp$.
 Finally, a \jsil variable store, $\store \in \stores$, is a mapping from JSIL program variables $\jvar \in \jvars$ to JSIL values.

We introduce the \jsil semantic judgement for program behaviour; %needed to state our soundness result. 
the full \jsil semantics is given in the Appendix. 
The semantics of \jsil is defined in a small-step style. 
Transitions have the form:  $\semtrans{\heap, \store, \ctx[i]}{\heap', \store', \ctx'[j]}$, 
meaning that the evaluation of the $i$-th command of the current procedure in 
the heap $\heap$ and store $\store$ generates the heap $\heap'$ and store $\store'$ 
and $j$ is the next command to be evaluated. 
A context $\ctx$ is  a list of tuples of form $(\pid, \sstore, \jvar, i, j)$, where: 
\dtag{1} $\pid$ is the identifier of a procedure whose code is currently being executed, 
\dtag{2} $\sstore$ is the store of the procedure that called $\pid$, \dtag{3} $\jvar$ is 
the variable to which the return of $\pid$ must be assigned, \dtag{4} $i$ is the index 
of the command to which the control must jump after the execution of $\pid$ in 
case of normal return, and \dtag{5} $j$ the index to which it must jump in case of 
error return. 


\myparagraph{\jsil: Symbolic Evaluation}
In order to symbolically execute \jsil programs, we extend the syntax of \jsil expressions with 
symbolic strings $\sstring \in \sstrings$ and symbolic numbers $\snumber \in \snumbers$. 
For convenience, we use $\svars$ to denote the union of $\sstrings$ and $\snumbers$ 
and $\svar$ to range over $\svars$. 
A symbolic expression $\sexpr$ is a \jsil expression with no program variables. 
\begin{equation}
\jsilexpr \triangleq \lit \mid \jvar \mid \sstring \mid \snumber \mid \unoper\ \jsilexpr \mid \jsilexpr \binoper \jsilexpr
\qquad 
\sexpr \triangleq \lit \mid \sstring \mid \snumber \mid \unoper\ \sexpr \mid \sexpr \binoper \sexpr
\end{equation}

We extend heaps, stores, and contexts with symbolic values, obtaining symbolic 
heaps, stores, and contexts, respectively ranged by $\sheap$, $\sstore$, and $\sctx$. 
A symbolic heap, $\sheap \in \sheaps$, is a partial function mapping pairs of  
object locations, and symbolic expressions to symbolic expressions. 
A symbolic store, $\sstore \in \sstores$, is a mapping from program variables 
$\jvar \in \jvars$ to symbolic expressions.
%
A \emph{symbolic state} $\sstate = (\sheap, \sstore, \pc)$ is a triple consisting of a 
symbolic heap $\sheap$, a symbolic store $\sstore$, and a path condition $\pc$. 
The path condition is a first order quantifier free formula over symbolic strings and 
numbers, which accumulates constraints on the given symbolic inputs that trigger 
the execution to follow the path that led to the current symbolic state. 
Path conditions are given by the following grammar: 
\begin{equation}
\pc \triangleq \sexpr_1 = \sexpr_2 \mid \sexpr_1 \leq \sexpr_2 \mid \pc_1 \, \wedge \, \pc_2 \mid \pc_1 \vee \pc_2 \mid \neg \pc \mid \ltrue \mid \lfalse
\end{equation}

Figure~\ref{fig:symbexe:bcmds} presents the symbolic execution rules for \jsil basic commands. 
Rules have the form $\symbtrans{\sheap, \sstore, \bcmd, \pc}{\sheap', \sstore', \pc'}$, 
where: \dtag{1} $\sheap$ and $\sstore$ are the symbolic heap and store on which to evaluate $\bcmd$, 
\dtag{2} $\pc$ the current \emph{path condition}, and \dtag{3} $\sheap'$, $\sstore'$, and $\pc'$
the resulting symbolic heap, store, and path condition. 
Figure~\ref{fig:symbexe:cmds} presents the symbolic execution rules for \jsil commands. 
Rules have the form $\symbtrans{\sheap, \sstore, \sctx[i], \pc}{\sheap', \sstore', \sctx'[j], \pc'}$; 
they differ from the rules for basic commands in that: \dtag{i} instead of the basic command to 
be executed on the left, they have the index of the command to be executed paired up 
with its associated execution context, and, \dtag{ii} on the right, they additionally have the 
index of the next command to be executed paired with the resulting execution context. 

%\begin{display}{}
\begin{figure}[ht!]
{\scriptsize
\begin{mathpar} 
%
\inferrule[\textsc{Skip}]{}
	{ \symbtrans{\sheap, \sstore, \jsilskip, \pc}{\sheap, \sstore, \pc}} 
 \and
 %
\inferrule[\textsc{Assignment}]
  {
      \symbeval{\jsilexpr}{\sstore} =  \sexpr
      \quad
      \sstore' = \sstore[\jvar \mapsto \sexpr]
  }{\symbtrans{\sheap, \sstore, \jvar := \jsilexpr, \pc}{\sheap, \sstore', \pc}} 
%
\and 
%
\inferrule[\textsc{Object Creation}]
  { 
    \sheap' = \sheap \dunion \hcell{\loc}{\protop}{\jsnull}
    \and (\loc,-) \notin \domain (\sheap)
  }{\symbtrans{\sheap, \sstore, \jvar := \jsilnew(), \pc}{\sheap', \sstore[\jvar \mapsto \loc], \pc}}
\\
%
\inferrule[\textsc{Property Access}]
  { 
 	\symbeval{\jsilexpr_1}{\sstore} =  \loc
  	\quad 
        \symbeval{\jsilexpr_2}{\sstore} =  \sexpr_p
        \quad
        \sheap = \sheap' \, \uplus \, \big((l, \sexprp_i) \mapsto \sexprv_i\big)\mid_{i = 0}^n   
        \quad
        (l, -) \not\in \domain(\sheap')
        \quad 
        0 \leq k \leq n
        \\\\
        \pc' = \pc \ \wedge \, \big( (\sexprp_k = \sexpr_p) \ \wedge \bigwedge_{i = 0, i \neq k}^n (\sexprp_i \neq \sexpr_p) \big)
  }{ \symbtrans{\sheap, \sstore, \jvar := [\jsilexpr_1, \jsilexpr_2], \pc}{\sheap,  \sstore[\jvar \mapsto \sexprv_k], \pc'}}
 %
\\
%
\inferrule[\textsc{Property Assignment - Found}]
  {     \symbeval{\jsilexpr_1}{\sstore} =  \loc
  	\quad 
        \symbeval{\jsilexpr_2}{\sstore} =  \sexpr_p
        \quad
        \symbeval{\jsilexpr_3}{\sstore} =  \sexpr_v
       \quad 
        \sheap = \sheap' \, \uplus \, \big((l, \sexprp_i) \mapsto \sexprv_i\big)\mid_{i = 0}^n   
        \quad
        (l, -) \not\in \domain(\sheap')
        \quad 
        0 \leq k \leq n
        \\
          \pc' = \pc \ \wedge \, \big( (\sexprp_k = \sexpr_p) \ \wedge \bigwedge_{i = 0, i \neq k}^n (\sexprp_i \neq \sexpr_p) \big)
         \quad
         \sheap'' = \sheap' \, \uplus \,  \big((l, \sexprp_i) \mapsto \sexprv_i\big)\mid_{i = 0, i \neq k}^n \, \uplus \,  (l, \sexpr_p) \mapsto \sexpr_v
  }{\symbtrans{\sheap, \sstore,  [\jsilexpr_1, \jsilexpr_2] := \jsilexpr_3, \pc}{\sheap'', \sstore, \pc'}} 
\\
%
\inferrule[\textsc{Property Assignment - Not Found}]
  {     \symbeval{\jsilexpr_1}{\sstore} =  \loc
  	\quad 
        \symbeval{\jsilexpr_2}{\sstore} =  \sexpr_p
        \quad
        \symbeval{\jsilexpr_3}{\sstore} =  \sexpr_v
       \quad 
        \sheap = \sheap' \, \uplus \, \big((l, \sexprp_i) \mapsto \sexprv_i\big)\mid_{i = 0}^n   
        \quad
        (l, -) \not\in \domain(\sheap')
        \quad 
        0 \leq k \leq n
        \\
          \pc' = \pc \ \wedge \, \bigwedge_{i = 0}^n (\sexprp_i \neq \sexpr_p)
         \quad
         \sheap' = \sheap \, \uplus \,  (l, \sexpr_p) \mapsto \sexpr_v
  }{\symbtrans{\sheap, \sstore, [\jsilexpr_1, \jsilexpr_2] := \jsilexpr_3, \pc}{\sheap', \sstore, \pc'}}   
%
\\
%
\inferrule[\textsc{Property Deletion}]
  { 
        \symbeval{\jsilexpr_1}{\sstore} =  \loc
  	\quad 
        \symbeval{\jsilexpr_2}{\sstore} =  \sexpr_p
       \quad 
        \sheap = \sheap' \, \uplus \, \big((l, \sexprp_i) \mapsto -\big)\mid_{i = 0}^n   
        \quad
        (l, -) \not\in \domain(\sheap')
        \quad 
        0 \leq k \leq n
     \\ 
      \pc' = \pc \ \wedge \, \big( (\sexprp_k = \sexpr_p) \ \wedge \bigwedge_{i = 0, i \neq k}^n (\sexprp_i \neq \sexpr_p) \big)
     \quad 
      \sheap'' = \sheap' \, \uplus \,  \big((l, \sexprp_i) \mapsto \sexprv_i\big)\mid_{i = 0, i \neq k}^n
   }{\symbtrans{\sheap, \sstore, \jsildelete(\jsilexpr_1, \jsilexpr_2), \pc}{\sheap'', \sstore, \pc'}}
 \\
 %
\inferrule[\textsc{Member Check - True}]
  { 
      \symbeval{\jsilexpr_1}{\sstore} =  \loc
  	\quad 
        \symbeval{\jsilexpr_2}{\sstore} =  \sexpr_p
       \quad 
        \sheap = \sheap' \, \uplus \, \big((l, \sexprp_i) \mapsto -\big)\mid_{i = 0}^n   
        \quad
        (l, -) \not\in \domain(\sheap')
        \quad 
        0 \leq k \leq n
     \\ 
     \pc' = \pc \ \wedge \, \big( (\sexprp_k = \sexpr_p) \ \wedge \bigwedge_{i = 0, i \neq k}^n (\sexprp_i \neq \sexpr_p) \big)
  }{\symbtrans{\sheap, \sstore, \jvar := \hasfield(\jsilexpr_1, \jsilexpr_2), \pc}{\sheap, \sstore[\jvar \mapsto \jtrue], \pc'}}
%
\\
%
\inferrule[\textsc{Member Check - False}]
  { 
      \symbeval{\jsilexpr_1}{\sstore} =  \loc
  	\quad 
        \symbeval{\jsilexpr_2}{\sstore} =  \sexpr_p
       \quad 
        \sheap = \sheap' \, \uplus \, \big((l, \sexprp_i) \mapsto -\big)\mid_{i = 0}^n   
        \quad
        (l, -) \not\in \domain(\sheap')
        \quad 
        0 \leq k \leq n
     \\ 
     \pc' = \pc \ \wedge \,  \bigwedge_{i = 0}^n (\sexprp_i \neq \sexpr_p) \big)
  }{\symbtrans{\sheap, \sstore, \jvar := \hasfield(\jsilexpr_1, \jsilexpr_2), \pc}{\sheap, \sstore[\jvar \mapsto \jfalse], \pc'}}
\\
%
\inferrule[\textsc{Assert - True}]
  { 
      \symbeval{\jsilexpr}{\sstore} =  \sexpr
     \quad 
     \pc \vdash \sexpr 
  }{\symbtrans{\sheap, \sstore, \assert(\jsilexpr), \pc}{\sheap, \sstore, \pc}} 
\quad
\inferrule[\textsc{Assert - False}]
  { 
      \symbeval{\jsilexpr}{\sstore} =  \sexpr
     \quad 
     \pc \not\vdash \sexpr 
  }{\symbtranserr{\sheap, \sstore, \assert(\jsilexpr), \pc}} 
\end{mathpar}}
\caption{Symbolic Execution for Basic Commands: {\scriptsize$\symbtrans{\sheap, \sstore, \bcmd, \pc}{\sheap', \sstore', \pc'}$}\label{fig:symbexe:bcmds}}
\end{figure}
%\end{display}  


\begin{figure}[ht]
{\scriptsize
\begin{mathpar} 
\inferrule[\textsc{Basic Command}]
   { 
     \prog_{\pid}(i) = \bcmd 
     \quad
     \symbtrans{\sheap, \sstore, \bcmd, \pc}{\sheap', \sstore', \pc'} 
   }{\symbtrans{\sheap, \sstore, \sctx[i], \pc}{\sheap', \sstore', \sctx[i+1], \pc'}}
%
   \qquad
  %
  \inferrule[\textsc{Basic Command - Fail}]
   { 
     \prog_{\pid}(i) = \bcmd 
     \quad
     \symbtranserr{\sheap, \sstore, \bcmd, \pc} 
   }{\symbtranserr{\sheap, \sstore, \sctx[i], \pc}}
 %
   \qquad
  %
  \inferrule[\textsc{Goto}]
   { \prog_{\pid}(i) = \goto \, j \quad}
   {\symbtrans{\sheap, \sstore, \sctx[i], \pc}{\sheap, \sstore, \sctx[j], \pc}}
  \\ 
  \inferrule[\textsc{Cond. Goto - True}]
   { \prog_{\pid}(i) =  \ifgoto{\jsilexpr}{j}{k} \quad
     \symbeval{\jsilexpr}{\sstore} =  \sexpr
   }
   {\symbtrans{\sheap, \sstore, \sctx[i], \pc}{\sheap, \sstore, \sctx[j],  \pc \, \wedge \, \sexpr}}
  \and 
    \inferrule[\textsc{Cond. Goto - False}]
   { \prog_{\pid}(i) =  \ifgoto{\jsilexpr}{j}{k} \quad
     \symbeval{\jsilexpr}{\sstore} =  \sexpr
   }
   {\symbtrans{\sheap, \sstore, \sctx[i], \pc}{\sheap, \sstore, \sctx[k], \pc \, \wedge \, \neg\sexpr}}
   \\
    \inferrule[\textsc{Procedure Call}]
   { 
    \prog_{\pid}(i) =   \jsilcall{\jvar}{\jsilexpr}{\jsilexpr_i \mid_{i = 0}^{n}}{j}
     \quad
    \symbeval{\jsilexpr}{\sstore} =  \pid' 
    \quad
      \symbeval{\jsilexpr_i}{\sstore} =  \sexpr_i \mid_{i = 0}^{n} 
     \quad
     \args(\pid') = \jsillist{\jvar_1, ..., \jvar_{m}} 
     \quad 
      \sexpr_i = \jsundefined \mid_{i = n+1}^{m}  
   }
   {\symbtrans{\sheap, \sstore, \sctx[i], \pc}{\sheap, [ \jvar_i \mapsto \sexpr_i \mid_{i = 0}^{m}], ((\pid', \sstore, \jvar, i+1, j)::\sctx)[0], \pc}}
    \\ 
  \inferrule[\textsc{Normal Return}]
   {
       \sctx = (-, \sstore', \jvar, i, -) :: \sctx' 
       \quad 
       \sstore(\procretvar) = \sexpr
   }  
   {\symbtrans{\sheap, \sstore, \sctx[\procretlab], \pc}{\sheap, \sstore'[\jvar \mapsto \sexpr], \sctx'[i], \pc}}
   \and 
     \inferrule[\textsc{Error Return}]
   {
       \sctx = (-, \sstore', \jvar, -, j) :: \sctx' 
       \quad 
       \sstore(\procerrvar) = \sexpr
   }  
   {\symbtrans{\sheap, \sstore, \sctx[\procerrlab], \pc}{\sheap, \sstore'[\jvar \mapsto \sexpr], \sctx'[j], \pc}}
 \end{mathpar}}
\caption{Symbolic Execution for Control Flow Commands: {\scriptsize$\symbtrans{\sheap, \sstore, \sctx[i], \pc}{\sheap', \sstore', \sctx'[j], \pc'}$}\label{fig:symbexe:cmds}}
\end{figure}

\begin{figure}[ht!]
{
\begin{tabular}{l}
$\quad${\bf Symbolic Expressions:}  \\
$
\quad
\semexpr{\lit}{\senv} \semeq \lit
\quad 
\semexpr{\svar}{\senv} \semeq \senv(\svar)
\quad 
\semexpr{\unoper\ \sexpr}{\senv} \semeq \unoper (\semexpr{\sexpr}{\senv})
\quad 
\semexpr{\sexpr_1 \binoper \sexpr_2}{\senv} \semeq \binoper(\semexpr{\sexpr_1}{\senv}, \semexpr{\sexpr_2}{\senv}) 
$
\\[3pt]
$\quad${\bf Symbolic Heaps:}  \\
$
\quad
 \semexpr{\hemp}{\senv} \semeq \hemp
\quad
\semexpr{\hcell{\loc}{\sexpr_p}{\sexpr_v}}{\senv} \semeq  \hcell{\loc}{\semexpr{\sexpr_p}{\senv}}{\semexpr{\sexpr_v}{\senv}}
\quad
\semexpr{\sheap_1 \dunion \sheap_2}{\senv} \semeq  \semexpr{\sheap_1}{\senv} \dunion \semexpr{\sheap_2}{\senv}
$%
%%
%%
\\[3pt]
$\quad${\bf Symbolic Stores:}  \\
$
\quad
 \semexpr{\storeemp}{\senv} \semeq \storeemp
\quad 
 \semexpr{(\jvar: \sexpr) \dunion \sstore}{\senv} \semeq (\jvar: \semexpr{\sheap_1}{\senv}) \dunion \semexpr{\sstore}{\senv}
$%
\\[3pt]
$\quad${\bf Symbolic Contexts:}  \\
$
\quad
 \semexpr{\lstemp}{\senv} \semeq \lstemp
\quad 
 \semexpr{(\pid, \sstore, \jvar, i, j) \lstcons \sctx}{\senv} \semeq (\pid, \semexpr{\sstore}{\senv}, \jvar, i, j) \lstcons \semexpr{\sctx}{\senv}
$%

\\[3pt]
$\quad${\bf Symbolic States:}  $\semexpr{(\sheap, \sstore, \sctx)}{\senv} \semeq (\semexpr{\sheap}{\senv}, \semexpr{\sstore}{\senv}, \semexpr{\sctx}{\senv})$
\end{tabular}
}
\caption{Interpretation of symbolic expressions, heaps, stores, and contexts.\label{fig:symbolic:interp}}
\end{figure}

\myparagraph{Soundness} To establish the soundness of symbolic execution we need to relate 
symbolic states to concrete states. To this end, we make use of \emph{symbolic environments} 
$\senv : \svars \rightharpoonup \lits$ mapping symbolic values to \jsil literals. 
A symbolic environment is said to be \emph{consistent} if it maps symbolic 
values to concrete values of the appropriate type (e.g. symbolic strings are mapped to strings 
and symbolic numbers are mapped to numbers). In the following, we will always 
assume consistent symbolic environments. 
%
Given a symbolic environment $\senv$, we define the interpretation of a symbolic 
expressions, heaps, stores, and contexts as shown in Figure~\ref{fig:symbolic:interp}. 

In the following, we write $\senv \vdash \pc$  if and only if $\semexpr{\pc}{\senv} = \ltrue$. 
For convenience, we define: 
\begin{align}
\smodels{\sheap, \sstore}{\pc} = \left\{ (\heap, \store) \mid \exists \senv \, . \,  \semexpr{(\sheap, \sstore)}{\senv} = (\heap, \store) \, \wedge \,  \senv \vdash \pc  \right\}  
\\
\smodels{\sheap, \sstore, \sctx}{\pc} = \left\{ (\heap, \store, \ctx) \mid \exists \senv \, . \,  \semexpr{(\sheap, \sstore, \sctx)}{\senv} = (\heap, \store, \ctx) \, \wedge \,  \senv \vdash \pc  \right\} 
\end{align}

\begin{theorem}[Soundess of \jsil symbolic execution]\label{teo:soundness:jsil:symb:exe}
$$
\begin{array}{l}
\symbtrans{\sheap, \sstore, \sctx[i], \pc}{\sheap', \sstore', \sctx'[j], \pc'} 
   \ \wedge \ 
      (\heap, \store, \ctx) \in \smodels{\sheap, \sstore, \sctx}{\pc'} \\ \quad \quad
      	 \ \Rightarrow \ \exists (\heap', \store', \ctx') \, . \, 
	 	 \semtrans{\heap, \store, \ctx[i]}{\heap', \store', \ctx'[j]}
		\, \wedge \, 
		(\heap', \store', \ctx') \in \smodels{\sheap', \sstore', \sctx'}{\pc'}  
\end{array}
$$
\end{theorem}



\subsection{Implementation}

\polish{The point here is to explain how writing a correct concrete \jsil interpreter in Rosette
yields the symbolic environments presented in the previous subsection.} 

Ideally I would like to talk about: 
\begin{itemize}
   \item how do we represent the \jsil state in Rosette? why did we choose this representation? 
   \item how do we represent \jsil programs as s-expressions? 
   \item ...
\end{itemize}

\begin{display}{Rosette implementation of \jsil symbolic state}
{\scriptsize
\begin{mathpar}
\inferrule[\textsc{Empty Heap}]
  {}{\roscomp{\hemp} \semeq (\racketlist)} 
\and 
\inferrule[\textsc{Non-empty Heap}]
  {
  	 \sheap_1 = \big((l, \sexprp_i) \mapsto \sexprv_i\big)\mid_{i = 0}^n   
	 \quad 
	 (\loc, -) \not\in \domain(\sheap_2)
  }{\roscomp{\sheap_1 \dunion \sheap_2} \semeq  (\racketcons (\racketcons \loc \, (\racketlist \, (\racketcons \sexprp_0 \, \sexprv_0) \cdots   (\racketcons \sexprp_n \, \sexprv_n)))  \ \roscomp{\sheap_2})} 
 \\
\inferrule[\textsc{Empty Store}]
  {}{\roscomp{\storeemp} \semeq (\racketlist)} 
\and 
\inferrule[\textsc{Non-Empty Store}]
  {}{\roscomp{(\jvar: \sexpr) \dunion \sstore} \semeq (\racketcons \, (\racketcons \, (\racketquote \jvar) \ \sexpr) \,  \roscomp{\sstore})} 
\\ 
\inferrule[\textsc{Empty Context}]
  {}{\roscomp{\lstemp} \semeq (\racketlist)} 
\quad 
\inferrule[\textsc{Non-Empty Context}]
  {}{\roscomp{(\fid, \sstore, \jvar, i, j) \lstcons \sctx} \semeq  (\racketcons \,  (\racketlist \, (\racketquote \fid) \, \roscomp{\sstore} \, (\racketquote \jvar) \, i \, j) \, \roscomp{\sctx})} 
\end{mathpar}}
\end{display}

\lstset{language=Scheme}

\begin{figure}
\begin{lstlisting}
(define (mutate-prop-val-list prop-val-list prop new-val)
  (cond
    [(null? prop-val-list)
     (list (cons prop new-val))]
    [(equal? (car (car prop-val-list)) prop)
     (cons (cons prop new-val) (cdr prop-val-list))]
    [ else
     (cons (car prop-val-list) (mutate-prop-val-list (cdr prop-val-list) prop new-val))]))

(define (mutate-heap heap loc prop val)
  (define (mutate-heap-pulp h-pulp loc prop val)
    (cond
      [(null? h-pulp)
       (list (cons loc (list (cons prop val))))]
      [(equal? (car (car h-pulp)) loc)
       (cons (cons loc (mutate-prop-val-list (cdr (car h-pulp)) prop val)) (cdr h-pulp))]
      [ else
       (cons (car h-pulp) (mutate-heap-pulp (cdr h-pulp) loc prop val))]))
  (let ((new-heap-pulp (mutate-heap-pulp (unbox heap) loc prop val)))
    (set-box! heap new-heap-pulp)))


(define (run-bcmd bcmd heap store)
  (let ((cmd-type (first bcmd)))
    (cond
    	[(eq? cmd-type 'h-assign)
      	 (let* ((loc-val (run-expr (second bcmd) store))
                (prop-val (run-expr (third bcmd) store))
                (rhs-val (run-expr  (fourth bcmd) store)))
            (mutate-heap heap loc-val prop-val rhs-val)
            rhs-val)]
         ...)))
\end{lstlisting}
\caption{Fragment of \jsil Interpreter in Rosette}
\end{figure}


\section{Symbolic Execution for JavaScript}

\subsection{Symbolic Execution by Compilation} 

\subsection{Motivating Example} 

We illustrate how Jilette is used to write symbolic tests for JavaScript code by using the JavaScript implementation 
of a  \emph{key-value map} given in Figure~\ref{map:example} (left). 
It contains four functions: 
\jsinline|Map|, for constructing an empty map;
\jsinline|get|, for retrieving the value associated with the key given as input;
\jsinline|put|, for inserting a new \emph{key-value pair} into the map and updating existing keys; and
\jsinline|validKey|, for deciding whether a key is valid.
This library implements a \emph{key-value map} as an object with property \jsinline|_contents|, denoting the object used to store the map contents.  
The named properties of \jsinline|_contents| and their value attributes correspond to the map keys and values, respectively.
As the functions \jsinline|get|, \jsinline|put|, and \jsinline|validKey| are to be shared between all map 
objects, they are defined as properties of \jsinline|Map.prototype|, which is the prototype 
of the objects that are created using \jsinline|Map| as a constructor (e.g.~using~\jsinline|new Map()|). 

 \begin{figure}[t!]
 \begin{lstjs}[firstnumber=1]
function Map () { this._contents = {} }

Map.prototype.get = function (k) {
    if (this._contents.hasOwnProperty(k)) {  return this._contents[k] } 
    	else { return null }  
}

Map.prototype.put = function (k, v) {
   var contents = this._contents;
   if (this.validKey(k)) {  contents[k] = v   } 
   	else { throw new Error("Invalid Key") } 
} 

Map.prototype.validKey = function (k) { ... }
\end{lstjs}
\caption{Map Implementation in JavaScript}
\end{figure}

Note that one can insert a key-value pair with \jsinline|"hasOwnProperty"| as a key into the map. 
By doing this, \jsinline|"hasOwnProperty"| in the prototype chain of
\jsinline|_contents| is overridden and subsequent calls to \jsinline|get| will fail. 
Running the symbolic test below reveals this bug. In fact, \jilette gives 
\jsinline|__s1 = "hasOwnProperty"| as a failing model for the symbolic tetst below. 
%
 \begin{lstjs}[firstnumber=1]
var m = new Map();  m.put (__s1, __n1); var r = m.get(__s1);  
assert(__n1 = r)
\end{lstjs}

