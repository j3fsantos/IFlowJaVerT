%!TEX root = ../main.tex

\subsection{Compiling Specifications to Parametrised Tests}

\subsection{Example} 

In Figure~\ref{fig:map:example}, we define a \emph{map object predicate}, \jsinline|Map|, 
using the auxiliary predicate \jsinline|KVPairs|, which captures the resource of the key-value pairs in the map, 
and the \jsinline|validKey(k)| predicate, which holds if and only if the 
JavaScript function \jsinline|ValidKey(k)| returns \jsinline|true|\footnote{We treat the $\mathtt{ValidKey}$ predicate as a black box.}.
%
Intuitively, the \jsinline|Map(m, mp, kvs, keys)| predicate captures the resource 
of a map object \jsinline|m| with prototype \jsinline|mp|, keys \jsinline|keys| (a set of strings),
and key-value pairs \jsinline|kvs| (a set of string pairs\footnote{We model pairs as lists with two elements and, for clarity, use the pair notation.}). 
Observe that the definition of \jsinline|Map| does not include the resource of a map prototype, as
it is shared between all map objects, and therefore needs to be factored out.  
%
We write \jsinline|-u-| for set union and omit the brackets around singleton 
sets when the meaning is clear. % from the context. 

\begin{figure}[t!]
{\scriptsize
 \begin{verbatim}
Map (m, mp, kvs, keys) := JSObject(m, mp) * 
  DataProp(m, "_contents", c) * JSObject(c, Object.prototype) * KVPairs(c, kvs, keys) *
  (m, "get") -> None * (m, "put") ->  None * (m, "validKey") ->  None * 
  (c, "hasOwnProperty") ->  None *  emptyFields(c, keys -u- "hasOwnProperty")
  \end{verbatim}
  \vspace*{-0.3cm}
 \begin{verbatim}
KVPairs (o, kvs, keys) := 
  (kvs = { }) * (keys = { }),
  (kvs = (key, value) -u- kvs') * (keys = key -u- keys') * 
    ValidKey(key) * DataProp(o, key, value) * KVPairs(o, kvs', keys')
\end{verbatim}}
\caption{Map predicate \label{fig:map:example}}
\end{figure}

%In the following, we assume a \jsinline|MapProto| predicate specifying the resource of 
%a valid map prototype. In particular, the map prototype needs to define the methods 
%\jsinline|put|, \jsinline|get|, and \jsinline|validKey|. 


We are now in the position to specify the functions of the map library. In particular, below we show how to use 
the map object predicate to specify \jsinline|get(k)|.  
%
We consider the case in which the key whose value we  want to fetch is stored in the 
map.  The specification is given below. 
%
\begin{displaymath} 
{\footnotesize
\begin{array}{c}
\left\{ {\begin{array}{c}
 \text{\texttt{Map(this, mp, kvs -u- (k, v), ks) * ObjProtoF()}} \\ 
\end{array}} \right\} \\
%
\text{\bfseries \texttt{get(k)}} \\[0.2mm]
%
\left\{ {\begin{array}{c}
 \text{\texttt{Map(this, mp, kvs -u- (k, v), ks) * ObjProtoF() * (ret = v) }} \\
\end{array}} \right\}
\end{array}
} 
\end{displaymath}
%
The predicate \jsinline|ObjProtoF()| describes the resource captured by the \jsinline|Object.prototype| object. 
In particular, it is needed because \texttt{get} uses the \texttt{hasOwnProperty} function. .
