\documentclass{llncs}


\usepackage{amsmath}
\usepackage{listingsutf8}
\usepackage{hyperref}
\usepackage{graphicx}
\usepackage{tabularx}
\usepackage{xspace}
\usepackage{textcomp}

\usepackage{amssymb,amsfonts,textcomp,stmaryrd}
\usepackage{mathpartir}
\usepackage{url}
\usepackage{upgreek}
\usepackage{xparse}
\usepackage{booktabs}
\usepackage[utf8]{inputenc}
\usepackage{wrapfig}

\usepackage{xcolor}
\usepackage{macros_js}
\usepackage{gdshojs}

\usepackage{fancyvrb}


\makeatletter
\newif\ifFV@bgcolor
\newbox\FV@bgbox
\define@key{FV}{bgcolor}{\FV@bgcolortrue\def\FV@bgcolor{#1}}

\def\FV@BeginVBox{%
  \leavevmode\ifFV@bgcolor\setbox\FV@bgbox=\fi
  \hbox\ifx\FV@boxwidth\relax\else to\FV@boxwidth\fi\bgroup
  \ifcase\FV@baseline\vbox\or\vtop\or$\vcenter\fi\bgroup}
\def\FV@EndVBox{\egroup\ifmmode$\fi\hfil\egroup
  \ifFV@bgcolor\colorbox{\FV@bgcolor}{\box\FV@bgbox}\fi}
\makeatother


%\acmBadgeL[http://ctuning.org/ae/ppopp2016.html]{ae-logo}
%\acmBadgeR[http://ctuning.org/ae/ppopp2016.html]{ae-logo}

\newcommand{\shat}{\^{s}}

%JavaScript 
\definecolor{SkyBlue}{rgb}{0.20,0.39,0.64}
\definecolor{Plum}{rgb}{0.46,0.31,0.48}
\definecolor{Chocolate}{rgb}{0.75,0.49,0.07}
\definecolor{Aluminium5}{rgb}{0.33,0.34,0.32}
\definecolor{DarkGreen}{rgb}{0.2,0.5,0.2}
\definecolor{ltblue}{rgb}{0,0.4,0.4}
\definecolor{dkblue}{rgb}{0,0.2,0.7}
\definecolor{dkgreen}{rgb}{0,0.4,0}
\definecolor{dkviolet}{rgb}{0.3,0,0.5}
\definecolor{dkred}{rgb}{0.6,0,0}
\definecolor{talkred}{rgb}{0.69,.20,0.22}
\definecolor{talkblue}{rgb}{0.04,0.40,0.80}
\definecolor{talkgreen}{rgb}{0.34,.81,0.10}
\definecolor{oldtalkblue}{rgb}{0.22,.20,0.69}
\definecolor{greenish}{rgb}{.0,.65,.0}
\definecolor{mygray}{gray}{0.9}

\lstset{
	showstringspaces=false
}

\lstdefinelanguage{Scheme}{
  morekeywords=[1]{define, define-syntax, define-macro, lambda, define-stream, stream-lambda},
  morekeywords=[2]{begin, call-with-current-continuation, call/cc,
    call-with-input-file, call-with-output-file, case, cond,
    do, else, for-each, if,
    let*, let, let-syntax, letrec, letrec-syntax,
    let-values, let*-values,
    and, or, not, delay, force,
    quasiquote, quote, unquote, unquote-splicing,
    map, fold, syntax, syntax-rules, eval, environment, query },
  morekeywords=[3]{import, export},
  alsodigit=!\$\%&*+-./:<=>?@^_~,
  sensitive=true,
  morecomment=[l]{;},
  morecomment=[s]{\#|}{|\#},
  morestring=[b]",
  basicstyle=\scriptsize\ttfamily,
  keywordstyle=\bf\ttfamily\color[rgb]{0,.3,.7},
  commentstyle=\color[rgb]{0.133,0.545,0.133},
  stringstyle={\color[rgb]{0.75,0.49,0.07}},
  upquote=true,
  breaklines=true,
  breakatwhitespace=true,
  literate=*{`}{{`}}{1}
}

\def\schemeinline{\lstinline[language=Scheme, basicstyle=\small\ttfamily]}

\lstdefinelanguage{JavaScript}{
  morekeywords=[1]{typeof, new, true, false, catch,
    function, return, null, catch, switch, var,
    if, in, while, do, else, case, break, continue},
  morekeywords=[2]{class, export, boolean, throw, implements, import, this},
  numbers=left,
  numbersep=4pt,
  numberstyle=\tiny\color{dkblue},
  columns=fullflexible,
  sensitive=false,
  comment=[l]{//},
  captionpos=b,   
  morecomment=[s]{/*}{*/},
  morestring=[b]',
  morestring=[b]",
  basicstyle=\scriptsize\texttt,
  identifierstyle=\ttfamily\color{Aluminium5},
  keywordstyle=[1]\ttfamily\color{Plum},
  keywordstyle=[2]\ttfamily\color{SkyBlue},
  stringstyle=\ttfamily\color{DarkGreen},
  commentstyle=\ttfamily, 
%  commandchars=\$\{\}
}[keywords,comments,strings]

\lstnewenvironment{lstjs}{\lstset{language=JavaScript,basicstyle=\fontsize{8}{8}\ttfamily,escapeinside={~}{~}}}{}
\def\jsinline{\lstinline[language=JavaScript, basicstyle=\small]}


% The Acronyms of the project and some other stuff
\newcommand{\jsil}{JSIL\xspace}
\newcommand{\jsverify}{JSVerify\xspace}
\newcommand{\JSComp}{JS-2-JSIL\xspace}
\newcommand{\jsilverify}{JSILVerify\xspace}

\newcommand{\jilette}{Cosette\xspace}
\newcommand{\rosette}{Rosette\xspace}

\newcommand{\myparagraph}[1]{\smallskip\noindent {\bf #1.}\hspace{1pt}}
\newcommand{\myparagraphq}[1]{\smallskip\noindent {\bf #1?}\hspace{1pt}}

% COMMENTS

\newcommand{\polish}[1]{{\color{red}#1}}

\newcommand{\pginline}[1]{ {\color{red} *** PG : #1 ***} }
\newcommand{\pmaxinline}[1]{ {\color{blue} *** PM : #1 ***} }
\newcommand{\jfsinline}[1]{ {\color{green} *** JFS : #1 ***} }
\newcommand{\jdinline}[1]{ {\color{purple} *** JD : #1 ***} }

\newif\ifComments
\Commentstrue

\newcommand{\pg}[1]{%
\ifComments
\begin{center}
\fbox{\begin{minipage}{0.95\textwidth} \color{red}
{\rm PG: \small #1}
\end{minipage}}
\end{center}
\fi}

\newcommand{\pmax}[1]{%
\ifComments
\begin{center}
\fbox{\begin{minipage}{0.95\textwidth} \color{blue}
{\rm PM: \small #1}
\end{minipage}}
\end{center}
\fi}

\newcommand{\jfs}[1]{%
\ifComments
\begin{center}
\fbox{\begin{minipage}{0.95\textwidth} \color{green}
{\rm JFS: \small #1}
\end{minipage}}
\end{center}
\fi}

\newcommand{\jd}[1]{%
\ifComments
\begin{center}
\fbox{\begin{minipage}{0.95\textwidth} \color{purple}
{\rm JD: \small #1}
\end{minipage}}
\end{center}
\fi}

\begin{document}
\title{\jilette: A Symbolic Framework for Testing JavaScript Programs}

\maketitle

\begin{abstract}
We present \jilette, a framework for symbolically executing JavaScript code (ECMAScript 5, ES5), which precisely follows the language standard. 
%At the core of \jilette is a sound symbolic interpreter for \jsil, an intermediate language well-suited for verification and analysis. 
%This interpreter is written in \underline{Rosette}, a symbolic virtual machine that enables the design of new solver-aided languages. 
\jilette works by first compiling JavaScript programs to \jsil using \underline{\JSComp}, a well-tested, standard-compliant compiler from 
JavaScript to \jsil, and then symbolically executing the compiled \jsil code in the \jsil symbolic interpreter. 
%
We study two complementary uses of \jilette. 
First, we show how \jilette can be used for symbolic testing of JavaScript programs by finding concrete executions that trigger assertion and test failures. 
We highlight the range of \jilette by giving examples using strings, regular expressions, and the notorious \jsinline|eval| statement.
Second, building on \jilette, we develop a tool for debugging separation logic specifications by compiling them to symbolic tests in order to find witnesses for bugs in both specification and code.
\end{abstract}

\section{Introduction}
%!TEX root = ../main.tex

JavaScript is the most widespread dynamic language: it is the de facto language for client-side Web applications (used by 94.8\% of websites \cite{JS948percent});
it is used for server-side scripting via Node.js; and it is even run on small embedded devices with limited 
memory. It is the most active language in both GitHub \cite{GithubActive} and StackOverflow \cite{SOActive}.
The dynamic nature of JavaScript and its complex semantics make it a difficult target for
program analysis, such as logic-based symbolic execution.  
We present \jilette, a symbolic framework for testing JavaScript code (ECMAScript 5, ES5~\cite{ecma}). 
%
\jilette is aimed at the general developer as it only requires them to write simple tests to check their code. 


The standard approach for validating code is running it against 
human-made test batteries, which check that given some concrete inputs, the code produces the expected
outputs. The main drawback of this approach is that tests are often
too incomplete. With \jilette, JS developers can write symbolic tests: instead of 
using concrete inputs, the developer uses symbolic inputs and states the 
constraints that the outputs need to satisfy in the form of simple, intuitive 
first-order assertions over the inputs. 
Furthermore, if a test fails, \jilette provides the concrete inputs that cause it 
to fail, exposing bugs in the tested code. 
We highlight the capabilities of \jilette through examples that showcase
JS specific features, such as: prototype-based inheritance, 
the for-in statement, and JS arrays.

\polish{We need a paragraph saying what is new... 
\begin{enumerate}
  \item \jilette follows the 
\end{enumerate}
}




\myparagraph{Architecture}
The core of \jilette consists of a symbolic interpreter for
\jsil~\cite{javert}, a simple intermediate goto language. 
We obtain this symbolic interpreter \emph{for free}, 
by implementing a concrete \jsil interpreter in Rosette~\cite{Rosette2,Rosette1},~a 
symbolic virtual machine that facilitates generation of solver-aided languages.
We design the concrete interpreter so that all of Rosette's natively supported solver-aided
features, such as advanced string and regular-expression reasoning, 
are lifted to the \jsil symbolic interpreter. 
In~\S\ref{sec:jsil:symb:exec}, we give a formalisation of the \jsil concrete and symbolic executions, linking them together with a {\em soundness result}. We also provide insights on how to correctly design the concrete \jsil interpreter in Rosette.


The second component that \jilette uses is \JSComp~\cite{javert}, 
a well-tested, standard-compliant compiler from JavaScript to \jsil. We extend
\JSComp with support for the non-strict mode of ES5, as well as
regular expressions and the entire \jsinline|String| built-in library.
\JSComp allows us to lift the \jsil symbolic execution to JavaScript by first compiling JavaScript code to \jsil code, and
then symbolically executing the compiled code in the 
\jsil symbolic interpreter. This process, described in \S\ref{symb:exec:comp},
involves extending JavaScript syntax and the \JSComp compiler to support symbolic values and 
constructs for reasoning about them. These constructs are intuitive
and allow the general developer to easily write assertions about the behaviour
of their program. 
Moreover, we adjust the \jsil symbolic interpreter so that the abstraction level 
of the generated \jsil code precisely matches the abstraction level of Rosette, 
 maximising the use of Rosette's native reasoning capabilities.


\myparagraphq{Why \jilette} 
\jilette is \emph{useful}: it has tangible applications. 
It can report bugs in JavaScript programs, producing concrete witnesses triggering the bugs. It can also be used as a helper tool for developers of logic-based functional correctness specifications of JavaScript code.
\jilette is \emph{approachable}: it can easily be used by a general JavaScript developer. The annotation burden of \jilette is minimal and the assertion language is simple and intuitive. \polish{Sweet spot?}
\jilette is \emph{trustworthy}: its components come with correctness guarantees. 
The correctness of the \JSComp compiler ensures full adherence to the real semantics of JavaScript. The \jilette symbolic execution engine is based on a sound symbolic
analysis for \jsil, guaranteeing the absence of false positives. \polish{Sentence about unification.}
Finally, \jilette is \emph{extensible}: its coverage can easily be extended in a modular way. This gives us the mechanism for supporting built-in libraries not covered by \JSComp, or adding support for standard-external runtime libraries, such as the DOM.

\section{Symbolic Execution for \jsil}\label{sec:jsil:symb:exec}
%!TEX root = ../main.tex

\begin{itemize}
  \item 2.0  - intro - explain the basic ideas of Jilette (use a diagram) - 
 
  \item 2.1 - describe the jsil language. give its formal syntax (extended with assert and solve). 
                   formally define symbolic execution for JSIL commands. soundness lemma. 
 
  \item 2.2 Implementation:  
               - encoding \jsil heaps in Rosette 
               - explain the \jsil interpreter implemented in Rosette and its connection to the \jsil semantics (as defined in appendix) 
               - give snippets of the interpreter 
               - discuss soundness, trust, and other issues
 
  \item 2.3 Symbolic execution for JavaScript 
              - explain that we have to extend the syntax of JavaScript with asserts  as well as constructs for creating symbolic values
              - give the example 
              - discuss challenges: abstraction level of the generated code needs to match the abstraction level of Rosette 
\end{itemize}

\subsection{Formalisation}

\myparagraph{\jsil: Formal Semantics}
The basic memory model of \jsil is as follows. 
\jsil values contain: numbers, $\jnumber$; booleans, $\jbool$; strings, $\jstring$;  the special values \jsinline|undefined| and \jsinline|null|; and object locations,  $\loc \in \locs$.
A \jsil heap, $\heap \in \heaps$, is a partial function mapping pairs of  object locations, and strings to heap values. 
 Given a heap $\heap$, we denote: a heap cell by $\hcell{\loc}{\jstring}{\val}$, meaning that  $h(\loc,\jstring) = \val$; the union of two disjoint heaps by $\oheap_1 \dunion \oheap_2$; heap lookup by $\hread{\oheap}{\loc}{\jstring}$; and the empty heap by $\hemp$.
 Finally, a \jsil variable store, $\store \in \stores$, is a mapping from JSIL program variables $\jvar \in \jvars$ to JSIL values.

\jsil semantics is defined in small-step style. Transitions for basic commands, given in Figure \ref{fig:sem:basic:commands}, are of the form $\semtrans{\heap, \store, \bcmd}{\heap', \store'}$, meaning that the execution of the basic command $\bcmd$ in the heap $\heap$ and store $\store$ results in the heap $\heap'$ and $\store'$. We also allow a transition of a basic command to fail, denoted by $\semtranserr{\sheap, \sstore, \bcmd}$.

\begin{figure}[ht!]
{\scriptsize
\begin{mathpar} 
%
\inferrule[Interpretation of expressions]{}{
\semexpr{\lit}{\store} \semeq \lit
\quad 
\semexpr{\jvar}{\store} \semeq \store(\jvar)
\quad 
\semexpr{\unoper\ \jexpr}{\store} \semeq \unoper (\semexpr{\jexpr}{\store})
\quad 
\semexpr{\jexpr_1 \binoper \jexpr_2}{\store} \semeq \binoper(\semexpr{\jexpr_1}{\store}, \semexpr{\jexpr_2}{\store})}
\\

\inferrule[\textsc{Skip}]{}
	{ \semtrans{\heap, \store, \jsilskip}{\heap, \store}} 
 \qquad
 %
\inferrule[\textsc{Assignment}]
  {
      \symbeval{\jsilexpr}{\store} =  \val
      \quad
      \store' = \store[\jvar \mapsto \val]
  }{\semtrans{\heap, \store, \jvar := \jsilexpr}{\heap, \store'}} 
%
\qquad 
%
\inferrule[\textsc{Object Creation}]
  { 
    \heap = \heap \dunion \hcell{\loc}{\protop}{\jsnull}
    \quad (\loc,-) \notin \domain (\heap)
  }{\semtrans{\heap, \store, \jvar := \jsilnew()}{\heap, \store[\jvar \mapsto \loc]}}
\\
%
\inferrule[\textsc{Property Access}]
  { 
 	\symbeval{\jsilexpr_1}{\store} =  \loc
  	\quad 
        \symbeval{\jsilexpr_2}{\store} =  \jstring
        \quad
        \heap = - \dunion \hcell{\loc}{\jstring}{\val}
  }{ \semtrans{\heap, \store, \jvar := [\jsilexpr_1, \jsilexpr_2]}{\heap,  \store[\jvar \mapsto \val]}}
 \and 
 \inferrule[\textsc{Property Deletion}]
  { 
        \symbeval{\jsilexpr_1}{\store} =  \loc
  	\quad 
        \symbeval{\jsilexpr_2}{\store} =  \jstring
        \quad
        \heap = \heap' \dunion \hcell{\loc}{\jstring}{-}
  }{\semtrans{\heap, \store, \jsildelete(\jsilexpr_1, \jsilexpr_2)}{\heap', \store}}
 %
\\
%
\inferrule[\textsc{Property Assignment - Found}]
  {     \symbeval{\jsilexpr_1}{\store} =  \loc
  	\quad 
        \symbeval{\jsilexpr_2}{\store} =  \jstring
        \quad
        \symbeval{\jsilexpr_3}{\store} =  \val
       \\\\
        \heap = \heap' \dunion  \hcell{\loc}{\jstring}{-}
  }{\semtrans{\heap, \store, [\jsilexpr_1, \jsilexpr_2] := \jsilexpr_3}{\heap' \dunion  \hcell{\loc}{\jstring}{\val}, \store}} 
 \and 
 \inferrule[\textsc{Property Assignment - Not Found}]
  {     \symbeval{\jsilexpr_1}{\store} =  \loc
  	\quad 
        \symbeval{\jsilexpr_2}{\store} =  \jstring
        \quad
        \symbeval{\jsilexpr_3}{\store} =  \val
       \\\\
        \heap = \heap' 
        \quad 
        (\loc, \jstring) \not\in \domain(\heap)
  }{\semtrans{\heap, \store, [\jsilexpr_1, \jsilexpr_2] := \jsilexpr_3}{\heap \dunion  \hcell{\loc}{\jstring}{\val}, \store}} 
\\
%
\inferrule[\textsc{Member Check - True}]
  { 
      \symbeval{\jsilexpr_1}{\store} =  \loc
  	\quad 
        \symbeval{\jsilexpr_2}{\store} =  \jstring
       \quad 
   	(\loc, \jstring) \in \domain(\heap) 
  }{\semtrans{\heap, \store,\jvar := \hasfield(\jsilexpr_1, \jsilexpr_2)}{\heap, \store[\jvar \mapsto \jtrue]}}
  \and 
 \inferrule[\textsc{Member Check - False}]
  { 
      \symbeval{\jsilexpr_1}{\store} =  \loc
  	\quad 
        \symbeval{\jsilexpr_2}{\store} =  \jstring
       \quad 
   	(\loc, \jstring) \not\in \domain(\heap) 
  }{\semtrans{\heap, \store,\jvar := \hasfield(\jsilexpr_1, \jsilexpr_2)}{\heap, \store[\jvar \mapsto \jfalse]}}
%
\\
%
\inferrule[\textsc{Assert - True}]
  { 
      \symbeval{\jsilexpr}{\store} =  \jtrue
  }{\semtrans{\heap, \store, \assert(\jsilexpr)}{\heap, \store}} 
\and
\inferrule[\textsc{Assert - False}]
  { 
      \symbeval{\jsilexpr}{\store} = \jfalse
  }{\semtranserr{\heap, \store, \assert(\jsilexpr)}} 
\end{mathpar}}
\vspace*{-0.5cm}
\caption{Semantics of \jsil Basic Commands: {$\semtrans{\heap, \store, \bcmd}{\heap', \store'}$}\label{fig:sem:basic:commands}}
\end{figure}

To describe transitions for \jsil commands, we introduce call stacks, denoted~$\ctx$. Call stacks are lists of tuples of the form $(\pid, \sstore, \jvar, i, j)$, where: 
\dtag{1}~$\pid$~is a procedure identifier, 
\dtag{2}~$\sstore$~is the store of the procedure that called $\pid$, \dtag{3}~$\jvar$~is 
the variable to which the return of $\pid$ must be assigned, \dtag{4} $i$ is the index 
of the command to which the control must jump after the execution of $\pid$ in 
case of normal return, and \dtag{5} $j$ the index to which it must jump in case of 
error return. Transitions for control flow commands have the form:  $\semtrans[\prog]{\heap, \store, i}{\heap', \store', i'}[\ctx][\ctx']$, meaning that, in the context of the entire program $\prog$, the evaluation of the $i$-th command of the first procedure in the call stack $\ctx$, in
the heap $\heap$ and store $\store$, generates the heap $\heap'$, store $\store'$, call stack $\ctx'$,   
and the next command to be evaluated is the $i'$-th command of the first procedure of the call stack~$\ctx'$. Due to space constraints and as the transitions for JSIL symbolic execution are  similar, we give the full semantics for JSIL control flow commands in the Appendix.

\myparagraph{\jsil: Symbolic Evaluation}
In order to symbolically execute \jsil programs, we extend the syntax of \jsil expressions with 
symbolic strings $\sstring \in \sstrings$ and symbolic numbers $\snumber \in \snumbers$. 
For convenience, we use $\svars$ to denote the union of $\sstrings$ and $\snumbers$ 
and $\svar$ to range over $\svars$. 
The syntax of symbolic expressions $\sexpr$ is as follows: $\sexpr \triangleq \lit \mid \sstring \mid \snumber \mid \unoper\ \sexpr \mid \sexpr \binoper \sexpr$.

We extend heaps, stores, and contexts with symbolic values, obtaining symbolic 
heaps, stores, and contexts, respectively ranged over by $\sheap$, $\sstore$, and $\sctx$. 
A symbolic heap, $\sheap \in \sheaps$, is a partial function mapping pairs of  
object locations, and symbolic expressions to symbolic expressions. 
A symbolic store, $\sstore \in \sstores$, is a mapping from program variables 
$\jvar \in \jvars$ to symbolic expressions.
%
A \emph{symbolic state} $\sstate = (\sheap, \sstore, \pc)$ is a triple consisting of a 
symbolic heap $\sheap$, a symbolic store $\sstore$, and a path condition $\pc$. 
The path condition is a first-order quantifier-free formula over symbolic strings and 
numbers, which accumulates constraints on the given symbolic inputs that trigger 
the execution to follow the path that led to the current symbolic state. 
Path conditions are given by the following grammar: 
\begin{equation*}
\pc \triangleq \sexpr_1 = \sexpr_2 \mid \sexpr_1 \leq \sexpr_2 \mid \pc_1 \, \wedge \, \pc_2 \mid \pc_1 \vee \pc_2 \mid \neg \pc \mid \ltrue \mid \lfalse
\end{equation*}

Figure~\ref{fig:symbexe:bcmds} presents the symbolic execution rules for \jsil basic commands. 
Rules have the form $\symbtrans{\sheap, \sstore, \bcmd, \pc}{\sheap', \sstore', \pc'}$, 
where: \dtag{1} $\sheap$ and $\sstore$ are the symbolic heap and store on which to evaluate $\bcmd$, 
\dtag{2} $\pc$ the current \emph{path condition}, and \dtag{3} $\sheap'$, $\sstore'$, and $\pc'$
the resulting symbolic heap, store, and path condition. Notice that the rules are non-deterministic.

Figure~\ref{fig:symbexe:cmds} presents the symbolic execution rules for \jsil commands. 
Rules have the form $\symbtrans[\prog]{\sheap, \sstore, i, \pc}{\sheap', \sstore', i', \pc'}[\sctx][\sctx']$; 
they are analogous to the semantic rules for \jsil commands, except that the heap, store, and call stack are symbolic, there is the additional path condition, and their execution can fail. For clarity, we keep the program and the context implicit wherever possible, and make use of a function $\ccmd{\prog, \ctx, i}$, which returns the $i$-th command of the procedure that is first in $\ctx$. We write $\ccmd{i}$ when $\prog$ and $\ctx$ are implicit.

%\begin{display}{}
\begin{figure}[ht!]
{\scriptsize
\begin{mathpar} 
%
\inferrule[\textsc{Skip}]{}
	{ \symbtrans{\sheap, \sstore, \jsilskip, \pc}{\sheap, \sstore, \pc}} 
 \and
 %
\inferrule[\textsc{Assignment}]
  {
      \symbeval{\jsilexpr}{\sstore} =  \sexpr
      \quad
      \sstore' = \sstore[\jvar \mapsto \sexpr]
  }{\symbtrans{\sheap, \sstore, \jvar := \jsilexpr, \pc}{\sheap, \sstore', \pc}} 
%
\and 
%
\inferrule[\textsc{Object Creation}]
  { 
    \sheap' = \sheap \dunion \hcell{\loc}{\protop}{\jsnull}
    \and (\loc,-) \notin \domain (\sheap)
  }{\symbtrans{\sheap, \sstore, \jvar := \jsilnew(), \pc}{\sheap', \sstore[\jvar \mapsto \loc], \pc}}
\\
%
\inferrule[\textsc{Property Access}]
  { 
 	\symbeval{\jsilexpr_1}{\sstore} =  \loc
  	\quad 
        \symbeval{\jsilexpr_2}{\sstore} =  \sexpr_p
        \quad
        \sheap = \sheap' \, \uplus \, \big((l, \sexprp_i) \mapsto \sexprv_i\big)\mid_{i = 0}^n   
        \quad
        (l, -) \not\in \domain(\sheap')
        \quad 
        0 \leq k \leq n
        \\\\
        \pc' = \pc \ \wedge \, \big( (\sexprp_k = \sexpr_p) \ \wedge \bigwedge_{i = 0, i \neq k}^n (\sexprp_i \neq \sexpr_p) \big)
  }{ \symbtrans{\sheap, \sstore, \jvar := [\jsilexpr_1, \jsilexpr_2], \pc}{\sheap,  \sstore[\jvar \mapsto \sexprv_k], \pc'}}
 %
\\
%
\inferrule[\textsc{Property Assignment - Found}]
  {     \symbeval{\jsilexpr_1}{\sstore} =  \loc
  	\quad 
        \symbeval{\jsilexpr_2}{\sstore} =  \sexpr_p
        \quad
        \symbeval{\jsilexpr_3}{\sstore} =  \sexpr_v
       \quad 
        \sheap = \sheap' \, \uplus \, \big((l, \sexprp_i) \mapsto \sexprv_i\big)\mid_{i = 0}^n   
        \quad
        (l, -) \not\in \domain(\sheap')
        \quad 
        0 \leq k \leq n
        \\
          \pc' = \pc \ \wedge \, \big( (\sexprp_k = \sexpr_p) \ \wedge \bigwedge_{i = 0, i \neq k}^n (\sexprp_i \neq \sexpr_p) \big)
         \quad
         \sheap'' = \sheap' \, \uplus \,  \big((l, \sexprp_i) \mapsto \sexprv_i\big)\mid_{i = 0, i \neq k}^n \, \uplus \,  (l, \sexpr_p) \mapsto \sexpr_v
  }{\symbtrans{\sheap, \sstore,  [\jsilexpr_1, \jsilexpr_2] := \jsilexpr_3, \pc}{\sheap'', \sstore, \pc'}} 
\\
%
\inferrule[\textsc{Property Assignment - Not Found}]
  {     \symbeval{\jsilexpr_1}{\sstore} =  \loc
  	\quad 
        \symbeval{\jsilexpr_2}{\sstore} =  \sexpr_p
        \quad
        \symbeval{\jsilexpr_3}{\sstore} =  \sexpr_v
       \quad 
        \sheap = \sheap' \, \uplus \, \big((l, \sexprp_i) \mapsto \sexprv_i\big)\mid_{i = 0}^n   
        \quad
        (l, -) \not\in \domain(\sheap')
        \quad 
        0 \leq k \leq n
        \\
          \pc' = \pc \ \wedge \, \bigwedge_{i = 0}^n (\sexprp_i \neq \sexpr_p)
         \quad
         \sheap'' = \sheap \, \uplus \,  (l, \sexpr_p) \mapsto \sexpr_v
  }{\symbtrans{\sheap, \sstore, [\jsilexpr_1, \jsilexpr_2] := \jsilexpr_3, \pc}{\sheap'', \sstore, \pc'}}   
%
\\
%
\inferrule[\textsc{Property Deletion}]
  { 
        \symbeval{\jsilexpr_1}{\sstore} =  \loc
  	\quad 
        \symbeval{\jsilexpr_2}{\sstore} =  \sexpr_p
       \quad 
        \sheap = \sheap' \, \uplus \, \big((l, \sexprp_i) \mapsto -\big)\mid_{i = 0}^n   
        \quad
        (l, -) \not\in \domain(\sheap')
        \quad 
        0 \leq k \leq n
     \\ 
      \pc' = \pc \ \wedge \, \big( (\sexprp_k = \sexpr_p) \ \wedge \bigwedge_{i = 0, i \neq k}^n (\sexprp_i \neq \sexpr_p) \big)
     \quad 
      \sheap'' = \sheap' \, \uplus \,  \big((l, \sexprp_i) \mapsto \sexprv_i\big)\mid_{i = 0, i \neq k}^n
   }{\symbtrans{\sheap, \sstore, \jsildelete(\jsilexpr_1, \jsilexpr_2), \pc}{\sheap'', \sstore, \pc'}}
 \\
 %
\inferrule[\textsc{Member Check - True}]
  { 
      \symbeval{\jsilexpr_1}{\sstore} =  \loc
  	\quad 
        \symbeval{\jsilexpr_2}{\sstore} =  \sexpr_p
       \quad 
        \sheap = \sheap' \, \uplus \, \big((l, \sexprp_i) \mapsto -\big)\mid_{i = 0}^n   
        \quad
        (l, -) \not\in \domain(\sheap')
        \quad 
        0 \leq k \leq n
     \\ 
     \pc' = \pc \ \wedge \, \big( (\sexprp_k = \sexpr_p) \ \wedge \bigwedge_{i = 0, i \neq k}^n (\sexprp_i \neq \sexpr_p) \big)
  }{\symbtrans{\sheap, \sstore, \jvar := \hasfield(\jsilexpr_1, \jsilexpr_2), \pc}{\sheap, \sstore[\jvar \mapsto \jtrue], \pc'}}
%
\\
%
\inferrule[\textsc{Member Check - False}]
  { 
      \symbeval{\jsilexpr_1}{\sstore} =  \loc
  	\quad 
        \symbeval{\jsilexpr_2}{\sstore} =  \sexpr_p
       \quad 
        \sheap = \sheap' \, \uplus \, \big((l, \sexprp_i) \mapsto -\big)\mid_{i = 0}^n   
        \quad
        (l, -) \not\in \domain(\sheap')
        \quad 
        0 \leq k \leq n
     \\ 
     \pc' = \pc \ \wedge \,  \bigwedge_{i = 0}^n (\sexprp_i \neq \sexpr_p) \big)
  }{\symbtrans{\sheap, \sstore, \jvar := \hasfield(\jsilexpr_1, \jsilexpr_2), \pc}{\sheap, \sstore[\jvar \mapsto \jfalse], \pc'}}
\\
%
\inferrule[\textsc{Assert - True}]
  { 
      \symbeval{\jsilexpr}{\sstore} =  \sexpr
     \quad 
     \pc \vdash \sexpr 
  }{\symbtrans{\sheap, \sstore, \assert(\jsilexpr), \pc}{\sheap, \sstore, \pc}} 
\quad
\inferrule[\textsc{Assert - False}]
  { 
      \symbeval{\jsilexpr}{\sstore} =  \sexpr
     \quad 
     \pc \not\vdash \sexpr 
  }{\symbtranserr{\sheap, \sstore, \assert(\jsilexpr), \pc}} \\
  \inferrule[\textsc{Assume}]
  {\symbeval{\jsilexpr}{\sstore} =  \sexpr}{\symbtrans{\sheap, \sstore, \assume(\jsilexpr), \pc}{\sheap, \sstore, \pc \land \sexpr}} 
\end{mathpar}}
\vspace*{-0.6cm}
\caption{Symbolic Execution for \jsil Basic Commands: {$\symbtrans{\sheap, \sstore, \bcmd, \pc}{\sheap', \sstore', \pc'}$}\label{fig:symbexe:bcmds}}
\end{figure}
%\end{display}  


\begin{figure}[ht]
{\scriptsize
\begin{mathpar} 
\inferrule[\textsc{Basic Command}]
   { 
     \ccmd{i} = \bcmd 
     \quad
     \symbtrans{\sheap, \sstore, \bcmd, \pc}{\sheap', \sstore', \pc'} 
   }{\symbtrans{\sheap, \sstore, i, \pc}{\sheap', \sstore', i+1, \pc'}}
%
   \qquad
  %
  \inferrule[\textsc{Basic Command - Fail}]
   { 
     \ccmd{i} = \bcmd 
     \quad
     \symbtranserr{\sheap, \sstore, \bcmd, \pc} 
   }{\symbtranserr{\sheap, \sstore, i, \pc}}
 %
   \qquad
  %
  \inferrule[\textsc{Goto}]
   { \ccmd{i} = \goto \, j \quad}
   {\symbtrans{\sheap, \sstore, i, \pc}{\sheap, \sstore, j, \pc}}
  \\ 
  \inferrule[\textsc{Cond. Goto - True}]
   { \ccmd{i} =  \ifgoto{\jsilexpr}{j}{k} \quad
     \symbeval{\jsilexpr}{\sstore} =  \sexpr
   }
   {\symbtrans{\sheap, \sstore, i, \pc}{\sheap, \sstore, j,  \pc \, \wedge \, \sexpr}}
  \and 
    \inferrule[\textsc{Cond. Goto - False}]
   { \ccmd{i} =  \ifgoto{\jsilexpr}{j}{k} \quad
     \symbeval{\jsilexpr}{\sstore} =  \sexpr
   }
   {\symbtrans{\sheap, \sstore, i, \pc}{\sheap, \sstore, k, \pc \, \wedge \, \neg\sexpr}}
   \\
    \inferrule[\textsc{Procedure Call}]
   { 
    \ccmd{i} =   \jsilcall{\jvar}{\jsilexpr}{\jsilexpr_i \mid_{i = 0}^{n}}{j}
     \quad
    \symbeval{\jsilexpr}{\sstore} =  \pid' 
    \quad
      \symbeval{\jsilexpr_i}{\sstore} =  \sexpr_i \mid_{i = 0}^{n} 
     \quad
     \args(\pid') = \jsillist{\jvar_1, ..., \jvar_{m}} 
     \quad 
      \sexpr_i = \jsundefined \mid_{i = n+1}^{m}  
   }
   {\symbtrans{\sheap, \sstore, i, \pc}{\sheap, [ \jvar_i \mapsto \sexpr_i \mid_{i = 0}^{m}], 0, \pc}[\sctx][(\pid', \sstore, \jvar, i+1, j)::\sctx]}
    \\ 
  \inferrule[\textsc{Normal Return}]
   {
       \sctx = (-, \sstore', \jvar, i, -) :: \sctx' 
       \quad 
       \sstore(\procretvar) = \sexpr
   }  
   {\symbtrans{\sheap, \sstore, \procretlab, \pc}{\sheap, \sstore'[\jvar \mapsto \sexpr], i, \pc}[\sctx][\sctx']}
   \and 
     \inferrule[\textsc{Error Return}]
   {
       \sctx = (-, \sstore', \jvar, -, j) :: \sctx' 
       \quad 
       \sstore(\procerrvar) = \sexpr
   }  
   {\symbtrans{\sheap, \sstore, \procerrlab, \pc}{\sheap, \sstore'[\jvar \mapsto \sexpr], j, \pc}[\sctx][\sctx']}
 \end{mathpar}}
 \vspace*{-0.4cm}
\caption{Symbolic Execution for \jsil Commands: {$\symbtrans{\sheap, \sstore, i, \pc}{\sheap', \sstore', j, \pc'}[\sctx][\sctx']$}\label{fig:symbexe:cmds}}
\end{figure}

\begin{figure}[ht!]
{
\begin{tabular}{l}
$\quad${\bf Symbolic Expressions:}  \\
$
\quad
\semexpr{\lit}{\senv} \semeq \lit
\quad 
\semexpr{\svar}{\senv} \semeq \senv(\svar)
\quad 
\semexpr{\unoper\ \sexpr}{\senv} \semeq \unoper (\semexpr{\sexpr}{\senv})
\quad 
\semexpr{\sexpr_1 \binoper \sexpr_2}{\senv} \semeq \binoper(\semexpr{\sexpr_1}{\senv}, \semexpr{\sexpr_2}{\senv}) 
$
\\[3pt]
$\quad${\bf Symbolic Heaps:}  \\
$
\quad
 \semexpr{\hemp}{\senv} \semeq \hemp
\quad
\semexpr{\hcell{\loc}{\sexpr_p}{\sexpr_v}}{\senv} \semeq  \hcell{\loc}{\semexpr{\sexpr_p}{\senv}}{\semexpr{\sexpr_v}{\senv}}
\quad
\semexpr{\sheap_1 \dunion \sheap_2}{\senv} \semeq  \semexpr{\sheap_1}{\senv} \dunion \semexpr{\sheap_2}{\senv}
$%
%%
%%
\\[3pt]
$\quad${\bf Symbolic Stores:}  
$
 \semexpr{\storeemp}{\senv} \semeq \storeemp
\quad 
 \semexpr{(\jvar: \sexpr) \dunion \sstore}{\senv} \semeq (\jvar: \semexpr{\sexpr}{\senv}) \dunion \semexpr{\sstore}{\senv}
$%
\\[3pt]
$\quad${\bf Symbolic Contexts:}  
$ \semexpr{\lstemp}{\senv} \semeq \lstemp
\quad 
 \semexpr{(\pid, \sstore, \jvar, i, j) \lstcons \sctx}{\senv} \semeq (\pid, \semexpr{\sstore}{\senv}, \jvar, i, j) \lstcons \semexpr{\sctx}{\senv}
$%

\\[3pt]
$\quad${\bf Symbolic States:}  $\semexpr{(\sheap, \sstore, \sctx)}{\senv} \semeq (\semexpr{\sheap}{\senv}, \semexpr{\sstore}{\senv}, \semexpr{\sctx}{\senv})$
\end{tabular}
}
\caption{Interpretation of symbolic expressions, heaps, stores, and contexts.\label{fig:symbolic:interp}}
\end{figure}

\myparagraph{Soundness} To establish the soundness of symbolic execution, we need to relate 
symbolic states to concrete states. To this end, we make use of \emph{symbolic environments} 
$\senv : \svars \rightharpoonup \lits$ mapping symbolic values to \jsil literals. 
A symbolic environment is said to be \emph{consistent} if it maps symbolic 
values to concrete values of the appropriate type (e.g. symbolic strings are mapped to strings 
and symbolic numbers are mapped to numbers). In the following, we will always 
assume consistent symbolic environments. 
%
Given a symbolic environment $\senv$, we define the interpretation of symbolic 
expressions, heaps, stores, and contexts as shown in Figure~\ref{fig:symbolic:interp}. 
In the following, we write $\senv \vdash \pc$  if and only if $\semexpr{\pc}{\senv} = \ltrue$. For convenience, we define: 
\begin{align}
\smodels{\sheap, \sstore}{\pc} = \left\{ (\heap, \store) \mid \exists \senv \, . \,  \semexpr{(\sheap, \sstore)}{\senv} = (\heap, \store) \, \wedge \,  \senv \vdash \pc  \right\}  
\\
\smodels{\sheap, \sstore, \sctx}{\pc} = \left\{ (\heap, \store, \ctx) \mid \exists \senv \, . \,  \semexpr{(\sheap, \sstore, \sctx)}{\senv} = (\heap, \store, \ctx) \, \wedge \,  \senv \vdash \pc  \right\} 
\end{align}

\begin{theorem}[Soundness of the \jsil symbolic execution]\label{teo:soundness:jsil:symb:exe}
$$
\begin{array}{l}
\symbtranstrans{\sheap, \sstore, i, \pc}{\sheap', \sstore', i', \pc'}[\sctx][\sctx'] 
   \ \wedge \ 
      (\heap, \store, \ctx) \in \smodels{\sheap, \sstore, \sctx}{\pc'} \\ \quad \quad
      	 \ \Rightarrow \ \exists (\heap', \store', \ctx') \, . \, 
	 	 \semtranstrans{\heap, \store, i}{\heap', \store', i'}[\ctx][\ctx']
		\, \wedge \, 
		(\heap', \store', \ctx') \in \smodels{\sheap', \sstore', \sctx'}{\pc'}  
\end{array}
$$
\end{theorem}



\subsection{Implementation}

\polish{The point here is to explain how writing a correct concrete \jsil interpreter in Rosette
yields the symbolic environments presented in the previous subsection.} 

Ideally I would like to talk about: 
\begin{itemize}
   \item how do we represent the \jsil state in Rosette? why did we choose this representation? 
   \item how do we represent \jsil programs as s-expressions? 
   \item ...
\end{itemize}

\begin{display}{Rosette implementation of \jsil symbolic state}
{\scriptsize
\begin{mathpar}
\inferrule[\textsc{Empty Heap}]
  {}{\roscomp{\hemp} \semeq (\racketlist)} 
\and 
\inferrule[\textsc{Non-empty Heap}]
  {
  	 \sheap_1 = \big((l, \sexprp_i) \mapsto \sexprv_i\big)\mid_{i = 0}^n   
	 \quad 
	 (\loc, -) \not\in \domain(\sheap_2)
  }{\roscomp{\sheap_1 \dunion \sheap_2} \semeq  (\racketcons (\racketcons \loc \, (\racketlist \, (\racketcons \sexprp_0 \, \sexprv_0) \cdots   (\racketcons \sexprp_n \, \sexprv_n)))  \ \roscomp{\sheap_2})} 
 \\
\inferrule[\textsc{Empty Store}]
  {}{\roscomp{\storeemp} \semeq (\racketlist)} 
\and 
\inferrule[\textsc{Non-Empty Store}]
  {}{\roscomp{(\jvar: \sexpr) \dunion \sstore} \semeq (\racketcons \, (\racketcons \, (\racketquote \jvar) \ \sexpr) \,  \roscomp{\sstore})} 
\\ 
\inferrule[\textsc{Empty Context}]
  {}{\roscomp{\lstemp} \semeq (\racketlist)} 
\quad 
\inferrule[\textsc{Non-Empty Context}]
  {}{\roscomp{(\fid, \sstore, \jvar, i, j) \lstcons \sctx} \semeq  (\racketcons \,  (\racketlist \, (\racketquote \fid) \, \roscomp{\sstore} \, (\racketquote \jvar) \, i \, j) \, \roscomp{\sctx})} 
\end{mathpar}}
\end{display}

\lstset{language=Scheme}

\begin{figure}
\begin{lstlisting}
(define (mutate-prop-val-list prop-val-list prop new-val)
  (cond
    [(null? prop-val-list)
     (list (cons prop new-val))]
    [(equal? (car (car prop-val-list)) prop)
     (cons (cons prop new-val) (cdr prop-val-list))]
    [ else
     (cons (car prop-val-list) (mutate-prop-val-list (cdr prop-val-list) prop new-val))]))

(define (mutate-heap heap loc prop val)
  (define (mutate-heap-pulp h-pulp loc prop val)
    (cond
      [(null? h-pulp)
       (list (cons loc (list (cons prop val))))]
      [(equal? (car (car h-pulp)) loc)
       (cons (cons loc (mutate-prop-val-list (cdr (car h-pulp)) prop val)) (cdr h-pulp))]
      [ else
       (cons (car h-pulp) (mutate-heap-pulp (cdr h-pulp) loc prop val))]))
  (let ((new-heap-pulp (mutate-heap-pulp (unbox heap) loc prop val)))
    (set-box! heap new-heap-pulp)))


(define (run-bcmd bcmd heap store)
  (let ((cmd-type (first bcmd)))
    (cond
    	[(eq? cmd-type 'h-assign)
      	 (let* ((loc-val (run-expr (second bcmd) store))
                (prop-val (run-expr (third bcmd) store))
                (rhs-val (run-expr  (fourth bcmd) store)))
            (mutate-heap heap loc-val prop-val rhs-val)
            rhs-val)]
         ...)))
\end{lstlisting}
\caption{Fragment of \jsil Interpreter in Rosette}
\end{figure}


\section{Symbolic Execution for JavaScript}

\subsection{Symbolic Execution by Compilation} 

\subsection{Motivating Example} 

We illustrate how Jilette is used to write symbolic tests for JavaScript code by using the JavaScript implementation 
of a  \emph{key-value map} given in Figure~\ref{map:example} (left). 
It contains four functions: 
\jsinline|Map|, for constructing an empty map;
\jsinline|get|, for retrieving the value associated with the key given as input;
\jsinline|put|, for inserting a new \emph{key-value pair} into the map and updating existing keys; and
\jsinline|validKey|, for deciding whether a key is valid.
This library implements a \emph{key-value map} as an object with property \jsinline|_contents|, denoting the object used to store the map contents.  
The named properties of \jsinline|_contents| and their value attributes correspond to the map keys and values, respectively.
As the functions \jsinline|get|, \jsinline|put|, and \jsinline|validKey| are to be shared between all map 
objects, they are defined as properties of \jsinline|Map.prototype|, which is the prototype 
of the objects that are created using \jsinline|Map| as a constructor (e.g.~using~\jsinline|new Map()|). 

 \begin{figure}[t!]
 \begin{lstjs}[firstnumber=1]
function Map () { this._contents = {} }

Map.prototype.get = function (k) {
    if (this._contents.hasOwnProperty(k)) {  return this._contents[k] } 
    	else { return null }  
}

Map.prototype.put = function (k, v) {
   var contents = this._contents;
   if (this.validKey(k)) {  contents[k] = v   } 
   	else { throw new Error("Invalid Key") } 
} 

Map.prototype.validKey = function (k) { ... }
\end{lstjs}
\caption{Map Implementation in JavaScript}
\end{figure}

Note that one can insert a key-value pair with \jsinline|"hasOwnProperty"| as a key into the map. 
By doing this, \jsinline|"hasOwnProperty"| in the prototype chain of
\jsinline|_contents| is overridden and subsequent calls to \jsinline|get| will fail. 
Running the symbolic test below reveals this bug. In fact, \jilette gives 
\jsinline|__s1 = "hasOwnProperty"| as a failing model for the symbolic tetst below. 
%
 \begin{lstjs}[firstnumber=1]
var m = new Map();  m.put (__s1, __n1); var r = m.get(__s1);  
assert(__n1 = r)
\end{lstjs}




\newpage
\section{Evaluation}\label{sec:evaluation}
%!TEX root = ../main.tex

We discuss the trustworthiness of \cosette and demonstrate that~our implementation, despite being  a proof-of-concept, has already proven useful for the debugging of real-world JavaScript code.
We elaborate on the results presented below in more detail in the Appendix. 

\myparagraph{Trustworthiness: JavaScript Semantics}
To ensure that Cosette follows the semantics of JavaScript without any simplifications, we tested \JSComp and our instrumented \jsil interpreter implemented in Rosette using Test262, the JavaScript official test suite~\cite{test262}. 
Out of the 10469 tests for ES5 Strict, we have identified 8330 tests appropriate for our coverage, of which we pass 100\%.

\myparagraph{Trustworthiness: Symbolic Interpreter} To make certain that the symbolic \jsil interpreter obtained by the Rosette lifting of the implemented instrumented interpreter is consistent with the symbolic semantics of \S\ref{subsec:symb:semantics}, we systematically constructed and successfully ran symbolic unit tests for each \jsil command, assuming the premises and asserting the conclusion of the appropriate rule of the symbolic semantics.

%With \cosette, we can write symbolic tests, in which some of the concrete values of the program are replaced with symbolic values.
%Symbolic tests improve on concrete tests for two main reasons.
%First, they are by construction more comprehensive than concrete tests, because symbolic tests can account for the whole range of values that a variable can take, instead of focusing on a few specific examples.
%Second, when \cosette finds a failing assertion inside a symbolic test, it can concretize the symbolic values into a counter-model that the developer can actually run in node, making debugging much easier compared to (the other things that we mention before).

\myparagraph{Whole-program Symbolic Testing: JS-Specific Features}
We created a number of symbolic tests to demonstrate that Cosette can reason about essential JavaScript features, such as prototype inheritance, function closures, arrays, strings, as well as the substantially more challenging for-in statement and dynamic dispatch. 

\myparagraph{Whole-program Symbolic Testing: Real-World Libraries}
We used \cosette to analyse the code of two JavaScript data structure libraries: Buckets.js~\cite{buckets}, and queue-pri~\cite{priq}.
We chose these libraries because reasoning about data structure code requires a precise description of the control flow features of JavaScript, because they come equipped with unit test suites, and because they do not have external dependencies (\cosette is a whole-program analysis); Buckets.js has over 65k downloads on npm.

For these two libraries, we wrote symbolic tests with the aim of obtaining a line coverage of 100\%, in order to compare them with the concrete unit tests that ship with the libraries.
In both cases, we were able to reduce the length of the tests by up to an average factor of 3, while increasing line coverage from around 90\% to a full 100\%.
We also discovered one bug in the Buckets.js library, as well as one in the queue-pri library.


The results are presented in table~\ref{cosette:res}.
For each file in the library, we report the number of JS executable lines in the code itself and including dependencies (slash-separated), the corresponding numbers of JSIL lines, the number of symbolic and concrete test cases, the number of JS lines in the symbolic and concrete tests, the coverage measured as percentage of lines and the average \cosette run time for the symbolic tests.
The files in Buckets.js are separated by a line from the unique file in queue-pri.

For the testing, we used a machine with an Intel Core i7-4980HQ CPU 2.80 GHz and DDR3 RAM 16GB. We measured the execution time of each symbolic test and averaged the times across tests for each library file. The times that we obtained reflect the fact that Rosette code is interpreted, rather than being run natively. We aim at implementing our own symbolic execution tool from scratch in the future, which, given our experience with JaVerT, should reduce execution times by at least an order of magnitude.

\begin{table}[!t]
{
\small
%\begin{center}
\setlength\tabcolsep{4pt}
\begin{tabular*}{\linewidth}{l@{\;\;}rrrrrr}
\toprule
% Name || JS Loc/loc* || JSIL Loc/loc* || #tests || symb/conc loc || symb/conc cov || time
Name & \makecell{JS lines} & \makecell{JSIL lines} & \# Tests & \makecell{Test lines} & \makecell{Line\\Cov.~(\%)} & \makecell{Avg.\\time} \\
\midrule
\texttt{arrays} & 44/71 & 1251/1942 & 9/24 & 166/329 & 100/100 & 20s \\
\texttt{bag} & 69/237 & 2041/7194 & 7/18 & 78/265 & 100/76.8 & 74s \\
\texttt{bstree} & 143/326 & 3819/8052 & 11/31 & 216/759 & 100/98.6 & 5m27s \\
\texttt{dict} & 57/84 & 1683/2374 & 7/14 & 116/170 & 100/80.7 & 15s \\
\texttt{heap} & 57/128 & 2059/4001 & 4/15 & 92/626 & 100/96.5 & 5m29s \\
\texttt{llist} & 126/153 & 2447/3138 & 9/21 & 149/370 & 100/94.4 & 24s \\
\texttt{multidict} & 56/184 & 1871/5496 & 6/16 & 118/189 & 100/74.1 & 1m15s \\
\texttt{pqueue} & 26/154 & 1066/5067 & 5/12 & 70/283 & 100/96.2 & 5m49s \\
\texttt{queue} & 30/183 & 1095/4233 & 6/9 & 111/146 & 100/96.7 & 20s \\
\texttt{set} & 40/124 & 1528/3902 & 6/12 & 86/271 & 100/70.0 & 1m01s \\
\texttt{stack} & 23/176 & 941/4079 & 4/7 & 91/104 & 100/87.0 & 26s \\
\midrule 
\texttt{queue-pri} & 19/164 & 872/5086 & 2/9 & 26/80 & 100/100 & xy.z \\
\bottomrule
%\end{center}
\end{tabular*}
}
\caption{Tests for the Buckets.js and queue-pri libraries}
\vspace*{-0.95cm}
\label{cosette:res}
\end{table}

%\pmax{Say something about how bugs work - sometimes coverage, sometimes semantics.}

\smallskip
\noindent \emph{Bug: MultiDictionary in Buckets.js.}
We have discovered a bug in the implementation of the Buckets.js multi-dictionary library.
A multi-dictionary is a key-value map in which a single key holds an array of distinct values. 
Our symbolic tests for the \jsinline|remove(key, value)| function, which removes a given key-value pair from the multi-dictionary, have revealed that the library wrongly treats the case in which we try to remove a key-value pair for a key with no associated values.
Concretely, a runtime error is thrown instead of \jsinline|remove| returning \jsinline|false|. 
This bug was not detected by the concrete unit tests associated with the library due to their incomplete coverage;
we have fixed it and submitted an appropriate pull request.




\smallskip
\noindent \emph{Bug: queue-pri.} This library implements a priority queue that stores data with an optional priority value.
The priority can either be a number (the lower the value, the higher the priority) or the default \jsinline{null} value if no priority is provided, in which case the associated element is put at the end of the queue. Our symbolic tests of the \jsinline{enqueue(data, pri)} method of the library have shown that that elements enqueued with priority \jsinline{0} were wrongly being always enqueued at the end of the queue. We traced the bug to the way in which priority was calculated inside \jsinline{enqueue}: \jsinline{priority = pri || null}, which evaluates to \jsinline|null| if the priority is not supplied, but also, due to the semantics of JavaScript, if it is equal to 0. This bug was not caught by the unit tests of the library because the developer had not considered inserting nodes with priority 0. This shows that \cosette is a useful tool for symbolic testing, because it fully follows the semantics of JavaScript and will expose corner cases that a developer may not be aware of.

\myparagraph{Specification-directed Bug-finding} 

\pmax{JaVerT examples - testing the unfolding, introducing bugs}

\bibliographystyle{abbrv}
\bibliography{javert}

\end{document}
