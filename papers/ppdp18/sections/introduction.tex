%!TEX root = ../main.tex

JavaScript is the most widespread dynamic language: it is the de facto language for client-side Web applications (used by 94.8\% of websites \cite{JS948percent});
it is used for server-side scripting via Node.js; and it is even run on small embedded devices with limited 
memory. It is the most active language in both GitHub \cite{GithubActive} and StackOverflow \cite{SOActive}.
The dynamic nature of JavaScript and its complex semantics make it a difficult target for
symbolic analysis and logic-based verification. 
This paper presents \jilette, a symbolic execution tool for JavaScript (ECMAScript 5, ES5~\cite{ecma}).
%
We highlight two relevant use cases for \jilette. First, we show how \jilette can be used as \dtag{i}~a tool for running symbolic tests for JavaScript programs; and \dtag{ii} a debugging tool for separation logic specifications of JavaScript programs. 

\myparagraph{Architecture}
The core of \jilette consists of a symbolic interpreter for
\jsil~\cite{javert}, a simple intermediate goto language. 
We obtain this symbolic interpreter \emph{for free}, 
by implementing a concrete \jsil interpreter in Rosette~\cite{Rosette2,Rosette1},~a 
symbolic virtual machine that facilitates generation of solver-aided languages.
We design the concrete interpreter so that all of Rosette's natively supported solver-aided
features, such as advanced string and regular-expression reasoning, 
are lifted to the \jsil symbolic interpreter. 
In~\S\ref{sec:jsil:symb:exec}, we give a formalisation of the \jsil concrete and symbolic executions, linking them together with a {\em soundness result}. We also provide insights on how to correctly design the concrete \jsil interpreter in Rosette.

The second component that \jilette uses is \JSComp~\cite{javert}, 
a well-tested, standard-compliant compiler from JavaScript to \jsil. We extend
\JSComp with support for the non-strict mode of ES5, as well as
regular expressions and the entire \jsinline|String| built-in library.
\JSComp allows us to lift the \jsil symbolic execution to JavaScript by first compiling JavaScript code to \jsil code, and
then symbolically executing the compiled code in the 
\jsil symbolic interpreter. This process, described in \S\ref{symb:exec:comp},
involves extending JavaScript syntax and the \JSComp compiler to support symbolic values and 
constructs for reasoning about them. These constructs are intuitive
and allow the general developer to easily write assertions about the behaviour
of their program. 
Moreover, we adjust the \jsil symbolic interpreter so that the abstraction level 
of the generated \jsil code precisely matches the abstraction level of Rosette, 
 maximising the use of Rosette's native reasoning capabilities.

\myparagraph{Application: Symbolic Testing} A commonly used 
approach to obtaining trust in JavaScript code is running it against 
adhoc test batteries---verifying that given concrete inputs, the code produces the expected
output. The main drawback of this approach is that tests, in general,
cannot guarantee exhaustiveness. % we also cant guarantee exhaustiveness 
In \S\ref{symbolic:testing}, we show how to use \jilette
for symbolic testing of JavaScript code: instead of 
tests with concrete 
inputs, the developer uses symbolic inputs and states the 
constraints that the output needs to satisfy as simple, intuitive 
first-order assertions over these inputs. 
Furthermore, if a test fails, \jilette provides the concrete inputs that cause it 
to fail, exposing bugs in the tested code. 
We highlight the capabilities of \jilette through examples that showcase
challenging reasoning on strings, regular expressions, and the \jsinline|eval|
statement.

\myparagraph{Application: Debugging Separation Logic Specifications}
Due to the complexity of JavaScript semantics, functional correctness 
specifications of JS programs are highly intricate. 
There are only a few tools (for example, \javert \cite{javert} and KJS \cite{Park:2015,stefanescu-park-yuwen-li-rosu-2016-oopsla}) that support such expressivity. They target the specialist developer wanting rich, 
mechanically verified specifications of critical JavaScript code.
However, when these 
tools cannot prove that a given function satisfies a specification, to discover the error, 
the developer needs to understand in detail a complicated proof trace (\javert), or even act with almost no feedback~(KJS). 

In \S\ref{sec:specs}, we show how \jilette can be used as an auxiliary mechanism for debugging 
separation logic specifications of JavaScript programs in \javert. 
Our approach consists of: translating the separation logic specifications 
into symbolic tests 
and running these tests using \jilette. 
Then, if a symbolic test generated from a given specification fails, we can 
be sure that the code to be verified does not satisfy its specification. 
More importantly, \jilette then generates a concrete witness that 
invalidates the specification. This information greatly simplifies the debugging of 
both specifications and code. 

\myparagraphq{Why \jilette} 
\jilette is \emph{useful}: it has tangible applications. 
It can report bugs in JavaScript programs, producing concrete witnesses triggering the bugs. It can also be used as a helper tool for developers of logic-based functional correctness specifications of JavaScript code.
\jilette is \emph{approachable}: it can easily be used by a general JavaScript developer. The annotation burden of \jilette is minimal and the assertion language is simple and intuitive. \polish{Sweet spot?}
\jilette is \emph{trustworthy}: its components come with correctness guarantees. 
The correctness of the \JSComp compiler ensures full adherence to the real semantics of JavaScript. The \jilette symbolic execution engine is based on a sound symbolic
analysis for \jsil, guaranteeing the absence of false positives. \polish{Sentence about unification.}
Finally, \jilette is \emph{extensible}: its coverage can easily be extended in a modular way. This gives us the mechanism for supporting built-in libraries not covered by \JSComp, or adding support for standard-external runtime libraries, such as the DOM.