%!TEX root = ../main.tex

\section{Evaluation}

\subsection{Challenging JavaScript examples}

\cosette is able to verify JavaScript programs that use nontrivial parts of the JavaScript semantics, such as dynamic dispatch and property enumeration (via \jsinline{for-in} loops), in the symbolic world.

\myparagraph{Dynamic dispatch}
The following example demonstrates that \cosette is able to bring together complex symbolic string and mathematical reasoning with JavaScript's memory model.

\begin{lstjs}
var o = Object.create(null);

o.plusOne = function(x) { return x + 1 };
o.minusOne = function(x) { return x - 1 };

var s1 = symb_string(), s2 = symb_string();
var n1 = symb_number(), n2 = symb_number();

Assume(not (n1 = n2));

var total1 = o[s1](n1)
var total2 = o[s2](n2);

Assert(total1 = total2);
\end{lstjs}

In this example, we create an object \jsinline{o} that contains two functions, \jsinline{o.plusOne} and \jsinline{o.minusOne}, which respectively add and subtract $1$ from their argument.
We create two symbolic strings \jsinline{s1} and \jsinline{s2}, and two distinct symbolic numbers \jsinline{n1} and {n2}.
Then, we compute the result of applying function \jsinline{o[s1]} to \jsinline{n1}, and \jsinline{o[s2]} to \jsinline{n2}, stating with the \jsinline{Assert} statement that we want both these values to be equal.

According to the different values that the strings \jsinline{s1} and \jsinline{s2} take,  different outcomes can happen.
If \jsinline{s1} (or \jsinline{s2}) is not a valid property name of \jsinline{o}, the program will give a runtime error, because \jsinline{o[s1]} (or \jsinline{o[s2]}) evaluates to \jsinline{undefined}, which is not a function, and cannot be applied; when running the program with \cosette, we indeed get a concrete model for this failing case.

If both \jsinline{s1} and \jsinline{s2} are valid property names for \jsinline{o}, the program is always executed until its end.
In that case, \cosette is able to present us with a model for the final assertion to hold (for example, \jsinline{s1 = "minusOne", s2 = "plusOne", n1 = 0}, and \jsinline{n2 = -2}), as well as a model that invalidates the final assertion (e.g.~\jsinline{s1 = "minusOne"}, \jsinline{s2 = "minusOne", n1 = 0} and \jsinline{n2 = 1}).

This example demonstrates that \cosette has a fine understanding of the mechanics that underly the dynamic dispatch system in JavaScript, and is able to not only infer possible dynamic function names for a property function call to work, but is also able to relate the dynamic name to the function output.


\myparagraph{Property enumeration}
The following example demonstrates that \cosette is able to reason about the interplay of static and dynamic properties of JavaScript objects.

\begin{lstjs}
function nbProp(o) {
  var count = 0;
  for (var p in o) { count++ }
  return count;
}

var o = {a: 1, b: 2, c: 3};

var s = symb_string();

o[s] = 4;
var res = nbProp(o);
var expectedRes = 3;

Assert(res = expectedRes);
\end{lstjs}

In this example, we create a JavaScript object \jsinline{o} that has three properties \jsinline{a, b}, and \jsinline{c}.
Then, we create a symbolic string \jsinline{s} and assign a value to the property of \jsinline{o} corresponding to {s}.
Finally, we count the actual number of properties in \jsinline{o}.
There are two possible cases here: either \jsinline{s} is equal to one of the concrete property names of \jsinline{o}, and the assignment actually overwrites the value associated with that property, or \jsinline{s} is a fresh string and the assignment creates a new concrete property.

\cosette is able to reason about both cases, and finds a model for the assertion; either shadowing the property if \jsinline{expectedRes = 3}, giving a fresh string if \jsinline{expectedRes = 4}, or saying that the assertiong is unsatisfiable if \jsinline{expectedRes} is different from these two values.

%This shows that \cosette is able to reason about the interplay between static and dynamic properties of JavaScript objects.
 
\subsection{Bug: Multidictionary in Buckets.js}
We found one bug in the implementation of the Buckets.js multi-dictionary (a key-value map in which a single key holds an array of distinct values). 
%\myparagraph{\jsinline{multidictionary}}
%The implementation of the Buckets.js \jsinline{multidictionary} essentially comprises the \jsinline{get(key)} (which returns an array of values), \jsinline{set(key, value)}, and \jsinline{remove(key, value)} methods.
The implementation of the Buckets.js multi-dictionary essentially consists of the \jsinline{get(key)}, \jsinline{set(key, value)}, and \jsinline{remove(key, value)} methods.
Both \jsinline{set} and \jsinline{remove} either add or remove a single value from the array held by the key, and \jsinline{get} returns this array.
More precisely, in the \jsinline{remove} function, the \jsinline{value} argument can either be \jsinline{undefined}, in which case the key and associated array are completely removed from the dictionary, or it can be an actual value, in which case it is removed from the array if present.
However, the library does not consider the case where the array is not found (for example, if all elements associated with the key have already been removed); in that case, the library tries to remove an element from \jsinline{undefined}, which raises an error.
A simplified version of the failing code is shown below:

\begin{lstjs}
multidict.remove = function (key, value) {
    if (value === undefined) { ... }
    // parent is an internal dictionary that maps keys to arrays
    var array = parent.get(key);
    if (arrays.remove(array, value)) { ... }
    return false;
}
\end{lstjs}
We were able to expose this bug with the following symbolic test:

\begin{lstjs}
var dict = new buckets.multidict()
var s = symb_string(), x1 = symb_number(), x2 = symb_number();
dict.set(s, x1); dict.set(s, x2);
dict.remove(s, x1);
var didRemove = dict.remove(s, x2);
assert((x1 != x2 && didRemove) || (x1 == x2 && !didRemove));
\end{lstjs}

In this test, we create a new dictionary \jsinline{dict}, and insert two symbolic values \jsinline{x1} and \jsinline{x2} at key \jsinline{s}, then remove them in order.
However, if \jsinline{x1} is actually equal to \jsinline{x2}, the implementation does not store it twice.
This means that the key in the second call to \jsinline{remove} on line 6 is not in the dictionary any more, which triggers the bug.

When running \cosette on this test, we obtain the counter-model \jsinline{x1 = x2 = 0} (any value of \jsinline{s} works), and running the instantiation of the test in Node raises an error.
We were able to fix the bug by adding a check for \jsinline{undefined} after line 3 in the code of \jsinline{remove}, after which \cosette was successfully able to discharge the assertion.


