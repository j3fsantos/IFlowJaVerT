\documentclass[acmsmall,review,anonymous]{acmart}\settopmatter{printfolios=true,printccs=false,printacmref=false}

\acmJournal{PACMPL}
\acmVolume{1}
\acmNumber{CONF} % CONF = POPL or ICFP or OOPSLA
\acmArticle{1}
\acmYear{2018}
\acmMonth{1}
\acmDOI{} % \acmDOI{10.1145/nnnnnnn.nnnnnnn}
\startPage{1}

\setcopyright{none}

\bibliographystyle{ACM-Reference-Format}
\citestyle{acmauthoryear}  

\usepackage{enumitem}
\setlist[description]{leftmargin=\parindent,labelindent=\parindent}

\usepackage{xcolor}
%\usepackage{amsmath}
\usepackage{listingsutf8}
\usepackage{hyperref}
\usepackage{graphicx}
\usepackage{tabularx}
\usepackage{xspace}
\usepackage{textcomp}
\usepackage{wasysym,stmaryrd}
\usepackage{mathpartir}
\usepackage{url}
%\usepackage{upgreek}
\usepackage{xparse}
\usepackage{booktabs}
\usepackage[utf8]{inputenc}
\usepackage[T1]{fontenc}
%\usepackage{stix}

%\usepackage{algorithm}
%\usepackage{algpseudocode}

\usepackage{wrapfig}
\usepackage{fancyvrb}

\makeatletter
\newif\ifFV@bgcolor
\newbox\FV@bgbox
\define@key{FV}{bgcolor}{\FV@bgcolortrue\def\FV@bgcolor{#1}}

\def\FV@BeginVBox{%
  \leavevmode\ifFV@bgcolor\setbox\FV@bgbox=\fi
  \hbox\ifx\FV@boxwidth\relax\else to\FV@boxwidth\fi\bgroup
  \ifcase\FV@baseline\vbox\or\vtop\or$\vcenter\fi\bgroup}
\def\FV@EndVBox{\egroup\ifmmode$\fi\hfil\egroup
  \ifFV@bgcolor\colorbox{\FV@bgcolor}{\box\FV@bgbox}\fi}
\makeatother

\newcommand{\shat}{\^{s}}

%JavaScript 
\definecolor{SkyBlue}{rgb}{0.20,0.39,0.64}
\definecolor{Plum}{rgb}{0.46,0.31,0.48}
\definecolor{Chocolate}{rgb}{0.75,0.49,0.07}
\definecolor{Aluminium5}{rgb}{0.33,0.34,0.32}
\definecolor{DarkGreen}{rgb}{0.2,0.5,0.2}
\definecolor{ltblue}{rgb}{0,0.4,0.4}
\definecolor{dkblue}{rgb}{0,0.2,0.7}
\definecolor{dkgreen}{rgb}{0,0.4,0}
\definecolor{dkviolet}{rgb}{0.3,0,0.5}
\definecolor{dkred}{rgb}{0.6,0,0}
\definecolor{talkred}{rgb}{0.69,.20,0.22}
\definecolor{talkblue}{rgb}{0.04,0.40,0.80}
\definecolor{talkgreen}{rgb}{0.34,.81,0.10}
\definecolor{oldtalkblue}{rgb}{0.22,.20,0.69}
\definecolor{greenish}{rgb}{.0,.65,.0}
\definecolor{mygray}{gray}{0.9}

\lstset{
	showstringspaces=false
}

\lstdefinelanguage{JavaScript}{
  morekeywords=[1]{typeof, new, true, false, catch,
    function, return, null, catch, switch, var,
    if, in, while, do, else, case, break, continue},
  morekeywords=[2]{class, export, boolean, throw, implements, import, this},
  numbers=left,
  numbersep=4pt,
  numberstyle=\tiny\color{dkblue},
  columns=fullflexible,
  sensitive=false,
  comment=[l]{//},
  captionpos=b,   
  morecomment=[s]{/*}{*/},
  morestring=[b]',
  morestring=[b]",
  basicstyle=\scriptsize\texttt,
  identifierstyle=\ttfamily\color{Aluminium5},
  keywordstyle=[1]\ttfamily\color{Plum},
  keywordstyle=[2]\ttfamily\color{SkyBlue},
  stringstyle=\ttfamily\color{DarkGreen},
  commentstyle=\ttfamily, 
%  commandchars=\$\{\}
}[keywords,comments,strings]

\lstdefinelanguage{Scheme}{
  morekeywords=[1]{define, define-syntax, define-macro, lambda, define-stream, stream-lambda},
  morekeywords=[2]{begin, call-with-current-continuation, call/cc,
    call-with-input-file, call-with-output-file, case, cond,
    do, else, for-each, if,
    let*, let, let-syntax, letrec, letrec-syntax,
    let-values, let*-values,
    and, or, not, delay, force,
    quasiquote, quote, unquote, unquote-splicing,
    map, fold, syntax, syntax-rules, eval, environment, query },
  morekeywords=[3]{import, export},
  alsodigit=!\$\%&*+-./:<=>?@^_~,
  sensitive=true,
  morecomment=[l]{;},
  morecomment=[s]{\#|}{|\#},
  morestring=[b]",
  basicstyle=\scriptsize\ttfamily,
  keywordstyle=\bf\ttfamily\color[rgb]{0,.3,.7},
  commentstyle=\color[rgb]{0.133,0.545,0.133},
  stringstyle={\color[rgb]{0.75,0.49,0.07}},
  upquote=true,
  breaklines=true,
  breakatwhitespace=true,
  literate=*{`}{{`}}{1}
}

\def\schemeinline{\lstinline[language=Scheme, basicstyle=\small\ttfamily]}

\lstnewenvironment{lstjs}{\lstset{language=JavaScript,basicstyle=\fontsize{8}{8}\ttfamily,escapeinside={~}{~}}}{}
\def\jsinline{\lstinline[language=JavaScript, basicstyle=\small]}


% The Acronyms of the project and some other stuff
\newcommand{\jsil}{JSIL\xspace}
\newcommand{\jsverify}{JSVerify\xspace}
\newcommand{\JSComp}{JS-2-JSIL\xspace}
\newcommand{\jsilverify}{JSILVerify\xspace}


% Tikz 
\usepackage{tikz}
\usetikzlibrary{calc,positioning,arrows,shapes,decorations.pathmorphing}
\usetikzlibrary{arrows,positioning} 
\tikzset{
    %Define standard arrow tip
    >=stealth',
    % Define arrow style
    pil/.style={
           ->,
           shorten <=2pt,
           shorten >=2pt,}
}

\newcommand{\runpic}{\includegraphics[width=0.06\picwidth]{running.pdf}}
\newcommand{\tickpic}{\resizebox{0.06\picwidth}{!}{\(\color{greenish} \checkmark \)}}
\tikzset{
  box/.style = {rectangle, draw=black,align=center,font=\scriptsize},
  sbox/.style = {rectangle,draw=black,align=left,font=\scriptsize,text width=1.7cm},
  p/.style = {-latex},
  dp/.style = {latex-latex},
  sz/.style n args={2}{minimum width=#2, minimum height=#1},
  m/.style = {midway,inner sep=0pt,fill=white},
  ll/.style = {font=\scriptsize,anchor=south west}
}



% Polishing...
\newcommand{\polish}[1]{{\color{red}#1}}


\usepackage{cosette_macros}


% macros_js as for Jose Santos
%\usepackage{macros_js}
%\usepackage{gdshojs}

\newcommand{\jilette}{Cosette\xspace}
\newcommand{\rosette}{Rosette\xspace}

\newcommand{\myparagraph}[1]{\smallskip\noindent {\bf #1.}\hspace{1pt}}
\newcommand{\myparagraphq}[1]{\smallskip\noindent {\bf #1?}\hspace{1pt}}

% COMMENTS

\newcommand{\pginline}[1]{ {\color{red} *** PG : #1 ***} }
\newcommand{\pmaxinline}[1]{ {\color{blue} *** PM : #1 ***} }
\newcommand{\jfsinline}[1]{ {\color{green} *** JFS : #1 ***} }
\newcommand{\jdinline}[1]{ {\color{purple} *** JD : #1 ***} }

\newif\ifComments
\Commentstrue

\newcommand{\pg}[1]{%
\ifComments
\begin{center}
\fbox{\begin{minipage}{0.95\textwidth} \color{red}
{\rm PG: \small #1}
\end{minipage}}
\end{center}
\fi}

\newcommand{\pmax}[1]{%
\ifComments
\begin{center}
\fbox{\begin{minipage}{0.95\textwidth} \color{blue}
{\rm PM: \small #1}
\end{minipage}}
\end{center}
\fi}

\newcommand{\jfs}[1]{%
\ifComments
\begin{center}
\fbox{\begin{minipage}{0.95\textwidth} \color{SkyBlue}
{\rm JFS: \small #1}
\end{minipage}}
\end{center}
\fi}

\newcommand{\jd}[1]{%
\ifComments
\begin{center}
\fbox{\begin{minipage}{0.95\textwidth} \color{purple}
{\rm JD: \small #1}
\end{minipage}}
\end{center}
\fi}

%% Title information
\title[Short Title]{\jilette:~Compositional Symbolic Execution for JavaScript}         
%\titlenote{with title note}           
%\subtitle{Subtitle}                     
%\subtitlenote{with subtitle note}   

%% Author with single affiliation.
\author{First1 Last1}
\authornote{with author1 note}          %% \authornote is optional;
                                        %% can be repeated if necessary
\orcid{nnnn-nnnn-nnnn-nnnn}             %% \orcid is optional
\affiliation{
  \position{Position1}
  \department{Department1}              %% \department is recommended
  \institution{Institution1}            %% \institution is required
  \streetaddress{Street1 Address1}
  \city{City1}
  \state{State1}
  \postcode{Post-Code1}
  \country{Country1}                    %% \country is recommended
}
\email{first1.last1@inst1.edu}          %% \email is recommended

%% Author with two affiliations and emails.
\author{First2 Last2}
\authornote{with author2 note}          %% \authornote is optional;
                                        %% can be repeated if necessary
\orcid{nnnn-nnnn-nnnn-nnnn}             %% \orcid is optional
\affiliation{
  \position{Position2a}
  \department{Department2a}             %% \department is recommended
  \institution{Institution2a}           %% \institution is required
  \streetaddress{Street2a Address2a}
  \city{City2a}
  \state{State2a}
  \postcode{Post-Code2a}
  \country{Country2a}                   %% \country is recommended
}
\email{first2.last2@inst2a.com}         %% \email is recommended
\affiliation{
  \position{Position2b}
  \department{Department2b}             %% \department is recommended
  \institution{Institution2b}           %% \institution is required
  \streetaddress{Street3b Address2b}
  \city{City2b}
  \state{State2b}
  \postcode{Post-Code2b}
  \country{Country2b}                   %% \country is recommended
}
\email{first2.last2@inst2b.org}         %% \email is recommended

%% Abstract
%% Note: \begin{abstract}...\end{abstract} environment must come
%% before \maketitle command
\begin{abstract}
We present \jilette, a framework for trustworthy, compositional symbolic execution of JavaScript programs.
\jilette is the first symbolic analysis tool for JavaScript that precisely models the semantics of the language.
Its aim is to assist the general programmer in debugging their code; the programmer writes symbolic tests for which \jilette provides concrete counter-models.
We prove that the symbolic execution underpinning \jilette is sound and that it does not generate false positives. Moreover, this symbolic execution is compositional, which allows us to use \jilette to find bugs in separation logic specifications of JavaScript programs that are not detectable by standard symbolic execution techniques.
%\pmaxinline{Now compositionality.}
%
We evaluate \jilette on an array of examples, including tests from the official JavaScript Test262 test suite and real-world Node.js libraries. These examples involve JavaScript-specific features, such as prototype inheritance, function closures, the for-in statement, and dynamic dispatch. We highlight the range of \jilette by using it to find bugs in a number of non-trivial separation logic specifications of JavaScript programs.


%
%It works by compiling JavaScript programs to \jsil using \JSComp, a well-tested, standard-compliant compiler from JavaScript to \jsil, and then symbolically executing the obtained \jsil code using a novel symbolic interpreter for \jsil. We prove that the \jsil symbolic interpreter is \emph{sound} and that it does not generate false positives.
%
%\bigskip
%We present \jilette, a symbolic execution tool for JavaScript (ECMAScript 5, ES5), which precisely follows the language standard. At the core of \jilette is a sound symbolic interpreter for \jsil, an intermediate language well-suited for verification and analysis. This interpreter is written in \underline{Rosette}, %~\cite{Rosette2,Rosette1}, 
%a symbolic virtual machine that enables the design of new solver-aided languages. 
%\jilette works by first compiling JavaScript programs to \jsil using \underline{\JSComp}, %~\cite{javert}, 
%a well-tested, standard-compliant compiler from JavaScript to \jsil, and then symbolically executing the compiled \jsil code in the \jsil symbolic interpreter. 
%We study two complementary uses of \jilette. 
%First, we show how \jilette can be used for symbolic testing of JavaScript programs by finding concrete executions that trigger assertion and test failures. 
%We highlight the range of \jilette by giving examples using strings, regular expressions, and the notorious \jsinline|eval| statement.
%Second, building on \jilette, we develop a tool for debugging separation logic specifications by compiling them to symbolic tests in order to find 
%witnesses for bugs in both specification and code.
%
%\bigskip
%We present \jilette, a framework for bounded symbolic execution of JavaScript (JS) code (ECMAScript 5, ES5). 
%\jilette is the first symbolic execution tool for JS that models the language semantics precisely.  
%It works by first compiling JS programs to \jsil using \JSComp, a well-tested, standard-compliant compiler from 
%JS to \jsil, and then symbolically executing the obtained \jsil code using a novel symbolic interpreter for \jsil. 
%We prove that the \jsil symbolic interpreter is \emph{sound} and that it does not to generate false positives. 
%%every time the tool reports a bug, it provides a concrete model that triggers that bug. 
%
%We demonstrate how \jilette can be used for the symbolic testing of JS programs by 
%finding concrete executions that trigger assertion and test failures. We highlight the range of \jilette 
%by giving examples involving JS-specific features, such as: prototype-based inheritance, 
%the for-in statement, and JS arrays. Finally, we thoroughly evaluate \jilette against a 
%representative fragment of test262 adapted to include symbolic~values.
%
\end{abstract}


%% 2012 ACM Computing Classification System (CSS) concepts
%% Generate at 'http://dl.acm.org/ccs/ccs.cfm'.
\begin{CCSXML}
<ccs2012>
<concept>
<concept_id>10011007.10011006.10011008</concept_id>
<concept_desc>Software and its engineering~General programming languages</concept_desc>
<concept_significance>500</concept_significance>
</concept>
<concept>
<concept_id>10003456.10003457.10003521.10003525</concept_id>
<concept_desc>Social and professional topics~History of programming languages</concept_desc>
<concept_significance>300</concept_significance>
</concept>
</ccs2012>
\end{CCSXML}

\ccsdesc[500]{Software and its engineering~General programming languages}
\ccsdesc[300]{Social and professional topics~History of programming languages}
%% End of generated code


%% Keywords
%% comma separated list
%\keywords{keyword1, keyword2, keyword3}  %% \keywords are mandatory in final camera-ready submission

%\usepackage{amsmath, amssymb, amscd,mathrsfs}
%\DeclareMathAlphabet{\mathbbm}{U}{bbm}{m}{n}


\begin{document}
%

\maketitle 

\section{Introduction}


In current program analysis research, there exists a gap between bug-finding techniques and verification techniques. In bug-finding, the aim is to discover concrete executions that cause a given program to behave incorrectly, and the emphasis is on precision, that is, the absence of false positive bug reports. In verification, the aim is to produce a proof that a given program always behaves correctly, and the emphasis is on soundness, which mandates adherence to the language standard and the exploration of all possible paths. 
%
Soundness, however, comes at a cost: verification techniques are often too conservative, rejecting many programs that are safe, and have prohibitive scalability issues.
%
There is a growing consensus among experts that soundness is, in fact, not a necessity for most modern analysis applications~\cite{.}. Instead, the property they advocate and observe in the  majority of tools is \emph{soundiness}---maintaining soundness as much as possible, but without detrimental effects to precision and/or scalability.

The main issue with most bug-finding tools is not that they are soundy, but that they are not \emph{trustworthy}: their soundiness and precision are not formally characterised. They are usually not justified with respect to the semantics of the targeted language, often relying on unstated simplifying assumptions, if not explicitly departing from the semantics of the language. Consequentially, the fact that a bug-finding tool cannot find a bug carries no guarantees with respect to the correctness of the input program. We trust these tools because they appear to work empirically.

Another difference between bug-finding and verification is that bug-finding tools are often based on whole-program analysis, whereas verification techniques are normally \emph{compositional}, meaning that the results of an analysis for a part of a program can be re-used when analysing the entire program. Compositionality is essential for tools that need to run regularly on large codebases. For instance, it would not be possible for Infer, a bug-finding tool based on separation logic, to be part of the continuous integration system at Facebook if it were not compositional. 

We present \jilette, the first trustworthy, compositional symbolic execution framework for JavaScript (ECMAScript 5 Strict~\cite{ecma}) analysis. \jilette precisely follows the ES5 English standard, without introducing any simplifications. 
It aims at the sweet spot between bug-finding and verification, 
%It is not done directly on JavaScript, but instead on \jsil, an intermediate representation previously introduced for JavaScript verification~\cite{javert}. \jsil comes with a trusted infrastructure that allows us to provably lift the results of symbolic analyses on compiled \jsil code back up to JavaScript. 
in that it comes with formal correctness guarantees and a precise result stating the conditions under which symbolic execution gives soundness. 

%Additionally, its compositionality allows us to catch a broader class of bugs than the current symbolic execution tools and also allows us to state the conditions under which the soundiness of \jilette becomes soundness, that is, when the absence of found bugs implies verification.

%However, compositionality is not trivial to achieve when the targeted programming language does not exhibit the frame property, as is the case for most dynamic languages including JavaScript.




%JavaScript is the most widespread dynamic language: it is the de facto language for client-side Web applications, used by 94.8\% of websites\footnote{\url{https://w3techs.com/technologies/details/cp-javascript/all/all}}; it is used for server-side scripting via Node.js; and it is even run on small embedded devices with limited memory. 
%It is the most active language in both GitHub and StackOverflow.\footnote{\url{http://githut.info}; \url{https://exploratory.io/viz/Hidetaka-Ko/94368d12800a?cb=1469037012628}}
%Due to its dynamic nature and complex semantics, JavaScript remains a difficult target for symbolic analysis and logic-based verification. 

%\pmax{Now, sth about the state-of-the-art}

% TO RELATED WORK
%Currently, several symbolic analysis tools for JavaScript are available, such as \pmaxinline{list them, cite them - are we doing static or dynamic also?}. \pmaxinline{Say something positive, don't know what, mention bug-finding}. However, there exists a gap that needs to be addressed. The symbolic analysis engines of these tools are not formalised and do not come with correctness guarantees \pmaxinline{check, especially Jalangi}. Their analyses are also not compositional \pmaxinline{We don't know what this means yet, also check, but possibly omit}. Moreover, each of these tools is tailored for catching a specific category of bugs, rather than targeting bug-finding in general \pmaxinline{give evidence - also, why do we care?}.



%We evaluate \jilette on an array of examples, including tests from the official JavaScript Test262 test suite and real-world Node.js libraries. These examples involve JavaScript-specific features, such as prototype inheritance, function closures, the for-in statement, and dynamic dispatch. We highlight the range of \jilette by using it to find bugs in a number of non-trivial separation logic specifications of JavaScript programs.



%In \S\ref{???}, we give an abstract semantics of \jsil, which we instantiate to obtain the concrete %, instrumented (discussed shortly), 
%and symbolic semantics, connected via a {\em soundness result}. Moreover, we prove that, when used for symbolic testing (discussed shortly), \jilette never produces false counter-models. 
%These theoretical results, combined with the correctness of \JSComp, allows us to lift the results of analyses done on compiled \jsil code back up to JavaScript. 

%To our knowledge, this is the first formalisation of a symbolic execution used for JavaScript analysis, and is the first symbolic execution that precisely follows the semantics of the language.

\myparagraph{Architecture} Something.

\myparagraph{Novelty: Trustworthiness} Designing a tractable symbolic analysis for JavaScript that is faithful to the language standard and providing correctness results for such an analysis is known to be a difficult task, due to the complexity of the language. The current approaches \pmaxinline{cite stuff} normally do not give a formal account of the analysis and/or simplify the language semantics. Our approach to designing \jilette is grounded on adherence to the standard and establishing trust. 
To contain the complexity of JavaScript, we move the analysis to \jsil, an intermediate representation for logic-based analysis of JavaScript~\cite{javert}. 
\jsil comes with a trusted compiler, \JSComp, which has been extensively tested against the official test suite, Test262,\footnote{\url{https://github.com/tc39/test262}} and produces \jsil code that corresponds line-by-line to the standard. 
In \S\ref{???}, we give a novel abstract semantics of \jsil, which we instantiate to obtain the concrete %, instrumented (discussed shortly), 
and symbolic semantics, connected via a {\em soundness result}. Moreover, we prove that, when used for symbolic testing (discussed shortly), \jilette never produces false counter-models. 
These theoretical results, combined with the correctness of \JSComp and the fact that the memory models of JavaScript and \jsil are the same by design, allow us to  provably lift the results of analyses done on compiled \jsil code back up to JavaScript. To our knowledge, this is the first formalisation of a symbolic execution used for JavaScript analysis, and is the first symbolic execution for JavaScript that precisely follows the semantics of the language. 

\myparagraph{Novelty: Compositionality} 
Static languages, such as C, C++, or Java, observe the so-called \emph{frame property}, first introduced formally for separation logic~\cite{???}, but intuitively known to language designers and programmers. 
What the frame property means, in a nutshell, that if a program produces a given output when run from a given state, then we can extend this state arbitrarily, run the program from this extended state, and the program will produce the same output. This also means that we can analyse parts of the program separately and then compose the obtained results to obtain the analysis of the entire program. For static languages, this \emph{compositionality} is a property of the language semantics and symbolic analysis tools for such languages leverage on it for free.
%
On the other hand, dynamic languages, such as JavaScript and \jsil, do not observe the frame property. 
This means that the behaviour of a JavaScript/\jsil program may change if the state in which it is run is extended; such extensions may introduce bugs, to which we refer as \emph{frame bugs}. 
%
Current symbolic execution tools for JavaScript~\cite{???} are not compositional. They employ \emph{whole-program analyses}, that is, they assume access to the entire program, meaning that the results they obtain when analysing functions in isolation cannot be reused, as they would not account for the interaction between the function and all of its possible frames.
%
In contrast, the symbolic execution of \jilette is compositional. \jilette can analyse a fragment of a program at a time, and the obtained results can be re-used in the analysis of the entire program. This has two important benefits.
%
First, it allows us to apply function summaries~\cite{???} instead of symbolically executing function bodies at each call site, speeding up execution time. This benefit is independent of the analysed language.
%
Second, it allows us to catch frame bugs, which are not reachable by non-compositional symbolic execution tools. This benefit is specific to dynamic languages.
%
We achieve compositionality by instrumenting the concrete semantics to keep track of properties that we know are {\em not present} in a given object. This we describe in detail in \S\ref{???}. To our knowledge, \jilette is the first compositional symbolic execution tool for dynamic languages.



%dynamic languages: they feature extensible objects, dynamic property access, and dynamic function calls. In terms of separation logic, they do not have the frame property~\cite{???}. What this effectively means is that is possible to introduce bugs into a JS/\jsil program by only extending the state for which it behaved correctly. We refer to such bugs as {\em frame bugs}.  This benefit is independent of the analysed language. Second, it allows us to catch frame bugs, which are not reachable by non-compositional symbolic execution tools. This benefit is specific to dynamic languages. We achieve compositionality by keeping track of properties that we know are {\em not present} in a given object. This we describe in detail in \S\ref{???}. To our knowledge, \jilette is the first compositional symbolic execution tool for dynamic languages.


\myparagraph{Application: Symbolic Testing} A commonly used 
approach to obtaining trust in JavaScript code is running it against 
adhoc test batteries---verifying that, given some concrete inputs, the code produces the expected
outputs. The main drawback of this approach is that manually created test suites often have a high degree of incompleteness that is not properly characterised. % we also cant guarantee exhaustiveness 
In \S\ref{???}, we show how to use \jilette
for symbolic testing of JavaScript code: instead of 
tests with concrete 
inputs, the developer uses symbolic inputs and states the 
constraints that the output needs to satisfy as simple, intuitive 
first-order assertions over these inputs. 
Then, if a test fails, \jilette provides the concrete inputs that cause it 
to fail, exposing bugs in the tested code. 
%We give a {\em proof} that \jilette does not produce false positives. 
Moreover, the compositionality of \jilette allows us to catch bugs that are not reachable by standard symbolic execution tools.
We illustrate the capabilities of \jilette on an array of examples, which include tests from the Test262 test suite and real-world Node.js libraries, and which make use of JavaScript-specific features, such as prototype inheritance, function closures, the for-in statement, and dynamic dispatch. \pmaxinline{Revisit after evaluation.}

\myparagraph{Application: Specification-directed Bug-finding} \pmaxinline{Very stuck here.} Due to the complexity of JavaScript semantics, functional correctness 
specifications of JavaScript programs are highly intricate. 
There are only a few tools (for example, \javert \cite{javert} and KJS \cite{Park:2015,stefanescu-park-yuwen-li-rosu-2016-oopsla}) that support such expressivity. They target the specialist developer wanting rich, 
mechanically verified specifications of critical JavaScript code.
However, when these 
tools cannot prove that a given function satisfies a specification, to discover the error, 
the developer needs to understand in detail a complicated proof trace (\javert), or  act with essentially no feedback~(KJS). In \S\ref{???}, we show how \jilette can be used as an auxiliary mechanism for debugging 
separation logic specifications of JavaScript programs in \javert. 
Our approach consists of translating the separation logic specifications 
into symbolic tests 
and running these tests using \jilette. 
Then, if a symbolic test generated from a given specification fails, we can 
be sure that the code to be verified does not satisfy its specification. 
More importantly, \jilette then generates a concrete witness that 
invalidates the specification. This information greatly simplifies the debugging of 
both specifications and code.

%\pmax{Say more clearly what the novelty is. Say in a very pretty way the connection between seplogic and non-seplogic, verification and symbolic execution, etc.}

%We highlight two relevant use cases for \jilette. First, we show how \jilette can be used as \dtag{i}~a tool for running symbolic tests for JavaScript programs; and \dtag{ii} a debugging tool for separation logic specifications of JavaScript programs.

%\myparagraph{Architecture}
%The core of \jilette consists of a symbolic interpreter for
%\jsil~\cite{javert}, a simple intermediate goto language. 
%We obtain this symbolic interpreter \emph{for free}, 
%by implementing a concrete \jsil interpreter in Rosette~\cite{Rosette2,Rosette1},~a 
%symbolic virtual machine that facilitates generation of solver-aided languages.
%We design the concrete interpreter so that all of Rosette's natively supported solver-aided features, such as advanced string and regular-expression reasoning, 
%are lifted to the \jsil symbolic interpreter. 
%In~\S\ref{sec:jsil:symb:exec}, we give a formalisation of the \jsil concrete and symbolic executions, linking them together with a {\em soundness result}. We also provide insights on how to correctly design the concrete \jsil interpreter in Rosette.

%The second component that \jilette uses is \JSComp~\cite{javert}, 
%a well-tested, standard-compliant compiler from JavaScript to \jsil. We extend
%\JSComp with support for the non-strict mode of ES5, as well as
%regular expressions and the entire \jsinline|String| built-in library.
%\JSComp allows us to lift the \jsil symbolic execution to JavaScript by first compiling JavaScript code to \jsil code, and
%then symbolically executing the compiled code in the 
%\jsil symbolic interpreter. This process, described in \S\ref{symb:exec:comp},
%involves extending JavaScript syntax and the \JSComp compiler to support symbolic values and 
%constructs for reasoning about them. These constructs are intuitive
%and allow the general developer to easily write assertions about the behaviour
%of their program. 
%Moreover, we adjust the \jsil symbolic interpreter so that the abstraction level 
%of the generated \jsil code precisely matches the abstraction level of Rosette, 
% maximising the use of Rosette's native reasoning capabilities.

%\myparagraph{Application: Symbolic Testing} A commonly used 
%approach to obtaining trust in JavaScript code is running it against 
%adhoc test batteries---verifying that given concrete inputs, the code produces the expected
%output. The main drawback of this approach is that tests, in general,
%cannot guarantee exhaustiveness. % we also cant guarantee exhaustiveness 
%In \S\ref{symbolic:testing}, we show how to use \jilette
%for symbolic testing of JavaScript code: instead of 
%tests with concrete 
%inputs, the developer uses symbolic inputs and states the 
%constraints that the output needs to satisfy as simple, intuitive 
%first-order assertions over these inputs. 
%Furthermore, if a test fails, \jilette provides the concrete inputs that cause it 
%to fail, exposing bugs in the tested code. 
%We highlight the capabilities of \jilette through examples that showcase
%challenging reasoning on strings, regular expressions, and the \jsinline|eval|
%statement.

%\myparagraph{Application: Debugging Separation Logic Specifications}
%Due to the complexity of JavaScript semantics, functional correctness 
%specifications of JS programs are highly intricate. 
%There are only a few tools (for example, \javert \cite{javert} and KJS \cite{Park:2015,stefanescu-park-yuwen-li-rosu-2016-oopsla}) that support such expressivity. They target the specialist developer wanting rich, 
%mechanically verified specifications of critical JavaScript code.
%However, when these 
%tools cannot prove that a given function satisfies a specification, to discover the error, 
%the developer needs to understand in detail a complicated proof trace (\javert), or even act with almost no feedback~(KJS). 
%
%In \S\ref{sec:specs}, we show how \jilette can be used as an auxiliary mechanism for debugging 
%separation logic specifications of JavaScript programs in \javert. 
%Our approach consists of: translating the separation logic specifications 
%into symbolic tests 
%and running these tests using \jilette. 
%Then, if a symbolic test generated from a given specification fails, we can 
%be sure that the code to be verified does not satisfy its specification. 
%More importantly, \jilette then generates a concrete witness that 
%invalidates the specification. This information greatly simplifies the debugging of 
%both specifications and code. 

%Jilette has the following benefits: 
%
%\dtag{1} it is \emph{useful}, in that it has tangible applications:
%	it can report bugs in JavaScript programs, producing concrete witnesses that trigger these  bugs; 
%	%
%	it can be used as a helper tool for developers of logic-based functional correctness specifications of JavaScript code; 
%	%
%	and it has support for advanced string reasoning, critical for reasoning about commonly used JavaScript code;
%
%\dtag{2} it is \emph{accessible}, in that it can easily be used by a general JavaScript developer: 
%	the annotation burden of \jilette is minimal; 
%	%
%	and the assertion language is simple and intuitive;
%\dtag{3} it is \emph{trustworthy}, in that its components come with correctness guarantees: 
%	the correctness of the \JSComp compiler ensures full adherence to the real semantics of JavaScript;
%	%
%	the soundness result for the symbolic execution used in \jilette guarantees absence of false positives;
%	and \polish{sentence about unification;}
%and \dtag{4} it is \emph{extensible}, in that its coverage can easily be extended in a modular way, allowing support for: 
%	built-in libraries not covered by \JSComp; 
%	%
%	and widely used runtime libraries that are not part of the standard, such as the DOM.

%\pmax{What's the story?
%\begin{enumerate}
%\setlength{\itemsep}{0.5em}
%\item 
%	{\bfseries Slogan}: Symbolic execution for JavaScript that precisely follows the language standard. \\ 
%	{\bfseries Goal}: Symbolic testing, bug-finding, concrete counter-models. \\
%	{\bfseries Novelty}: The precision wrt semantics, formal correctness guarantees. \\ 
%	{\bfseries Benefits}: Trustworthy, sound analysis. \\
%	{\bfseries Limitations}: No loop invariants, bounded. No eval.
%
%\item 
%	{\bfseries Slogan}: Compositional execution for dynamic languages in general, and JavaScript in particular. \\ 
%	{\bfseries Novelty}: Compositionality. \\ 
%	{\bfseries Benefits}: summaries, frame-related bugs.   
%	
%\item 
%	{\bfseries Application}: Symbolic testing of JavaScript programs. \\
%	{\bfseries Novelty}: None? \\ 
%	{\bfseries Evaluation}: Tests for the symbolic execution rules; JS programs using prototype inheritance, arrays, function closures, for-in, dynamic dispatch, etc.; test262 tests; node.js libraries for data structures
%	
%\item 
%	{\bfseries Application}: Debugging of separation logic specifications. \\ 
%	{\bfseries Novelty}: Counter-models for separation logic assertions. \\
%	{\bfseries Evaluation}: JaVerT specifications of this and that.
%\end{enumerate}}




%
%\myparagraphq{Why \jilette} 
%\jilette is \emph{useful}: it has tangible applications. 
%It can report bugs in JavaScript programs, producing concrete witnesses triggering the bugs. It can also be used as a helper tool for developers of logic-based functional correctness specifications of JavaScript code.
%\jilette is \emph{approachable}: it can easily be used by a general JavaScript developer. The annotation burden of \jilette is minimal and the assertion language is simple and intuitive. \polish{Sweet spot?}
%\jilette is \emph{trustworthy}: its components come with correctness guarantees. 
%The correctness of the \JSComp compiler ensures full adherence to the real semantics of JavaScript. The \jilette symbolic execution engine is based on a sound symbolic
%analysis for \jsil, guaranteeing the absence of false positives. \polish{Sentence about unification.}
%Finally, \jilette is \emph{extensible}: its coverage can easily be extended in a modular way. This gives us the mechanism for supporting built-in libraries not covered by \JSComp, or adding support for standard-external runtime libraries, such as the DOM.

\newpage

%\newpage
%
%\myparagraph{What's in the paper}
%
%\bigskip
%\polish{TO GO IN SOMEWHERE \\
%
%Clarify ES5 Strict
%
%JaVerT targets the specialist
%developer wanting rich, mechanically verified specifications of critical JavaScript code.
%Functional correctness, yes, and it works, but paid for by a heavy annotation burden.
%}



%We show how  to use Jilette for writing symbolic tests for client side 
%JavaScript code calling Web APIs. In particular, we demonstrate how to 
%checking the conformance of Web API requests with their specified signatures. 
%The existing solutions for this problem are still imprecise due to the 
%dynamicity of JavaScript combined with the difficulty of reasoning about
%operations on symbolic strings \cite{Idontknow}. Jilette is an excellent fit for
%this task as it leverages on Rosette's back-end
%constraint solver, Z3, which supports reasoning on symbolic strings
%and regular expressions, whereas JS-2-JSIL successfully
%contains the complexity of JavaScript itself.

\newpage
\section{Overview}\label{sec:overview}
%!TEX root = ../main.tex

We illustrate how \cosette can be used both for whole-program and compositional symbolic testing of JavaScript programs. We demonstrate how \cosette  explicitly exposes the resilience of JavaScript programs to the environment, not considered by standard symbolic execution tools, but essential for compositional analysis.

 \begin{figure*}[t]
 \centering
 %
 \begin{subfigure}[b]{0.33\textwidth}
 {\lstset{language=JavaScript,basicstyle=\fontsize{7}{7}\ttfamily,escapeinside={~}{~}}
 \begin{lstlisting}
function Map () { this._contents = {} }

Map.prototype.get = function (k) {
  var c = this._contents;
  if (this.validKey(k)) {
    return (c.hasOwnProperty(k) ? c[k] : null)
  } else throw new Error("Invalid Key");
}

Map.prototype.put = function (k, v) {
  if (this.validKey(k)) {  
    this._contents[k] = v   
  } else throw new Error("Invalid Key");
} 

Map.prototype.validKey = function (k) { ... }
\end{lstlisting}}
\vspace*{-0.2cm}
\caption{Library implementation}
\label{fig:2a}
\end{subfigure}
%
 \begin{subfigure}[b]{0.33\textwidth}
 \includegraphics[width=0.93\textwidth]{figures/mapDiagram.png}
 \vspace*{0.5cm}
 \caption{General library heap}
 \label{fig:2b}
 \end{subfigure}
 %
 \begin{subfigure}[b]{0.3\textwidth}
 \centering 
 {\lstset{xleftmargin=.17\textwidth,language=JavaScript,basicstyle=\fontsize{7}{7}\ttfamily,escapeinside={~}{~}}
\begin{lstlisting}
var k = symb_string();
var v = symb_number();
assume(validKey(k));
var m = new Map(); m.put(k, v); 
var result = m.get(k);
assert(result = v)
\end{lstlisting}}
\vspace*{0.1cm}
 \includegraphics[width=0.78\textwidth]{figures/heapfail.png}
 \captionsetup{format=nastyCaption}
\caption{Simple symbolic test (above); \\the \jsinline|"hasOwnProperty"| bug (below)}
\label{fig:2c}
\end{subfigure}
\vspace*{-0.25cm}
\caption{Running example: JavaScript key-value map library}
\label{fig:two}
 \vspace*{-0.4cm}
\end{figure*}

Our running example is a \emph{key-value map} implementation, given in Figure~\ref{fig:2a}. It contains four functions: 
\jsinline|Map|, for constructing an empty map;
\jsinline|get|, for retrieving the value associated with a given key;
\jsinline|put|, for inserting/updating key-value pairs; and \jsinline|validKey|, for deciding whether or not a key is valid.
The map library implements a \emph{key-value map} as an object with property \jsinline|_contents|, denoting the object storing the map contents.  
The named properties of \jsinline|_contents| and their value attributes correspond to the map keys and values, respectively.
The functions \jsinline|get|, \jsinline|put|, and \jsinline|validKey| are shared between all map 
objects and are, therefore, defined in \jsinline|Map.prototype|, which is the prototype\footnote{In JavaScript, inheritance is modelled through \emph{prototype chains}. On property lookup, $\mathtt{o.p}$, we first check if the property $\mathtt{p}$ is present in the object $\mathtt{o}$, in which case its value is returned. Otherwise, we  check if $\mathtt{p}$ is present in the prototype of $\mathtt{o}$, and so  forth.} of all objects created using \jsinline|Map| as a constructor (i.e.,~using~\jsinline|new Map()|). 
The \jsinline|get| function returns the value associated with a given key in the map, or \jsinline|null| if the key is not in the map. 
Note that, in order to check that the given key is in the map, \jsinline|get| uses the built-in function \jsinline|hasOwnProperty|, which lives in \jsinline|Object.prototype|, the prototype of all objects.
The \jsinline|put| function updates the map if the supplied key is valid, and otherwise throws an error. 
The \jsinline|validKey| function describes the conditions under which a given key is valid. 

In Figure \ref{fig:2b}, we show a general heap of key-value maps. There is the \jsinline|map| object, with its \jsinline|_contents| property pointing to the \jsinline|contents| object and its prototype being \jsinline|Map.prototype|. There is the \jsinline|contents| object, which holds the key-value pairs, and whose prototype is \jsinline|Object.prototype|. There is the \jsinline|Map.prototype| object, which holds the \jsinline|get|, \jsinline|put|, and \jsinline|validKey| functions,\footnote{In JavaScript, functions are modelled as objects in the heap. As their structure does not add to this example, we only show the locations of the appropriate objects.} and whose prototype is also \jsinline|Object.prototype|. Finally, there is \jsinline|Object.prototype|, which holds the \jsinline|hasOwnProperty| function that is called by \jsinline|Map.prototype.get|.

%Observe that a naive implementation of the function \jsinline|validKey| may result in potential bugs. In particular, one can insert a key-value pair with \jsinline|"hasOwnProperty"| as a key into the map. By doing this, \jsinline|"hasOwnProperty"| in the prototype chain of \jsinline|_contents| is overridden and subsequent calls to \jsinline|get| will fail. 

%\myparagraph{Prototype chains and $\mathtt{Object.prototype}$}
%In order to better understand the implementation of the map library as well as its possible bugs, 
%one must first understand the \emph{prototype-based inheritance} mechanism of JavaScript. 
%Every JavaScript object has a prototype, which (for presentation purposes) we assume to 
%be stored  in an internal property \jsinline|@proto|. In order to determine the value of a property
%\jsinline|p| of an object \jsinline|o|, the semantics first checks if \jsinline|o| has a 
%property named \jsinline|p|, in which case the property look-up yields its value. Otherwise, the 
%semantics checks if \jsinline|p| belongs to the properties of the prototype of \jsinline|o| and so 
%forth. Hence, in the example, when looking up the value of the property \jsinline|hasOwnProperty|
%of the object \jsinline|contents|, one gets the value associated with the property  \jsinline|hasOwnProperty|
%of its prototype.
%The sequence of objects that can be accessed from a given object through the inspection 
%of the respective prototypes is called a \emph{prototype chain}.
%Prototype chains typically finish with the object \jsinline|Object.prototype| from which JavaScript 
%programs can access a number of built-in functions, which are part of the language runtime environment and are used for inspecting and manipulating objects.
%An example of such a function is \jsinline|hasOwnProperty(p)|, which checks whether or not the object 
%on which it is invoked has the property \jsinline|p| (e.g. {\small \jsinline|map.hasOwnProperty("_contents")|}
%evaluates to \jsinline|true| when evaluated in the heap shown in Fig.~\ref{map:example}-(right), 
%because the object \jsinline|map| has a property named~\jsinline|"_contents"|). 

\lstnewenvironment{lstjsex}{\lstset{language=JavaScript,basicstyle=\fontsize{8}{8}\ttfamily,escapeinside={~}{~}, numbers=none}}{}

\vspace*{-0.2cm}
\subsection{Whole-program Symbolic Testing}
\label{subsec:st}

Developers are used to writing unit tests for their code---verifying that, given some concrete inputs, the code produces the expected outputs. Using \cosette, they can write unit tests with \emph{symbolic} inputs and outputs, systematically testing a broad range of behaviours with a single symbolic test. For example, one meaningful unit test for the \jsinline|put| function consists of inserting a valid key-value pair \jsinline|(k, v)| into a map and then verifying that the pair has been inserted correctly. In Cosette, this test can be written as in Figure~\ref{fig:2c}. First, we declare \jsinline|k| to be a symbolic string and \jsinline|v| to be an symbolic number, using \cosette's constructs for creating symbolic variables. Next, we assume that \jsinline|k| is a valid key. Next, we create a new map, put the (symbolic) key-value pair \jsinline|(k, v)| into the map and then retrieve the value corresponding to the key~\jsinline|k|. Finally, we assert that the retrieved value is equal to the one we had previously put.

% 
When running \cosette on this test, if the \jsinline|validKey(k)| function was implemented incorrectly,\footnote{For instance, $\mathtt{validKey(k)}$ may only require that $\mathtt{k}$ is a string, which is a reasonable implementation, in the sense that it disallows JavaScript's implicit coercions.}
we will obtain the counter-model \jsinline|k = "hasOwnProperty"|. To understand this error, recall the heap and the implementation of \jsinline|get| from Figure~\ref{fig:two}. We can see that, if we were to put the key \jsinline|"hasOwnProperty"| into the contents object of a map, then the lookup of \jsinline|c.hasOwnProperty| done by \jsinline|get| will not reach \jsinline|Object.prototype| as intended, resolving instead to the \jsinline|hasOwnProperty| property of the \jsinline|contents| object (Figure~\ref{fig:2c}, below).

This example highlights how \cosette does not require specialist knowledge and can 
be used as a testing tool by a general JavaScript developer. The annotations amount to the creation of 
symbolic variables and the writing of assumptions and assertions, remaining minimal and intuitive, in contrast with the standard annotation burden of verification tools.

\vspace*{-0.2cm}
\subsection{Specification-driven Bug-finding}
\label{subsec:sdbf}

%\pmax{
%\begin{itemize}
%\item Compositionality = resilience to frame
%\item Summaries have to be resilient to frame
%\item BUT there are also NONES, and this is what is new
%\item Now guide through
%\item POINT - resilient to ALL frames, for whole-program analysis we are resilient only to one frame - we do get to create it, but it's only one after all
%\item Then, we can have more general specs, but needn't necessarily
%\end{itemize}
%}

As well as for whole-program analysis, \cosette can be used for compositional symbolic analysis of JavaScript functions in isolation, where the user specifies the functions in terms of their pre- and post-conditions. These specifications may account for only the parts of the heap required for running the function and can involve predicates, both recursive and non-recursive. We deal with recursive predicates by unfolding them to a  bound specified by the user.

Much like symbolic tests generalise concrete tests, specifications generalise symbolic tests. Given a JavaScript function, its specification, and the unfold depth for predicates, \cosette generates symbolic tests to verify that the function conforms to the specification up to that given depth. If this is not the case, \cosette will return a concrete counter-model that invalidates the specification. Unlike whole-program analysis tools, \cosette also tests if the given specification is compositional, that is, if it is resilient against all possible contexts in which the function can be run, and reports back to the user any found sources of non-compositionality.

\cosette supports the specification of symbolic states via simple separation logic assertions in the style of JaVerT~\cite{javert}. The developer has at their disposal a number of built-in predicates that capture the fundamental concepts of JavaScript (discussed throughout the text), and can define their own predicates as well. For instance, learning from the previous symbolic test, we could define the following predicate for describing valid keys:
\begin{Verbatim}[fontsize=\footnotesize,commandchars=\\\{\}]
    ValidKey(k) := types(k : Str) * (k <> "hasOwnProperty"),
\end{Verbatim}
\noindent meaning that \jsinline|k| is a valid key if it is a string that is not equal to \jsinline|"hasOwnProperty"|. From there, if we wanted to do a full functional correctness specification of the \jsinline|Map| library, we could, guided by the heap in Figure~\ref{fig:2b}, define the following two predicates:

% \textcolor{red}{(m, "get") -> None * (m, "put") -> None * (m, "validKey") -> None} * 
% * NoProps(c, keys)

\smallskip
\begin{Verbatim}[fontsize=\footnotesize,commandchars=\\\{\}]
 Map (m, mp, kvs) := JSObjectWithProto(m, mp) * 
   DataProp(m, "_contents", c) * JSObject(c) * KVPairs(c, kvs) * 
     \textcolor{red}{NoProp(m, "get")} * \textcolor{red}{NoProp(m, "put")} * \textcolor{red}{NoProp(m, "validKey")} * 
       \textcolor{blue}{NoProps(c, FProj(kvs))}
\end{Verbatim}
\begin{Verbatim}[fontsize=\footnotesize,commandchars=\\\{\}]
  KVPairs (c, kvs) := (kvs = \{ \}),
                      (kvs = \{(k, v)\} U kvs') * ValidKey(k) * 
                        DataProp(c, k, v) * KVPairs(c, kvs')
\end{Verbatim}

\smallskip
The \jsinline|Map| predicate states that a map object is a standard JavaScript object with a given prototype \jsinline|mp|, and that it has the property \jsinline|_contents|, which points to a  JavaScript object \jsinline|c|.
Using the \jsinline|KVPairs| predicate, % (explained shortly), 
it also states that \jsinline|c| holds the key-value pairs \jsinline|kvs|. 
%Finally, it obtains the set of keys \jsinline|keys| from the set of key-value pairs using the first projection predicate \jsinline|First|, and then, using the \jsinline|NoProps| predicate, states that all other properties are absent from \jsinline|c|.
The \jsinline|KVPairs(c, kvs)| predicate is defined recursively: \jsinline|kvs| is either empty or contains at least one key-value pair \jsinline|(k, v)|, 
in which case we state that the key \jsinline|k| must be valid, that object \jsinline|o| has  property \jsinline|k| with value \jsinline|v|, and proceed recursively.
The highlighted parts of \jsinline|Map| describe the compositionality requirements: in red, we state that map objects must not have the properties \jsinline|"get"|, \jsinline|"put"|, and \jsinline|"validKey"|; in blue, we state that the object~\jsinline|c| has no other properties except for the map keys.\footnote{$\mathtt{NoProp(o, p)}$  states that the object $\mathtt{o}$ does not have property $\mathtt{p}$; $\mathtt{NoProps(o, props)}$ states that the object $\mathtt{o}$ has no properties outside of those from the set $\mathtt{props}$; the $\mathtt{FProj}$ operator extracts the set of keys from the set of key-value pairs.}
%
%The uniqueness of keys is guaranteed by the \jsinline|DataProp| predicate of \jsinline|KVPairs| and the separating conjunction.
We also require a \jsinline|MapProto(mp)| predicate, describing that \jsinline|mp| is a valid map prototype, that is, that it defines the \jsinline|put|, \jsinline|get|, and \jsinline|validKey| methods. To avoid clutter, we keep its definition opaque.

\begin{wrapfigure}{R}{0.23\textwidth}
\vspace*{-0.3cm}
\hspace*{-0.8cm}
$
{\scriptsize
\begin{array}{c}
\left\{ {\begin{array}{c}
 \text{\texttt{Map(m, mp, kvs) * MapProto(mp) *}} \\
 \text{\texttt{ValidKey(k) * (k $\notin$ FProj(kvs)) *}} \\
 \text{\textcolor{blue}{\texttt{Writable(Object.prototype, k)}}}
\end{array}} \right\} \\
%
\text{\bfseries \texttt{m.put(k, v)}} \\[0.2mm]
%
\left\{ {\begin{array}{c}
 \text{\texttt{Map(m, kvs -u- (k, v)) * MapProto(mp) *}} \\
 \text{\textcolor{blue}{\texttt{Writable(Object.prototype, k)}}} \\
\end{array}} \right\}
\end{array}
} 
$
\vspace*{-0.4cm}
\end{wrapfigure}

On the right, we show one compositional specification of \jsinline|put(k, v)|. 
We assume a map object \jsinline|m|, with key-value pairs \jsinline|kvs| and prototype \jsinline|Map.Prototype|. We also assume that \jsinline|k| is a valid key not already in the map. For compositionality, highlighted in blue, we have to state that the property \jsinline|k| is not non-writable in \jsinline|Object.prototype|.\footnote{In JavaScript, object properties can be non-writable, meaning that their value cannot be changed. In this specification, we assume that the property exists and is writable. There is an analogous specification for $\mathtt{put}$, in which the property does not exist.}
In the end, the key-value pair has been inserted into the map, while the remaining part of the heap remains unchanged.

% for example, \jsinline|JSObject(o)| states that \jsinline|o| is a standard JavaScript object; \jsinline|JSObjectWithProto(o, op)| states that \jsinline|o| is a JavaScript object with prototype \jsinline|op|; and \jsinline|DataProp(o, p, v)| states that the property \jsinline|p| of~\jsinline|o| has value \jsinline|v|. One can also describe the absence of object properties: \jsinline|NoProp(o, p)| states that the object~\jsinline|o| does not have property \jsinline|p|, and \jsinline|NoProps(o, props)| states that the object \jsinline|o| has no other properties except those in the set \jsinline|props|. There also exist predicates for describing function objects, prototype chains, function closures, etc.,

%\smallskip
%\begin{minipage}{0.475\textwidth}
%\begin{displaymath} 
%{\scriptsize
%\begin{array}{c}
%\left\{ {\begin{array}{c}
% \text{\texttt{Map(m, mp, kvs -u- (k, v')) * MapProto(mp)}}
%\end{array}} \right\} \\
%%
%\text{\bfseries \texttt{m.put(k, v)}} \\[0.2mm]
%%
%\left\{ {\begin{array}{c}
% \text{\texttt{Map(m, kvs -u- (k, v)) * MapProto(mp)}}
%\end{array}} \right\}
%\end{array}
%} 
%\end{displaymath}
%\end{minipage}
%\quad
%\begin{minipage}{0.48\textwidth}
%%
%\begin{displaymath} 
%{\scriptsize
%\begin{array}{c}
%\left\{ {\begin{array}{c}
% \text{\texttt{Map(m, mp, kvs) * MapProto(mp) *}} \\
% \text{\texttt{ValidKey(k) * (k $\notin$ First(kvs))}}
%\end{array}} \right\} \\
%%
%\text{\bfseries \texttt{m.put(k, v)}} \\[0.2mm]
%%
%\left\{ {\begin{array}{c}
% \text{\texttt{Map(m, kvs -u- (k, v)) * MapProto(mp)}}
%\end{array}} \right\}
%\end{array}
%} 
%\end{displaymath}
%\end{minipage}

%\pmax{Explain specs a bit}

When symbolically testing the specifications of the \jsinline|Map| library functions, the \jsinline|"hasOwnProperty"| bug will not be triggered again, as \jsinline|ValidKey(k)| has been adjusted appropriately. However, if we were to forget the highlighted parts in the \jsinline|Map| definition and \jsinline|put| specification, we would encounter other issues, exposing the tension between compositionality and dynamic languages. 

\begin{wrapfigure}{R}{0.18\textwidth}
\vspace*{-0.4cm}
\hspace*{-0.6cm}
\centering
\includegraphics[width=0.21\textwidth]{figures/compositional.png}
\vspace*{-0.58cm}
\caption*{\hspace*{-0.58cm}{\small Figure 2. Compositionality}}
\vspace*{-0.4cm}
\end{wrapfigure}

\setcounter{figure}{2} 
First, if we omitted the part of the \jsinline|Map| predicate highlighted in red, \cosette would complain, when testing the \jsinline|put| function, that it has no information about the property \jsinline|"put"| of the map object~\jsinline|m| and cannot perform the lookup \jsinline|m.put| (Fig.~2, above). This means that the library code is not resilient to frames in which a map object \jsinline|m| has the property \jsinline|put|.
Analogous issues would arise for the \jsinline|"get"| and \jsinline|"validKey"| properties.% which also must be absent from all map objects. 

Second, if we omitted the parts highlighted in blue, upon execution of the line \jsinline|this._contents[k] = v| (Fig.~\ref{fig:2a}, line 12) \cosette would complain that it has no information about the property \jsinline|k| in the prototype chain of the contents object (Fig.~2, below). This is required as the semantics of JavaScript has to look up the value of the property \jsinline|k| of the contents object before performing the actual assignment, to check if the assignment is allowed (i.e.,~ that the property is non-writable). As \jsinline|k| is symbolic, what this means is that the library is not resilient to frames in which there are non-writable properties in \jsinline|Object.prototype|. A similar issue arises for the \jsinline|Map| constructor (Fig.~\ref{fig:2a}, line 1), which requires the property \jsinline|"_contents"| not to be non-writable in the prototype chains of map objects.

%Now, we can give the complete definition of the \jsinline|Map| predicate and the corrected, compositional specification of \jsinline|put(k, v)| when \jsinline|k| is a valid key that is not in the map, with the compositionality-related changes highlighted in red:
%
%\noindent
%\begin{minipage}{0.58\textwidth}
%\begin{Verbatim}[fontsize=\footnotesize,commandchars=\\\{\}]
%  Map (m, mp, kvs) := 
%    JSObjectWithProto(m, mp) * \textcolor{red}{NoProp(m, "get")} * 
%    \textcolor{red}{NoProp(m, "put")} * \textcolor{red}{NoProp(m, "validKey")} * 
%    DataProp(m, "_contents", c) * JSObject(c) *
%    KVPairs(c, kvs) * \textcolor{red}{First(kvs, keys)} * \textcolor{red}{NoProps(c, keys)}
%\end{Verbatim} 
%\end{minipage}
%\begin{minipage}{0.42\textwidth}
%%
%\begin{displaymath} 
%{\scriptsize
%\begin{array}{c}
%\left\{ {\begin{array}{c}
% \text{\texttt{Map(m, mp, kvs) * MapProto(mp) *}} \\
% \text{\texttt{ValidKey(k) * (k $\notin$ First(kvs)) *}} \\
% \text{\textcolor{red}{\texttt{Writable(Object.prototype, k)}}}
%\end{array}} \right\} \\
%%
%\text{\bfseries \texttt{m.put(k, v)}} \\[0.2mm]
%%
%\left\{ {\begin{array}{c}
% \text{\texttt{Map(m, kvs -u- (k, v)) * MapProto(mp) *}} \\
% \text{\textcolor{red}{\texttt{Writable(Object.prototype, k)}}} \\
%\end{array}} \right\}
%\end{array}
%} 
%\end{displaymath}
%\end{minipage}

What this example illustrates is that, in order to be compositional, specifications of programs written in dynamic languages have to explicitly state which parts of the heap must not be present. \cosette is able to detect and report compositionality-related issues, such as those presented above, which are highly likely to remain unnoticed in whole-program analysis, as there we always have complete information about the entire contents of the heap.

%\subsection{Compositional Symbolic Testing}
%
%%\begin{wrapfigure}{R}{0.45\textwidth}
%%\vspace*{-0.5cm}
%%\centering
%%\begin{lstjsex}
%%[ Map(m, mp) * MapProto(mp) * ValidKey(k) ]
%%    m.put(k, v); 
%%    var result = m.get(k);
%%[ Precondition * (result = v) ]
%%\end{lstjsex}
%%\vspace*{-0.4cm}
%%\caption{Revisited test for \jsinline|Map|}
%%\label{test:map:2}
%%\vspace*{-0.35cm}
%%\end{wrapfigure}
%
%A general developer can write concrete and symbolic tests for their code, but cannot be expected to write full functional correctness specifications in separation logic. Using \cosette, they can combine the best of both worlds by describing only the shape of their \polish{heap/memory/data structure} using separation logic and then testing the behaviour of the code using symbolic tests. Using this approach, one can not only find the same bugs as in whole-program symbolic testing, but also detect compositionality issues triggered by the tests. 



%Refer to Figure 2 - the developer knows the heap and can describe it easily. If they forget, . Ideally, this would be automatic.
%
%We illustrate the compositional symbolic execution of \cosette using the \jsinline|get(k)| function of the key-value map example. Below, we revisit the code of \jsinline|get| and describe the heap before and after \jsinline|get(k)| is called with a valid key that is in the map. In order to do this, the developer needs to know the structure of the heap (Figure~\ref{map:example}) and use the built-in predicates.
%
%\smallskip
%\begin{minipage}{0.52\textwidth}
% \begin{lstjs}
%Map.prototype.get = function (k) {
%  var c = this._contents;
%  if this.validKey(k) {
%    return (c.hasOwnProperty(k) ? 
%               c[k] : null)
%  } else throw new Error("Invalid Key");
%}
%\end{lstjs}
%\end{minipage}
%\begin{minipage}{0.47\textwidth}
%$
%{\scriptsize
%\begin{array}{c}
%\left\{{\begin{array}{c}
% \text{\texttt{JSObjectWithProto(this, mp) * MapProto(mp) * }} \\
% \text{\texttt{DataProp(m, "\_contents", c) * JSObject(c) * }} \\
% \text{\texttt{ValidKey(k) * \color{blue}{DataProp(c, k, v)}}}
%\end{array}}\right\} \\
%%
%\text{\bfseries \texttt{get(k)}} \\
%%
%\left\{ {\begin{array}{c}
% \text{\texttt{StartingState * (ret = v)}} 
%\end{array}} \right\}
%\end{array}
%}
%$
%\end{minipage}
%
%\smallskip
%To describe the starting state, we first note that the map object on which \jsinline|get| was called has prototype \jsinline|mp|, representing \jsinline|Map.prototype|.\footnote{Some stuff.} Next, we state that the map object itself has property \jsinline|"_contents"| pointing to a standard JavaScript object \jsinline|c|, that the key \jsinline|k| is valid, and that the object \jsinline|c| has the property \jsinline|k| with value \jsinline|v|. In the final state, we additionally know that the return value of the function should be equal to \jsinline|v|.
%
%
%
%
%
%\vspace*{5cm}
%catch how the spec is not compositional, reveal frame non-resilience shit
%
%It is meant to be run as a method call, the this, it is in the prototype. validKey is also meant to be in the prototype. Refer to figure 2. the this has the contents field, then that is an object, the key k is valid and then it may or may not have the k field.
%
%Now, frame bugs will pop up and you will see how the function is not resilient to the frame.
%
%you can catch these bugs in real-world tests
%
%...and if they want, they can write a more general abstraction of the entire data structure, which may be recursive or not, then ask for it to be unfolded to a given depth and create symbolic tests from that.
%
%
%
%\newpage
%In order for a specification of a program to be compositional, it must be resilient against all the possible frames, that is, all possible contexts in which the program can be run. This is in contrast to whole-program analysis, which considers only one frame at a time. 
%
%we demonstrate how \cosette explicitly exposes the resilience of JavaScript programs to the environment, not  considered by standard symbolic execution tools, but essential for compositional analysis. Therefore, this .
%
%For that, we revisit the \javert specification of key-value maps. This specification involves several predicates, shown below, which use JavaScript-specific abstractions that hide the internals of the language, such as \jsinline|JSObject(c)|, which states that \jsinline|c| is a standard JavaScript object, and \jsinline|DataProp(o, p, v)|, which states that the property \jsinline|p| of \jsinline|o| has value \jsinline|v|.
%
%% \textcolor{red}{(m, "get") -> None * (m, "put") -> None * (m, "validKey") -> None} * 
%
%\begin{Verbatim}[fontsize=\footnotesize,commandchars=\\\{\}]
%    Map (m, kvs) := DataProp(m, "_contents", c) * JSObject(c) * 
%                      KVPairs(c, kvs) * first(kvs, keys) * emptyFields(c, keys)
%\end{Verbatim}
% \begin{Verbatim}[fontsize=\footnotesize,commandchars=\\\{\}]
%KVPairs (o, kvs) := (kvs = \{ \}),
%                    (kvs = (k, v) -u- kvs') * ValidKey(k) * DataProp(o, k, v) * KVPairs(o, kvs')
%\end{Verbatim}
%\begin{Verbatim}[fontsize=\footnotesize,commandchars=\\\{\}]
%    ValidKey (k) := types(k : Str) * \textcolor{red}{(k <> "hasOwnProperty")}
%\end{Verbatim}
%
%
%Refer to Figure 2 - the developer knows the heap and can describe it easily. If they forget, . Ideally, this would be automatic.
%
%The \jsinline|Map| predicate captures the resource corresponding to a map object. 
%Concretely, it first states that the map object has the property \jsinline|_contents|, which points to a  JavaScript object \jsinline|c|.
%Next, using the \jsinline|KVPairs| predicate (explained shortly), it states that \jsinline|c| holds the key-value pairs \jsinline|kvs|. Finally, it obtains the set of keys \jsinline|keys| from the set of key-value pairs using the first projection predicate \jsinline|first|, and then, via the \jsinline|emptyFields| assertion, states that all other properties are absent from \jsinline|c|.
%
%The \jsinline|KVPairs(o, kvs)| predicate talks about key-value pairs of an object \jsinline|o|. 
%It is defined recursively on the structure of \jsinline|kvs| and it has two definitions, separated by a comma. 
%We have that \jsinline|kvs| is either empty or that it contains at least one key-value pair \jsinline|(k, v)|,\footnote{We write $\mathtt{-u-}$ for set union and omit the brackets around singleton sets.} 
%in which case we state that the key \jsinline|k| must be valid, that object \jsinline|o| has  property \jsinline|k| with value \jsinline|v|, and proceed recursively.
%The uniqueness of keys is guaranteed by the \jsinline|DataProp| predicate of \jsinline|KVPairs| and the separating conjunction.
%
%\begin{wrapfigure}{R}{0.3\textwidth}
%\vspace*{-0.45cm}
%\centering
%\includegraphics[width=0.29\textwidth]{figures/symbvsass.png}
%\vspace*{-0.3cm}
%\caption{Unfolded assertion {\scriptsize$\mathtt{Map(map, \{ (\hat{s}_1, \hat{n}_1), (\hat{s}_2, \hat{n}_2) \} )}$}}\label{fig:symb:state:versus:assertion}
%\label{fig:unfolded}
%\vspace*{-0.4cm}
%\end{wrapfigure}
%
%The \jsinline|ValidKey(k)| predicate captures the validity of a given key and holds \emph{iff} the corresponding JavaScript function \jsinline|validKey(k)| returns \jsinline|true|.
%In the definition of \jsinline|ValidKey|, we highlight in red a potential bug in the specification, already seen in the symbolic testing example.
%% source of errors on which we will focus shortly.
%
%To give a better intuition of how the \jsinline|Map| predicate works, we show the full unfolding of {\small$\mathtt{Map(map, \{ (k_1, v_1), (k_2, v_2) \} )}$} in Figure \ref{fig:symb:state:versus:assertion}. 
%
%\noindent
%\begin{minipage}{0.475\textwidth}
%\begin{displaymath} 
%{\scriptsize
%\hspace*{-0.2cm}
%\begin{array}{c}
%\left\{ {\begin{array}{c}
% \text{\texttt{Map(this, kvs -u- (k, v)) * ObjProtoF() *}} \\
% \text{\texttt{(this, "@proto") -> mp * MapProto(mp) * ...}}
%\end{array}} \right\} \\
%%
%\text{\bfseries \texttt{get(k)}} \\[0.2mm]
%%
%\left\{ {\begin{array}{c}
% \text{\texttt{Precondition * (ret = v)}} 
%\end{array}} \right\}
%\end{array}
%} 
%\end{displaymath}
%\end{minipage}
%\quad
%\begin{minipage}{0.48\textwidth}
%%
%\begin{displaymath} 
%{\scriptsize
%\begin{array}{c}
%\left\{ {\begin{array}{c}
% \text{\texttt{Map(this, kvs -u- (k, v')) * ObjProtoF() *}} \\
% \text{\texttt{(this, "@proto") -> mp * MapProto(mp) * ...}}
%\end{array}} \right\} \\
%%
%\text{\bfseries \texttt{put(k, v)}} \\[0.2mm]
%%
%\left\{ {\begin{array}{c}
% \text{\texttt{Map(this, kvs -u- (k, v)) * ObjProtoF() *}} \\
% \text{\texttt{(this, "@proto") -> mp * MapProto(mp) * ...}}
%\end{array}} \right\}
%\end{array}
%} 
%\end{displaymath}
%\end{minipage}
%
%\vspace{5pt}
%The predicate \jsinline|ObjProtoF()| describes the \jsinline|Object.prototype| object. It is needed because \jsinline|get| uses the \jsinline|hasOwnProperty| function, defined in \jsinline|Object.prototype|. 
%The predicate \jsinline|MapProto| specifies the resource of a valid map prototype: in particular, it defines the \jsinline|put|, \jsinline|get|, and \jsinline|validKey| methods.
%
%Given a JavaScript function, its separation logic specification, and the depth to which the unfold recursive predicates (non-recursive predicates are unfolded automatically), \cosette generates symbolic tests to verify that the function conforms to the specification up to that given depth.
%Now, if we forgot to state the part of the \jsinline|ValidKey(k)| predicate highlighted in red, that is, if we did not state that \jsinline|k <> "hasOwnProperty"|, the symbolic test generated for the specification of \jsinline|get| would fail for depth $\geq 1$, with the counter-model \jsinline|k = "hasOwnProperty"|, triggering the same bug previously described in the context of symbolic testing.

%\subsection{Catching \polish{procedure-local} bugs}
%
%The bug associated with the shadowing of the \jsinline|hasOwnProperty| property of \jsinline|Object.prototype| illustrates how a JavaScript library can be broken by only using its own functions. However, as JavaScript does not observe the frame property, there exists an additional class of bugs that can be triggered by the environment in which the library is run. These bugs expose how the library is not resilient against the possible frames and signal which properties of which objects must not be present in order for the library to behave correctly.
%
%To illustrate such bugs, recall the symbolic test from \S\ref{subsec:st}. This symbolic test creates an empty map on which it checks whether or not the behaviour of \jsinline|put| is correct. After catching the \jsinline|hasOwnProperty| bug, one might want to construct a more general test, starting from an arbitrary map. For this, one would need to use the \jsinline|Map| predicate from \S\ref{subsec:sdbf}:
%
%\begin{Verbatim}[fontsize=\footnotesize,commandchars=\\\{\}]
%         Map (m, kvs) := DataProp(m, "_contents", c) * JSObject(c) * 
%                           KVPairs(c, kvs) * first(kvs, keys) * emptyFields(c, keys)
%\end{Verbatim}
%
%\noindent as part of the initial state in which to run the symbolic test. Then, however, on execution of the test, when we reach the \jsinline|m.put(k, v)| command, we will get an error. The symbolic execution will not be able to determine if the property \jsinline|put|, which is supposed to be found in \jsinline|Map.prototype|, exists in the object~\jsinline|m| or not. This means that an environment can break the map library by putting into a map object the properties that are meant to be found in its prototype, and also that the specification of maps needs to be strengthened to forbid this explicitly:
%\begin{Verbatim}[fontsize=\footnotesize,commandchars=\\\{\}]
%   Map (m, kvs) := DataProp(m, "_contents", c) * JSObject(c) * 
%                     \textcolor{red}{((m, "get) -> none)} * \textcolor{red}{((m, "put") -> none)} * \textcolor{red}{((m, "validKey") -> none)} *
%                       KVPairs(c, kvs) * first(kvs, keys) * emptyFields(c, keys).
%\end{Verbatim}
%
%Bugs such as this will very rarely be caught by whole-program analyses, because there the entire state of the program is known and the test needs to be especially crafted with these bugs in mind. The reason that \cosette can catch them easily is because it is compositional and can run in partially described states.



\newpage
\section{Symbolic Execution for \jsil}\label{sec:jsil:symb:exec}
%!TEX root = ../main.tex

\pmax{Think story!}

We describe our novel symbolic execution engine for \jsil, an intermediate
representation for JavaScript analysis previously introduced by \citet{javert}. 
In \S\ref{subsec:jsil:analysis:formalism}, we define an abstract semantics for 
\jsil, which we then instantiate to obtain its concrete and symbolic semantics.
In \S\ref{sex:formal:guarantees}, we present the formal guarantees of 
our symbolic analysis, including: \dtag{i} a soundness result for the \jsil symbolic 
execution, \dtag{ii} a guarantee of absence of false positives for bug-finding, \pmaxinline{Do we know what bug-finding is here?}
\dtag{iii} a justification result for the lifting of analyses on compiled \jsil code back to JavaScript;
and \dtag{iv} \polish{a verification result that precisely states the conditions 
under which symbolic execution gives us verification guarantees.}
Finally, in \S\ref{subsec:jsil:analysis:implementation}, we give a brief overview
of our implementation. %in \rosette~\cite{Rosette1,Rosette2} NO NO NO! Burn the witch!


\subsection{Symbolic Semantics}\label{subsec:jsil:analysis:formalism}

\subsubsection{Syntax} 
\jsil is a simple goto language featuring top-level procedures and commands operating on object heaps. It was purposefully designed to natively support the main dynamic features of JavaScript: extensible objects; dynamic property access; and dynamic procedure calls. The syntax of \jsil is shown below. 

\vspace{5pt}
\begin{display}{Syntax of the \jsil Language}{
\begin{tabular}{llllll}
	 Numbers: $\jnumber \in \numbers$ &  Booleans: $\jbool \in \bools$ & \  \ Strings: $\jstring \in \strings$ & \  \ Locations: $\loc \in \locs$ & \  Vars: $\jvar \in \jvars$ & \ Types: $\jtype \in \jtypes$ \\[0.1cm]
\multicolumn{6}{l}{Values: $\val \in \vals$ \defeq \ $\jnumber \! \mid \! \jbool \! \mid \! \jstring \! \mid \! \jundefined \! \mid \! \jnull \! \mid \! \jempty \! \mid \! \loc \! \mid \! \jtype \! \mid \!  \pid$ \qquad Expressions: $\jsilexpr \in \exprs$ \defeq \ $\val \mid \jvar \mid \ominus\ \jsilexpr \mid \jsilexpr \binop{} \jsilexpr $} \\[0.1cm]
	\multicolumn{6}{l}{Basic Commands: $\bcmd \in \bcmds$ \defeq\ $\jsilskip \mid \jvar := \jsilexpr  \mid \jvar := \jsilnew() \mid \jvar := [\jsilexpr, \jsilexpr] \mid [\jsilexpr, \jsilexpr] := \jsilexpr \mid$} \\[0.1cm]
	\multicolumn{6}{l}{\hspace{3.29cm} $\jsildelete(\jsilexpr, \jsilexpr) \mid \jvar := \hasfield(\jsilexpr, \jsilexpr) \mid \jvar := \getfields(\jsilexpr) \mid \jvar := \makesymbolic(\jtype)$} \\[0.1cm]
	% Commands
	\multicolumn{6}{l}{Commands: $\jcmd \in \cmds$ \defeq \ $ \bcmd \mid \goto \ i \mid  \ifgoto{\jsilexpr}{i}{j} \mid \jsilcall{\jvar}{\jsilexpr}{\jvec{\jsilexpr}}{j} 
	         \mid \assume(\jsilexpr) \mid \assert(\jsilexpr)$} \\[0.1cm]
	\multicolumn{6}{l}{Procedures : $\proc \in \procs$ \defeq \ $\procedure{\pid}{\jvec{\jvar}}{\jvec{\jcmd}}$}
 \end{tabular}}
\end{display}

\vspace{5pt}
\noindent \jsil \emph{values}, $\val \in \vals$, include numbers, booleans, strings, the special values $\jundefined$, $\jnull$, and $\jempty$, as well as types~$\jtype$, and procedure identifiers $\pid$.
\jsil~\emph{expressions}, $\jsilexpr \in \exprs$, include \jsil values, \jsil program variables $\jvar$, and various unary and binary operators, which, for instance, provide support for sets and lists. 

\jsil \emph{basic commands} enable the manipulation of extensible objects and do not affect control flow. 
They include $\jsilskip$, variable assignment, object creation, property access, property assignment, property deletion, membership check,  property collection, and symbolic variable creation. 

\jsil \emph{commands} include \jsil basic commands and commands related to control flow: conditional and unconditional gotos, dynamic procedure calls, and two special commands, $\assume$ and $\assert$, essential for symbolic execution, but with trivial concrete semantics.\footnote{\jsil also has a $\phi$-node assignment, which allows \JSComp to produce code directly in Static-Single-Assignment (SSA) \cite{SSA}. To avoid clutter, we omit the $\phi$-node assignment from the formal presentation as it does not impact the reasoning. Details can be found in \cite{javert}.} 
The two goto commands are straightforward: $\goto \ i$ jumps to the $i$-th command of the active procedure, and $\ifgoto{\jsilexpr}{i}{j}$ jumps to the $i$-th command if $\jsilexpr$ evaluates to $\jtrue$, and to the $j$-th if it evaluates to $\jfalse$. 
The dynamic procedure call $\jsilcall{\jvar}{\jsilexpr}{\jvec{\jsilexpr}}{j}$ first evaluates  $\jsilexpr$ and $\jvec{\jsilexpr}$ to obtain the procedure identifier and arguments, respectively, executes the appropriate procedure with these arguments, and, in the end, assigns its return value to $\jvar$.
If the procedure raises an error, control is transferred to the $j$-th command; otherwise, it simply follows to the next command. 

A \jsil program $\prog \in \progs$ can be seen as a set of top-level procedures of the form $\procedure{\pid}{\jvec{\jvar}}{\jvec{\jcmd}}$, where $\pid$ is the procedure identifier, $\jvec{\jvar}$ are its formal parameters, and its body ${\jvec{\jcmd}}$  is a sequence of \jsil commands.
Every \jsil program contains a special procedure $\jsilmain$\hspace{-2pt}, denoting the entry point of the program. 
\jsil procedures return via two dedicated indexes, $\retlab$ and $\errlab$, using two dedicated variables, $\retvar$ and $\errvar$. If a procedure reaches the $\retlab$ index, it returns normally with the return value denoted by $\retvar$; when it reaches $\errlab$, it returns an error, with the error value denoted by $\errvar$.

%

\subsubsection{Abstract Semantics}
We first define an abstract semantics for \jsil, which is parametric on a \jsil \emph{state signature}. 
Then, we instantiate this abstract semantics with a concrete state signature to obtain the \emph{concrete semantics} for \jsil, and with a symbolic state signature to obtain its symbolic counterpart.
This approach to the design of symbolic analyses has the following main benefits: \dtag{1} it streamlines the formalism, avoiding redundancy,
\dtag{2} it makes explicit the choices in the design of the symbolic semantics,
and \dtag{3} it leads to modular implementations, avoiding code duplication.


The abstract semantics is parametric on abstract states $\absstate \in \absstates$ and abstract values $\absval \in \absvals$.
Abstract states are assumed to have a store component $\absstore: \jvars \partialmap \absvals$, mapping program variables $\jvar \in \jvars$ to abstract values. Abstract values must include: \jsil values, $\val$; abstract locations, $\absloc$; abstract property names, $\absprop$; sets of abstract values; and the special value $\none$, used to indicate the absence of a property in an object. Abstract states expose the following functions and relations: 
\begin{description}
\setlength{\itemsep}{0.2em}
  \item[\jsil Expression Evaluation,] $\evalexpr{} :  (\jvars \partialmap \absvals)  \partialmap \exprs \partialmap \absvals$. 
  	The expression $\evalexpr{}(\absstore, \jsilexpr)$ evaluates to the abstract value of the \jsil expression $\jsilexpr$ under store $\absstore$. 
            
  \item[Store Selector,] $\stosel : \absstates \rightarrow (\jvars \partialmap \absvals)$. The expression $\stosel(\absstate)$ evaluates to the abstract store associated with a given abstract state $\absstate$. For clarity, we write $\absstate.\stosel$ instead of $\stosel(\absstate)$. 
             
  \item[Store Update,] $\stupdt{}: \absstates \rightarrow \jvars \rightarrow \absvals \rightarrow \absstates$. 
             The expression $\stupdt{}(\absstate, \jvar, \absval)$ denotes the abstract state obtained from $\absstate$ 
             by updating the abstract value corresponding to $\jvar$ in $\absstate.\stosel$ to $\absval$. 

   \item[Heap Update,] $\hpupdt{} : \absstates \rightarrow \absvals \partialmap \absvals \partialmap \absvals \partialmap \absstates$. 
             The expression $\hpupdt{}(\absstate, \absloc, \absprop, \absval)$ denotes the abstract state obtained from $\absstate$ 
             by updating the value of property $\absprop$ of the object at location $\absloc$ to $\absval$. 
                   
  
  \item[GetCell,] $\getcell \subseteq \absstates \times \exprs \times \exprs \times \absstates \times \absvals \times \absvals \times \absvals$. 
          We pretty-print $(\absstate, \jsilexpr_1, \jsilexpr_2, \absstate', \absloc, \absprop, \absval) \in \getcell$ as $\absgetcellrule{}{\absstate, \jsilexpr_1, \jsilexpr_2}{\absstate', (\absloc, \absprop, \absval)}$. 
          Informally, GetCell retrieves the value associated with a given property of a given object. More formally, if $\absgetcellrule{}{\absstate, \jsilexpr_1, \jsilexpr_2}{\absstate', (\absloc, \absprop, \absval)}$ holds, 
          then \dtag{1} $\absloc$ denotes the abstract location resulting from the evaluation of $\jsilexpr_1$, 
          \dtag{2} $\absprop$ denotes the abstract property name resulting from the evaluation of $\jsilexpr_2$, 
          \dtag{3} $\absval$ denotes the abstract value of the property $\absprop$ of the object at location $\absloc$, 
          and \dtag{4} $\absstate'$ denotes a (potential) modification of $\absstate$ that allows for the property inspection (discussed shortly). 
          Note that $\getcell$ is non-deterministic and is, therefore, a relation.
             
  \item[GetDomain,] $\getdomain \subseteq \absstates \rightarrow \exprs \partialmap \absvals$. 
           We pretty-print $(\absstate, \jsilexpr, \absval) \in \getdomain$ as $\absgetdomainrule{}{\absstate, \jsilexpr}{\absval}$. 
           If $\absgetdomainrule{}{\absstate, \jsilexpr}{\absval}$ holds, then $\absval$ denotes the 
           set of abstract property names associated with the object at the location resulting from the evaluation of $\jsilexpr$ 
           in $\absstate$. 
   
   \item[Assumption,] $\absassume{} : \absstates \rightarrow \exprs \partialmap \absvals$. 
            The expression $\absassume{}(\absstate, \jsilexpr)$ denotes the abstract state obtained from the 
            abstract state $\absstate$ by assuming that $\jsilexpr$ evaluates to the boolean $\jtrue$. 
  
   \item[Satisfiability,] $\abssat{} : \absstates \rightarrow \exprs \partialmap \bools$. 
            The expression $\abssat{}(\absstate, \jsilexpr)$ evaluates to $\jtrue$ if the \jsil expression $\jsilexpr$ is \polish{satisfiable} in the abstract 
            state $\absstate$, and $\jfalse$ otherwise.
             
   \item[Symbolic Value Creation,] $\absmakesymbolic : \jtypes \rightarrow \absvals$. 
            The expression $\absmakesymbolicrule{}{\jtype}$ evaluates to a fresh symbolic value of type $\jtype$.
\end{description}

\begin{figure}[ht!]
{\scriptsize
\begin{mathpar}
  \inferrule[\textsc{Skip}]{}
	{ \absbsemrule{\absstate, \jsilskip}{\absstate}{}}  	
\and 
\inferrule[\textsc{Property Collection}]
  {
           \absgetdomainrule{}{\absstate, \jsilexpr}{\absval}  
  }{\absbsemrule{\absstate, \jvar := \getfields(\jsilexpr)}{\stupdt{}(\absstate, \jvar, \absval)}{}} 
\and 
\inferrule[\textsc{Assignment}]
  {
        \absstore = \absstate.\stosel
        \quad 
        \absval = \evalexpr{}(\absstore, \jsilexpr)
  }{\absbsemrule{\absstate, \jvar := \jsilexpr}{\stupdt{}(\absstate, \jvar, \absval)}{}} 
\\
\inferrule[\textsc{Object Creation}]
  { 
   \loc \text{ fresh}
  }{\absbsemrule{\absstate, \jvar := \jsilnew()}{\hpupdt{}(\absstate, \loc, \protop, \jnull)}{}}
\and 
\inferrule[\textsc{Property Access}]
  { 
  	\absgetcellrule{}{\absstate, \jsilexpr_1, \jsilexpr_2}{\absstate', (-, -, \absval)}
  }{ \absbsemrule{\absstate, \jvar := [\jsilexpr_1, \jsilexpr_2]}{\stupdt{}(\absstate', \jvar, \absval)}{}}
%
\\
\inferrule[\textsc{Property Assignment}]
  {    
      \absgetcellrule{}{\absstate, \jsilexpr_1, \jsilexpr_2}{\absstate', (\absloc, \absprop, -)} \quad \absstore = \absstate.\stosel \quad \absval = \evalexpr{}(\absstore, \jsilexpr_3)
  }{\absbsemrule{\absstate, [\jsilexpr_1, \jsilexpr_2] := \jsilexpr_3}{\hpupdt{}(\absstate', \absloc, \absprop, \absval)}{}} 
\and
 \inferrule[\textsc{Property Deletion}]
  { 
        \absgetcellrule{}{\absstate, \jsilexpr_1, \jsilexpr_2}{\absstate', (\absloc, \absprop, \absval)}
        \quad
     	\absval \neq \none
  }{\absbsemrule{\absstate, \jsildelete(\jsilexpr_1, \jsilexpr_2)}{\hpupdt{}(\absstate', \absloc, \absprop, \none)}{}}
\\
\inferrule[\textsc{Member Check - True}]
  { 
     \absgetcellrule{}{\absstate, \jsilexpr_1, \jsilexpr_2}{\absstate', (-, -, \absval)} 
     \quad 
    \absval \neq \none
  }{\absbsemrule{\absstate, \jvar := \hasfield(\jsilexpr_1, \jsilexpr_2)}{	{}(\absstate', \jvar, \jtrue)}{}}
 \and 
\inferrule[\textsc{Member Check - False}]
  { 
     \absgetcellrule{}{\absstate, \jsilexpr_1, \jsilexpr_2}{\absstate', (-, -, \none)} 
  }{\absbsemrule{\absstate, \jvar := \hasfield(\jsilexpr_1, \jsilexpr_2)}{\stupdt{}(\absstate', \jvar, \jfalse)}{}} \\
  \inferrule[\textsc{Make Symbolic}]
  { 
     \absval = \absmakesymbolicrule{}{\jtype}
  }{\absbsemrule{\absstate, \jvar := \makesymbolic(\jtype)}{\stupdt{}(\absstate, \jvar, \absval)}{}}
\end{mathpar}}
 \vspace*{-0.6cm}
\caption{Abstract Semantics for Basic Commands: {\scriptsize$\absbsemrule{\absstate, \bcmd}{\absstate'}{}$}\label{abs:sem:bcmds:fig}}
\end{figure}

\begin{figure}[ht!]
{\scriptsize
\begin{mathpar} 
\inferrule[\textsc{Basic Command}]
   { 
     \ccmd[\prog][\abscs]{i} = \bcmd 
     \\\\
     \absbsemrule{\absstate, \bcmd}{\absstate'}{} 
   }{\abssemrule{\absstate, \abscs, i}{\absstate', \abscs, i{+}1}{\top}{\top}{}}
  %
  \and
  %
  \inferrule[\textsc{Cond. Goto - True}]
   { \ccmd[\prog][\abscs]{i} =  \ifgoto{\jsilexpr}{j}{k} 
     \\\\
     %   \absval = \evalexpr{}(\absstate.\stosel, \jsilexpr)
     %   \quad 
     \absstate' = \absassume{}(\absstate, \jsilexpr) 
   }
   {\abssemrule{\absstate, \abscs, i}{\absstate', \abscs, j}{\top}{\top}{}}
  %
  \and 
  % 
  \inferrule[\textsc{Cond. Goto - False}]
   { \ccmd[\prog][\abscs]{i} =  \ifgoto{\jsilexpr}{j}{k} 
      \\\\ 
       % \absval = \evalexpr{}(\absstate.\stosel, \jsilexpr)
       % \quad 
      \absstate' = \absassume{}(\absstate, \neg\jsilexpr) 
   }
   {\abssemrule{\absstate, \abscs, i}{\absstate', \abscs, k}{\top}{\top}{}}
  %
  \\
  %
  \inferrule[\textsc{Goto}]
   { 
   \ccmd[\prog][\abscs]{i} = \goto \, j \quad}
   {\abssemrule{\absstate, \abscs, i}{\absstate, \abscs, j}{\top}{\top}{}}
   %
   \qquad 
   %
   \inferrule[\textsc{Normal Return}]
   {
       \abscs = (-, \absstore', \jvar, i, -) :: \abscs'  
       \quad 
       \absstore = \absstate.\stosel
       \\\\
       \absstate' = \absstate[\stosel \mapsto \absstore'[\jvar \mapsto \absstore(\retvar)]]
   }  
   {\abssemrule{\absstate, \abscs, \retlab}{\absstate', \abscs', i}{\top}{\top}{}}
     %
   \qquad 
   %
      \inferrule[\textsc{Error Return}]
   { 
       \abscs = (-, \absstore', \jvar, -, j) :: \abscs'
       \quad 
       \absstore = \absstate.\stosel
       \\\\
       \absstate' = \absstate[\stosel \mapsto \absstore'[\jvar \mapsto \absstore(\errvar)]]
   }  
   {\abssemrule{\absstate, \abscs, \errlab}{\absstate', \abscs', j}{\top}{\top}{}}
      \\
    \inferrule[\textsc{Procedure Call}]
   { 
    \ccmd[\prog][\abscs]{i} =   \jsilcall{\jvar}{\jsilexpr}{\jsilexpr_i \mid_{i = 0}^{n}}{j}
     \and
     \absstore = \absstate.\stosel 
     \and 
    \evalexpr{}(\absstore, \jsilexpr) =  \pid' 
        \and
     \args(\prog, \pid') = \jsillist{\jvar_1, ..., \jvar_{m}} 
      \\\\
      \absval_i = \evalexpr{}(\absstore, \jsilexpr_i) \mid_{i = 0}^{n} 
     \and
      \absval_i = \jundefined \mid_{i = n+1}^{m}  
      \and 
      \absstore' = [ \jvar_i \mapsto \absval_i \mid_{i = 0}^{m}]
      \and 
      \abscs' = (\pid', \absstore, \jvar, i{+}1, j) :: \abscs 
   }
   {\abssemrule{\absstate, \abscs, i}{\absstate[\stosel \mapsto \absstore'], \abscs', 0}{\top}{\top}{}}
    \\ 
%
\inferrule[\textsc{Assume}]
  { 
      \ccmd[\prog][\abscs]{i}  = \assume(\jsilexpr) 
      \\\\ 
      %\absstore = \absstate.\stosel
      %\quad
      %\evalexpr{}(\absstore, \jsilexpr) =  \absval
      %\\\\
     \absstate' = \absassume{}(\absstate, \jsilexpr) 
  }{\abssemrule{\absstate, \abscs, i}{\absstate', \abscs, i{+}1}{\top}{\top}{}} 
\and
\inferrule[\textsc{Assert - True}]
  { 
      \ccmd[\prog][\abscs]{i}  = \assert(\jsilexpr)
      \\\\ 
      %\absstore = \absstate.\stosel
      %\quad
      %\absval = \evalexpr{}(\absstore, \jsilexpr)
      % \\\\
       \abssat(\absstate, \neg \jsilexpr) = \jfalse
  }{\abssemrule{\absstate, \abscs, i}{\absstate, \abscs, i{+}1}{\top}{\top}{}} 
\and 
\inferrule[\textsc{Assert - False}]
  { 
      \ccmd[\prog][\abscs]{i}  = \assert(\jsilexpr)
      \\\\ 
      %\absstore = \absstate.\stosel
      %\quad
      %  \absval = \evalexpr{}(\absstore, \jsilexpr)
      % \\\\
      \abssat(\absstate, \neg \jsilexpr) = \jtrue
  }{\abssemrule{\absstate, \abscs, i}{\absstate, \abscs, i}{\top}{\bot}{}} 
 \end{mathpar}}
% \pmax{What happens when we exit from main, how do we stop? Basically, cmd returns nothing and we can't reduce?}
 \vspace*{-0.6cm}
\caption{Abstract Execution for Control Flow Commands: $\abssemrule{\absstate, \abscs, i}{\absstate', \abscs', j}{\mode}{\mode'}{}$}
\label{abs:sem:cmds:fig}
\end{figure}

Transitions for basic commands and control flow commands are given 
in Figures~\ref{abs:sem:bcmds:fig} and~\ref{abs:sem:cmds:fig}. Transitions for 
basic commands are of the form 
$\absbsemrule{\absstate, \bcmd}{\absstate'}{}$, meaning that the execution of the basic command 
$\bcmd$ in the abstract state $\absstate$ results in the abstract state $\absstate'$. 
%
To describe transitions for control flow commands, we introduce \emph{execution modes} and 
\emph{call stacks}.  \jsil has two execution modes, ranged over by $\mode$: 
$\top$, which indicates that the execution can proceed; and~$\bot$, which indicates that a failure 
has occurred and the execution must stop. Call stacks, ranged over by $\abscs$, are lists of tuples of the form $(\pid, \absstore, \jvar, i, j)$, where: 
\dtag{1}~$\pid$~is the identifier of the procedure that is currently executing;
\dtag{2}~$\absstore$~is the store of the procedure that called $\pid$; 
\dtag{3}~$\jvar$~is the variable to which the return of $\pid$ must be assigned in $\absstore$; 
\dtag{4} $i$ is the index of the command to which the control must jump after the execution of $\pid$ in case of normal return; 
and \dtag{5} $j$ is the index to which it must jump in case of error return. 
Transitions for control flow commands have the form:  $\abssemrule{\absstate, \abscs, i}{\absstate', \abscs', j}{\mode}{\mode'}{}$, 
meaning that, in the context of the entire program $\prog$, the evaluation of the $i$-th command of the first procedure in the 
call stack $\abscs$, in state $\absstate$ and execution mode $\mode$, generates 
the state $\absstate'$, call stack $\abscs$,  and the next command to be evaluated is the $j$-th command of the first procedure 
of the call stack~$\abscs'$, in execution mode $\mode'$. 
For clarity, we keep the executing program implicit and make use of a function $\ccmd{\prog, \cs, i}$, which 
returns the $i$-th command of the procedure that is first in $\cs$. %We write $\ccmd{i}$ when $\prog$ and $\ctx$ are implicit.

\jfs{
\begin{enumerate}
   \item Should we explain the rules?
   \item We shou
\end{enumerate}
}

\subsubsection{Concrete Semantics}
A \jsil concrete state $\jstate$ is a pair $(\heap, \store)$ consisting of a heap and a store. 
A heap, $\heap \in \heaps$, is a partial function mapping pairs of  object locations and property names (strings) to \polish{\jsil values},
whereas a store, $\store \in \stores$, maps \jsil program variables to \jsil values. 
Given a heap~$\heap$, we denote: the empty heap by $\hemp$; heap lookup by $\heap(\loc, \jstring)$; 
a heap cell by $\hcell{\loc}{\jstring}{\val}$, meaning that  $\heap(\loc,\jstring) = \val$; 
heap update by $\heap[(\loc, p) \mapsto \val]$, denoting the heap $\heap$ extended with $\heap(\loc,\jstring) = \val$;
and the union of two disjoint heaps by $\heap_1 \dunion \heap_2$.
We instantiate the abstract semantics for the concrete case by providing the appropriate definitions 
for the required abstract functions/relations below, noting that abstract values are instantiated to
\jsil values extended with the property absence indicator $\none$.
We write $\absbsemrule{\jstate, \bcmd}{\jstate'}{\concrete}$ for the concrete semantic 
judgement for basic commands and $\abssemrule{\jstate, \cs, i}{\jstate', \cs', j}{\mode}{\mode'}{\concrete}$ 
for control flow commands. 

\begin{display}{Concrete Semantics Rules}
\text{
{\scriptsize
\begin{mathpar} 
  \inferrule[\textsc{Expression Evaluation}]
  {}{\evalexpr{c}(\store, \val) = \val \qquad 
  	\evalexpr{c}(\store, \jvar) = \store(\jvar) \qquad 
	\evalexpr{c}(\store, \ominus \jsilexpr) = \semop{\ominus} \ \evalexpr{c}(\store, \jsilexpr) \qquad
	\evalexpr{c}(\store, \jsilexpr_1 \oplus \jsilexpr_2) = \evalexpr{c}(\store, \jsilexpr_1) \ \semop{\oplus} \ \evalexpr{c}(\store, \jsilexpr_2)}
  \\
  \inferrule[\textsc{GetDomain}]
   { 
        \evalexpr{c}(\store, \jsilexpr) = \loc
      \quad
       \heap = \heap' \, \uplus \, \big((\loc, p_i) \mapsto \val_i \big)\mid_{i = 0}^m  
      \quad
          (\loc,-) \notin \domain (\heap')  
       }{  \absgetdomainrule{\concrete}{(\heap, \store), \jsilexpr}{\jsilset{p_1, ..., p_m}}}
  %
  \and
  %
    \inferrule[\textsc{GetCell - Explicit Cell}]
   { 
         \evalexpr{c}(\store, \jsilexpr_1) = \loc
        \quad 
         \evalexpr{}(\store, \jsilexpr_2) = p
        \quad
       \heap = - \, \uplus \, (\loc, p) \mapsto \val
        }{  \absgetcellrule{\concrete}{(\heap, \store), \jsilexpr_1, \jsilexpr_2}{(\heap, \store), (\loc, p, \val)}}
 %
 \\
   \inferrule[\textsc{Store Selector}]
   {}{  \stosel((\heap, \store)) =  \store}
  	\qquad
   \inferrule[\textsc{Store Update}]
   { 
         \store' = \store[\jvar \mapsto \val]
   }{  \stupdt{\concrete}((\heap, \store), \jvar, \val) =  (\heap, \store')}
   \qquad
       \inferrule[\textsc{Heap Update - Non-$\none$}]
   { 
         \heap' = \heap[(\loc, p) \mapsto \val]
         \quad 
         \val \neq \none
   }{  \hpupdt{\concrete}((\heap, \store), \loc, p, \val) =  (\heap', \store)}
 \qquad
       \inferrule[\textsc{Heap Update - $\none$}]
   { 
         \heap = \heap' \dunion (\loc, p) \mapsto -
   }{  \hpupdt{\concrete}((\heap, \store), \loc, p, \none) =  (\heap', \store)}
  %

   \\
      \inferrule[\textsc{Assumption}]
   {}{  \absassume{\concrete}(\jstate, \jtrue) =  (\jstate, \jtrue) }
\and 
    \inferrule[\textsc{Satisfiability}]
   {}{  \abssat{\concrete}(\jstate, \jbool) = \jbool}
  \and
      \inferrule[\textsc{Symbolic Value Creation }]
   {\val \in \vals \text{ is of type } \jtype}{\absmakesymbolicrule{\concrete}{\jtype} = \val}
 \end{mathpar}
  }}
 \end{display}
  
\pmax{Perhaps a comment on the semantics - eval of expressions and something?}

\subsubsection{Symbolic Semantics}
In order to symbolically execute \jsil programs, we extend the syntax of \jsil expressions with 
symbolic strings $\sstring \in \sstrings$ and symbolic numbers $\snumber \in \snumbers$. 
For convenience, we use $\svars$ to denote the union of $\sstrings$ and $\snumbers$ 
and $\svar$ to range over $\svars$. We introduce: \jsil symbolic expressions, $\sexpr \in \sexprs$, defined as follows: $\sexpr \triangleq \val \mid \svar \mid \unoper\ \sexpr \mid \sexpr \binoper \sexpr$. 
Furthermore, we redefine \jsil expressions, $\pvsexpr \in \pvsexprs$ to take into account symbolic values
 as follows: $\pvsexpr \triangleq \val \mid \jvar \mid \svar \mid \unoper\ \pvsexpr \mid \pvsexpr \binoper \pvsexpr$.
Extended \jsil expressions differ from symbolic expressions in that they can contain program variables.
We extend heaps, stores, and call stacks with symbolic expressions, obtaining symbolic 
heaps, stores, and call stacks, respectively ranged over by $\sheap$, $\sstore$, and $\scs$.
A symbolic heap, $\sheap \in \sheaps$, is a partial function mapping pairs of  
object locations and symbolic expressions to symbolic expressions. 
A symbolic store, $\sstore \in \sstores$, is a mapping from program variables 
$\jvar \in \jvars$ to symbolic expressions. Therefore, an evaluation of a \jsil extended expression $\pvsexpr$ in a symbolic store $\sstore$ always yields a 
symbolic expression $\sexpr$.
A symbolic call stack $\scs$ only differs from a concrete call stack in that it contains 
symbolic stores instead of concrete stores.


A \emph{symbolic state} $\sstate = (\sheap, \sdom, \sstore, \pc)$ is a 4-tuple consisting of a 
symbolic heap $\sheap$, a symbolic domain $\sdom$, a symbolic store $\sstore$, and a path condition $\pc$. 
The \emph{heap domain} maps object locations to set of properties that may be defined in the corresponding object.  
The \emph{path condition}~\cite{symb:exec:survey} is a first-order quantifier-free formula over symbolic strings and 
numbers, which accumulates constraints on the given symbolic inputs that trigger 
the execution to follow the path that led to the current symbolic state.
To avoid clutter, we conflate logical values with \jsil logical values and \jsil logical 
operators with the boolean logical operators. Alternatively, we could have chosen to 
have two different types, \jsil logical expressions and logical expressions, together with a lifting 
function for converting the former to the latter. Our choice simplifies both reasoning 
and~presentation, in that a path condition is simply a \jsil symbolic expression of type boolean. 

We instantiate the abstract semantics for the symbolic case by providing the appropriate definitions 
for the required abstract functions/relations. 
We write $\absbsemrule{\jstate, \bcmd}{\jstate'}{\symbolic}$ for the symbolic semantic 
judgement for basic commands and $\abssemrule{\jstate, \cs, i}{\jstate', \cs', j}{\mode}{\mode'}{\symbolic}$ 
for control flow commands. 

\vspace{5pt}
\begin{display}{Symbolic Semantics Rules}
\text{
{\scriptsize
\begin{mathpar} 
  \inferrule[\textsc{GetDomain}]
   { 
        \evalexpr{}(\sstore, \jsilexpr) = \sloc
      \quad
       \sheap = \sheap' \, \uplus \, \big((\sloc, \sexprp_i) \mapsto \sexprv_i \big)\mid_{i = 0}^m  
        \\\\
        %
          (\sloc,-) \notin \domain (\sheap')  
        \quad
         \pc \vdash \jsilset{\sexprp_1, ..., \sexprp_m} = \sdom(\sloc)
         \\\\
        \forall_{0 \leq i \leq n} \, \sexprv_i \neq \none 
         \quad 
           \forall_{n < i \leq m} \, \sexprv_i = \none 
   }{  \absgetdomainrule{\symbolic}{(\sheap, \sdom, \sstore, \pc), \jsilexpr}{\jsilset{\sexprp_1, ..., \sexprp_n}}}
  \and
 % 
     \inferrule[\textsc{GetCell - Not Found}]
   { 
         \evalexpr{}(\sstore, \jsilexpr_1) = \sloc
        \quad 
        \evalexpr{}(\sstore, \jsilexpr_2) = \sexprp
      \quad
       \sheap = \sheap'' \, \uplus \, \big((\sloc, \sexprp_i) \mapsto \sexprv_i \big)\mid_{i = 0}^m
      \\\\
       (\sloc,-) \notin \domain (\sheap'')
        \quad
        \sdom(\sloc) = \sexprv_d
        \quad 
        \sheap' = \sheap[(\sloc, \sexprp) \mapsto \none]
       \\\\ 
       \sdom' = \sdom[\sloc \mapsto \sexprv_d \cup \jsilset{\sexprp}] 
       \quad 
       \pc' = \pc \wedge \sexprp \not\in (\sexprv_d \cup \jsilset{\sexprp_i \mid_{i=0}^m})
   }{  \absgetcellrule{\symbolic}{(\sheap, \sdom, \sstore, \pc), (\jsilexpr_1, \jsilexpr_2)}{(\sheap', \sdom', \sstore, \pc'), (\sloc, \sexprp, \none)}}
  %
  \\
  %
    \inferrule[\textsc{GetCell - Found}]
   { 
         \evalexpr{}(\sstore, \jsilexpr_1) = \sloc
        \quad 
        \evalexpr{}(\sstore, \jsilexpr_2) = \sexprp
      \\\\
       \sheap = - \, \uplus \, (\sloc, \sexprp') \mapsto \sexprv
        \\\\
        %
       %\pc \not\vdash \sexprp' \neq \sexprp 
       %\quad 
       \sstate' = (\sheap, \sdom, \sstore, \pc \wedge (\sexprp = \sexprp'))
   }{  \absgetcellrule{\symbolic}{(\sheap, \sdom, \sstore, \pc), (\jsilexpr_1, \jsilexpr_2)}{\sstate', (\sloc, \sexprp, \sexprv)}}
 %
 \and 
   \inferrule[\textsc{Store Update}]
   { 
         \sstate = (\sheap, \sdom, \sstore, \pc) 
         \\\\
         \sstore' = \sstore[\jvar \mapsto \val]
         \\\\
         \sstate' = (\sheap, \sdom, \sstore', \pc)
   }{  \stupdt{\symbolic}(\sstate, \jvar, \val) = \sstate'  }
   %
   \and 
       \inferrule[\textsc{Heap Update}]
   { 
         \sstate = (\sheap, \sdom, \sstore, \pc) 
         \\\\ 
         \sheap' = \sheap[(\sloc, \sexprp) \mapsto \sexprv]
         \\\\
         \sstate' = (\sheap', \sdom, \sstore, \pc)
   }{  \hpupdt{\symbolic}(\sstate, (\sloc, \sexprp), \sexprv) = \sstate' }
\\
  %
      \inferrule[\textsc{Assume}]
   { 
         \sstate = (\sheap, \sdom, \sstore, \pc) 
   }{  \absassume{\symbolic}(\sstate, \sexprv) =  (\sheap, \sdom, \sstore, \pc \wedge (\sexprv = \jtrue)) }
  \qquad 
    \inferrule[\textsc{Satisfiable - True}]
   { 
         \pc = \sstate.\pcsel 
         \quad 
         (\pc \wedge (\sexprv = \jtrue)) \text{ SAT}
   }{  \abssat{\symbolic}(\sstate, \sexprv) = \jtrue}
 \qquad 
    \inferrule[\textsc{Satisfiable - False}]
   { 
         \pc = \sstate.\pcsel 
         \quad 
         (\pc \wedge (\sexprv = \jtrue)) \text{ UNSAT}
   }{  \abssat{\symbolic}(\sstate, \sexprv) = \jfalse}
 \end{mathpar}}}
 \end{display}

\vspace{5pt}
\begin{wrapfigure}{R}{0.4\textwidth}
\vspace*{-0.25cm}
{\small
\hspace*{0.25cm} $\mathtt{0\quad o := new\ ()}$ \\
\hspace*{0.25cm} $\mathtt{1\quad o[\hat{s}] := 42};$ \\
\hspace*{0.25cm} $\mathtt{2\quad x := getFields(o);}$ \\
\hspace*{0.25cm} $\mathtt{3\quad assert\ (card \ x == 2)}$
}
\vspace*{-0.3cm}
\end{wrapfigure}
\myparagraph{Example:}
To get a better intuition of how symbolic execution works, let us take a look at the snippet of code shown on the right. 
This code: 
	\dtag{0}~creates a new object $\mathtt{o}$;
	\dtag{1}~assigns 42 to a symbolic property $\mathtt{\hat{s}}$ of $\mathtt{o}$; 
	\dtag{2}~collects all the properties of $\mathtt{o}$ into a set and assigns this set to $\mathtt{x}$; and
	\dtag{3}~asserts that the cardinality of the set in $\mathtt{x}$ is 2, i.e.~that~$\mathtt{o}$ has two properties in the end. 
	     This last assertion will produce a failing symbolic execution. Let us understand why.
%
We start from an empty heap, empty domain, empty store, and path condition $\jtrue$: 
$\sstate_0 = \tuple{\hemp, \demp, \storeemp, \jtrue}$. After the execution of the first command, $\mathtt{o := new\ ()}$, using the \textsc{Basic Command} and \textsc{Object Creation} rules, we get to the state {\small $\sstate_1 = \tuple{\{ \mathtt{l_o : \{ ``@proto" : null} \} \}, \mathtt{l_o : \{ ``@proto"} \} \}, \{ \mathtt{o : l_o} \} , \jtrue}$}, illustrated at the top of Figure~\ref{fig:sexecexample}.
The next command to be executed is the property assignment $\mathtt{o[\hat{s}] := 42}$. The semantic rule for \textsc{Property Assignment} makes use 
of the abstract function \textsc{GetCell}. There are two potential \textsc{Get Cell} rules (\textsc{Not Found} and \textsc{Found}), and in our case, both of them are applicable. The key strategy is to branch on the targeted property of the object (in our case, the symbolic property $\hat{s}$ of object at location $\mathtt{l_o}$) being equal to any one or none of the already existing properties of the object (in our case, we have only $\mathtt{``@proto"}$), adding the appropriate equalities and/or inequalities to the path condition, and proceeding with the symbolic execution for all obtained branches. In this case, this means that the symbolic execution will branch on whether or not $\hat{s} = \mathtt{``@proto"}$. We obtain two symbolic states, shown in the second row of Figure \ref{fig:sexecexample}. The left branch corresponds to the (\textsc{Found}) case, when $\hat{s} = \mathtt{``@proto"}$: this equality is added to the path condition and the value of the property $\mathtt{``@proto"}$ is updated to 42. In the right branch, we have that $\hat{s} \neq \mathtt{``@proto"}$ (\textsc{Not Found}), hence object $\mathtt{o}$ has two properties: $ \mathtt{``@proto"}$, with value $\jnull$; and $\hat{s}$, with value 42.
The execution then continues in both branches with the property collection command $\mathtt{x := getFields(o)}$, which assigns the set of properties of the object $\mathtt{o}$ to the variable~$\mathtt{x}$ (last row of Figure~\ref{fig:sexecexample}). Finally, we execute $\mathtt{assert\ (card \ x = 2)}$, asserting that $\mathtt{o}$ has exactly two properties, which we observe to hold in the right branch, but not in the left.
Therefore, following the \textsc{Assert - False} rule, we obtain a failing symbolic execution trace, from which a concrete counter-model can be derived ($\hat{s} = \mathtt{``@proto"}$).

\begin{figure}[!t]
\centering
\includegraphics[width=0.83\textwidth]{symbSemEx.png}
\vspace*{-0.1cm}
\caption{Example of a \jilette symbolic execution}
\label{fig:sexecexample}
\vspace*{-0.3cm}
\end{figure}




%\caption{Concrete Semantics Rules}
%\end{figure}


% Given a heap~$\heap$, we denote: a heap cell by $\hcell{\loc}{\jstring}{\val}$, meaning that  $h(\loc,\jstring) = \val$; the union of two disjoint heaps by $\oheap_1 \dunion \oheap_2$; heap lookup by $\hread{\oheap}{\loc}{\jstring}$; and the empty heap by $\hemp$.
%A \jsil variable store, $\store \in \stores$, is a mapping from JSIL program variables $\jvar \in \jvars$ to JSIL values. 



%the heap $\heap'$, store $\store'$, call stack $\ctx'$,   
%and the next command to be evaluated is the $i'$-th command of the first procedure of the call stack~$\ctx'$, in execution mode $\mode'$

%the heap $\heap$, store $\store$, . Due to space constraints and as the transitions for JSIL symbolic execution are  similar, we give the full semantics for JSIL control flow commands in the~Appendix. % So far, so boring.




%We denote the semantic interpretation of unary operators $\unoper$ by $\semop{\unoper}$, the semantic interpretation of binary operators $\binoper$ by $\semop{\binoper}


\subsection{Formal Guarantees}\label{sex:formal:guarantees}

\subsubsection{Preliminaries}
To establish the soundness of symbolic execution, we need to relate 
symbolic states to concrete states. To this end, we make use of \emph{symbolic environments} 
$\senv : \svars \rightharpoonup \vals$, mapping symbolic values to concrete values. 
A symbolic environment is \emph{well-formed} if it maps symbolic 
values to concrete values of the appropriate type (i.e.~symbolic strings are mapped to strings 
and symbolic numbers are mapped to numbers). In the following, we always 
assume well-formed symbolic environments. 
%%
Given a symbolic environment $\senv$, we define the interpretation of symbolic states as follows. 

\begin{display}{Symbolic State Interpretation}
{\scriptsize 
\begin{mathpar}
\inferrule[Symbolic Expressions]
{}{
\interpret{\symbconc}{\senv}(\val) \semeq \val 
%
\qquad
%
\interpret{\symbconc}{\senv}(\svar) \semeq \senv(\svar)
%
\qquad
%
\interpret{\symbconc}{\senv}(\unoper\ \sexpr) \semeq \semop{\unoper} (\interpret{\symbconc}{\senv}(\sexpr))
%
\qquad 
\interpret{\symbconc}{\senv}(\sexpr_1 \binoper \sexpr_2) \semeq \semop{\binoper}(\interpret{\symbconc}{\senv}(\sexpr_1), \interpret{\symbconc}{\senv}(\sexpr_2)) 
}
\\
%
\inferrule[Symbolic Stores]{}{
 \interpret{\symbconc}{\senv}(\storeemp) \semeq \storeemp   
\qquad
\frac{
   \val = \interpret{\symbconc}{\senv}(\sexpr) 
   \quad 
   \store = \interpret{\symbconc}{\senv}(\sstore)
}{
  \interpret{\symbconc}{\senv}((\jvar: \sexpr) \dunion \sstore) \semeq (\jvar: \val) \dunion \store 
}}
\and
\inferrule[Symbolic Contexts]{}{
\interpret{\symbconc}{\senv}(\lstemp) \semeq \lstemp
\qquad
\frac{
    \store = \interpret{\symbconc}{\senv}(\sstore)
    \quad 
    \cs = \interpret{\symbconc}{\senv}(\scs)
}{
\interpret{\symbconc}{\senv}((\pid, \sstore, \jvar, i, j) \lstcons \scs) \semeq (\pid, \store, \jvar, i, j) \lstcons \cs
}}
 \\
 %
\inferrule[Symbolic Heaps]{}{
\interpret{\symbconc}{\senv}(\hemp) \semeq \hemp, \emptyset
\qquad
\frac{
  \begin{array}{c}
   l = \interpret{\symbconc}{\senv}(\sexprl) \quad 
    p =  \interpret{\symbconc}{\senv}(\sexprp) \\
    v =  \interpret{\symbconc}{\senv}(\sexprv) \quad 
    \heap = \hcell{l}{p}{v}
  \end{array}
}{
\interpret{\symbconc}{\senv}(\hcell{\sexprl}{\sexprp}{\sexprv}) \semeq  \heap, \emptyset
}
\qquad
\frac{
  \begin{array}{c}
   l = \interpret{\symbconc}{\senv}(\sexprl) \quad 
   p = \interpret{\symbconc}{\senv}(\sexprp)  \\ 
   \hdom = \jsilset{ (l, p) }
    \end{array}
}{
  \interpret{\symbconc}{\senv}(\hcell{\sexprl}{\sexprp}{\none}) \semeq \hemp, \hdom
}
\qquad
\frac{
  \begin{array}{c}
   \interpret{\symbconc}{\senv}(\sheap_i) = h_i, \hdom_i \mid_{i =1,2} \\ 
   \heap = h_1 \dunion h_2 \quad 
   \hdom = \hdom_1 \dunion \hdom_2
   \end{array}
}{
\interpret{\symbconc}{\senv}(\sheap_1 \dunion \sheap_2) \semeq \heap, \hdom
}}
\\
\inferrule[Symbolic Domains]
{
  \hdom =  \{ (l, p) \mid \sexprl \in \domain(\sdom) \, \wedge \, l = \interpret{\symbconc}{\senv}(\sexprl) 
                 \, \wedge \, p \not\in \interpret{\symbconc}{\senv}(\sdom(\sexprl)) \}
}{
\interpret{\symbconc}{\senv}(\sdom) 
    \semeq \hdom 
}
\and
\inferrule[Symbolic States]{
    \sstate = (\sheap, \sdom, \sstore, \pc) 
    \quad 
   \interpret{\symbconc}{\senv}(\sheap) = \heap, \hdom_1
   \quad
   \interpret{\symbconc}{\senv}(\sdom) = \hdom_2
   \\\\
   \interpret{\symbconc}{\senv}(\sstore) = \store
    \quad 
    \interpret{\symbconc}{\senv}(\pc) = \jtrue 
    \quad 
    \hdom = \domain(h) \cup \hdom_1 \cup \hdom_2 
}{
\interpret{\symbconc}{\senv}(\sstate) \semeq \{ ((\heap , \store), \heap_f) \mid \domain(\heap_f) \cap \hdom = \emptyset \} 
}

\end{mathpar}}
\end{display}

\jfs{
\begin{enumerate}
\item Explain the rules. 
\item Explain how the interpretation of symbolic states is not the small footprint interpretation - it takes all the possible compatible frames into account. 
\item Explain how the nones + domains restrict all the possible frames. 
\end{enumerate}
}
 


\subsubsection{Main Results}
The \emph{soundness theorem} (Theorem~\ref{teo:soundness:jsil:symb:exe}) states that if we have a symbolic trace given by 
$\transabssemrule{\sstate, \scs, i}{\sstate', \scs', j}{\mode}{\mode'}{\symbolic}$ and a concrete CF-state $(\jstate, \cs)$
in the models of the initial symbolic CF-state filtered by the final path condition $\sstate'.\pcsel$,  
then there exists a concrete CF-state $(\jstate', \cs')$ in the models of the final symbolic CF-state, such that: 
$\transabssemrule{\jstate, \cs, i}{\jstate', \cs', j}{\mode}{\mode'}{\concrete}$. 
We use the final path condition $\sstate'.\pcsel$ for both the models of the initial and final 
symbolic states because we only care about the initial concrete CF-states for which 
the concrete execution follows the same path as the symbolic~execution. 

\begin{theorem}[Bounded Soundness]\label{teo:soundness:jsil:symb:exe}
$$
\begin{array}{l}
\transabssemrule{\sstate, \scs, i}{\sstate', \scs', j}{\mode}{\mode'}{\symbolic}  \ \wedge \ (\jstate, \heap_f, \cs) \in \interpret{\symbconc}{\senv}(\sstate \, \wedge \, \sstate'.\pcsel, \scs) \\ \quad \quad 
    \ \implies \ \exists \senv', \jstate', \cs' \, . \, 
        \transabssemrule{\jstate \dunion \heap_f, \cs, i}{\jstate' \dunion \heap_f, \cs', j}{\mode}{\mode'}{\concrete}
               \ \wedge \ (\jstate', \heap_f, \cs') \in \interpret{\symbconc}{\senv'}(\sstate', \heap_f, \scs')
\end{array}
$$
\end{theorem}

The \emph{bug-finding} corollary (Corollary~\ref{bug:finding}) states that if 
we find a symbolic trace that results in a failed assertion, 
then there also exists a concrete execution that will cause that assertion to fail.
Observe that the analysis is designed in such a way that there are no false positives, 
meaning that if we find a failing symbolic trace,
we can always instantiate its symbolic values obtaining a concrete counter-model for the 
failing assertion. This is essential, as \jilette is primarily a \emph{bug-finding} tool.


\begin{corollary}[Bug-finding]\label{bug:finding}
$$
\transabssemrule{\sstate, \scs, i}{\sstate', \scs', j}{\mode}{\bot}{\symbolic}  
      \ \implies \  \exists \jstate, \cs \, . \, \transabssemrule{\jstate, \cs, i}{-, -, j}{\mode}{\bot}{\concrete} 
$$
\end{corollary}

Finally, the \emph{verification} corollary (Corollary~\ref{corollary:verification})
states that if we have symbolically explored all the possible execution paths
starting from a given symbolic CF-state $(\sstate, \scs)$,  
then the execution of the program starting from  any concrete state in the models 
of the initial symbolic state (under the initial path condition) will result in a final concrete state
in the models of one of the final symbolic states (under its associated path condition).  
As \jilette does not infer loop invariants, if a \jsil program contains loops that cannot be 
unrolled statically, we will never be in the case of the verification corollary. 

\begin{corollary}[Verification]\label{corollary:verification}
$$
\begin{array}{l}
\wedge_{k=1}^n \, \big( \transabssemrule{\sstate, \scs, i}{\sstate_k, \scs_k, j_k}{\mode}{\mode_k}{\symbolic}  \big)
    \, \wedge \, \big( \sstate.\pcsel \vdash \vee_{k=1}^n \sstate_k.\pcsel \big) 
    \, \wedge \, (\jstate, \heap_f, \cs) \in \interpret{\symbconc}{\senv}(\sstate, \scs)
    \\ \quad \quad
%
      \ \implies \ \exists 1 \leq l \leq n, \senv', \jstate', \cs' \, . \, 
           \transabssemrule{\jstate, \cs, i}{\jstate', \cs', j_l}{\mode}{\mode_l}{\concrete}
           \, \wedge \, 
           (\jstate', \heap_f, \cs') \in \interpret{\symbconc}{\senv'}(\sstate_l, \scs_l)
\end{array}
$$
\end{corollary}

%
The {\bf proofs} of the above results are given in the Appendix. \polish{What do these results mean with respect to related work? How is this maintainable?} 

\subsubsection{Proof Methodology}

\vspace{5pt}
\begin{display}{Instrumented Semantics Rules}
\text{
{\scriptsize
\begin{mathpar} 
 \inferrule[\textsc{GetDomain}]
   { 
        \semexpr{\jsilexpr}{\store} = \loc
      \quad
       \iheap = \iheap' \, \uplus \, \big((\loc, p_i) \mapsto - \big)\mid_{i = 0}^m  
        \\\\
        %
          (\loc,-) \notin \domain (\iheap')  
        \quad
          \jsilset{p_1, ..., p_m} = \idom(\loc)
          \\\\
        \forall_{0 \leq i \leq n} \, \val_i \neq \none 
         \quad 
           \forall_{n < i \leq m} \, \val_i = \none 
   }{  \absgetdomainrule{\instrumented}{(\iheap, \idom, \store), \jsilexpr}{\jsilset{p_1, ..., p_n}}}
  \and
 % 
     \inferrule[\textsc{GetCell - Not Found}]
   { 
         \semexpr{\jsilexpr_1}{\store} = \loc
        \quad 
         \semexpr{\jsilexpr_2}{\store} = p
      \quad
       \iheap = \iheap'' \, \uplus \, \big((\loc, p_i) \mapsto \val_i \big)\mid_{i = 0}^m
      \\\\
       (\loc,-) \notin \domain (\iheap'')
        \quad
        \idom(\loc) = \val_d
        \quad 
       p \not\in \val_d \cup \jsilset{p_i \mid_{i=0}^m})
       \\\\
        \iheap' = \iheap[(\loc, p) \mapsto \none]
       \quad 
       \idom' = \idom[\loc \mapsto \val_d \cup \jsilset{p}]
     }{  \absgetcellrule{\instrumented}{(\iheap, \idom, \store), (\jsilexpr_1, \jsilexpr_2)}{(\iheap', \idom', \store), (\loc, p, \none)}}
  %
  \\
  %
    \inferrule[\textsc{GetCell - Found}]
   { 
         \semexpr{\jsilexpr_1}{\store} = \loc
        \quad 
         \semexpr{\jsilexpr_2}{\store} = p
      \quad
       \iheap = - \, \uplus \, (\loc, p) \mapsto \val
        }{  \absgetcellrule{\instrumented}{(\iheap, \idom, \store), (\jsilexpr_1, \jsilexpr_2)}{(\iheap, \idom, \store), (\loc, p, \val)}}
%
  \and 
   %
    \inferrule[\textsc{Store Update}]
   { 
         \store' = \store[\jvar \mapsto \val]
   }{  \stupdt{\instrumented}((\iheap, \idom, \store), \jvar, \val) =  (\heap, \idom, \store')}
 %
 \and
 % 
       \inferrule[\textsc{Heap Update}]
   { 
         \istate = (\iheap, \idom, \store) 
         \quad 
         \iheap' = \iheap[(\loc, p) \mapsto \val]
   }{  \hpupdt{\instrumented}(\istate, (\loc, p), \val) =  (\iheap, \idom, \store) }
\\
  %
      \inferrule[\textsc{Assume}]
   {}{  \absassume{\instrumented}(\istate, \jtrue) =  \istate }
  \and
    \inferrule[\textsc{Satisfiable}]
   {}{  \abssat{\instrumented}(\istate, \jbool) = \jbool}
     \and
      \inferrule[\textsc{Symbolic Value Creation }]
   {\val \in \vals \text{ is of type } \jtype}{\absmakesymbolicrule{\concrete}{\jtype} = \val}
 \end{mathpar}}}
 \end{display}

\newpage

\begin{display}{Symbolic State Instrumented Interpretation}
{\scriptsize 
\begin{mathpar}
\inferrule[Symbolic Heaps]{}{
\interpret{\symbinst}{\senv}(\hemp) \semeq \hemp
\qquad
\frac{
  \begin{array}{c}
   l = \interpret{\symbconc}{\senv}(\sexprl) \quad 
    p =  \interpret{\symbconc}{\senv}(\sexprp) \\
    v =  \interpret{\symbconc}{\senv}(\sexprv) \quad 
    \heap = \hcell{l}{p}{v}
  \end{array}
}{
\interpret{\symbinst}{\senv}(\hcell{\sexprl}{\sexprp}{\sexprv}) \semeq  \heap
}
\qquad
\frac{
  \begin{array}{c}
   l = \interpret{\symbconc}{\senv}(\sexprl) \quad 
   p = \interpret{\symbconc}{\senv}(\sexprp)  \\ 
      \heap = \hcell{l}{p}{\none}
    \end{array}
}{
  \interpret{\symbinst}{\senv}(\hcell{\sexprl}{\sexprp}{\none}) \semeq \heap
}
\qquad
\frac{
  \begin{array}{c}
   \interpret{\symbconc}{\senv}(\sheap_i) = \iheap_i, \hdom_i \mid_{i =1,2} \\ 
   \iheap = \iheap_1 \dunion \iheap_2 
   \end{array}
}{
\interpret{\symbinst}{\senv}(\sheap_1 \dunion \sheap_2) \semeq \iheap
}}
\\
\inferrule[Symbolic Domains]
{
   \idom(\loc) = \val  \iff  \exists \sexprl \in \domain(\sdom) \, . \,  \loc = \semexpr{\sexprl}{\senv} \, \wedge \, \val = \semexpr{\sdom(\sexprl)}{\senv}
}{
\interpret{\symbinst}{\senv}(\sdom) \semeq \idom
}
\and
\inferrule[Symbolic States]{
    \sstate = (\sheap, \sdom, \sstore, \pc) 
    \quad 
    \interpret{\symbinst}{}(\senv, \sheap) = \iheap
   \quad
    \interpret{\symbinst}{}(\senv, \sdom) = \idom
    \\\\
    \semexpr{\scs}{\senv} = \cs  
    \quad
    \semexpr{\sstore}{\senv} = \store 
    \quad
    \semexpr{\pc}{\senv} = \jtrue 
}{
\jmodels{\symbinst}(\senv, \sstate, \scs) \semeq ((h, \idom, \store), \cs)
}
\end{mathpar}}
\end{display}

\begin{lemma}[Bounded-Soundness - Instrumented Semantics]\label{lemma:soundness:jsil:symb:exe:instrumented:instrumented}
$$
\begin{array}{l}
\transabssemrule{\sstate, \scs, i}{\sstate', \scs', j}{\mode}{\mode'}{\symbolic}  \ \wedge \ (\istate, \cs) = \jmodels{\symbinst}(\sstate[\pcsel \mapsto \sstate'.\pcsel], \scs) \\ \quad \quad 
    \ \implies \ \exists \istate', \cs' \, . \, 
        \transabssemrule{\istate, \cs, i}{\jstate', \cs', j}{\mode}{\mode'}{\instrumented} \ \wedge \ (\istate', \cs') = \jmodels{\symbinst}(\sstate', \scs')
\end{array}
$$
\end{lemma}


\begin{display}{Instrumented State Interpretation}
{\scriptsize 
\begin{mathpar}
\inferrule[Symbolic Heaps]{}{
\interpret{\instrumented}{}(\hemp) \semeq \hemp, \emptyset
\qquad
\interpret{\instrumented}{}(\hcell{\loc}{p}{\val}) \semeq  \hcell{\loc}{p}{\val}, \emptyset
\qquad
  \interpret{\instrumented}{}(\hcell{\loc}{p}{\none}) \semeq \hemp, \jsilset{ (l, p) }
\qquad
\frac{
   \interpret{\instrumented}{}(\iheap_i) = h_i, \hdom_i, i =1,2
}{
\interpret{\instrumented}{}(\iheap_1 \dunion \iheap_2) \semeq  h_1 \dunion h_2, \hdom_1 \dunion \hdom_2
}}
\\
\inferrule[Instrumented Domains]
{}{
\interpret{\instrumented}{}(\idom) 
    \semeq \{ (l, p) \mid l \in \domain(\idom) \, \wedge \, p \not\in \idom(l) \}
}
\and
\inferrule[Symbolic States]{
    \istate = (\iheap, \idom, \store) 
    \quad 
    \interpret{\instrumented}{}(\iheap) = h, \hdom_1
   \quad
    \interpret{\instrumented}{}(\idom) = \hdom_2
    \quad 
    \hdom = \domain(h) \cup \hdom_1 \cup \hdom_2 
}{
\interpret{\instrumented}{}(\istate) \semeq \{ ((h, \store), h_F) \mid \domain(h_F) \cap \hdom = \emptyset \} 
}
\end{mathpar}}
\end{display}

\begin{lemma}[Frame Property - Instrumented Semantics]\label{lemma:frame:property}
$$
\begin{array}{l}
\transabssemrule{\istate, \cs, i}{\istate', \cs', j}{\mode}{\mode'}{\instrumented}  \ \wedge \
      ((h, \store), h_F)  \in \interpret{\instrumented}{}(\istate) \\ \quad \quad 
    \ \implies \  \exists h' \, . \, 
        \transabssemrule{(h \dunion h_F, \store), \cs, i}{((h' \dunion h_F), \store'), \cs', j}{\mode}{\mode'}{\concrete} \ 
               \wedge \ ((h', \store'), h_F) \in \interpret{\instrumented}{}(\istate')
\end{array}
$$
\end{lemma}

\begin{lemma}[Modular Symbolic Interpretation]\label{lemma:symb:interpretation}
$$
\begin{array}{l}
(\jstate, \cs) \in \jmodels{\symbolic}(\senv, \sstate, \scs) 
   \iff 
   \exists \iheap, h, h_F, \store \, . \,   \\ \quad \quad 
       ((\iheap, \store), \cs)  = \interpret{\symbinst}{}(\senv, \sstate, \scs)
       \ \wedge \
       ((h, \store), h_F) \in \interpret{\instrumented}{}((\iheap, \store))
       \ \wedge \ 
       \jstate = (h \dunion h_F, \store)
\end{array}
$$
\end{lemma}





\subsection{Implementation}\label{subsec:jsil:analysis:implementation}


Implementing a symbolic execution engine for \jsil is a non-trivial 
task, requiring a substantial engineering effort. 
% 
% 
Hence, instead of implementing the symbolic semantics of \jsil from scratch, we leverage on 
\rosette~\cite{Rosette1,Rosette2}, a symbolic virtual machine designed to 
enable swift development of new 
solver-aided languages. 
%
\rosette is a small extension of Racket~\cite{racket} equipped with a symbolic compiler with support 
for symbolic values and first order assertions. Because \rosette is itself solver-aided, languages 
implemented in \rosette can also make use of the solver-aided facilities provided by \rosette. 
Hence, by implementing a \emph{concrete} \jsil interpreter in \rosette, we obtain \emph{for free} a symbolic 
interpreter for \jsil. %consistent with the symbolic semantics described in  \S\ref{subsec:jsil:analysis:formalism}. 
%The idea of turning a concrete interpreter into a symbolic interpreter by embedding it in 
%
The implementation of the concrete interpreter in \rosette must fulfil the following criteria:

\begin{itemize}          
   \item \emph{Efficiency:} the implementation must promote \rosette's efficient behaviour;
   
   \item \emph{Termination:} the user must be given a way to establish a bound for the symbolic execution 
            of programs that loop on symbolic values; 
  
   \item \emph{Adequacy:} the symbolic execution of the concrete interpreter in \rosette 
            must be consistent with the symbolic semantics described in \S\ref{subsec:jsil:analysis:formalism}. 
\end{itemize}

\subsubsection{Efficiency}
It is often possible for more than one transition of the symbolic 
semantics to be applicable during symbolic execution, 
giving rise to a potentially intractable number of possible symbolic states. 
To counter this problem, \rosette uses a sophisticated 
symbolic state merging algorithm, which factors out the common 
part between multiple symbolic states  in order to expose more 
opportunities for concrete evaluation. The non-mergeable portions of the state 
are represented as \emph{guarded symbolic unions}. 
Our goal is to write the interpreter code in a way that helps 
\rosette merge sets of possible symbolic states, yielding minimal 
guarded symbolic unions.

\subsubsection{Termination} The \jsil symbolic execution engine does not 
include the abstraction mechanisms which would allow it to finitise the symbolic 
execution of loops depending on symbolic values~\cite{abstract:symbolic:exec}. 
Hence, the user is asked to specify an upper bound on the number of times
the symbolic execution is allowed to branch on symbolic values by using
conditional gotos. 
Once that upper bound is reached, if a conditional goto that branches 
on a symbolic value is encountered, the symbolic execution stops.  






\section{Debugging Separation Logic Specifications}\label{sec:specs}
%%!TEX root = ../main.tex

We show how to use \cosette for debugging \jsil code annotated with 
separation logic (SL) specifications. Tools that allow for SL-reasoning about
functional correctness properties in general, and those targeting 
JavaScript in particular, require the user to have substantial expertise 
and to go through a long and complex proof trace, whenever verification
is not successful. \cosette substantially simplifies this process by providing
concrete counter-models that invalidate the input specifications.

In \S\ref{subsec:sep:assertions}, we extend the 
 \jsil symbolic interpreter with a mechanism for asserting
SL-assertions. 
%
In \S\ref{subsec:fip}, we show how to implement this mechanism by giving 
a sound decision procedure for solving the frame inference problem (FIP)~\cite{}
in the context of symbolic execution.
%
Unlike verification tools, our emphasis is in the generation of counter-models 
for failing cases. 
%
Finally, in \S\ref{specs:to:symbolic:tests}, we present an algorithm  
for generating symbolic tests from SL-specifications, which guarantees 
that whenever a symbolic test fails, \cosette produces a concrete 
counter-model that invalidates the corresponding specification.

\vspace{-5pt}
\subsection{Symbolic Execution with SL-Assertions}\label{subsec:sep:assertions}

\jsil Logic assertions~\cite{javert}
provide a compositional way of describing \emph{partial} symbolic states. 
\jsil assertions include: boolean operations; the separating conjunction; 
and assertions for describing heaps. The $\lemp$ assertion describes 
an empty heap. The cell assertion, $(\lexpr_1,\lexpr_2) \pointsto \lexpr_3$,  describes an object 
at the location denoted by $\lexpr_1$ with a property denoted by $\lexpr_2$ that has the value 
denoted by $\lexpr_3$. The object domain assertion $\emptyfields{\lexpr_1}{\lexpr_2}$ states that the object at 
the location denoted by $\lexpr_1$ has no properties other than possibly those included in the
set denoted by $\lexpr_2$. The syntax of assertions is given below. 
We refer to assertions different from $- \sep -$ and $\lemp$ as \emph{simple assertions}
and use $\spass$ and $\sqass$ to range over them.

\vspace{2pt}
\begin{display}{\jsil Logic Assertions}
%
{\small
\begin{tabular}{l}
  %%%%
  $\lexpr \quad \ \triangleq \val \mid \jvar \mid \svar \mid \unoper\ \lexpr \mid \lexpr \binoper \lexpr \quad \quad \quad \quad \quad \quad \quad \ \ $   \text{ Logical Expressions} \\
  $\rass, \sass \ \triangleq \jtrue \mid \jfalse \mid  \neg \rass \mid \rass \land \sass \mid \rass \lor \sass  \mid \lexpr = \lexpr \mid \lexpr \leq \lexpr$  \quad \text{\hfill{Pure Asrts.}} \\
  $\pass, \qass \triangleq \sass \mid \lemp \mid (\lexpr, \lexpr)\pointsto \lexpr \mid \pass \sep \qass  \mid \emptyfields{\lexpr}{\lexpr} $ \quad \quad \quad \quad \ \  \text{\hfill Asrts.} \\
\end{tabular}}
\end{display}

\noindent Without loss of generality, we implicitly assume that different symbolic locations 
denote different concrete locations.\footnote{In order to express aliasing, the user has to write multiple assertions.}
 Furthermore, given a cell assertion $(\lexpr_1,\lexpr_2) \pointsto \lexpr_3$, we always assume 
 $\lexpr_1$ to be either a concrete location $\loc$ or a symbolic location $\sloc$. 
%
Unsurprisingly, every symbolic state can be thought of as an assertion. In particular, 
the symbolic state $\sstate = (\sheap, \sdom, \sstore, \pc)$ corresponds to the assertion: 
\begin{equation*}
{\small \begin{array}{l}
\big(\varoast_{(\sloc, \sexprp) \in \domain(\sheap)} (\sloc, \sexprp) \mapsto \sheap(\sloc, \sexprp)\big) 
  \sep \big(\varoast_{\sloc \in \domain(\sdom)} \, \emptyfields{\sloc}{\idom(\sloc)}\big)  \\
 %
 \qquad \sep \big(\bigwedge_{\jvar \in \domain(\sstore)} \, \jvar = \sstore(\jvar)\big) \sep \pc
\end{array}}
\end{equation*}

\noindent where $\varoast$ denotes the iterated separating conjunction~\cite{}. 
Analogously, every assertion can be trivially re-organised into a symbolic state: 
\dtag{1} cell assertions form the heap part of the state, 
\dtag{2} object domain assertions form the domain part,
\dtag{3} equalities involving program variables form the store, and 
\dtag{4} pure assertions form the path condition. 
Finally,  all the occurrences of program variables in the heap, domain, and path condition 
are replaced with their corresponding symbolic expressions given by the store. 
We use $\normaliser(\pass)$ to refer to the symbolic state corresponding 
to $\pass$. 
%
In the following, we use $\interpret{}{}(\sstate)$ for denoting the set 
$\{ (\jstate, \heap_f) \mid \exists \senv \, . \, (\jstate, \heap_f) \in \interpret{\symbconc}{\senv}(\sstate) \}$ 
and $\interpret{}{}(\pass)$ for denoting $\interpret{}{}(\normaliser(\pass))$. 

\myparagraph{Inductive Predicates}
\cosette does not support symbolic execution over inductive predicates, which are commonplace 
in SL-style specifications~\cite{smallf, berdine:aplas:2005}. 
As in~\cite{korat}, we deal with user-defined inductive predicates by \emph{unfolding} 
those predicates up to a fixed bound, given by the user. This unfolding mechanism 
is routine and is, therefore, omitted from the paper. 

\myparagraph{Asserting SL-Assertions}
We extend \jsil with a special construct, $\sepassert(P)$, for stating that 
the assertion $P$ must hold whenever that command is evaluated. 
The symbolic semantics is given below. 
\begin{mathpar}
\inferrule[\textsc{SL-Assert - True}]
  { 
     \ccmd{i}  = \sepassert(\pass)  \\\\
     %
     \unificationfun(\sstate, \pass) = \success{\subst, \sstate_f}
  }{
    \abssemrule{\sstate, \scs, i}{\sstate, \scs, i{+}1}{\top}{\top}{\symbolic}
}
\qquad
\inferrule[\textsc{SL-Assert - False}]
  { 
     \ccmd{i}  = \sepassert(\pass)  \\\\
     %
    \unificationfun(\sstate, \pass) = \fail{\pc_f}  \\\\ 
    %
     (\sstate.\pcsel \wedge \pc_f) \text{ SAT}
  }{
    \abssemrule{\sstate, \scs, i}{\sstate, \scs, i}{\top}{\bot}{\symbolic}
}
\end{mathpar}

\vspace{-3pt}
\noindent The rules make use of a partial decision procedure $\unificationfun(\sstate, \pass)$ for 
determining whether or not a given symbolic state $\sstate$ satisfies an assertion $P$, 
which is in general undecidable \cite{citemeplease}. More concretely, 
the decision procedure outputs: 
\dtag{1} $\success{\subst, \sstate_f}$ when it finds a substitution $\subst$ and 
a symbolic state frame $\sstate_f$ for which it holds that $\interpret{}{}(\sstate) \subseteq \interpret{}{}(\sstate_f \statecompose \normaliser(\subst(\pass)))$, 
where we use $\statecompose$ for the composition of two symbolic states, 
and \dtag{2} $\fail{\pc_f}$ when it finds a first order formula $\pc_f$ such that 
$\interpret{}{}(\sstate \, \wedge \, \pc_f) \cap \interpret{}{}(\pass) = \emptyset$. 
Note that every concrete state and heap frame in $\interpret{}{}(\sstate \, \wedge \, \pc_f)$ are counter models 
for $P$. By requiring that $(\sstate.\pcsel \, \wedge \, \pc_f)$ be satisfiable, 
the semantics only triggers an assertion failure when it finds a concrete witness for the failure ---
any instantiation of $(\sstate \, \wedge \, \pc_f)$. 



\subsection{Frame Inference Problem}\label{subsec:fip}

We describe a partial decision procedure, which we  
implement as part of the \jsil symbolic interpreter, for proving entailments 
between symbolic states \underline{and} finding counter 
models in case of failure.  
As it is customary~\cite{javert,jacobs2011verifast,sepwithsmt}, the decision procedure works by first using \emph{pattern-matching} 
on the spatial part of the symbolic state, and then discharging the pure part of the 
entailment to an external constraint solver (in our case, \rosette). 

\begin{table} 
{\small \begin{tabular}{@{}c@{}ccc@{}c@{}}\toprule
\emph{Argument} & & \textbf{IN}  & & \textbf{OUT}  \\
\cmidrule{1-1} \cmidrule{3-3} \cmidrule{5-5}

$\svar$                                                       & & $\{ \svar \}$                                                          & & $\{ \svar \}$    \\
$\jsillist{\lexpr_1, ..., \lexpr_n}$                     & & $\upin(\lexpr_1) \cup ... \cup \upin(\lexpr_n)$      & & $\upout(\lexpr_1) \cup ... \cup \upout(\lexpr_n)$ \\
$\lexpr_1 + \lexpr_2$                                    & & $\upin(\lexpr_1) \cup \upin(\lexpr_2)$                  & & $\emptyset$ \\
$\cdots$ & & $\cdots$ & & $\cdots$ \\[1pt]
%%
$(\lexpr_1, \lexpr_2)\pointsto \lexpr_3$   & & $\upin(\lexpr_1) \cup \upin(\lexpr_2)$ & & $\upout(\lexpr_3)$  \\
%
$\emptyfields{\lexpr_1}{\lexpr_2}$           &  & $\upin(\lexpr_1) \cup \upin(\lexpr_2)$ & & $\emptyset$ \\
%
$\lexpr_1 = \lexpr_2$                               & & $\upin(\lexpr_1)$/$\upin(\lexpr_2)$      & & $\upout(\lexpr_2)$/$\upout(\lexpr_1)$ \\
%
$\sass$                                                    & & $\fv(\sass)$                                           & & $\emptyset$ \\
\bottomrule
\end{tabular}}
\caption{\emph{in} and \emph{out} sets for assertions and logical expressions}
\vspace{-25pt}
\end{table}

\myparagraph{Unification Plan}
When solving $\unificationfun(\sstate, \pass)$, the symbolic variables of $\pass$ that are not 
in $\sstate$ are assumed to be existentially quantified. 
In order to find the appropriate bindings for these variables, we 
introduce the notion of \emph{unification plan} (Definition~\ref{def:up}).
Informally, a unification plan is an ordering of the simple assertions in $\pass$ that 
guarantees that the unification algorithm does not have to backtrack at unification~time. 

We define for each assertion an \emph{in} set and an \emph{out} set (these 
sets are reminiscent of the predicate parameter modes in~\cite{nguyen:vmcai:2008}).
Intuitively, the variables in the \emph{out} set of an assertion are those that can be computed 
using the variables in the \emph{in} set, together with the current state. 
For instance, given the assertion $(\svar_1, \svar_2) \pointsto \svar_3$, if we know the bindings
for $\svar_1$ and $\svar_2$, we can compute the bindings for $\svar_3$, given 
the current symbolic state. 
%
Analogously, we define \emph{in} and \emph{out} sets for logical expressions. 
The variables in the \emph{out} set of a logical expression are those that can be computed
given the value of the whole expression, whereas the variables in the \emph{in} set 
are those that need to be known for us to compute the value of the whole expression.  
Finally, the definition of unification plan is given below. 

\begin{definition}[Unification Plan]\label{def:up}
A \emph{unification plan} $\up$ is a sequence of simple assertions $\sqass_i \mid_{i=0}^n$ such that: 
$$
 \forall 0 \leq i \leq n \, . \, \upin(\sqass_i) \subseteq \big(\cup_{j = 0}^{i-1} \upout(\sqass_j)\big) \cup \upin_0
$$
where $\upin_0$ is the set of \underline{non}-existentially quantified logical variables. 
\end{definition}

\vspace{-3pt}
\noindent It is not always possible to generate a unification plan for an SL-assertion. We only 
consider assertions that admit a unification plan. 


\myparagraph{Unification Algorithm}
Given a symbolic state $\sstate$ and an assertion $\pass$, $\unificationfun(\sstate, \pass)$: 
\dtag{1} replaces all the occurrences of program variables in $\pass$ with their bindings 
given by $\sstate.\stosel$;
\dtag{2} computes an initial substitution $\subst_0$ mapping all the non-existentially quantified symbolic 
variables in $\pass$ to themselves (put formally: $\subst_0 = \identity_{\fv(\sstate) \cap \fv(\pass)}$); 
\dtag{3} creates a unification plan $\up$ for $\pass$; and, 
\dtag{4} calls Algorithm~\ref{fip:symb:states}. 
%
Algorithm~\ref{fip:symb:states} makes use of the following auxiliary functions: 

\begin{description}
\setlength{\itemsep}{0.2em}
  \item[FIP GetCell.] In case of success, $\GetCellV{\sstate, \sloc, \sexprp}$ returns 
          a pair consisting of the symbolic expression $\sexprv$ associated with 
          $(\sloc, \sexprp)$ in the heap component of $\sstate$ \underline{and} 
          the state obtained by removing that cell from $\sstate.\hpsel$.
  
  \item[FIP GetDomain.] In case of success, $\GetDomainV{\sstate, \sloc}$ returns 
          a pair consisting of the symbolic expression $\sexprv_d$ denoting the domain 
          of the object at location $\sloc$ in $\sstate$ \underline{and} 
          the state obtained by removing all the negative resource associated with 
          $\sloc$ from $\sstate$.  
  
  \item[Expression Unification.]  In case of success, $\unifylexpr(\sexprv, \sexprv', \subst, \pc)$ 
          returns a substitution $\subst'$ that extends $\subst$ such that $\pc \vdash \sexprv = \subst'(\sexprv')$. 
\end{description}
In case of failure, all auxiliary functions return a constraint $\pc_f$ under which 
the unification is guaranteed not to be possible. 
Below, we give selected rules for $\GetCellV{\sstate, \sloc, \sexprp}$ and $\GetDomainV{\sstate, \sloc}$.  

{\small \begin{algorithm}[t!]
\algblock[Name]{match}{end}
\caption{Frame Inference for Symbolic States}\label{fip:symb:states}
\begin{algorithmic}[1]
\Function{Unification}{$\sstate$, $\up$, $\subst$}
    \State $\textbf{match}$ $\up$ $\textbf{with}$
    \State $\mid~\lstemp:$ \Return $\success{\subst, \sstate}$
   % Cell ASS
    \State $\mid~(\sloc, \sexprp) \pointsto \sexprv \lstcons \up' :$ 
    \State $\qquad \textbf{match} \ \GetCellV{\sstate, \subst(\sloc), \subst(\sexprp)}$ $\textbf{with}$
    \State $\qquad \mid~\success{\sexprv', \sstate'}:$
    \State $\qquad \qquad \textbf{match} \ \unifylexpr(\sexprv, \sexprv', \subst, \sstate.\pcsel)$ $\textbf{with}$
     \State $\qquad \qquad \mid~ \success{\subst'}:$ \Return \Call{Unification}{$\sstate'$, $\up'$, $\subst'$}
      \State $\qquad \qquad \mid~ \fail{\pc_f} \ \, :$ \Return $\fail{\pc_f}$
      \State $\qquad \mid~\fail{\pc_f}:$ \Return $\fail{\pc_f}$
      % EF ASS
     \State $\mid~\emptyfields{\sloc}{\sexprv} \lstcons \up' :$  
       \State $\qquad \textbf{match} \ \GetDomainV{\sstate, \subst(\sloc)}$ $\textbf{with}$
       \State $\qquad \mid~\success{\sexprv', \sstate'}:$
         \State $\qquad \qquad \textbf{match} \ \unifylexpr(\sexprv, \sexprv', \subst, \sstate.\pcsel)$ $\textbf{with}$
       \State $\qquad \qquad \mid~ \success{\subst'}:$ \Return \Call{Unification}{$\sstate'$, $\up'$, $\subst'$}
       \State $\qquad \qquad \mid~ \fail{\pc_f} \ \, :$ \Return $\fail{\pc_f}$
       \State $\qquad \mid~\fail{\pc_f}:$ \Return $\fail{\pc_f}$
     % Logical Equality  
     \State $\mid~(\sexprv = \sexprv') \lstcons \up' :$  
        \State $\qquad \textbf{match} \ \unifylexpr(\sexprv, \sexprv', \subst, \sstate.\pcsel)$ $\textbf{with}$
         \State $\qquad \qquad \mid~ \success{\subst'}:$ \Return \Call{Unification}{$\sstate$, $\up'$, $\subst'$}
       \State $\qquad \qquad \mid~ \fail{\pc_f} \ \, :$ \Return $\fail{\pc_f}$
     % OTHER PURE ASS
     \State $\mid~\sass \lstcons \up' :$   
      \State $\qquad \textbf{if} \,(\sstate.\pcsel \vdash \subst(\sass))$
       \State $\qquad \qquad \textbf{then}$ \Return  \Call{Unification}{$\sstate$, $\up'$, $\subst$}
      \State $\qquad \qquad \textbf{else}$  \Return $\fail{\subst(\sass)}$
\EndFunction
\end{algorithmic}
\end{algorithm}}


\vspace{2pt}
\begin{display}{Selected FIP Rules}
\text{
{\scriptsize
\begin{mathpar} 
  \inferrule[\textsc{GetDomain}]
   { 
       \sheap = \sheap' \, \uplus \, \big((\sloc, \sexprp_i) \mapsto \sexprv_i \big)\mid_{i = 0}^m  
         \quad 
          (\sloc,-) \notin \domain (\sheap')  
         \quad
        \forall_{0 \leq i \leq n} \,  \sexprv_i \neq \none 
         \quad
           \forall_{n < i \leq m} \, \sexprv_i = \none
           \\\\ 
           \sexprv =  \{ \sexprp_i \mid_{i = n{+}1}^m\}
           \and
           \sheap'' = (\sloc, \sexprp_i) \mapsto \sexprv_i  \mid_{i=0}^n
           \and
           \sdom = \sdom' \dunion (\sloc \mapsto \sexprv')
   }{  \GetDomainV{(\sheap, \sdom, \sstore, \pc), \sloc} \semeq \success{(\sheap' \, \uplus \,  \sheap'', \sdom', \sstore, \pc), \sexprv' \backslash \sexprv}}
  \\
      \inferrule[\textsc{GetCell - Found}]
   { 
      (\sheap, \sdom, \sstore, \pc) = \sstate    
       \quad
       \sheap = \sheap' \, \uplus \, (\sloc, \sexprp') \mapsto \sexprv 
       \\\\
       {\color{blue} \pc \vdash (\sexprp = \sexprp')}
       \quad
       \sstate' = (\sheap', \sdom, \sstore, \pc)
   }{  \GetCellV{\sstate, \sloc, \sexprp} \semeq \success{\sstate', \sexprv}}
\quad
     \inferrule[\textsc{GetCell - Not Found}]
   { 
     (\sheap, \sdom, \sstore, \pc) = \sstate  
     \quad 
        { \color{blue} \pc \vdash \sexprp \not\in \sdom(\sloc)}
        \\\\
        \sstate' = (\sheap, \sdom[\sloc \mapsto \sdom(\sloc) \cup \jsilset{\sexprp}], \sstore,  \pc)
   }{  \GetCellV{\sstate, \sloc, \sexprp} \semeq \success{\sstate', \none}}
\\
     \inferrule[\textsc{GetCell - Fail with Domain Info}]
   { 
       \sheap = \sheap'' \, \uplus \, \big((\sloc, \sexprp_i) \mapsto \sexprv_i \big)\mid_{i = 0}^m
      \qquad
       (\sloc,-) \notin \domain (\sheap'')
       \qquad
        { \color{blue} \pc \not\vdash \sexprp \not\in \sdom(\sloc)}
        \qquad
          { \color{blue} \pc \not\vdash \sexprp = \sexprp_i \mid_{i=0}^m}
   }{  \GetCellV{(\sheap, \sdom, \sstore, \pc), \sloc, \sexprp} \semeq \fail{{\color{red} (\sexprp \in \sdom(\sloc)) \, \wedge \, \wedge_{i=0}^m (\sexprp_i \neq \sexprp)}}}
 \\
 \\
     \inferrule[\textsc{GetCell - Fail without Domain Info}]
   { 
       \sheap = \sheap'' \, \uplus \, \big((\sloc, \sexprp_i) \mapsto \sexprv_i \big)\mid_{i = 0}^m
      \qquad
       (\sloc,-) \notin \domain (\sheap'')
       \qquad
       \sloc \not\in \domain(\sdom)
        \qquad
          { \color{blue} \pc \not\vdash \sexprp = \sexprp_i \mid_{i=0}^m}
   }{  \GetCellV{(\sheap, \sdom, \sstore, \pc), \sloc, \sexprp} \semeq \fail{{\color{red} (\wedge_{i=0}^m (\sexprp_i \neq \sexprp)}}}
 \end{mathpar}}}
 \end{display}

The rules are analogous to the rules in \S\ref{subsec:symb:semantics} except that they return a new symbolic state 
from which the matched resource is removed and their corresponding constraints are lifted to the premise. 

\myparagraph{Formal Guarantees}
Theorem~\ref{teo:fip:soundness} states that the unification algorithm is sound: 
given an SL-assertion $\pass$ and a symbolic state $\sstate$, if
$\unificationfun(\sstate, \pass) = \success{\subst, \sstate_f}$, then there is a symbolic state
$\sstate'$ such that $\sstate = \sstate' \statecompose \sstate_f$ and $\sstate'$ satisfies $\pass$. 
The bug-finding theorem, Theorem~\ref{teo:fip:bugfinding}, is more subtle. It states that, 
in case of failure, to find a counter-model for $\pass$, one has to pick a concretisation of the 
symbolic state that is consistent with the failing constraint generated by the unification algorithm.


\begin{theorem}[Soundness of FIP]\label{teo:fip:soundness}
$$
\begin{array}{l}
	\unificationfun(\sstate, \pass) = \success{\subst, \sstate_f}
        \implies 
        \interpret{}{}(\sstate) \subseteq \interpret{}{}(\sstate_f \statecompose \normaliser(\subst(\pass)))
\end{array}
$$ 
\end{theorem}

\begin{theorem}[Bug-finding for SL]\label{teo:fip:bugfinding}
$$
\begin{array}{l}
\unificationfun(\sstate, \pass) = \fail{\pc_f} 
   \implies
   \interpret{}{}(\sstate \, \wedge \, \pc_f) \cap \interpret{}{}(\pass) = \emptyset
\end{array}
$$ 
\end{theorem}



\subsection{From Specifications to Symbolic Tests}\label{specs:to:symbolic:tests}

\jsil Logic specifications have the form $\specsig{\pass}{\pid(\jvec{x})}{\qass}{\flag}$, where $\pass$ and $\qass$ are the 
pre- and postconditions of the procedure with identifier $\pid$ and formal parameters $\jvec{x}$. 
Each specification is associated with a return mode $\flag \in \{ \fnormal, \ferror \}$, indicating if the function
 returns normally or with an error. 
 %If it returns normally, then its return value can be accessed  via a dedicated variable 
% $\retvar$, and $\errvar$ otherwise. 
 Intuitively, a specification $\specsig{\pass}{\pid(\jvec{x})}{\qass}{\flag}$ is 
valid for a given \jsil program $\prog$, if $\prog$ contains a procedure with identifier 
$\pid$ and ``whenever $\pid$ is executed in a state satisfying $P$, then, 
if it terminates, it does so in a state satisfying $Q$, with return mode $\flag$''.
The formal definition is given below. 


\begin{definition}[Validity of \jsil Logic Specifications]
A \jsil logic specification $\specsig{\pass}{\pid(\jvec{x})}{\qass}{\flag}$ is valid with respect to a program 
$\prog$, written $\prog \satisfies \specsig{\pass}{\pid(\jvec{x})}{\qass}{\flag}$, if and only if, for all logical 
contexts $(\iheap, \store, \senv)$, heaps $\heap_f$, stores $\store_f$, and flags $\flag'$, it holds that: 
$$
\begin{array}{l}
   (\jstate, \heap_f, \cs) \in \interpret{}{}(P) 
   \ \wedge \ 
    \abssemrule{\jstate \dunion \heap_f, \cs, 0}{\jstate', \cs', i_{\flag'}}{\top}{\top}{\concrete} \\ \quad \
   \implies
      \flag' = \flag \ \wedge \ \exists \jstate'' \, . \, \jstate' = \jstate'' \dunion \heap_f
          \ \wedge \   (\jstate'', \heap_f, \cs') \in \interpret{}{}(Q) 
\end{array}
$$
\end{definition}

\begin{figure}
{\small
$$
\begin{array}{lll}
\testify{}(\specsig{P}{\pid(\jvar_1, ..., \jvar_n)}{Q}{\flag}) \ \semeq                           &  \testify{\fnormal}(\pid, \svar_i\mid_{i=0}^n, Q) \ \semeq \\
%
\ \  \mathbf{let} \ \sstore =  [ \jvar_i \mapsto \svar_i \mid_{i=0}^n] \ \mathbf{in}        &  \ \  \darkmath{\sf proc} \jsilmain () \{    \\
%
 \ \  \mathbf{let} \ \sstate = \normaliser(\sstore(P)) \ \mathbf{in}                               &   \ \ \ \ 0_{\phantom{\sf nm}}: \jsilcall{\jvar}{\pid}{\svar_0, ..., \svar_n}{\errlab} \\
 %
\ \  \mathbf{let} \ Q' = \sstore(Q) \ \mathbf{in}                                                           &    \ \ \ \ \retlab \, : \sepassert(Q[\jvar/\retvar])  \\
 %
 \ \  \mathbf{let} \ \proc = \testify{\flag}(\pid, \svar_i\mid_{i=0}^n, Q')  \ \mathbf{in}  &    \ \ \ \ \errlab \, \, \, : \jassert(\jfalse)   \\
 %
 \ \ \ \ (\proc, \sstate)                                                                                                 &    \ \ \}  
\end{array}
$$}
\vspace*{-0.2cm}
\caption{Symbolic Test Generation Algorithm~\label{fig:test:generation}}
\vspace*{-0.2cm}
\end{figure}

Given a \jsil program $\prog$ containing a procedure $\pid$ with spec {\small $\specsig{\pass}{\pid(\jvec{x})}{\qass}{\flag}$}, 
our goal is to construct a symbolic test for checking whether or not $\pid$ behaves as its specification mandates.
A symbolic test is a pair $(\proc, \sheap)$ consisting of a \jsil procedure with the code of the test and the initial 
symbolic heap on which to execute the test. 
%
Figure~\ref{fig:test:generation} presents the test generation procedure. Intuitively, $\testify{}(\specsig{\pass}{\pid(\jvec{x})}{\qass}{\flag})$ 
returns the symbolic test for $\specsig{\pass}{\pid(\jvec{x})}{\qass}{\flag}$. The test generation function $\testifyfun{} \ $ is defined in terms 
of two auxiliary functions, $\testifyfun{\fnormal}$ and $\testifyfun{\ferror}$, for generating tests for $\fnormal$-mode and 
$\ferror$-mode specifications, respectively. 
For space reasons, we only present $\testifyfun{\fnormal}$, $\testifyfun{\ferror}$ is equivalent. 
The test program $\prog'$, denoted by $\prog[\jsilmain \mapsto \proc]$, is obtained from the original program $\prog$ and the test procedure $\proc$ by replacing the 
$\jsilmain$ of $\prog$ with the new test procedure, $\proc$. 

Finally, Theorem~\ref{teo:bug:finding:sl} states that if the symbolic execution of the 
test generated for $\specsig{\pass}{\pid(\jvec{x})}{\qass}{\flag}$ finds a bug, then the specification 
is not~valid.

\begin{theorem}[Bug-finding for SL Specifications]\label{teo:bug:finding:sl}
$$
\begin{array}{l}
\testify{}(\specsig{\pass}{\pid(\jvec{x})}{\qass}{\flag})  = (\proc, \sstate) \, \wedge \, 
  \prog' :  \transabssemrule{\sstate, \csmain, 0}{-}{\top}{\bot}{\symbolic} \\ \quad \quad 
    \implies  
         \prog \not\satisfies \specsig{\pass}{\pid(\jvec{x})}{\qass}{\flag}
\end{array}
$$
\noindent where:  $\csmain = [ (\jsilmain, -, -, -) ]$ 
and $\prog' = \prog[\jsilmain \mapsto \proc]$.
\end{theorem}





\newpage
\section{Evaluation}\label{sec:evaluation}
%%!TEX root = ../main.tex

\myparagraph{Methodology}
Paragraph about the test262 tests

With \cosette, we can write symbolic tests, in which some of the concrete values of the program are replaced with symbolic values.
Symbolic tests improve on concrete tests for two main reasons.
First, they are by construction more comprehensive than concrete tests, because symbolic tests can account for the whole range of values that a variable can take, instead of focusing on a few specific examples.
Second, when \cosette finds a failing assertion inside a symbolic test, it can concretize the symbolic values into a counter-model that the developer can actually run in node, making debugging much easier compared to (the other things that we mention before).

Paragraph about spec-driven debugging

\myparagraph{Symbolic testing}
We used \cosette to analyse the code of two JavaScript data structure libraries: Buckets.js~\cite{buckets}, and queue-pri~\cite{priq}.
We chose these libraries because reasoning about data structure code requires a precise description of the control flow features of JavaScript, because they come equipped with unit test suites, and because they do not have external dependencies (\cosette is a whole-program analysiso); Buckets.js has over 65k downloads on npm.
For these two libraries, we wrote symbolic tests with the aim of obtaining 100\% line coverage, in order to compare them with the concrete unit tests that ship with the libraries.
In both cases, we were able to reduce the length of the tests by up to an average factor of 3, while retaining full coverage; we also discovered one bug in the Buckets.js library, and one in the queue-pri library.


\myparagraph{Results}
The results are presented in table~\ref{cosette:res}.
For each file in the library, we report the number of JS executable lines in the code itself / including dependencies, the corresponding numbers of JSIL lines, the number of symbolic and concrete test cases, the number of JS lines in the symbolic and concrete tests, the coverage measured as percentage of lines and the average \cosette run time for the symbolic tests.
Most of the execution time is spent in \cosette itself, with negligible time spent in solver calls (except for a limited number of cases).


\begin{table}[h]
{
\small
%\begin{center}
\setlength\tabcolsep{4pt}
\begin{tabular*}{\linewidth}{lrrrrrr}
\toprule
% Name || JS Loc/loc* || JSIL Loc/loc* || #tests || symb/conc loc || symb/conc cov || time
Name & \makecell{JS loc} & \makecell{JSIL loc} & \# Tests & \makecell{Test loc} & \makecell{Line\\cov. (\%)} & \makecell{Avg.\\time} \\
\midrule
\texttt{arrays} & 44/71 & 1251/1942 & 9/24 & 166/329 & 100/\_ & 20.2s \\
\texttt{bag} & 69/237 & 2041/7194 & 7/18 & 78/265 & 100/\_ & 1m14s \\
\texttt{bstree} & 143/326 & 3819/8052 & 11/31 & 216/759 & 100/\_ & 5m27s \\
\texttt{dict} & 57/84 & 1683/2374 & 7/14 & 116/170 & 100/\_ & 15.6s \\
\texttt{heap} & 57/128 & 2059/4001 & 4/15 & 92/626 & 100/\_ & 5m29s \\
\texttt{llist} & 126/153 & 2447/3138 & 9/21 & 149/370 & 100/\_ & 24.0s \\
\texttt{multidict} & 56/184 & 1871/5496 & 6/16 & 118/189 & 100/\_ & 1m15s \\
\texttt{pqueue} & 26/154 & 1066/5067 & 5/12 & 70/283 & 100/\_ & 5m49s \\
\texttt{queue} & 30/183 & 1095/4233 & 6/9 & 111/146 & 100/\_ & 20.7s \\
\texttt{set} & 40/124 & 1528/3902 & 6/12 & 86/271 & 100/\_ & 1m02s \\
\texttt{stack} & 23/176 & 941/4079 & 4/7 & 91/104 & 100/\_ & 26.4s \\
\midrule 
\texttt{queue-pri} & 0/0 & 0/0 & 0/0 & 0/0 & 100/0 & 42.0 \\
\bottomrule
%\end{center}
\end{tabular*}
}
\caption{Tests for the Buckets.js and queue-pri libraries}
\label{cosette:res}
\end{table}


\myparagraph{Bug-finding}
With \cosette, we found a bug in the implementation of the Buckets.js \jsinline{multidictionary} (a key-value map in which a single key might hold several values), as well as a bug in \jsinline{queue-pri}, a JS priority queue implementation~(\cite{priq}). 

%\myparagraph{\jsinline{multidictionary}}
%The implementation of the Buckets.js \jsinline{multidictionary} essentially comprises the \jsinline{get(key)} (which returns an array of values), \jsinline{set(key, value)}, and \jsinline{remove(key, value)} methods.
The implementation of the Buckets.js \jsinline{multidictionary} essentially comprises the \jsinline{get(key)}, \jsinline{set(key, value)}, and \jsinline{remove(key, value)} methods.
In the \jsinline{remove} function, the \jsinline{value} argument can either be \jsinline{undefined}, in which case the key is completely removed from the dictionary, or it can be an actual value.
In that case, the array associated with the key \jsinline{s} is retrieved, and, if the value is present in the array, it is removed.
However, the library does not consider the case where this array is not found; in that case, the library tries to remove an element from \jsinline{undefined}, which raises an error.
\begin{lstjs}
multiDict.remove = function (key, value) {
    if (value === undefined) { ... }
    var array = parent.get(key);
    if (buckets.arrays.remove(array, value, equalsF)) { ... }
    return false;
}
\end{lstjs}
We were able to expose this behavior with the following symbolic test:

\begin{lstjs}
var dict = new buckets.MultiDictionary()
var s = symb_string(s);
var x1 = symb_number(x1), x2 = symb_number(x2);
dict.set(s, x1); dict.set(s, x2);
dict.remove(s, x1);
var res = dict.remove(s, x2);
assert((x1 != x2 && res) || (x1 == x2 && !res));
\end{lstjs}

In this test, we create a new dictionary \jsinline{dict}, and insert two symbolic values \jsinline{x1} and \jsinline{x2} at key \jsinline{s}, then remove them in order.
However, if \jsinline{x1} is actually equal to \jsinline{x2}, the implementation does not store it twice.
This means that the key in the second call to \jsinline{remove} on line 6 is not in the dictionary any more, which triggers the bug.

When running \cosette on this test, we obtain the counter-model \jsinline{x1 = x2 = 0}, and running the instantiation of the test in Node triggers an error.
Moreover, we were able to easily fix the bug by adding a check for \jsinline{undefined} after line 3 in the code of \jsinline{remove}, after which \cosette was successfully able to discharge the assertion.

%\myparagraph{\jsinline{queue-pri}}
The other bug that we found is in the \jsinline{queue-pri} library, which implements a priority queue that stores data with an optional priority value.
The priority value is either an integer (the lower the value, the higher the priority) or the default \jsinline{null} value if no priority is provided, which puts the associated element at the end of the queue.
With symbolic tests of the \jsinline{enqueue} and \jsinline{dequeue} methods of the library, we found that elements enqueued with a priority value of \jsinline{0} were actually being enqueued at the end of the queue, when they should have had the highest priority instead.

We identified the bug to come from the following part of the \jsinline{enqueue} function: \jsinline!var payload = { data: data, priority: pri || null }!
Indeed, in JavaScript, the expression \jsinline{0 || null} actually evaluates to \jsinline{null}, whereas the developer believed that it reduced to 0.
Even though the developer wrote tests, they never used a priority value of 0 and therefore never triggered the bug.
This shows that \cosette is a useful tool for debugging JavaScript code, because it analyses all possible code paths, and does not make any assumptions about the subtle semantics of the language (type coercions in this case).






%We demonstrate the practical validity of \cosette as a tool for bug-finding in JavaScript.
%We wrote symbolic tests fully covering the code of real-world Node.js libraries, and found several  implementation bugs.
%Moreover, we measured evaluation times, including solver times, to assess the performance of \cosette.
%
%Throughout this section, we will be using the Buckets.js~\cite{buckets} (which has around 65k downloads on npm) data structure library as a running example.
%It implements most common data structures, comes with unit tests, and has no external dependencies, which is necessary for the whole-program analysis of \cosette.
%
%\subsection{Making Concrete Tests Symbolic}
%
%When writing libraries, developers more and more often include extensive test suites, called unit tests, along with actual code (JavaScript has several well-known unit test libraries, such as Jasmine~\cite{jasmine}, Ava~\cite{ava}, or Mocha~\cite{mocha}).
%These test suites allow the developer to have reasonable confidence in the well-behavedness of their code, up to the level of detail of the tests.
%However, being concrete by nature, these tests might miss corner cases or code paths that the developer hasn't reasoned about.
%
%The example code on the right shows the expressiveness of symbolic testing.
%We adapted this code from one of the tests bundled with the array library in Buckets.js.
%The snippet tests the \jsinline{swap} function, which swaps two elements of a JavaScript array, and returns \jsinline{true} if the swap was successful (the two index arguments were within the bounds of the array), or \jsinline{false} if the indices were out of bounds.
%In the original test code, the developer had to write several examples, each reflecting one of these particular cases.
%With the help of symbolic values, we are able to reduce this to a single example which encompasses all possible cases.
%We take \jsinline{ar} to be a JS array with some values, and \jsinline{i} is a symbolic number that we use as an index (for simplicity, we only make one of the two indices symbolic here).
%The assertion on line 7 accurately describes all the possible cases that can happen, \jsinline{i} being either inside the bounds or outside.
%Running \cosette on this code tells us that the assertion always holds, which means that \jsinline{swap} behaves as expected.
%
%Writing symbolic tests with \cosette therefore allows the developer to reduce the burden of test writing and maintenance, by increasing the expressive power of tests without losing their simplicity.
%
%\begin{lstjs}
%var n1 = symb_number(n1); % 0
%var n2 = symb_number(n2);
%var i = symb_number(i);
%var ar = [n1, n1, n2, n2];
%var res = buckets.arrays.swap(ar, 0, i);
%var len = ar.length;
%assert((i >= 0 && i < len && res) || ((i < 0 || i >= len) && !res));
%\end{lstjs}

\newpage
\section{Related Work} 

The existing literature covers a wide range of analysis techniques for JavaScript programs, including: 
type systems~\cite{thiemann:esop:2005,anderson:ecoop:2005,jensen:sas:2009,typescript:toot:2014,feldthaus:oopsla:2014,bierman:ecoop:2014,rastogi:popl:2015},
control flow analysis~\cite{feldthaus2013efficient}, pointer analysis~\cite{jang2009points,sridharan:ecoop:12} and abstract
interpretation~\cite{kashyap:fse:14,jensen:sas:2009,andreasen:oopsla:2014,park:ecoop:15}, among others. 
Here, we focus on the existing work on logic-based analysis and symbolic execution for JavaScript. 

\myparagraph{Symbolic Execution} Ooga. Booga. Boo.




\myparagraph{Logic-based Analysis} 
%
\cite{gardner:popl:2012} have developed a separation logic for a small fragment of ECMAScript 3, to reason about the variable store emulated in the JavaScript heap.
%
\cite{rosu-serbanuta-2010-jlap} have developed $\mathbb{K}$, a term-rewriting framework  for  formalising the operational
semantics of programming languages.
 In particular, they have developed KJS~\cite{Park:2015} which provides a $\mathbb{K}$-interpretation of the core language and part of the built-in libraries of the ES5 standard. KJS has been tested against the official ECMAScript Test262 test suite and passed all 2782 tests for the core language; the testing results for the built-in libraries are not reported. 
\cite{stefanescu-park-yuwen-li-rosu-2016-oopsla} introduce a language-independent verification infrastructure 
that can be instantiated with a $\mathbb{K}$-interpretation of a  language to automatically generate a symbolic verification tool for that language based on the $\mathbb{K}$ reachability logic. They apply this infrastructure to KJS to generate a verification tool for JavaScript, which they use to verify functional correctness properties of operations for manipulating data structures such as binary search trees, AVL trees, and lists.


\section{Conclusions}\label{conclusions}

\pmaxinline{Can we be more general, and say something like 'logic-based specifications'? It's all about translating to FOL, or even some version of PL. Also, we need to say at some point why we care about specifications written in separation logic.}

\newpage
\bibliography{oopsla18}

\newpage
\appendix

%!TEX root = ../main.tex

\newtheorem{lemmax}{}
\newtheorem{temax}{}

\section{\jsil Syntax and Semantics}


\begin{figure}[ht!]
{\scriptsize
\begin{mathpar} 
%
\inferrule[\textsc{Skip}]{}
	{ \semtrans{\heap, \store, \jsilskip}{\heap, \store}} 
 \qquad
 %
\inferrule[\textsc{Assignment}]
  {
      \symbeval{\jsilexpr}{\store} =  \val
      \quad
      \store' = \store[\jvar \mapsto \val]
  }{\semtrans{\heap, \store, \jvar := \jsilexpr}{\heap, \store'}} 
%
\qquad 
%
\inferrule[\textsc{Object Creation}]
  { 
    \heap = \heap \dunion \hcell{\loc}{\protop}{\jsnull}
    \quad (\loc,-) \notin \domain (\heap)
  }{\semtrans{\heap, \store, \jvar := \jsilnew()}{\heap, \store[\jvar \mapsto \loc]}}
\\
%
\inferrule[\textsc{Property Access}]
  { 
 	\symbeval{\jsilexpr_1}{\store} =  \loc
  	\quad 
        \symbeval{\jsilexpr_2}{\store} =  \jstring
        \quad
        \heap = - \dunion \hcell{\loc}{\jstring}{\val}
  }{ \semtrans{\heap, \store, \jvar := [\jsilexpr_1, \jsilexpr_2]}{\heap,  \store[\jvar \mapsto \val]}}
 \and 
 \inferrule[\textsc{Property Deletion}]
  { 
        \symbeval{\jsilexpr_1}{\store} =  \loc
  	\quad 
        \symbeval{\jsilexpr_2}{\store} =  \jstring
        \quad
        \heap = \heap' \dunion \hcell{\loc}{\jstring}{-}
  }{\semtrans{\heap, \store, \jsildelete(\jsilexpr_1, \jsilexpr_2)}{\heap', \store}}
 %
\\
%
\inferrule[\textsc{Property Assignment - Found}]
  {     \symbeval{\jsilexpr_1}{\store} =  \loc
  	\quad 
        \symbeval{\jsilexpr_2}{\store} =  \jstring
        \quad
        \symbeval{\jsilexpr_3}{\store} =  \val
       \\\\
        \heap = \heap' \dunion  \hcell{\loc}{\jstring}{-}
  }{\semtrans{\heap, \store, [\jsilexpr_1, \jsilexpr_2] := \jsilexpr_3}{\heap' \dunion  \hcell{\loc}{\jstring}{\val}, \store}} 
 \and 
 \inferrule[\textsc{Property Assignment - Not Found}]
  {     \symbeval{\jsilexpr_1}{\store} =  \loc
  	\quad 
        \symbeval{\jsilexpr_2}{\store} =  \jstring
        \quad
        \symbeval{\jsilexpr_3}{\store} =  \val
       \\\\
        \heap = \heap' 
        \quad 
        (\loc, \jstring) \not\in \domain(\heap)
  }{\semtrans{\heap, \store, [\jsilexpr_1, \jsilexpr_2] := \jsilexpr_3}{\heap \dunion  \hcell{\loc}{\jstring}{\val}, \store}} 
\\
%
\inferrule[\textsc{Member Check - True}]
  { 
      \symbeval{\jsilexpr_1}{\store} =  \loc
  	\quad 
        \symbeval{\jsilexpr_2}{\store} =  \jstring
       \quad 
   	(\loc, \jstring) \in \domain(\heap) 
  }{\semtrans{\heap, \store,\jvar := \hasfield(\jsilexpr_1, \jsilexpr_2)}{\heap, \store[\jvar \mapsto \jtrue]}}
  \and 
 \inferrule[\textsc{Member Check - False}]
  { 
      \symbeval{\jsilexpr_1}{\store} =  \loc
  	\quad 
        \symbeval{\jsilexpr_2}{\store} =  \jstring
       \quad 
   	(\loc, \jstring) \not\in \domain(\heap) 
  }{\semtrans{\heap, \store,\jvar := \hasfield(\jsilexpr_1, \jsilexpr_2)}{\heap, \store[\jvar \mapsto \jfalse]}}
%
\\
%
\inferrule[\textsc{Assert - True}]
  { 
      \symbeval{\jsilexpr}{\store} =  \jtrue
  }{\semtrans{\heap, \store, \assert(\jsilexpr)}{\heap, \store}} 
\and
\inferrule[\textsc{Assert - False}]
  { 
      \symbeval{\jsilexpr}{\store} \neq \jtrue
  }{\semtranserr{\heap, \store, \assert(\jsilexpr)}} 
\end{mathpar}}
\caption{Symbolic Execution for Basic Commands: {\scriptsize$\semtrans{\heap, \store, \bcmd}{\heap', \store'}$}\label{fig:sem:basic:commands}}
\end{figure}




\begin{figure}[ht!]
{\scriptsize
\begin{mathpar} 
\inferrule[\textsc{Basic Command}]
   { 
     \prog_{\pid}(i) = \bcmd 
     \quad
     \semtrans{\heap, \store, \bcmd}{\heap', \store'} 
   }{\semtrans{\heap, \store, \ctx[i]}{\heap', \store', \ctx[i+1]}}
%
   \qquad
  %
  \inferrule[\textsc{Basic Command - Fail}]
   { 
     \prog_{\pid}(i) = \bcmd 
     \quad
     \semtranserr{\heap, \store, \bcmd} 
   }{\semtranserr{\heap, \store, \ctx[i]}}
 %
   \qquad
  %
  \inferrule[\textsc{Goto}]
   { \prog_{\pid}(i) = \goto \, j \quad}
   {\semtrans{\heap, \store, \ctx[i]}{\heap, \store, \ctx[j]}}
  \\ 
  \inferrule[\textsc{Cond. Goto - True}]
   { \prog_{\pid}(i) =  \ifgoto{\jsilexpr}{j}{k} \quad
     \symbeval{\jsilexpr}{\store} =  \jtrue
   }
   {\semtrans{\heap, \store, \ctx[i]}{\heap, \store, \ctx[j]}}
  \and 
    \inferrule[\textsc{Cond. Goto - False}]
   { \prog_{\pid}(i) =  \ifgoto{\jsilexpr}{j}{k} \quad
     \symbeval{\jsilexpr}{\store} =  \jfalse
   }
   {\semtrans{\heap, \store, \ctx[i]}{\heap, \store, \ctx[k]}}
   \\
    \inferrule[\textsc{Procedure Call}]
   { 
    \prog_{\pid}(i) =   \jsilcall{\jvar}{\jsilexpr}{\jsilexpr_i \mid_{i = 0}^{n}}{j}
     \quad
    \symbeval{\jsilexpr}{\sstore} =  \pid' 
        \quad
     \args(\pid') = \jsillist{\jvar_1, ..., \jvar_{m}} 
      \quad
      \val_i = \symbeval{\jsilexpr_i}{\sstore} \mid_{i = 0}^{n} 
     \ 
      \val_i = \jsundefined \mid_{i = n+1}^{m}  
   }
   {\semtrans{\heap, \store, \ctx[i]}{\heap, [ \jvar_i \mapsto \val_i \mid_{i = 0}^{m}] , ((\pid', \store, \jvar, i+1, j)::\ctx)[0]}}
    \\ 
  \inferrule[\textsc{Normal Return}]
   {
       \ctx = (-, \store', \jvar, i, -) :: \ctx' 
       \quad 
       \store(\procretvar) = \val
   }  
   {\semtrans{\heap, \store, \ctx[\procretlab]}{\heap, \store'[\jvar \mapsto \val], \ctx'[i]}}
   \and 
     \inferrule[\textsc{Error Return}]
   {
       \ctx = (-, \store', \jvar, -, j) :: \sctx' 
       \quad 
       \sstore(\procerrvar) = \val
   }
   {\semtrans{\heap, \store, \ctx[\procerrlab]}{\sheap, \store'[\jvar \mapsto \val], \ctx'[j]}}
 \end{mathpar}}
\caption{Symbolic Execution for Control Flow Commands: {\scriptsize$\semtrans{\heap, \store, \ctx[i]}{\heap', \store', \ctx'[j]}$}} 
\end{figure}


\begin{figure}[ht!]
{\scriptsize
\begin{mathpar} 
\inferrule[\textsc{Basic Command}]
   { 
     \ccmd[\prog][\ctx]{i} = \bcmd 
     \quad
     \semtrans{\heap, \store, \bcmd}{\heap', \store'} 
   }{\semtrans[\prog]{\heap, \store, i}{\heap', \store', i+1}[C]}
%
   \qquad
  %
  \inferrule[\textsc{Basic Command - Fail}]
   { 
     \ccmd[\prog][\ctx]{i} = \bcmd 
     \quad
     \semtranserr{\heap, \store, \bcmd} 
   }{\semtranserr[\prog]{\heap, \store, i}[C]}
 %
   \qquad
  %
  \inferrule[\textsc{Goto}]
   { \ccmd[\prog][\ctx]{i} = \goto \, j \quad}
   {\semtrans[\prog]{\heap, \store, i}{\heap, \store, j}[C]}
  \\ 
  \inferrule[\textsc{Cond. Goto - True}]
   { \ccmd[\prog][\ctx]{i} =  \ifgoto{\jsilexpr}{j}{k} \quad
     \symbeval{\jsilexpr}{\store} =  \jtrue
   }
   {\semtrans[\prog]{\heap, \store, i}{\heap, \store, j}[C]}
  \and 
    \inferrule[\textsc{Cond. Goto - False}]
   { \ccmd[\prog][\ctx]{i} =  \ifgoto{\jsilexpr}{j}{k} \quad
     \symbeval{\jsilexpr}{\store} =  \jfalse
   }
   {\semtrans[\prog]{\heap, \store, i}{\heap, \store, k}[C]}
   \\
    \inferrule[\textsc{Procedure Call}]
   { 
    \ccmd[\prog][\ctx]{i} =   \jsilcall{\jvar}{\jsilexpr}{\jsilexpr_i \mid_{i = 0}^{n}}{j}
     \quad
    \symbeval{\jsilexpr}{\sstore} =  \pid' 
        \quad
     \args(\pid') = \jsillist{\jvar_1, ..., \jvar_{m}} 
      \quad
      \val_i = \symbeval{\jsilexpr_i}{\sstore} \mid_{i = 0}^{n} 
     \ 
      \val_i = \jsundefined \mid_{i = n+1}^{m}  
   }
   {\semtrans[\prog]{\heap, \store, i}{\heap, [ \jvar_i \mapsto \val_i \mid_{i = 0}^{m}] , 0}[C][(\pid', \store, \jvar, i+1, j) :: \ctx]}
    \\ 
  \inferrule[\textsc{Normal Return}]
   {
       \ctx = (-, \store', \jvar, i, -) :: \ctx' 
       \quad 
       \store(\procretvar) = \val
   }  
   {\semtrans[\prog]{\heap, \store, \procretlab}{\heap, \store'[\jvar \mapsto \val], i}[C][C']}
   \and 
     \inferrule[\textsc{Error Return}]
   {
       \ctx = (-, \store', \jvar, -, j) :: \ctx' 
       \quad 
       \store(\procerrvar) = \val
   }
   {\semtrans[\prog]{\heap, \store, \procerrlab}{\heap, \store'[\jvar \mapsto \val], j}[C][C']}
 \end{mathpar}}
 \pmax{What happens when we exit from main, how do we stop? Basically, cmd, returns nothing and we can't reduce?}
 \vspace*{-0.4cm}
\caption{Symbolic Execution for Control Flow Commands: $\semtrans[\prog]{\heap, \store, i}{\heap', \store', j}[C][C']$}
\end{figure}

\begin{figure}[ht!]
{\scriptsize
\begin{mathpar} 
\inferrule[\textsc{Basic Command}]
   { 
     \ccmd{i} = \bcmd 
     \quad
     \semtrans{\heap, \store, \bcmd}{\heap', \store'} 
   }{\semtrans{\heap, \store, i}{\heap', \store', i+1}}
%
   \qquad
  %
  \inferrule[\textsc{Basic Command - Fail}]
   { 
     \ccmd{i} = \bcmd 
     \quad
     \semtranserr{\heap, \store, \bcmd} 
   }{\semtranserr{\heap, \store, i}}
 %
   \qquad
  %
  \inferrule[\textsc{Goto}]
   { \ccmd{i} = \goto \, j \quad}
   {\semtrans{\heap, \store, i}{\heap, \store, j}}
  \\ 
  \inferrule[\textsc{Cond. Goto - True}]
   { \ccmd{i} =  \ifgoto{\jsilexpr}{j}{k} \quad
     \symbeval{\jsilexpr}{\store} =  \jtrue
   }
   {\semtrans{\heap, \store, i}{\heap, \store, j}}
  \and 
    \inferrule[\textsc{Cond. Goto - False}]
   { \ccmd{i} =  \ifgoto{\jsilexpr}{j}{k} \quad
     \symbeval{\jsilexpr}{\store} =  \jfalse
   }
   {\semtrans{\heap, \store, i}{\heap, \store, k}}
   \\
    \inferrule[\textsc{Procedure Call}]
   { 
    \ccmd{i} = \jsilcall{\jvar}{\jsilexpr}{\jsilexpr_i \mid_{i = 0}^{n}}{j}
     \quad
    \symbeval{\jsilexpr}{\sstore} =  \pid' 
        \quad
     \args(\pid') = \jsillist{\jvar_1, ..., \jvar_{m}} 
      \quad
      \val_i = \symbeval{\jsilexpr_i}{\sstore} \mid_{i = 0}^{n} 
     \ 
      \val_i = \jsundefined \mid_{i = n+1}^{m}  
   }
   {\semtrans{\heap, \store, i}{\heap, [ \jvar_i \mapsto \val_i \mid_{i = 0}^{m}] , 0}[C][(\pid', \store, \jvar, i+1, j) :: \ctx]}
    \\ 
  \inferrule[\textsc{Normal Return}]
   {
       \ctx = (-, \store', \jvar, i, -) :: \ctx' 
       \quad 
       \store(\procretvar) = \val
   }  
   {\semtrans{\heap, \store, \procretlab}{\heap, \store'[\jvar \mapsto \val], i}[C][C']}
   \and 
     \inferrule[\textsc{Error Return}]
   {
       \ctx = (-, \store', \jvar, -, j) :: \ctx' 
       \quad 
       \store(\procerrvar) = \val
   }
   {\semtrans{\heap, \store, \procerrlab}{\heap, \store'[\jvar \mapsto \val], j}[C][C']}
 \end{mathpar}}
 \vspace*{-0.4cm}
\caption{Symbolic Execution for Control Flow Commands: $\semtrans[\prog]{\heap, \store, i}{\heap', \store', j}[C][C']$}
\end{figure}

\section{Proofs - Section~\ref{sec:jsil:symb:exec}}

\begin{lemma}[Soundess of symbolic execution for \jsil basic commands]
$$
\begin{array}{l}
\symbtrans{\sheap, \sstore, \bcmd, \pc}{\sheap', \sstore', \pc'}
   \ \wedge \ 
      (\heap, \store) \in \smodels{\sheap, \sstore}{\pc'} \\ \quad \quad
      	 \ \implies \ \exists (\heap', \store') \, . \, 
	 	 \semtrans{\heap, \store, \bcmd}{\heap', \store'}
		\, \wedge \, 
		(\heap', \store') \in \smodels{\sheap', \sstore'}{\pc'}  
\end{array}
$$
\end{lemma}
\begin{proof}
We proceed by case analysis on $\symbtrans{\sheap, \sstore, \bcmd, \pc}{\sheap', \sstore', \pc'}$. 
\vspace{5pt}

\noindent\prooflab{Skip} 
We conclude that $\bcmd = \jsilskip$, and 
that $\sheap' = \sheap$, $\sstore' = \sstore$, and $\pc' = \pc$. 
By picking $\heap' = \heap$, $\store' = \store$, the result follows. 
\vspace{6pt}

\noindent\prooflab{Assignment} 
We conclude that $\bcmd = \jvar := \jsilexpr$, for some variable $\jvar$ and expression $\jsilexpr$, 
and that $\sheap' = \sheap$, $\sstore' = \sstore[\jvar \mapsto \symbeval{\jsilexpr}{\store}]$, and $\pc' = \pc$. 
From $(\heap, \store) \in \smodels{\sheap, \sstore}{\pc'}$, we conclude that there is a symbolic environment 
$\senv$ such that $\heap = \semexpr{\sheap}{\senv}$ and $\store = \semexpr{\sstore}{\senv}$. 
Noting that: 
$$
 \semtrans{\heap, \store, \jvar := \jsilexpr}{\heap, \store[\jvar \mapsto \symbeval{\jsilexpr}{\store}]}
% \qquad 
 %\semexpr{\sstore[\jvar \mapsto \symbeval{\jsilexpr}{\store}]}{\senv} = \semexpr{\sstore}{\senv}[\jvar \mapsto \symbeval{\jsilexpr}{\store, \senv}]
$$
we pick $\heap' = \heap$ and $\store' =  \store[\jvar \mapsto \symbeval{\jsilexpr}{\store}]$. We 
now have to prove that $(\heap', \store') \in \smodels{\sheap', \sstore'}{\pc}$.
Observing that: 
$$
\heap' =  \semexpr{\sheap}{\senv} = \semexpr{\sheap'}{\senv} 
\quad 
\store' = \semexpr{\sstore}{\senv}[\jvar \mapsto \symbeval{\jsilexpr}{\semexpr{\sstore}{\senv}}]
   = \semexpr{\sstore[\jvar \mapsto \symbeval{\jsilexpr}{\store}]}{\senv} 
   = \semexpr{\sstore'}{\senv}
$$
%
the result follows. 
\vspace{6pt}

\noindent\prooflab{Object Creation}
We conclude that $\bcmd = \jvar := \jsilnew()$, for some variable $\jvar$, and that
$\sheap' = \sheap \dunion \hcell{\loc}{\protop}{\jsnull}$, $\sstore' = \sstore[\jvar \mapsto \loc]$, and $\pc' = \pc$, 
 where  $(\loc,-) \notin \domain (\sheap)$. 
 From $(\heap, \store) \in \smodels{\sheap, \sstore}{\pc'}$, we conclude that there is a symbolic environment
$\senv$ such that $\heap = \semexpr{\sheap}{\senv}$ and $\store = \semexpr{\sstore}{\senv}$. 
Noting that: 
$$
\semtrans{\heap, \store, \jvar := \jsilnew()}{\heap \dunion \hcell{\loc}{\protop}{\jsnull}, \store[\jvar \mapsto \loc]}
$$
where: $(\loc,-) \notin \domain (\heap)$, we pick $\heap' = \semexpr{\sheap}{\senv} \dunion \hcell{\loc}{\protop}{\jsnull}$ 
and $\store' = \semexpr{\sstore}{\senv}[\jvar \mapsto \loc]$. 
We now have to prove that $(\heap', \store') \in \smodels{\sheap', \sstore'}{\pc}$.
Noting that: 
$$
\begin{array}{l}
\heap' = \semexpr{\sheap}{\senv} \dunion \hcell{\loc}{\protop}{\jsnull} = \semexpr{\sheap \dunion \hcell{\loc}{\protop}{\jsnull}}{\senv}   
     = \semexpr{\sheap'}{\senv}  \\
%
\store' = \semexpr{\sstore}{\senv}[\jvar \mapsto \loc] = \semexpr{\sstore}{\senv}[\jvar \mapsto \symbeval{\loc}{\senv}] = 
      \semexpr{\sstore[\jvar \mapsto \loc]}{\senv} = \semexpr{\sstore'}{\senv} 
\end{array}
$$
the result follows. 
\vspace{6pt}

\noindent\prooflab{Property Access}
We conclude that $\bcmd = \jvar := [\jsilexpr_1, \jsilexpr_2]$, for some variable $\jvar$, and expressions $\jsilexpr_1$ and $\jsilexpr_2$, 
and that $\sheap' = \sheap$, $\sstore' = \sstore[\jvar \mapsto \sexprv_k]$, and: 
 $$\pc' =  \pc \ \wedge \, \big( (\sexprp_k = \sexpr_p) \ \wedge \bigwedge_{i = 0, i \neq k}^n (\sexprp_i \neq \sexpr_p) \big)$$
 where 
 $\symbeval{\jsilexpr_1}{\sstore} =  \loc$, $\symbeval{\jsilexpr_2}{\sstore} =  \sexpr_p$, 
 $\sheap = \sheap'' \, \uplus \, \big((l, \sexprp_i) \mapsto \sexprv_i\big)\mid_{i = 0}^n$, 
 $(l, -) \not\in \domain(\sheap')$, and $0 \leq k \leq n$. 
%
From $(\heap, \store) \in \smodels{\sheap, \sstore}{\pc'}$, we conclude that there is a symbolic environment
$\senv$ such that $\heap = \semexpr{\sheap}{\senv}$, $\store = \semexpr{\sstore}{\senv}$, and 
$\senv \vdash \pc'$. 
We now have to prove that we can apply the \prooflab{Property Access} rule in the concrete state.
To this end, we have to show that there is a concrete heap $\heap''$ such that:
$\heap = \heap'' \dunion \hcell{\symbeval{\jsilexpr_1}{\store}}{\symbeval{\jsilexpr_2}{\store}}{\symbeval{\jsilexpr_3}{\store}}$. 
Note that: 
$$
\begin{array}{l}
%
 \symbeval{\jsilexpr_1}{\store} = \symbeval{\jsilexpr_1}{\symbeval{\sstore}{\senv}} = \symbeval{\symbeval{\jsilexpr_1}{\sstore}}{\senv} 
    = \symbeval{\loc}{\senv} = \loc \\ 
 %
  \symbeval{\jsilexpr_2}{\store}  = \symbeval{\jsilexpr_2}{\semexpr{\sstore}{\senv}} =  \symbeval{\symbeval{\jsilexpr_2}{\sstore}}{\senv}
   =  \symbeval{\sexpr_p}{\senv} = \symbeval{\sexprp_k}{\senv}  \text{ (because $\senv \vdash \pc'$ and $\pc' \vdash \sexprp_k = \sexpr_p$)} \\
 %
 \heap = \semexpr{\sheap'' \, \uplus \, \big((l, \sexprp_i) \mapsto \sexprv_i\big)\mid_{i = 0}^n}{\senv} 
       =  \semexpr{\sheap'' \, \uplus \, \big((l, \sexprp_i) \mapsto \sexprv_i\big)\mid_{i = 0, i \neq k}^n}{\senv} \dunion \semexpr{(l, \sexprp_k) \mapsto \sexprv_k}{\senv} \\
         \qquad = \semexpr{\sheap'' \, \uplus \, \big((l, \sexprp_i) \mapsto \sexprv_i\big)\mid_{i = 0, i \neq k}^n}{\senv} \dunion (l, \semexpr{\sexprp_k}{\senv}) \mapsto \semexpr{\sexprv_k}{\senv}  \\ 
         \qquad =  \semexpr{\sheap'' \, \uplus \, \big((l, \sexprp_i) \mapsto \sexprv_i\big)\mid_{i = 0, i \neq k}^n}{\senv} \dunion (\symbeval{\jsilexpr_1}{\store}, \symbeval{\jsilexpr_2}{\store}) \mapsto \semexpr{\sexprv_k}{\senv}
\end{array}
$$
We can now apply the \prooflab{Property Access} rule of \jsil semantics, concluding: 
$$
   \semtrans{\heap, \store, \jvar := [\jsilexpr_1, \jsilexpr_2]}{\heap,  \store[\jvar \mapsto \semexpr{\sexprv_k}{\senv}]}
$$
meaning that: $\heap' = \heap$ and $\store' = \store[\jvar \mapsto \semexpr{\sexprv_k}{\senv}]$.
We have now to prove that $(\heap', \store') \in \smodels{\sheap', \sstore'}{\pc'}$.
Observe that: 
$$
\begin{array}{l}
\heap' = \heap = \semexpr{\sheap}{\senv}   = \semexpr{\sheap'}{\senv}  \text{ (because $\heap' = \heap$ and $ \sheap = \sheap'$)}
\\
 \store' =  \semexpr{\sstore}{\senv}[\jvar \mapsto \symbeval{\sexprv_k}{\senv}] 
    =  \semexpr{\sstore[\jvar \mapsto \sexprv_k]}{\senv} 
    =  \semexpr{\sstore'}{\senv}
\end{array}
$$
 which concludes the proof. 
\vspace{6pt}

\noindent\prooflab{Property Deletion}
We conclude that $\bcmd = \jsildelete(\jsilexpr_1, \jsilexpr_2)$, for some expressions $\jsilexpr_1$ and $\jsilexpr_2$
and that: 
$$
\begin{array}{l}
\sheap' = \sheap'' \, \uplus \,  \big((\loc, \sexprp_i) \mapsto \sexprv_i\big)\mid_{i = 0, i \neq k}^n
\quad 
\sstore' = \sstore
\\ 
 \pc' = \pc \ \wedge \, \big( (\sexprp_k = \sexpr_p) \ \wedge \bigwedge_{i = 0, i \neq k}^n (\sexprp_i \neq \sexpr_p) \big)
\end{array}
$$
where $\loc = \symbeval{\jsilexpr_1}{\sstore}$ and $\sexpr_p = \symbeval{\jsilexpr_2}{\sstore}$
From $(\heap, \store) \in \smodels{\sheap, \sstore}{\pc'}$, we conclude that there is a symbolic environment
$\senv$ such that $\heap = \semexpr{\sheap}{\senv}$, $\store = \semexpr{\sstore}{\senv}$, and 
$\senv \vdash \pc'$. 
We now have to prove that we can apply the \prooflab{Property Deletion} rule in the concrete state.
To this end, we have to show that:
$\heap = \heap' \dunion \hcell{\symbeval{\jsilexpr_1}{\store}}{\symbeval{\jsilexpr_2}{\store}}{-}$. 
Note that: 
$$
\begin{array}{l}
%
 \symbeval{\jsilexpr_1}{\store} = \symbeval{\jsilexpr_1}{\symbeval{\sstore}{\senv}} = \symbeval{\symbeval{\jsilexpr_1}{\sstore}}{\senv} 
    = \symbeval{\loc}{\senv} = \loc \\ 
 %
  \symbeval{\jsilexpr_2}{\store}  = \symbeval{\jsilexpr_2}{\semexpr{\sstore}{\senv}} =  \symbeval{\symbeval{\jsilexpr_2}{\sstore}}{\senv}
   =  \symbeval{\sexpr_p}{\senv} = \symbeval{\sexprp_k}{\senv}  \text{ (because $\senv \vdash \pc'$ and $\pc' \vdash \sexprp_k = \sexpr_p$)} \\
 %
 \heap = \semexpr{\sheap'' \, \uplus \, \big((l, \sexprp_i) \mapsto \sexprv_i\big)\mid_{i = 0}^n}{\senv} 
       =  \semexpr{\sheap'' \, \uplus \, \big((l, \sexprp_i) \mapsto \sexprv_i\big)\mid_{i = 0, i \neq k}^n}{\senv} \dunion \semexpr{(l, \sexprp_k) \mapsto \sexprv_k}{\senv} \\
         \qquad = \semexpr{\sheap'' \, \uplus \, \big((l, \sexprp_i) \mapsto \sexprv_i\big)\mid_{i = 0, i \neq k}^n}{\senv} \dunion (l, \semexpr{\sexprp_k}{\senv}) \mapsto \semexpr{\sexprv_k}{\senv}  \\ 
         \qquad =  \semexpr{\sheap'' \, \uplus \, \big((l, \sexprp_i) \mapsto \sexprv_i\big)\mid_{i = 0, i \neq k}^n}{\senv} \dunion (\symbeval{\jsilexpr_1}{\store}, \symbeval{\jsilexpr_2}{\store}) \mapsto \semexpr{\sexprv_k}{\senv} \\ 
         \qquad = \semexpr{\sheap'}{\senv} \dunion (\symbeval{\jsilexpr_1}{\store}, \symbeval{\jsilexpr_2}{\store}) \mapsto -
\end{array}
$$
We can now apply the \prooflab{Property Deletion} rule of \jsil semantics, concluding: 
$$
   \semtrans{\heap, \store, \jsildelete(\jsilexpr_1, \jsilexpr_2)}{\semexpr{\sheap'}{\senv},  \store}
$$
meaning that: $\heap' = \semexpr{\sheap'}{\senv}$ and $\store' = \store$.
We have now to prove that $(\heap', \store') \in \smodels{\sheap', \sstore'}{\pc'}$.
Noting that $\heap' = \semexpr{\sheap'}{\senv}$ and $\store' = \store = \semexpr{\sstore}{\senv} = \semexpr{\sstore'}{\senv}$, 
the result follows. 
\vspace{6pt}

\noindent\prooflab{Property Assignment - Found}
We conclude that  $\bcmd = [\jsilexpr_1, \jsilexpr_2] := \jsilexpr_3$ for some expressions $\jsilexpr_1$, $\jsilexpr_2$, 
and $\jsilexpr_3$, and that: 
$$
\begin{array}{l}
  \sheap =  \sheap'' \, \uplus \, \big((l, \sexprp_i) \mapsto \sexprv_i\big)\mid_{i = 0}^n    \\
  %
  \sheap' = \sheap'' \, \uplus \,  \big((l, \sexprp_i) \mapsto \sexprv_i\big)\mid_{i = 0, i \neq k}^n \, \uplus \,  (l, \sexpr_p) \mapsto \sexpr_v  \\
  %
  \sstore' = \sstore \\ 
  %
  \pc' = \pc \ \wedge \, \big( (\sexprp_k = \sexpr_p) \ \wedge \bigwedge_{i = 0, i \neq k}^n (\sexprp_i \neq \sexpr_p)
\end{array}
$$ 
where $\symbeval{\jsilexpr_1}{\sstore} =  \loc$, $\symbeval{\jsilexpr_2}{\sstore} =  \sexpr_p$, 
$\symbeval{\jsilexpr_3}{\sstore} =  \sexpr_v$.
From $(\heap, \store) \in \smodels{\sheap, \sstore}{\pc'}$, we conclude that there is a symbolic environment
$\senv$ such that $\heap = \semexpr{\sheap}{\senv}$, $\store = \semexpr{\sstore}{\senv}$, and 
$\senv \vdash \pc'$. 
We now have to prove that we can apply the \prooflab{Property Assignment - Found} rule in the concrete state.
To this end, we have to show that there is a concrete heap $\heap''$ such that:
$\heap = \heap'' \dunion \hcell{\symbeval{\jsilexpr_1}{\store}}{\symbeval{\jsilexpr_2}{\store}}{-}$. 
Note that: 
$$
\begin{array}{l}
%
 \symbeval{\jsilexpr_1}{\store} = \symbeval{\jsilexpr_1}{\symbeval{\sstore}{\senv}} = \symbeval{\symbeval{\jsilexpr_1}{\sstore}}{\senv} 
    = \symbeval{\loc}{\senv} = \loc \\ 
 %
  \symbeval{\jsilexpr_2}{\store}  = \symbeval{\jsilexpr_2}{\semexpr{\sstore}{\senv}} =  \symbeval{\symbeval{\jsilexpr_2}{\sstore}}{\senv}
   =  \symbeval{\sexpr_p}{\senv} = \symbeval{\sexprp_k}{\senv}  \text{ (because $\senv \vdash \pc'$ and $\pc' \vdash \sexprp_k = \sexpr_p$)} \\
 %
  \symbeval{\jsilexpr_3}{\store}  = \symbeval{\jsilexpr_3}{\semexpr{\sstore}{\senv}} =  \symbeval{\symbeval{\jsilexpr_3}{\sstore}}{\senv}
   =  \symbeval{\sexpr_v}{\senv} \\
 %
 \heap = \semexpr{\sheap'' \, \uplus \, \big((l, \sexprp_i) \mapsto \sexprv_i\big)\mid_{i = 0}^n}{\senv} 
       =  \semexpr{\sheap'' \, \uplus \, \big((l, \sexprp_i) \mapsto \sexprv_i\big)\mid_{i = 0, i \neq k}^n}{\senv} \dunion \semexpr{(l, \sexprp_k) \mapsto \sexprv_k}{\senv} \\
         \qquad = \semexpr{\sheap'' \, \uplus \, \big((l, \sexprp_i) \mapsto \sexprv_i\big)\mid_{i = 0, i \neq k}^n}{\senv} \dunion (l, \semexpr{\sexprp_k}{\senv}) \mapsto \semexpr{\sexprv_k}{\senv}  \\ 
         \qquad =  \semexpr{\sheap'' \, \uplus \, \big((l, \sexprp_i) \mapsto \sexprv_i\big)\mid_{i = 0, i \neq k}^n}{\senv} \dunion (\symbeval{\jsilexpr_1}{\store}, \symbeval{\jsilexpr_2}{\store}) \mapsto \semexpr{\sexprv_k}{\senv} \\ 
\end{array}
$$
We can now apply the \prooflab{Property Assignment - Found} rule of \jsil semantics, concluding: 
$$
   \semtrans{\heap, \store, [\jsilexpr_1, \jsilexpr_2] := \jsilexpr_3}
     {\semexpr{\sheap'' \, \uplus \, \big((l, \sexprp_i) \mapsto \sexprv_i\big)\mid_{i = 0, i \neq k}^n}{\senv} \dunion (\symbeval{\jsilexpr_1}{\store}, \symbeval{\jsilexpr_2}{\store}) \mapsto \symbeval{\jsilexpr_3}{\store},  \store}
$$
meaning that: 
$\heap' = \symbeval{\sheap'' \, \uplus \, \big((l, \sexprp_i) \mapsto \sexprv_i\big)\mid_{i = 0, i \neq k}^n}{\senv} \dunion (\symbeval{\jsilexpr_1}{\store}, \symbeval{\jsilexpr_2}{\store}) \mapsto \symbeval{\jsilexpr_3}{\store}$ and 
$\store' = \store$.
We have now to prove that $(\heap', \store') \in \smodels{\sheap', \sstore'}{\pc'}$.
Noting that:
$$
\begin{array}{l}
\heap' = \symbeval{\sheap'' \, \uplus \, \big((l, \sexprp_i) \mapsto \sexprv_i\big)\mid_{i = 0, i \neq k}^n}{\senv} \dunion (\symbeval{\jsilexpr_1}{\store}, \symbeval{\jsilexpr_2}{\store}) \mapsto \symbeval{\jsilexpr_3}{\store} \\ 
  \qquad = \symbeval{\sheap'' \, \uplus \, \big((l, \sexprp_i) \mapsto \sexprv_i\big)\mid_{i = 0, i \neq k}^n}{\senv} \dunion (\loc, \symbeval{\sexpr_p}{\senv}) \mapsto \symbeval{\sexpr_v}{\senv}  \\
    \qquad = \symbeval{\sheap'' \, \uplus \, \big((l, \sexprp_i) \mapsto \sexprv_i\big)\mid_{i = 0, i \neq k}^n \dunion (\loc, \sexpr_p) \mapsto \sexpr_v}{\senv}  \\
    \qquad = \symbeval{\sheap'}{\senv} \\[2pt]
 %
 \store' = \store = \symbeval{\sstore}{\senv} = \symbeval{\sstore'}{\senv} 
\end{array}
$$
the result follows. 
\vspace{6pt}

\noindent\prooflab{Property Assignment - Not Found}
We conclude that  $\bcmd = [\jsilexpr_1, \jsilexpr_2] := \jsilexpr_3$ for some expressions $\jsilexpr_1$, $\jsilexpr_2$, 
and $\jsilexpr_3$, and that: 
$$
\begin{array}{l}
  \sheap =   \sheap'' \, \uplus \, \big((l, \sexprp_i) \mapsto \sexprv_i\big)\mid_{i = 0}^n     \\
  %
  \sheap' =  \sheap \, \uplus \,  (l, \sexpr_p) \mapsto \sexpr_v  \\
  %
  \sstore' = \sstore \\ 
  %
    \pc' = \pc \ \wedge \, \bigwedge_{i = 0}^n (\sexprp_i \neq \sexpr_p)
\end{array}
$$ 
where $\symbeval{\jsilexpr_1}{\sstore} =  \loc$, $\symbeval{\jsilexpr_2}{\sstore} =  \sexpr_p$, 
$\symbeval{\jsilexpr_3}{\sstore} =  \sexpr_v$,  $(\loc, -) \not\in \domain(\sheap'')$, 
and $0 \leq k \leq n$. 
From $(\heap, \store) \in \smodels{\sheap, \sstore}{\pc'}$, we conclude that there is a symbolic environment
$\senv$ such that $\heap = \semexpr{\sheap}{\senv}$, $\store = \semexpr{\sstore}{\senv}$, and 
$\senv \vdash \pc'$. 
We have now to prove that we can apply the \prooflab{Property Assignment - Found} rule in the concrete state.
To this end, we have to show that:
$(\symbeval{\jsilexpr_1}{\store}, \symbeval{\jsilexpr_2}{\store}) \not\in \domain(\heap)$. 
Note that: 
$$
\begin{array}{l}
%
 \symbeval{\jsilexpr_1}{\store} = \symbeval{\jsilexpr_1}{\symbeval{\sstore}{\senv}} = \symbeval{\symbeval{\jsilexpr_1}{\sstore}}{\senv} 
    = \symbeval{\loc}{\senv} = \loc \\ 
 %
  \symbeval{\jsilexpr_2}{\store}  = \symbeval{\jsilexpr_2}{\semexpr{\sstore}{\senv}} =  \symbeval{\symbeval{\jsilexpr_2}{\sstore}}{\senv}
   =  \symbeval{\sexpr_p}{\senv} \\
 %
  \symbeval{\jsilexpr_3}{\store}  = \symbeval{\jsilexpr_3}{\semexpr{\sstore}{\senv}} =  \symbeval{\symbeval{\jsilexpr_3}{\sstore}}{\senv}
   =  \symbeval{\sexpr_v}{\senv} \\
 %
 \heap = \semexpr{\sheap'' \, \uplus \, \big((l, \sexprp_i) \mapsto \sexprv_i\big)\mid_{i = 0}^n}{\senv} \\
    \qquad = \semexpr{\sheap''}{\senv} \dunion \biguplus_{0 \leq i \leq n} ((l, \symbeval{\sexprp_i}{\senv}) \mapsto \symbeval{\sexprv_i}{\senv})
\end{array}
$$
From  $(\loc, -) \not\in \domain(\sheap'')$, we conclude that $(\loc, -) \not\in \domain(\semexpr{\sheap''}{\senv})$. 
Since $\senv \vdash \pc'$, we additionally conclude that: 
$
  \forall_{0 \leq i \leq n}  \, \symbeval{\sexprp_i}{\senv} \neq \symbeval{\sexpr_p}{\senv} 
$
Recalling that $\symbeval{\jsilexpr_2}{\store} = \symbeval{\sexpr_p}{\senv}$, we conclude that  
$
  \forall_{0 \leq i \leq n}  \, \symbeval{\sexprp_i}{\senv} \neq \symbeval{\jsilexpr_2}{\store}
$, from which it follows (together with $(\loc, -) \not\in \domain(\semexpr{\sheap''}{\senv})$) that 
$(\symbeval{\jsilexpr_1}{\store}, \symbeval{\jsilexpr_2}{\store}) \not\in \domain(\heap)$.
We can now apply the \prooflab{Property Assignment - Not Found} rule of \jsil semantics, concluding: 
$$
   \semtrans{\heap, \store, [\jsilexpr_1, \jsilexpr_2] := \jsilexpr_3}
     {\heap \dunion (\symbeval{\jsilexpr_1}{\store}, \symbeval{\jsilexpr_2}{\store}) \mapsto \symbeval{\jsilexpr_3}{\store},  \store}
$$
meaning that $\heap' = \heap \dunion (\symbeval{\jsilexpr_1}{\store}, \symbeval{\jsilexpr_2}{\store}) \mapsto \symbeval{\jsilexpr_3}{\store}$ 
and $\store' = \store$. 
%
We have now to prove that $(\heap', \store') \in \smodels{\sheap', \sstore'}{\pc'}$.
Noting that:
$$
\begin{array}{l}
\heap' = \symbeval{\sheap}{\senv} \dunion (\symbeval{\jsilexpr_1}{\store}, \symbeval{\jsilexpr_2}{\store}) \mapsto \symbeval{\jsilexpr_3}{\store} \\ 
  \qquad = \symbeval{\sheap}{\senv} \dunion (\loc, \symbeval{\sexpr_p}{\senv}) \mapsto \symbeval{\sexpr_v}{\senv}  \\
    \qquad = \symbeval{\sheap \dunion (\loc, \sexpr_p) \mapsto \sexpr_v}{\senv}  \\
    \qquad = \symbeval{\sheap'}{\senv} \\[2pt]
 %
 \store' = \store = \symbeval{\sstore}{\senv} = \symbeval{\sstore'}{\senv} 
\end{array}
$$
the result follows. 
\vspace{6pt}



\noindent\prooflab{Member Check - True}
We conclude that  $\bcmd = \jvar := \hasfield(\jsilexpr_1, \jsilexpr_2)$ for some variable $\jvar$ and expressions $\jsilexpr_1$ and $\jsilexpr_2$, and that: 
$$
\begin{array}{l}
  \sheap =   \sheap'' \, \uplus \, \big((l, \sexprp_i) \mapsto -\big)\mid_{i = 0}^n     \\
  %
  \sheap' =  \sheap \\
  %
  \sstore' = \sstore[\jvar \mapsto \jtrue] \\ 
  %
    \pc' = \pc \ \wedge \, \big( (\sexprp_k = \sexpr_p) \ \wedge \bigwedge_{i = 0, i \neq k}^n (\sexprp_i \neq \sexpr_p) \big)
\end{array}
$$ 
where $\symbeval{\jsilexpr_1}{\sstore} =  \loc$, $\symbeval{\jsilexpr_2}{\sstore} =  \sexpr_p$, 
$(\loc, -) \not\in \domain(\sheap'')$, and $0 \leq k \leq n$. 
%
From $(\heap, \store) \in \smodels{\sheap, \sstore}{\pc'}$, we conclude that there is a symbolic environment
$\senv$ such that $\heap = \semexpr{\sheap}{\senv}$, $\store = \semexpr{\sstore}{\senv}$, and 
$\senv \vdash \pc'$. 
We have now to prove that we can apply the \prooflab{Member Check - True} rule in the concrete state.
To this end, we have to show that:
$\heap = \heap'' \dunion (\symbeval{\jsilexpr_1}{\store}, \symbeval{\jsilexpr_2}{\store}) \mapsto -$, for 
some concrete heap $\heap''$. 
Note that: 
$$
\begin{array}{l}
%
 \symbeval{\jsilexpr_1}{\store} = \symbeval{\jsilexpr_1}{\symbeval{\sstore}{\senv}} = \symbeval{\symbeval{\jsilexpr_1}{\sstore}}{\senv} 
    = \symbeval{\loc}{\senv} = \loc \\ 
 %
  \symbeval{\jsilexpr_2}{\store}  = \symbeval{\jsilexpr_2}{\semexpr{\sstore}{\senv}} =  \symbeval{\symbeval{\jsilexpr_2}{\sstore}}{\senv}
   =  \symbeval{\sexpr_p}{\senv} \\
 %
 \heap = \semexpr{\sheap'' \, \uplus \, \big((l, \sexprp_i) \mapsto \sexprv_i\big)\mid_{i = 0}^n}{\senv} \\
    \qquad = \semexpr{\sheap'' \, \uplus \, \big((l, \sexprp_i) \mapsto \sexprv_i\big)\mid_{i = 0, i\neq k}^n \dunion (l, \sexprp_k) \mapsto \sexprv_k}{\senv} \\
    \qquad = \semexpr{\sheap'' \, \uplus \, \big((l, \sexprp_i) \mapsto \sexprv_i\big)\mid_{i = 0, i\neq k}^n}{\senv} \dunion \semexpr{(l, \sexprp_k) \mapsto \sexprv_k}{\senv} \\
    \qquad = \semexpr{\sheap'' \, \uplus \, \big((l, \sexprp_i) \mapsto \sexprv_i\big)\mid_{i = 0, i\neq k}^n}{\senv} \dunion (l, \semexpr{\sexprp_k}{\senv}) \mapsto \semexpr{\sexprv_k}{\senv} \\ 
     \qquad = \semexpr{\sheap'' \, \uplus \, \big((l, \sexprp_i) \mapsto \sexprv_i\big)\mid_{i = 0, i\neq k}^n}{\senv} \dunion (l, \semexpr{\sexpr_p}{\senv}) \mapsto \semexpr{\sexprv_k}{\senv}
      			\text{ (using $\senv \vdash \pc$)} \\ 
     \qquad = \semexpr{\sheap'' \, \uplus \, \big((l, \sexprp_i) \mapsto \sexprv_i\big)\mid_{i = 0, i\neq k}^n}{\senv} \dunion (\symbeval{\jsilexpr_1}{\store}, \symbeval{\jsilexpr_2}{\store}) \mapsto - \\
\end{array}
$$
We can now apply the \prooflab{Member Check - True} rule of \jsil semantics, concluding: 
$$
   \semtrans{\heap, \store, \jvar := \hasfield(\jsilexpr_1, \jsilexpr_2)}{\heap,  \store[\jvar \mapsto \jtrue]}
$$
meaning that $\heap' = \heap$ and $\store' = \store[\jvar \mapsto \jtrue]$. 
%
We have now to prove that $(\heap', \store') \in \smodels{\sheap', \sstore'}{\pc'}$.
Noting that:
$$
\begin{array}{l}
\heap' = \heap = \semexpr{\sheap}{\senv} = \semexpr{\sheap'}{\senv} \\
 %
 \store' = \store[\jvar \mapsto \jtrue] = \symbeval{\sstore}{\senv}[\jvar \mapsto \jtrue]  = \symbeval{\sstore[\jvar \mapsto \jtrue]}{\senv} = \symbeval{\sstore'}{\senv} 
\end{array}
$$
the result follows. 
\vspace{6pt}


\noindent\prooflab{Member Check - False}
We conclude that  $\bcmd = \jvar := \hasfield(\jsilexpr_1, \jsilexpr_2)$ for some variable $\jvar$ and expressions $\jsilexpr_1$ and $\jsilexpr_2$, and that: 
$$
\begin{array}{l}
  \sheap =  \sheap'' \, \uplus \, \big((l, \sexprp_i) \mapsto -\big)\mid_{i = 0}^n      \\
  %
  \sheap' =  \sheap \\
  %
  \sstore' = \sstore[\jvar \mapsto \jfalse] \\ 
  %
     \pc' = \pc \ \wedge \,  \bigwedge_{i = 0}^n (\sexprp_i \neq \sexpr_p) 
\end{array}
$$ 
where $\symbeval{\jsilexpr_1}{\sstore} =  \loc$, $\symbeval{\jsilexpr_2}{\sstore} =  \sexpr_p$, 
$(\loc, -) \not\in \domain(\sheap'')$, and $0 \leq k \leq n$. 
%
From $(\heap, \store) \in \smodels{\sheap, \sstore}{\pc'}$, we conclude that there is a symbolic environment
$\senv$ such that $\heap = \semexpr{\sheap}{\senv}$, $\store = \semexpr{\sstore}{\senv}$, and 
$\senv \vdash \pc'$. 
We have now to prove that we can apply the \prooflab{Member Check - False} rule in the concrete state.
To this end, we have to show that: $(\symbeval{\jsilexpr_1}{\store}, \symbeval{\jsilexpr_2}{\store}) \not\in \domain(\heap)$. 
Note that: 
$$
\begin{array}{l}
%
 \symbeval{\jsilexpr_1}{\store} = \symbeval{\jsilexpr_1}{\symbeval{\sstore}{\senv}} = \symbeval{\symbeval{\jsilexpr_1}{\sstore}}{\senv} 
    = \symbeval{\loc}{\senv} = \loc \\ 
 %
  \symbeval{\jsilexpr_2}{\store}  = \symbeval{\jsilexpr_2}{\semexpr{\sstore}{\senv}} =  \symbeval{\symbeval{\jsilexpr_2}{\sstore}}{\senv}
   =  \symbeval{\sexpr_p}{\senv} \\
 %
 \heap = \semexpr{\sheap'' \, \uplus \, \big((l, \sexprp_i) \mapsto \sexprv_i\big)\mid_{i = 0}^n}{\senv} \\
    \qquad = \semexpr{\sheap''}{\senv} \dunion \biguplus_{0 \leq i \leq n} (l, \semexpr{\sexprp_i}{\senv}) \mapsto \semexpr{\sexprv_i}{\senv}
\end{array}
$$
Since $(\loc, -) \not\in \domain(\sheap'')$, we conclude that $(\loc, -) \not\in \semexpr{\sheap''}{\senv}$. 
%
Since $\senv \vdash \pc'$, we additionally conclude that: 
$
  \forall_{0 \leq i \leq n}  \, \symbeval{\sexprp_i}{\senv} \neq \symbeval{\sexpr_p}{\senv} 
$
Recalling that $\symbeval{\jsilexpr_2}{\store} = \symbeval{\sexpr_p}{\senv}$, we conclude that  
$
  \forall_{0 \leq i \leq n}  \, \symbeval{\sexprp_i}{\senv} \neq \symbeval{\jsilexpr_2}{\store}
$, from which it follows (together with $(\loc, -) \not\in \domain(\semexpr{\sheap''}{\senv})$) that 
$(\symbeval{\jsilexpr_1}{\store}, \symbeval{\jsilexpr_2}{\store}) \not\in \domain(\heap)$.
%
We can now apply the \prooflab{Member Check - False} rule of \jsil semantics, concluding: 
$$
   \semtrans{\heap, \store, \jvar := \hasfield(\jsilexpr_1, \jsilexpr_2)}{\heap,  \store[\jvar \mapsto \jfalse]}
$$
meaning that $\heap' = \heap$ and $\store' = \store[\jvar \mapsto \jfalse]$. 
%
Now we have to prove that $(\heap', \store') \in \smodels{\sheap', \sstore'}{\pc'}$.
Noting that:
$$
\begin{array}{l}
\heap' = \heap = \semexpr{\sheap}{\senv} = \semexpr{\sheap'}{\senv} \\
 %
 \store' = \store[\jvar \mapsto \jfalse] = \symbeval{\sstore}{\senv}[\jvar \mapsto \jfalse]] = \symbeval{\sstore[\jvar \mapsto \jfalse]}{\senv} = \symbeval{\sstore'}{\senv} 
\end{array}
$$
the result follows. 
\vspace{6pt}


\noindent\prooflab{Assert - True}
We conclude that  $\bcmd = \assert(\jsilexpr)$ for some expression $\jsilexpr$, and that: 
$$
  \sheap' = \sheap 
  \quad
  \sstore' =  \sstore 
  \quad
  \pc' = \pc
  \quad
  \pc \vdash  \symbeval{\jsilexpr}{\sstore}
$$ 
From $(\heap, \store) \in \smodels{\sheap, \sstore}{\pc'}$, we conclude that there is a symbolic environment
$\senv$ such that $\heap = \semexpr{\sheap}{\senv}$, $\store = \semexpr{\sstore}{\senv}$, and 
$\senv \vdash \pc$. 
We have now to prove that we can apply the \prooflab{Assert - True} rule in the concrete state.
To this end, we have to show that: $\symbeval{\jsilexpr}{\store} = \jtrue$. 
Noting that:
$
  \symbeval{\jsilexpr}{\store} = \symbeval{\jsilexpr}{\symbeval{\sstore}{\senv}} 
         = \symbeval{\symbeval{\jsilexpr}{\sstore}}{\senv} 
$, we conclude (using $\senv \vdash \pc$ and $\pc \vdash  \symbeval{\jsilexpr}{\sstore}$) that 
$\symbeval{\jsilexpr}{\store} = \jtrue$. 
We can now apply the \prooflab{Assert - True} rule of \jsil semantics, concluding: 
$$
   \semtrans{\heap, \store, \assert(\jsilexpr)}{\heap,  \store}
$$
meaning that $\heap' = \heap$ and $\store' = \store$. 
%
Now we have to prove that $(\heap', \store') \in \smodels{\sheap', \sstore'}{\pc'}$.
Noting that:
$$
\begin{array}{l}
\heap' = \heap = \semexpr{\sheap}{\senv} = \semexpr{\sheap'}{\senv}
 %
 \quad 
 %
 \store' = \store = \symbeval{\sstore}{\senv} = \symbeval{\sstore}{\senv} = \symbeval{\sstore'}{\senv} 
\end{array}
$$
the result follows. 
\vspace{6pt}

\noindent\prooflab{Assert - False}
We conclude that  $\bcmd = \assert(\jsilexpr)$ for some expression $\jsilexpr$, and that: 
$\pc \not\vdash  \symbeval{\jsilexpr}{\sstore}$. 
From $(\heap, \store) \in \smodels{\sheap, \sstore}{\pc'}$, we conclude that there is a symbolic environment
$\senv$ such that $\heap = \semexpr{\sheap}{\senv}$, $\store = \semexpr{\sstore}{\senv}$, and 
$\senv \vdash \pc$. 
We have now to prove that we can apply the \prooflab{Assert - False} rule in the concrete state.
To this end, we have to show that: $\symbeval{\jsilexpr}{\store} = \jfalse$. 
Noting that:
$
  \symbeval{\jsilexpr}{\store} = \symbeval{\jsilexpr}{\symbeval{\sstore}{\senv}} 
         = \symbeval{\symbeval{\jsilexpr}{\sstore}}{\senv} 
$, we conclude (using $\senv \vdash \pc$ and $\pc \not\vdash  \symbeval{\jsilexpr}{\sstore}$) that 
$\symbeval{\jsilexpr}{\store} = \jfalse$, from which the result follows. 
\end{proof}


\begin{temax}[Theorem~\ref{teo:soundness:jsil:symb:exe} - Soundess of \jsil symbolic execution]
$$
\begin{array}{l}
\symbtrans{\sheap, \sstore, \sctx[i], \pc}{\sheap', \sstore', \sctx'[j], \pc'} 
   \ \wedge \ 
      (\heap, \store, \ctx) \in \smodels{\sheap, \sstore, \sctx}{\pc'} \\ \quad \quad
      	 \ \implies \ \exists (\heap', \store', \ctx') \, . \, 
	 	 \semtrans{\heap, \store, \ctx[i]}{\heap', \store', \ctx'[j]}
		\, \wedge \, 
		(\heap', \store', \ctx') \in \smodels{\sheap', \sstore', \sctx'}{\pc'}  
\end{array}
$$
\end{temax}
%
\begin{proof}
We proceed by case analysis on $\symbtrans{\sheap, \sstore, \sctx[i], \pc}{\sheap', \sstore', \sctx'[j], \pc'}$. 

\end{proof}


\end{document}
